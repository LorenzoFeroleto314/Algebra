\documentclass{article}

\usepackage{import}
\import{}{preamble_def.tex}

\begin{document}
\AddToShipoutPicture*{\BackgroundPic}

\title{\Huge{\textbf{Appunti del corso di Algebra II}} \\ \vspace{3mm}
	   \LARGE{\textbf{Dipartimento di Matematica e Applicazioni,}} \\ 
	   \LARGE{\textbf{Università di Milano-Bicocca}} \\ \vspace{3mm}
	   \Large{\textbf{A.A. 2019/2020}} % \\ \vspace{2.5mm} \large{\textbf{A cura di F. Clerici}}
		 }

\vspace{-4mm}
\date{Versione del \today}

\maketitle
\thispagestyle{empty}
\clearpage

\tableofcontents

\

\

\noindent \textbf{Changelog (versione del 7 Ottobre 2020):}
\begin{itemize}
\item Reworking completo di varie cose
\end{itemize}

\thispagestyle{empty}

\noindent \textbf{To do (in ordine di importanza):}
\begin{itemize}
	\item Tutte le annotazioni che ho messo sull'iPad di capitoli 1.1, 1.2, 1.3
	\item Teoria dei moduli (lezioni dal 06/11/2019 fino alla fine del corso)
	\item Estensione di campi (lezioni del 25-30/10/19)
	\item Campi di spezzamento e campi finiti (lezioni del 05-06/11/2019)
	\item Domini a valutazione discreta (lezioni del 22-23/10/19)
	\item Capitolo 1.7: sistemare spacing, anello locale che non è dominio, proposizione 1.7.10
	\item Capitolo 1.5: riduzione mod p, Eisenstein, ciclotomici $x^{p-1}+...+x+1$
	\item Capitolo 1.4: polinomi di Laurent e serie formali (fix i due rif in anelli locali)
	\item Introduzione?
\end{itemize}
\clearpage

\setcounter{page}{1}
\section{Complementi di teoria degli anelli}
\import{sezione_1__complementi_di_teoria_degli_anelli/}{Capitolo 1.1 - Anelli di polinomi in una variabile.tex}
\import{sezione_1__complementi_di_teoria_degli_anelli/}{Capitolo 1.2 - Anelli di polinomi in n variabili.tex}
\import{sezione_1__complementi_di_teoria_degli_anelli/}{Capitolo 1.3 - Anelli di polinomi in più variabili}
\import{sezione_1__complementi_di_teoria_degli_anelli/}{Capitolo 1.4 - Polinomi di Laurent e serie formali.tex}
\import{sezione_1__complementi_di_teoria_degli_anelli/}{Capitolo 1.6 - Anelli noetheriani.tex}
\import{sezione_1__complementi_di_teoria_degli_anelli/}{Capitolo 1.7 - Localizzazione.tex}
\import{sezione_1__complementi_di_teoria_degli_anelli/}{Capitolo 1.8 - Domini a valutazione discreta.tex}








\end{document}