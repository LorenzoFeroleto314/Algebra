
\vspace{1.75mm}
\subsection{Anelli di polinomi in una variabile}

Introduciamo la struttura algebrica dei polinomi, ponendo la convenzione che 
$R$ indichera' un anello commutativo unitario e $\N = \{1,2,\dots\}, N_0 = \N\cup\{0\}$.

\begin{defn}[Anello di polinomi in una variabile e operazioni]{anello dei polinomi in una var}
Sia $R$ un anello commutativo unitario. Denotiamo l'\emph{anello dei polinomi a coefficienti in R
nella variabile x} come il seguente insieme 

\[ R[x]:=\left\{\sum\limits_{i=0}^n a_i x^i : a_i\in R,\, n\in \N_0\right\}, \]

\noindent dove sono definite due operazioni binarie interne. \\
Presi infatti due elementi 
$f(x)=\sum\limits_{i=0}^m a_i x^i$ e $g(x)=\sum\limits_{j=0}^n b_j x^j$ di $R[x],$ 
definiamo le operazioni binarie di \emph{somma} 

\[ + : R[x] \times R[x] \to R[x],\ f(x)+g(x)=\sum\limits_{i=0}^s (a_i+b_i)x^i \] 

\noindent dove abbiamo posto $s=\max\{m,n\}$ e $a_i=b_j=0_R$ per $i>m$ e $j>n$, e \emph{prodotto}

\[ \cdot : R[x] \times R[x] \to R[x],\ f(x)\cdot g(x)=\sum\limits_{k=0}^{m+n} \left(\sum\limits_{i=0}^{k}a_i b_{k-i}\right) x^k\] 
\end{defn}

\noindent Come visto nel corso di Algebra I, si verifica facilmente che $R[x]$ dotato di tali operazioni di somma e prodotto 
è un anello commutativo\footnote{Infatti $a_ib_{k-i}=b_{k-i}a_i$ essendo $R$ un anello commutativo per ipotesi, 
da cui $f(x)\cdot g(x)=g(x)\cdot f(x)$.} con elemento neutro il polinomio identicamente nullo $0_{R[x]}=0_{R}$ 
e unità il polinomio costante $1_{R[x]}=1_{R}$. Vediamone qualche esempio.


\begin{example}[]{exmp1}
Se prendiamo $R=\Z,$ $f(x)=x^2+2x+3$ e $g(x)=4x+5,$ si ha che 

\[ f(x)+g(x)=(1+0)x^2+(2+4)x+(3+5)=x^2+6x+8, \] 

{\setlength{\belowdisplayskip}{-2pt}\setlength{\abovedisplayskip}{0pt}
\begin{align*}
f(x)\cdot g(x) 
  &= (3\cdot 0+2\cdot 0+1\cdot 4+0\cdot 5)x^3+(3\cdot 0+2\cdot 4+1\cdot 5)x^2+(3\cdot 4+2\cdot 5)x+3\cdot 5 \\ 
  &= 4x^3+13x^2+22x+15. \ 
\end{align*}}
\end{example}

\vspace{1.5mm}

\noindent Di qui in seguito, denoteremo il prodotto di polinomi semplicemente come $f(x)g(x)$ o $f\cdot g$.
\noindent Possiamo quindi definire su $R[x]$ il concetto di ``grado'' di un polinomio.


\begin{defn}[Funzione grado; grado di un polinomio]{funzione grado}
Sia $R$ un anello e sia $f(x)=\sum\limits_{i=0}^n a_i x^i\in R[x]$. 
La funzione $\deg^{\star}\colon R[x]\to \N_0\cup \{\infty\}$ definita come 

$\deg^{\star}(f)=\begin{cases} \max\{k\in \N_0: a_k\neq 0_R\} 
\text{ \ se }f(x)\not\equiv 0_R \\ \infty 
\text{ \ \ \ \ \ \ \ \ \ \ \ \ \ \ \ \ \ \ \ \ \ \ \ \ \ \ \ se } 
f(x)\equiv 0_R\end{cases}$è detta \underline{grado}.\footnotemark
\end{defn}

\footnotetext{Sarebbe più corretto scrivere $\deg^{\star}(f(x)),$ ma si preferisce evitare l'uso di troppe parentesi. 
Ricordiamo che con $f(x)\equiv k$ si intende il polinomio costante uguale a $k$. 
Tale notazione serve per non confondere un polinomio costante $p(x)\equiv 0$ con l'equazione algebrica $p(x)=0.$}

\noindent Tale definizione coincide con quella classica di grado di un polinomio tranne nel caso in cui 
$f(x)$ sia identicamente nullo. Infatti, per questa definizione $f(x)\equiv 0_R$ è l'unico polinomio di grado infinito, 
mentre secondo quella classica anch'esso ha grado $0$ in quanto costante. 

\begin{exm}Se consideriamo i polinomi $f(x)=x^2+1,$ $g(x)\equiv 1$ e $h(x)\equiv 0$ in $\Z[x]$, 
si ha che $\deg^{\star}(f)=2$ e $\deg^{\star}(g)=0,$ ma $\deg^{\star}(h)=\infty. \ \square$\end{exm}

\noindent Possiamo ora dimostrare un risultato che mette in relazione l'anello dei polinomi con quello dei suoi coefficienti, 
nel caso in cui quest'ultimo sia un dominio di integrità.
\footnote{Ricordiamo che un dominio di integrità è un anello commutativo unitario $R\neq \{0_R\}$ senza divisori dello zero, 
cioè in cui $ab=0_R$ se e solo se $a=0_R$ o $b=0_R$. Esempi di domini di integrità sono $\Z,$ 
le classi di resto $\mathbb{F}_p=\Z/p\Z$ con $p$ primo, gli interi gaussiani
$\Z[i]=\left\{a+bi: a,b\in\Z\right\}$ e $\Z[\sqrt{2}\,]=\{a+b\sqrt{2}: a,b\in \Z\}$.}

\begin{prop}[]{prop sul grado}
Sia $R$ un dominio di integrità. Allora, per ogni $f(x),\,g(x)\in R[x]$ vale \[\deg^{\star}(f\cdot g)=\deg^{\star}(f)+\deg^{\star}(g). \ \bf{(\star)}\] In particolare, $R[x]$ è un dominio di integrità se e solo se $R$ è un dominio di integrità.
\end{prop}
\vspace{-4mm}
\begin{proof} Osserviamo innanzitutto che se almeno uno tra $f(x)$ e $g(x)$ è identicamente nullo, 
allora $\bf{(\star)}$ è vera perché $f(x)g(x)\equiv 0_R$ e quindi 

\[\deg^{\star}(f\cdot g)=\infty=\deg^{\star}(f)+\deg^{\star}(g).\] 

\noindent D'altra parte, siano $f(x)=\sum\limits_{i=0}^m a_i x^i$ e $g(x)=\sum\limits_{j=0}^n b_j x^j$ 
non nulli con $a_m\neq 0_R$ e $b_n\neq 0_R$. Poiché $R$ è un dominio di integrità, $a_m b_n\neq 0_R$, 
cioè $a_m b_n x^{m+n}$ è il monomio di grado massimo nel prodotto $f(x)g(x)$. Per definizione di grado, 
concludiamo quindi che \[\deg^{\star}(f\cdot g)=m+n=\deg^{\star}(f)+\deg^{\star}(g).\]

\noindent Sia ora $R$ un dominio di integrità, e mostriamo che lo è anche $R[x]$. 
Osserviamo innanzitutto che $R[x]$ è un anello commutativo unitario, in quanto eredita tali proprietà da $R$. 
Inoltre, presi $f(x),\,g(x)\in R[x]$ tali che $f(x)g(x)\equiv 0_R$, per quanto appena mostrato vale 

\[\deg^{\star}(f)+\deg^{\star}(g)=\deg^{\star}(f\cdot g)=\deg^{\star}(0_R)=\infty.\] 

\noindent Dunque, almeno uno fra $f(x)$ e $g(x)$ ha grado infinito ed è quindi il polinomio nullo, 
cioè $R[x]$ non ha divisori dello zero ed è effettivamente un dominio di integrità.
\vspace{1mm}
\noindent Viceversa, sia $R[x]$ un dominio di integrità. Allora, $R\subseteq R[x]$ è commutativo e unitario in quanto sottoanello, 
e presi $a,b\in R,$ possiamo vedere $a$ e $b$ come polinomi costanti in $R[x]$. Essendo $R[x]$ un dominio di integrità, 
$ab=0_R$ se e solo se $a=0_R$ o $b=0_R$, da cui anche $R$ non ha divisori dello zero ed è quindi un dominio di integrità.
\end{proof}

\noindent Osserviamo che $\bf{(\star)}$ non vale quando l'anello $R$ non è un dominio di integrità.

\begin{example}[]{esempio sul grado}
Siano $f(x)=2x+1$ e $g(x)=3x+2$ in $\Z/6\Z\,[x]$. Allora, $\deg^{\star}(f)=\deg^{\star}(g)=1$, 
ma $f(x)g(x)=6x^2+7x+2\equiv_6 x+2$, da cui $\deg^{\star}(f\cdot g)=1\neq 2=\deg^{\star}(f)+\deg^{\star}(g).$
\end{example}

\noindent Più in generale, se $R$ non è un dominio di integrità, per definizione esistono $a,b\in R$ non nulli 
tali che $ab=0_R$. Allora, detti $f(x)=ax$ e $g(x)=bx$, si ha $f(x)g(x)=abx^2=0_Rx^2=0_R$, da cui, essendo 
$\deg^{\star}(f)=\deg^{\star}(g)=1$, l'uguaglianza $\bf{(\star)}$ non vale perché 

\[\deg^{\star}(f\cdot g)=\deg^{\star}(0_R)=\infty\neq 2=\deg^{\star}(f)+\deg^{\star}(g).\] 

\noindent Dunque, per la proposizione \ref{prop sul grado} segue che $\bf{(\star)}$ vale se e solo se $R$ è un dominio di integrità.
\vspace{1.5mm}

\noindent Prima di procedere nello studio degli anelli di polinomi, richiamiamo il concetto di elemento invertibile di un anello. 
Preso un anello $R$, sia $R^{\times}$ l'insieme degli elementi di $R$ che hanno inverso moltiplicativo, 
cioè l'insieme degli $a\in R$ per cui esiste $b\in R$ tale che $ab=1_R$. Se esiste, denotiamo l'inverso moltiplicativo 
di $a$ con $a^{-1}$. Allora, vale la proposizione seguente.

\begin{prop}[]{gruppo degli el inveribili e' gruppo}
Sia $R$ un anello. Allora, $R^{\times}$ è un gruppo rispetto al prodotto.
\end{prop}
\vspace{-4mm}
\begin{proof}
Osserviamo innanzitutto che il prodotto è associativo essendo $R$ un anello, e in particolare $1_R$ è l'unità anche di 
$R^{\times}$. Inoltre, presi $a,b\in R^{\times}$, per definizione esistono $c,d\in R$ tali che $ac=1_R$ e $bd=1_R$, 
dunque \[(ab)(dc)=a(bd)c=a1_Rc=ac=1_R,\] cioè $ab\in R^{\times}$ è invertibile con inverso $dc$, da cui $R^{\times}$ 
è chiuso rispetto al prodotto. Infine, se $ab=1_R$ è evidente che anche $a^{-1}=b\in R^{\times}$, 
dunque $(R^{\times},\cdot)$ è effettivamente un gruppo.
\end{proof}

\noindent Grazie a tale proposizione, la definizione seguente risulta quindi ben posta.

\begin{defn}[Gruppo moltiplicativo di un anello]{defn gruppo moltiplicativo}
Sia $R$ un anello unitario commutativo. L'insieme $R^{\times}$ degli elementi di $R$ 
che ammettono inverso moltiplicativo è un gruppo detto 
\emph{gruppo moltiplicativo di $R$}.\footnotemark
\end{defn}

\footnotetext{Tale gruppo viene spesso indicato anche con $\mathcal{U}(R)$ o $R^{\star}$ ed è anche detto 
``gruppo delle unità di $R$''.}

\noindent Se da una parte la proposizione \ref{prop sul grado} mostra che $R[x]$ può avere la struttura di un dominio di integrità, l'anello dei polinomi $R[x]$ non è mai un campo, nemmeno se lo è $R$ stesso.\footnote{Vedremo nel \emph{Capitolo 1.4} una generalizzazione degli anelli di polinomi con la struttura di un campo.} Infatti, $x\in R[x]$ non è un elemento invertibile perché il suo inverso $1/x$ non è un polinomio.\footnote{Più rigorosamente, se $f(x)=x$ fosse invertibile, esisterebbe $g(x)\in R[x]$ tale che $f(x)g(x)=1_R$, da cui $\deg^{\star}(f\cdot g)=\deg^{\star}(1_R)=0=\deg^{\star}(f)+\deg^{\star}(g)$, cioè $\deg^{\star}(g)=-\deg^{\star}(f)=-1<0$, assurdo.} Risulta quindi naturale chiedersi quali elementi di $R[x]$ siano effettivamente invertibili.


\begin{prop}[]{se R dom R[x] dom}
Sia $R$ un dominio di integrità. Allora, $R[x]^{\times}=R^{\times}$.
\end{prop}
\vspace{-4mm}
\begin{proof}
Poiché ogni elemento di $R^{\times}$ può essere visto come polinomio costante di~$R[x]$, è evidente che 
$R^{\times}\subseteq R[x]^{\times}$. D'altra parte, siano $f(x),\,g(x)\in R[x]^{\times}$ tali che 
$f(x)g(x)=1_R$. Allora, per la \emph{Proposizione 1.1.1} si ha che 

\[\deg^{\star}(f\cdot g)=\deg^{\star}(1_R)=0=\deg^{\star}(f)+\deg^{\star}(g),\] 

\noindent quindi $\deg^{\star}(f)=\deg^{\star}(g)=0$ essendo il grado non negativo. Questo prova che 
ogni elemento di $R[x]^{\times}$ è in realtà una costante invertibile, cioè 
$R[x]^{\times}\subseteq R^{\times}$, dunque $R[x]^{\times}=R^{\times}$.
\end{proof}


\noindent\emph{\underline{Come esprimere una relazione piu' generica di sottoanello?}}
\vspace{2mm}\\
Sia $R$ un anello, e supponiamo di voler aggiungere a $R$ un certo elemento $x\notin R$ 
senza alcuna relazione con gli altri elementi di $R$, in modo che la struttura algebrica risultante sia ancora un anello 
e sia la più piccola possibile. Come possiamo fare? 
\vspace{1.5mm}
\noindent Poiché ogni anello è chiuso rispetto a somma e prodotto, tale struttura conterrà anche tutte le potenze non negative 
$\{x^0, x^1, x^2, ...\}$ di $x$ e tutte le combinazioni lineari tra potenze di $x$ ed elementi di $R$, 
cioè tutti gli elementi della forma $a_nx^n + ... + a_1x+a_0$ con $a_0,...\,,a_n\in R$. 
Dunque, l'anello dei polinomi $R[x]$ sembra essere la struttura che soddisfa le nostre richieste, 
cioè il più piccolo anello contenente sia $R$ che $x$. Resta solo da formalizzare meglio il concetto di ``più piccolo anello'', 
cioè chiarire cosa significa che un anello ne contiene un altro.
\vspace{1.5mm}
\noindent A questo scopo, potremmo considerare sull'insieme degli anelli la relazione d'ordine data dall'inclusione, 
cioè dire che un anello $R$ è più piccolo di un altro anello $S$ se e solo se $R\subseteq S$. 
Tuttavia, questo non terrebbe conto dell'importanza algebrica degli isomorfismi: infatti, 
la struttura che stiamo cercando di costruire è definita a meno di isomorfismi, 
e anelli isomorfi potrebbero essere non confrontabili secondo l'inclusione.
\footnote{Ad esempio, si verifica facilmente che la mappa 

$\varphi\colon \C\to \Mat_{2\times 2}(\R)$, $a+bi \mapsto 
\begin{pmatrix} a & b \\ -b & a \end{pmatrix}$ 

è un isomorfismo di anelli, ma 
$\C\not\subseteq \operatorname{Mat}_{2\times 2}(\R)$ e $\operatorname{Mat}_{2\times 2}(\R)\not\subseteq \C$, 
cioè tali anelli non sono confrontabili secondo l'inclusione.} Per risolvere tale problema, 
ha quindi più senso definire che $R$ è più piccolo di $S$ se e solo se $S$ contiene una copia isomorfa dell'anello $R$, 
cioè se e solo se esiste un sottanello di $S$ isomorfo a $R$.
\clearpage

\begin{defn}[Anello piu' piccolo]{anello piu piccolo}
Siano $R$ e $S$ due anelli. Diciamo che $R$ è \underline{più piccolo} di $S$ (o anche che $S$ contiene $R$) 
se e solo se esiste un omomorfismo di anelli iniettivo $\varphi \colon R\to S$.
\end{defn} 

\noindent Si osservi che tale definizione è equivalente a quanto detto sopra: se esiste un monomorfismo 
(cioè un omomorfismo iniettivo) $\varphi\colon R\to S$, la restrizione $\varphi\colon R\to \varphi(R)$ è un isomorfismo, 
dunque l'immagine $\varphi(R)\subseteq S$ è un sottoanello di $S$ isomorfo a $R$.

\begin{example}[]{esempio anello piu piccolo}
Chiaramente $\R$ non è un sottoanello di $\R^2$, in quanto $\R\not\subseteq \R^2$. D'altra parte, la mappa 
$\varphi\colon \R\to \R^2$, $x\mapsto (x,x)$ è un omomorfismo iniettivo, quindi $\R^2$ contiene una copia isomorfa di $\R$, 
che geometricamente corrisponde alla bisettrice $y=x$.
\end{example}

\noindent\emph{Osservazione.} Tornando al problema iniziale, sia $X$ la struttura algebrica che stiamo cercando di costruire. 
Allora, possiamo riformulare le condizioni su $X$ come segue:

\begin{itemize}
\item $X$ contiene $R\Rightarrow$ esiste un monomorfismo $\iota\colon R\to X$;
\item $X$ è il più piccolo anello contenente sia $R$ che $x\notin R\Rightarrow$ per ogni altro anello $S$ 
con tali proprietà (cioè tale che esista un monomorfismo $\varphi\colon R\to S$ e contenente un $s\notin R$), 
abbiamo che $X$ è più piccolo di $S$, ossia esiste un monomorfismo $\phi\colon X\to S$. 
\end{itemize}

\noindent In particolare, richiediamo che tale mappa $\phi$ soddisfi $\phi(x)=s$ e $\phi(\iota(R))=\varphi(R)$, 
cioè che mandi l'elemento aggiunto $x$ nell'elemento aggiunto $s$ e la copia isomorfa $\iota(R)$ di $R$ in $X$ 
nella copia isomorfa $\varphi(R)$ di $R$ in $S$.

\[
\begin{tikzcd}
R \arrow[rr, "\varphi", hook] \arrow[dd, "\iota"', hook] &  & S \\
                                                         &  &   \\
X \arrow[rruu, "\phi"', dashed, hook]                    &  &  
\end{tikzcd}
\]

\noindent Osserviamo ora che l'anello dei polinomi $R[x]$ soddisfa effettivamente tali proprietà. 
Infatti, detta $\iota\colon R\to R[x]$ la mappa di inclusione che manda ogni elemento $r\in R$ 
nel corrispondente polinomio costante $r\in R[x]$, è evidente che $\iota$ sia un monomorfismo, 
e preso un qualunque monomorfismo $\varphi\colon R\to S$, basta definire $\phi\colon R[x]\to S$ 
ponendo $\phi(x)=s$ e $\phi(\iota(r))=\varphi(r)$ per ogni $r\in R$. 
Tale mappa si estende per linearità su tutto $R[x]$ ponendo 

$$\phi\left( \sum\limits_{i=0}^n r_ix^i \right)=\sum\limits_{i=0}^n \varphi(r_i)s^i$$ 

\noindent ed è facile verificare che $\phi$ sia un monomorfismo.
\footnote{Approfondiremo meglio questa questione nel \emph{Capitolo 2.1} quando tratteremo le estensioni di campi.} 
Più in generale, vale il teorema seguente.

\begin{teo}[Proprietà universale]{proprieta universale}
Siano $R$ e $S$ due anelli e sia $\varphi\colon R\to S$ un omomorfismo. Allora, per ogni $s\in S$ esiste 
un unico omomorfismo di anelli $\phi\colon R[x]\to S$ tale che $\phi(x)=s$ e $\phi \raisebox{-.5em}{$\vert_{R}$}= \varphi$.
\end{teo}
\vspace{-4mm}
\begin{proof}Siano $f(x)=\sum\limits_{i=0}^m a_i x^i$ e $g(x)=\sum\limits_{j=0}^n b_j x^j$ in $R[x]$ 
e sia $\phi(f)=\sum\limits_{i=0}^m \varphi(a_i) s^i$. Osserviamo innanzitutto che $\phi(f)$ è ben definita. 
Infatti, $\varphi(a_i)\in S$ e $\phi(f)\in S$ perché somma di prodotti di elementi dell'anello $S$, 
che è chiuso rispetto a somma e prodotto. Inoltre, $\phi(x)=\varphi(1_R)s^1=s$ e $\phi(r)=\varphi(r)s^0=\varphi(r)$ 
per ogni $r\in R$, quindi $\phi$ soddisfa le condizioni richieste. Mostriamo ora che $\phi$ preserva le operazioni. Infatti, 

\[ \phi(f+g)=\sum\limits_{i=0}^{\max\{m,n\}}\varphi(a_i+b_i)s^i=\sum\limits_{i=0}^{m}\varphi(a_i)s^i+\sum\limits_{j=0}^{n}\varphi(b_j)s^j=\phi(f)+\phi(g) \] 

\noindent per la distributività del prodotto rispetto alla somma e perché $\varphi(a_i+b_i)=\varphi(a_i)+\varphi(b_i)$, e 

\[ \phi(f\cdot g)=\sum\limits_{k=0}^{m+n}\left(\sum\limits_{i=0}^{k}\varphi(a_ib_{k-i})\right)
s^k=\left(\sum\limits_{i=0}^{m}\varphi(a_i)s^i\right)\left(\sum\limits_{j=0}^{n}\varphi(b_j)s^j\right)=\phi(f)\cdot \phi(g) \] 

\noindent per come è definito il prodotto tra polinomi e perché $\varphi(a_ib_{k-i})=\varphi(a_i)\varphi(b_{k-i})$ essendo $\varphi$ 
un omomorfismo. Poiché $\phi(0_{R[x]})=\varphi(0_R)=0_S$ e $\phi(1_{R[x]})=\varphi(1_R)=1_S,$ 
concludiamo che tale mappa $\phi$ è effettivamente un omomorfismo di anelli.

\vspace{2mm}

\noindent Mostriamo ora che $\phi$ è unico. Sia $\psi\colon R[x]\to S$ un altro omomorfismo di anelli 
tale che $\psi(x)=s$ e $\psi \raisebox{-.5em}{$\vert_{R}$}= \varphi$. 
Poiché $\psi$ preserva le operazioni, per ogni $f(x)=\sum\limits_{i=0}^m a_i x^i\in R[x]$ vale 

\[ \psi(f) = \psi\left( \sum\limits_{i=0}^m a_i x^i \right) = 
\sum\limits_{i=0}^m \psi(a_i) \psi(x^i)=\sum\limits_{i=0}^m \varphi(a_i) \psi(x)^i=\sum\limits_{i=0}^m \varphi(a_i) s^i=\phi(f) \] 

\noindent essendo $\psi(a_i)=\varphi(a_i)$ perché $a_i\in R$ e $\psi(x^i)=\psi(x)^i=s^i$. 
Dunque, $\psi$ coincide con $\phi$ per ogni polinomio $f(x)\in R[x]$, da cui $\phi$ è unico.
\end{proof}

\noindent Nel caso particolare in cui $\varphi=\id_R$ e quindi $R\subseteq S$, 
la mappa $\phi$ di cui sopra viene spesso denotata con $\phi_s$. 
In questo caso, $\phi_s(f)$ non è altro che il polinomio $f(x)$ calcolato in $x=s$, 
cioè $\phi_s(f)=f(s)$, il che spiega l'origine del nome ``valutazione in $s$'' per tale mappa.

\vspace{3mm}
\begin{defn}[Valutazione di un polinomio]{valutazione polinomio}
Tale omomorfismo di anelli $\phi_s$ è detto \emph{valutazione in $s$}.
\end{defn}
\clearpage

\begin{example}[]{esempio valutazione polinomio}
Se $R=\Z$, $S=\Z[\sqrt{2}\,]$ e $f(x)=x^2+2x+3\in \Z[x]$, detta $\phi_{\sqrt{2}}\colon \Z[x]\to \Z[\sqrt{2}\,]$ 
la valutazione in $\sqrt{2}$, abbiamo che $\phi_{\sqrt{2}}(f)=(\sqrt{2})^2+2\sqrt{2}+3=5+2\sqrt{2}\in \Z[\sqrt{2}\,]$.
\end{example}

\noindent Vogliamo ora dimostrare che la \emph{proprietà universale} è una caratteristica propria degli anelli di polinomi, 
cioè che se $T$ è un anello contenente sia $R$ che un elemento $t\notin R$ e dotato della \emph{proprietà universale}, 
allora $T\cong R[x]$. Nella dimostrazione ci limiteremo al caso in cui $R\subseteq T$ e $\varphi=\id_R$ (e quindi $R\subseteq S$), 
ma il caso generale è del tutto analogo.

\begin{teo}[]{asdf}
Sia $R$ un anello e sia $T\supseteq R$ un anello contenente un elemento $t\notin R$ e tale che per ogni anello $S\supseteq R$ e 
per ogni $s\in S$ esista un unico omomorfismo di anelli 
$\psi \colon T\to S$ con $\psi(t)=s$ e $\psi \raisebox{-.5em}{$\vert_{R}$}= \id_R$. Allora, $T\cong R[x]$.
\end{teo}
\vspace{-4mm}
\begin{proof}Poiché per ipotesi tale proprietà vale per ogni anello $S\supseteq R$, 
in particolare scegliamo $S=R[s]$ e siano $\phi_t\colon R[s]\to T$ la valutazione in $t$
\footnote{Ricordiamo che per il \emph{Teorema \ref{proprieta universale}} tale omomorfismo è l'unico che soddisfa 
$\phi_t(s)=t$ e $\phi_t \raisebox{-.5em}{$\vert_{R}$}= \operatorname{id}_R$.} e $\alpha=\phi_t\circ \psi\colon T\to T$. 
\vspace{-2mm}
\[
  \begin{tikzcd}
    T \arrow{r}{\psi} \arrow[swap]{dr}{\alpha} & R[s] \arrow{d}{\phi_t} \\
     & T
  \end{tikzcd}
\]
Osserviamo innanzitutto che $\alpha$ è ben definito ed è un omomorfismo in quanto composizione di omomorfismi. 
Inoltre, $\alpha(t) = \phi_t(\psi(t)) = \phi_t(s)=t$ e $\alpha(r)=\phi_t(\psi(r))=\phi_t(r)=r$ 
per ogni $r\in R$, cioè $\alpha \raisebox{-.5em}{$\vert_{R}$}= \operatorname{id}_R$. 
D'altra parte, poiché $T\supseteq R$, possiamo scegliere $S=T$ e $s=t$ nell'enunciato del teorema, 
così sappiamo che esiste un unico omomorfismo $\psi'\colon T\to T$ 
tale che $\psi'(t)=t$ e $\psi' \raisebox{-.5em}{$\vert_{R}$}= \id_R$. 
Poiché anche l'identità $\id_T\colon T\to T$ soddisfa tali proprietà, 
per l'unicità di $\psi'$ deve essere $\alpha=\id_T$. Sia ora $\beta=\psi\circ \phi_t\colon R[s]\to R[s]$.
\vspace{-2mm}
\[
  \begin{tikzcd}
    R[s] \arrow{r}{\phi_t} \arrow[swap]{dr}{\beta} & T \arrow{d}{\psi} \\
     & R[s]
  \end{tikzcd}
\]
Come sopra, osserviamo che $\beta$ è ben definito ed è un omomorfismo in quanto composizione di omomorfismi. 
Inoltre, $\beta(s) = \psi(\phi_t(s)) = \psi(t)=s$ e $\beta(r)=\psi(\phi_t(r))=\psi(r)=r$ 
per ogni $r\in R$, cioè $\beta \raisebox{-.5em}{$\vert_{R}$}= \operatorname{id}_R$. 
Poiché anche l'identità $\id_{R[s]}\colon R[s]\to R[s]$ soddisfa $\id_{R[s]}(s)=s$ 
e $\id_{R[s]} \raisebox{-.5em}{$\vert_{R}$}= \id_R$, e per il \emph{Teorema 1.1.4} 
esiste un unico omomorfismo con queste proprietà, deve essere $\beta=\operatorname{id}_{R[s]}$. 
Dunque, essendo $\phi_t\circ \psi=\operatorname{id}_T$ e $\psi\circ \phi_t=\operatorname{id}_{R[s]}$ isomorfismi, 
lo sono anche $\psi$ e $\phi_t$,\footnote{In generale, se $f\colon X\to Y$ e $g\colon Y\to X$ sono omomorfismi 
tali che $g\circ f=\operatorname{id}_X$ e $f\circ g=\operatorname{id}_Y,$ allora $f$ e $g$ sono isomorfismi. 
Infatti, $f$ è iniettivo perché $f(x)=f(x')\Rightarrow x=g(f(x))=g(f(x'))=x',$ ed è suriettivo perché 
preso $y\in Y,$ si ha che $g(y)\in X$ e $f(g(y))=y.$ In modo del tutto analogo si dimostra che anche $g$ è un isomorfismo, 
e in particolare risulta quindi che $g=f^{-1}.$} da cui concludiamo che $T\cong R[s]\cong R[x].
\footnote{Infatti $s$ è solo un nome qualunque per la variabile dei polinomi a coefficienti in $R$.}$
\end{proof}