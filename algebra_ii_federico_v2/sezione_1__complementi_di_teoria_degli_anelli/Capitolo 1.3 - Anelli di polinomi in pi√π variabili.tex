\subsection{Anello dei polinomi in più variabili}

In questo paragrafo vogliamo generalizzare il concetto di anello dei polinomi ad un numero qualsiasi di variabili $"x"$ appartenenti ad un insieme $\mathcal{X}$ anche infinito.
l'anello dei polinomi nel caso in cui le variabili
siano degli elementi $x \in \mathcal{X}$. Per fare ciò sono necessarie delle definizioni preliminari degli oggetti che verranno coinvolti.

%ho dovuto mettere una doppia S iniziale perchjè sennò me la faceva scomparire sul pdf...
\begin{defn}[supporto di una funzione]{supp(f)}
	Sia $f : X \longrightarrow Y$ una funzione tra due insiemi dotati di operazioni binarie ed elemento neutro.
	Si dice \emph{\underline{supporto di f}} il seguente insieme:
	\[supp(f) := \{x \in X \mid f(x) \neq 0_Y\}\]
\end{defn}

\noindent
In particolare, in questa sezione useremo un insieme di funzioni a supporto finito:
\[\mathcal{F}^\times(\mathcal{X},\N_0) := \{\alpha : \mathcal{X} \rightarrow \N_0 : \vert supp(\alpha) \vert < \infty\}\]
In questo modo $\alpha(x) \neq 0$ solo in un numero finito di $x \in \mathcal{X}$. Questo particolare insieme, 
d'ora in poi verrà denotato semplicemente come $\mathcal{F}^\times$ per evitare di appesantire la notazione.\\
Inoltre su $\mathcal{F}^\times$ è possibile definire una struttura algebrica di monoide abeliano tramite la somma puntuale:
\[(\alpha + \beta)(x) = \alpha(x) + \beta(x) \qquad \forall \alpha, \beta \in \mathcal{F}^\times\]
\[con \ elemento \ neutro \ 0_{\mathcal{F}^\times}(x) = 0 \qquad \forall x \in \mathcal{X}\]
La somma definita sopra è ben posta, infatti, se $\alpha, \beta \in \mathcal{F}^\times \Rightarrow \mid
 supp(\alpha) \mid, \mid supp(\beta) \mid < \infty$. Quindi se $x \in supp(\alpha + \beta) \Rightarrow \alpha(x)
  + \beta(x) \neq 0 \Rightarrow \alpha(x) \neq 0 \vee \beta(x) \neq 0 \Rightarrow x \in supp(\alpha) \cup supp(\beta)$ e quindi, 
  $supp(\alpha + \beta) \subseteq supp(\alpha) \cup supp(\beta)$ che passando alle cardinalità diventa:
  \[\mid supp(\alpha + \beta) \mid \ \leq \ \mid supp(\alpha) \cup supp(\beta) \mid \ \leq \ \mid supp(\alpha) \mid + \mid supp(\beta) \mid 
  < \infty \quad ovvero \quad\alpha + \beta \in \mathcal{F}^\times\]

\noindent
Fatte queste considerazioni è possibile definire formalmente \emph{l'insieme dei monomi monici nelle variabili $x \in \mathcal{X}$}:
\begin{defn}[monomi monici]{monomi monici}
	Sia $R$ un anello commutativo e sia $\mathcal{X}$ un insieme.\\
	Per ogni $\alpha \in \mathcal{F}^\times$ è definito monomio monico $X^\alpha$ la seguente produttoria:
	\[X^\alpha := \prod_{x \in \mathcal{X}}x^{\alpha(x)} \qquad con \ la \ convenzione \qquad x^0 = 1_R \ \forall x \in \mathcal{X}\]
	L'insieme di tutti i monomi monici su $\mathcal{X}$ è: $\{X^\alpha \mid \alpha \in \mathcal{F}^\times\} =: M$
\end{defn}

\noindent
E' importante osservare che $\mathcal{X}$ può essere anche un insieme infinito, ma dato che le $\alpha \in \mathcal{F}^\times$ 
si ha che ogni monomio sarà costituito solo da un numero finito di variabili (le altre sono tutte $1_R$ e pertanto trascurabili).

%%%%%%%%%%%%%%%%%
\newpage
%%%%%%%%%%%%%%%%%  

\begin{example}[]{esempio monomi monici}
	Sia per esempio $R = \Z, \ \mathcal{X} = \{x_1, x_2, x_3, x_4\}, \ \alpha(x_i) = i \ \forall \ i \in \{1, 2, 3\} \ e \ \alpha(x_4) = 0.$\\
	Allora si ha che:
	\[X^\alpha := \prod_{x \in \mathcal{X}}x^{\alpha(x)} = x_1 \cdot x_2^2 \cdot x_3^3 \cdot 1 = x_1x_2^2x_3^3\]
\end{example}


\noindent Sia $\mathcal{F}(\mathcal{F}^\times,R) := \{f : \mathcal{F}^\times \rightarrow R : \vert supp(f) \vert < \infty\}$, 
cioè l'insieme delle funzioni $f$ che ad ogni elemento di $\mathcal{F}^\times$ associano un elemento di $R$ e che non valgono $0_R$ 
solo per un numero finito di elementi di $\mathcal{F}^\times$.\\
Sia $r_{\_} \in \mathcal{F}(\mathcal{F}^\times,R)$ la funzione che per ogni $\alpha \in \mathcal{F}^\times$ associa l'elemento $r_\alpha \in R$. 
Osserviamo che ora possiamo definire un qualsiasi polinomio $"f"$ nelle variabili $x \in \mathcal{X}$ come 
\textbf{combinazione lineare di monomi a coefficienti in} $R$:
\[f = \sum_{\alpha \in \mathcal{F}^\times}r_\alpha X^\alpha\]
e dato che $r_{\_} \in \mathcal{F}(\mathcal{F}^\times,R)$ ho che il polinomio $f$ sarà composto da una somma di finita di termini non nulli.\\

\noindent
Siamo quindi pronti a definire l'anello dei polinomi nelle variabili $x \in \mathcal{X}$.
\begin{defn}[Anello dei polinomi nelle variabili $x \in \mathcal{X}$]{anello polinomi in più variabili}
	Sia $R$ un anello commutativo.\\
	Si definisce \emph{anello dei polinomi nelle variabili} $x \in \mathcal{X}$ l'insieme:
	\[R[\mathcal{X}] := \{f = \sum_{\alpha \in \mathcal{F}^\times}r_\alpha X^\alpha \mid r_{\_} \in \mathcal{F}(\mathcal{F}^\times,R)\}\]
  con le seguenti operazioni di somma e di prodotto. Dati $f, g \in R[\mathcal{X}]$:
  \[f(X) = \sum_{\alpha \in \mathcal{F}^\times}r_\alpha X^\alpha, \quad g(X) = \sum_{\beta \in \mathcal{F}^\times}s_\beta X^\beta\]\\
  \begin{align*}
    %SOMMA
    somma: \quad &+ : R[\mathcal{X}] \times R[\mathcal{X}] \longrightarrow R[\mathcal{X}] \quad 
    f(X) + g(X) = \sum\limits_{\alpha \in \mathcal{F}^\times}(r_\alpha + s_\alpha) X^\alpha\\
    %PRODOTTO
    prodotto: \quad &\cdot : R[\mathcal{X}] \times R[\mathcal{X}] \longrightarrow R[\mathcal{X}] \quad 
    f(X) \cdot g(X) = \sum\limits_{\gamma \in \mathcal{F}^\times}t_\gamma X^\gamma\\
    dove abbiamo posto $\gamma = \alpha + \beta$ e $t_\gamma = \sum\limits_{\alpha + \beta = \gamma} r_\alpha s_\beta$.
  \end{align*}
	%ZERO E UNO
	\[elemento \ neutro: \quad 0_R = \sum_{\alpha \in \mathcal{F}^\times} 0_\alpha X^\alpha, \quad unita' \ moltiplicativa: \quad 1_R = X^0 \]
\end{defn}

%%%%%%%%%%%%
\newpage
%%%%%%%%%%%%

\noindent
Anche per gli anelli di polinomi in più variabili vale la cosiddetta \textit{Proprietà Universale}.

%NON RIESCO A CREARE L'AMBIENTE TEOREMA. INOLTRE MI DA ERRORE DI COMANDI dopo "omorfismo di anelli", MA
%STICAZZI PLOTTA COMUNQUE
\begin{prop}[Proprietà Universale]{Proprietà univ. anello polinomi piu variabili}
	Sia $X$ un insieme e sia $R$ un anello commutativo.\\
	Allora, per ogni anello commutativo $S\supseteq R$ e per ogni mappa $\varphi\colon \mathcal{X} \to S$ esiste un unico omomorfismo di anelli 
	$\phi\colon R[\mathcal{X}]\to S$ tale che $\phi(X^{\delta_x})=\varphi(x) \ \forall x \in \mathcal{X} \ e \ \phi |_{\vert_{R}}= id_R$, dove:
  \[\delta_x \colon \mathcal{X} \to \N , \delta_x(y)=
  \begin{cases}
    1 \text{ \ se }y=x\\
    0 \text{ \ se } y\neq x
  \end{cases}\]
\end{prop}

\begin{proof}
	Siano $f=\sum\limits_{\alpha \in \mathcal{F}^\times} r_\alpha X^\alpha$, e $g=\sum\limits_{\beta \in \mathcal{F}^\times} s_\beta X^\beta$, due elementi di $R[\mathcal{X}]$. 
	Per ogni monomio $X^\alpha \in M$, sia $\phi(X^\alpha) = \prod\limits_{x \in X} \varphi(x)^{\alpha(x)}$, e sia quindi $\phi(f) = \sum\limits_{\alpha \in \mathcal{F}^\times} r_\alpha \phi(X^\alpha).$ 
	Poiché $r_\alpha \in R \subseteq S$ per ipotesi e $\phi(f) \in S$ perché somma di prodotti di elementi di $S$ (che in quanto anello è chiuso rispetto a somma e prodotto), $\phi$ è ben definita. 
	Inoltre, $\phi(X^{\delta_x}) = \varphi(x)$ e $\phi(\rho) = \rho$ per ogni $\rho \in R$, quindi $\phi$ soddisfa le condizioni richieste.
	%FOOTNOTE
	\footnote{$\delta_x$ è la funzione tale che per ogni $x \in X$ si abbia $X^{\delta_x} = x$. 
	Infatti, $X^{\delta_x} = \prod\limits_{y \in X}y^{\delta_x(y)} = x^{\delta_x(x)} = x^1 = x$ perché tutti gli altri termini del prodotto hanno esponente $0$, essendo per definizione $\delta_x(y) = 0$ se $y \neq x$.}

	%OMOMORFISMO DI ANELLI
	Mostriamo ora che è un omomorfismo di anelli. 
	Infatti, 
	
	\[\phi\left(f+g\right) = \sum\limits_{\alpha \in \mathcal{F}^\times}(r_\alpha + s_\alpha) \phi(X^\alpha) = \sum\limits_{\alpha \in \mathcal{F}^\times} r_\alpha \phi(X^\alpha) + \sum\limits_{\alpha \in \mathcal{F}^\times} s_\alpha \phi(X^\alpha) = \phi(f) + \phi(g)\]
	
	per la proprietà distributiva del prodotto rispetto alla somma, essendo $S$ un anello, e
	 
	\[\phi(f\cdot g) = 
	\sum\limits_{\gamma \in \mathcal{F}^\times} \sum\limits_{\alpha + \beta = \gamma} r_\alpha s_\beta \phi(X^\gamma) = 
	\left(\sum\limits_{\alpha \in \mathcal{F}^\times} r_\alpha \phi(X^\alpha) \right) \cdot \left(\sum\limits_{\beta \in \mathcal{F}^\times} s_\beta \phi(X^\beta) \right) = 
	\phi(f)\cdot \phi(g)\] 
	
	perché $\phi(X^\gamma) = 
	\prod\limits_{x \in X} \varphi(x)^{\gamma(x)} = 
	\prod\limits_{x\in X} \varphi(x)^{\alpha(x)} \cdot \prod\limits_{x \in X} \varphi(x)^{\beta(x)} = 
	\phi(X^\alpha) \cdot \phi(X^\beta)$. 
	Poiché $\phi(0_R)=0_S$ e $ \phi(1_R)=1_S$, concludiamo che $\phi$ è un omomorfismo di anelli.
	
	\vspace{1mm}
	
	%UNICITA'
	Mostriamo ora che $\phi$ è unico. 
	Sia $\psi\colon R[\mathcal{X}]\to S$ un omomorfismo di anelli tale che $\psi(X^{\delta_x})=\varphi(x)$ e $\psi \raisebox{-.5em}{$\vert_{R}$}= \operatorname{id}_R.$ 
	Allora, per ogni monomio $X^\alpha \in M$ vale 
	
	\[ \psi(X^\alpha) = 
	\psi\left( \prod\limits_{x \in X} x^{\alpha(x)} \right) = 
	\prod\limits_{x \in X} \psi\left(x^{\alpha(x)} \right) = 
	\prod\limits_{x \in X} \psi(X^{\delta_x})^{\alpha(x)} = 
	\prod\limits_{x \in X} \varphi(x)^{\alpha(x)} = 
	\phi(X^\alpha) \]. 
	
	Poiché $\psi$ è un omomorfismo, per ogni $f = \sum\limits_{\alpha \in \mathcal{F}^\times} r_\alpha X^\alpha \in R[\mathcal{X}]$ si ha che 
	
	\[ \psi(f) = 
	\psi\left(\sum\limits_{\alpha \in \mathcal{F}^\times} r_\alpha X^\alpha \right) = 
	\sum\limits_{\alpha \in \mathcal{F}^\times} \psi(r_\alpha X^\alpha) = 
	\sum\limits_{\alpha \in \mathcal{F}^\times} \psi(r_\alpha) \psi(X^\alpha) = 
	\sum\limits_{\alpha \in \mathcal{F}^\times} r_\alpha \phi(X^\alpha) = 
	\phi(f)\] 
	
	essendo $\psi(r_\alpha) = r_\alpha$ perché $r_\alpha \in R$ e $\psi(X^\alpha) = \phi(X^\alpha)$ per quanto appena mostrato. 
	Dunque, $\psi$ coincide con $\phi,$ che risulta quindi essere unico.
\end{proof}

\noindent Sia $R[x]$ l'anello dei polinomi a coefficienti in $R$ nella variabile $x.$ 
Possiamo considerare $R[x]$ stesso come anello dei coefficienti per l'anello dei polinomi nella variabile $y$, cioè 

\[(R[x])[y]=\left\{\sum\limits_{i=0}^n f_i\,y^i : f_i\in R[x],\, n\in \mathbb{N}\right\}.\] 

Poiché ogni polinomio di $(R[x])[y]$ può essere visto come un polinomio in due variabili di $R[x,y]$ e ogni polinomio di $R[x,y]$ 
può essere pensato come un polinomio di $(R[x])[y]$ raccogliendo i termini dello stesso grado in $y$, questo suggerisce che $(R[x])[y]\simeq R[x,y]$.

\begin{exm}
  Sia $f(y)=(x^2+1)y^2+(2x)y+3 \in (\mathbb{Z}[x])[y]$. Allora, possiamo vedere $f(y)$ come un polinomio in due variabili 
  $g(x,y)=x^2y^2+y^2+2xy+3 \in \mathbb{Z}[x,y]$. Viceversa, preso $p(x,y)=xy^2+2xy+3y+4\in \mathbb{Z}[x,y]$, 
  raccogliendo i termini dello stesso grado in $y$ possiamo pensare $p(x,y)$ come un polinomio $q(y)=(x)y^2+(2x+3)y+4 \in (\mathbb{Z}[x])[y]. \ \square$
\end{exm}

\noindent In generale, se $X$ e $Y$ sono insiemi non vuoti e $(R[X])[Y]$ è l'anello dei polinomi a coefficienti in $R[X]$ e a variabili in $Y$, 
detta $X \sqcup Y$ l'unione disgiunta,\footnote{Ricordiamo che l'unione disgiunta di una famiglia di insiemi $\{A_i\}_{i\in I}$ è l'insieme 
$\bigsqcup\limits_{i\in I} A_i = \bigcup\limits_{i\in I} (A \times \{i\})$. Ad esempio, presi $A_0=\{3,4,5\}$ e $A_1=\{5,6\}$, 
si ha che $A_0 \sqcup A_1=\{(3,0), (4,0), (5,0), (5,1), (6,1)\}$.} vale il teorema seguente.

\begin{teo}[]{asdf da sistemare1}
  Sia $R$ un anello commutativo e siano $X$ e $Y$ non vuoti. Allora, $R[X\sqcup Y]\simeq (R[X])[Y]$.
\end{teo}
\vspace{-4mm}
\begin{proof}
  Sia $S$ un anello commutativo tale che $R\subseteq R[X]\subseteq S$ e sia $\varphi_X\colon X\to S$ definita come $\varphi_X(x)=X^{\underline{\delta}_x}$. 
  Presa una qualunque funzione $\varphi_Y\colon Y\to S$, sia $\widetilde{\varphi}\colon X\sqcup Y\to S$ l'unica mappa tale che 
  $\widetilde{\varphi} \raisebox{-.5em}{$\vert_{X}$}=\varphi_X$ e $\widetilde{\varphi} \raisebox{-.5em}{$\vert_{Y}$}=\varphi_Y$. 
  Allora, per il \emph{Teorema 1.3.1} esiste un unico omomorfismo $\widetilde{\phi}\colon R[X\sqcup Y]\to S$ tale che 
  $\widetilde{\phi}(Z^{\underline{\delta}_z})=\widetilde{\varphi}(z)$ per ogni $z\in X\sqcup Y$ e 
  $\widetilde{\phi} \raisebox{-.5em}{$\vert_{R}$}= \operatorname{id}_R$. 
  Per ogni $\underline{\alpha}\in \mathcal{F}^{\times}(X,\mathbb{N})$, 
  sia $\underline{\widetilde{\alpha}}\in \mathcal{F}^{\times}(X \sqcup Y,\mathbb{N})$ 
  l'unica funzione tale che $\underline{\widetilde{\alpha}} \raisebox{-.5em}{$\vert_{X}$} = \underline{\alpha}$ e 
  $\underline{\widetilde{\alpha}} \raisebox{-.5em}{$\vert_{Y}$} = \underline{0}$. 
  Allora, possiamo pensare ogni monomio $X^{\underline{\alpha}}$ di $R[X]$ come monomio $Z^{\widetilde{\underline{\alpha}}}$ di $R[X\sqcup Y]$, da cui 
  {\setlength{\belowdisplayskip}{2.5pt} \setlength{\abovedisplayskip}{5pt}
  \begin{align*} 
    \widetilde{\phi}(Z^{\widetilde{\underline{\alpha}}}) &= \widetilde{\phi}\left( \prod\limits_{z\in X\sqcup Y} z^{\widetilde{\underline{\alpha}}(z)} \right) 
    = \prod\limits_{z\in X\sqcup Y} \widetilde{\phi}\left(z^{\widetilde{\underline{\alpha}}(z)}\right) 
    = \prod\limits_{z\in X\sqcup Y} \widetilde{\phi}\left(Z^{\underline{\delta}_z}\right)^{\widetilde{\underline{\alpha}}(z)} 
    = \prod\limits_{z\in X\sqcup Y} \widetilde{\varphi}(z)^{\widetilde{\underline{\alpha}}(z)} \\ 
    &= \prod\limits_{x\in X} \varphi_X(x)^{\underline{\alpha}(x)}\cdot \prod\limits_{y\in Y} \varphi_Y(y)^{\underline{0}} 
    = \prod\limits_{x\in X} (X^{\underline{\delta}_x})^{\underline{\alpha}(x)}\cdot 1_R = X^{\underline{\alpha}} 
  \end{align*}}
  \noindent per come abbiamo definito $\widetilde{\varphi}$ e $\underline{\widetilde{\alpha}}$ ed usando il fatto che $\widetilde{\phi}$ è un omomorfismo. 
  Quindi, preso $f=\sum\limits_{\underline{\alpha}\in \mathcal{F}^{\times}} r_{\underline{\alpha}} X^{\underline{\alpha}}\in R[X]$, 
  pensando $f$ come elemento $\widetilde{f}=\sum\limits_{\widetilde{\underline{\alpha}}\in \mathcal{F}^{\times}} r_{\widetilde{\underline{\alpha}}} 
  Z^{\widetilde{\underline{\alpha}}}\in R[X\sqcup Y]$ si ha che 
  
  \[ \widetilde{\phi}(\widetilde{f}\,) = \sum\limits_{\widetilde{\underline{\alpha}}
  \in \mathcal{F}^{\times}} \widetilde{\phi}(r_{\widetilde{\underline{\alpha}}}) \widetilde{\phi}(Z^{\widetilde{\underline{\alpha}}}) 
  = \sum\limits_{\widetilde{\underline{\alpha}}\in \mathcal{F}^{\times}} r_{\widetilde{\underline{\alpha}}}\,\widetilde{\phi}(Z^{\widetilde{\underline{\alpha}}}) 
  = \sum\limits_{\underline{\alpha}\in \mathcal{F}^{\times}} r_{\underline{\alpha}} X^{\underline{\alpha}} = f \] 
  
  perché $\widetilde{\phi}(r_{\widetilde{\underline{\alpha}}})=r_{\widetilde{\underline{\alpha}}}$ essendo 
  $\widetilde{\phi} \raisebox{-.5em}{$\vert_{R}$}= \operatorname{id}_R$, da cui $\widetilde{\phi} \raisebox{-.5em}{$\vert_{R[X]}$}=\operatorname{id}_{R[X]}$. 
  Inoltre, per ogni $y\in Y$ si ha che $\widetilde{\phi}(Z^{\underline{\delta}_y})=\widetilde{\varphi}(y)=\varphi_Y(y)$. 
  Poiché $R[X\sqcup Y]$ è un anello commutativo contenente $R[X]$ che soddisfa la proprietà universale di $(R[X])[Y]$,\footnote{Infatti, 
  abbiamo appena mostrato che per ogni anello $S\supseteq R[X]$ e per ogni mappa $\varphi_Y\colon Y\to S$, esiste un unico omomorfismo 
  $\widetilde{\phi}\colon R[X\sqcup Y]\to S$ tale che $\widetilde{\phi}(Z^{\underline{\delta}_y})=\varphi_Y(y)$ 
  per ogni $y\in Y$ e $\widetilde{\phi} \raisebox{-.5em}{$\vert_{R[X]}$}=\operatorname{id}_{R[X]}$.} per la generalizzazione del 
  \emph{Teorema REF DA SISTEMARE} possiamo effettivamente concludere che $R[X \sqcup Y]\simeq (R[X])[Y]$.
\end{proof}
\clearpage

\noindent Nel caso in cui l'insieme delle variabili sia finito, vale il corollario seguente.

\begin{cor}[]{label da sistesamrea dhjfasdf}
  Sia $n$ un intero positivo. Allora, $R[x_1,...\,,x_n]\simeq (\cdots((R[x_1])[x_2])\cdots )[x_n]$.
\end{cor}
\vspace{-4mm}
\begin{proof}
  Procediamo per induzione sul numero $n$ di variabili. Chiaramente, se $n=1$ allora $R[x_1]\simeq R[x_1]$. 
  Supponiamo quindi che la tesi sia vera per un certo intero $n\geq 1.$ Detti $X=\{x_1,...\,,x_n\}$ e $Y=\{x_{n+1}\}$, 
  per il \emph{Teorema 1.3.2 ref da sistemare} si ha che $R[X\sqcup Y]\simeq (R[X])[Y]$ da cui $R[x_1,...\,,x_{n+1}]\simeq 
  (R[x_1,...\,,x_n])[x_{n+1}]\simeq ((\cdots((R[x_1])[x_2])\cdots )[x_n])[x_{n+1}]$.
\end{proof}

\noindent Possiamo quindi estendere agli anelli di polinomi in più variabili anche la \emph{Proposizione 1.1.1}. Per fare ciò, osserviamo innanzitutto che ogni polinomio di $R[X]$ è la somma di un numero finito di monomi non nulli, ognuno con un numero finito di variabili. Dunque, ogni polinomio di $R[X]$ può essere pensato come un polinomio in un numero finito di variabili, o meglio, per ogni $f\in R[X]$ esiste un sottoinsieme delle variabili $X_f\subseteq X$ finito tale che $f\in R[X_f]$.\footnote{Più formalmente, preso $f=\sum\limits_{\underline{\alpha}\in \mathcal{F}^{\times}}r_{\underline{\alpha}} X^{\underline{\alpha}}\in R[X]$ sappiamo che $\Omega_f=\operatorname{supp}(r\underline{\, \, \,}\,)\subseteq \mathcal{F}^{\times}$ è finito, quindi esiste solo un numero finito di funzioni $\underline{\alpha}\in \mathcal{F}^{\times}$ per cui il monomio $X^{\underline{\alpha}}$ ha un coefficiente $r_{\underline{\alpha}}$ non nullo. Poiché ogni $\underline{\alpha}\in \mathcal{F}^{\times}$ ha supporto finito, $X_f=\bigcup\limits_{\underline{\alpha}\in \Omega_f}\operatorname{supp}(\underline{\alpha})$ è finito in quanto unione finita di insiemi finiti.}

\begin{prop}[]{Se R di integrita' R[x] di integrita}
  Sia $X$ un insieme non vuoto e sia $R$ un dominio di integrità. Allora, anche l'anello dei polinomi $R[X]$ è un dominio di integrità.
\end{prop}
\vspace{-4mm}
\begin{proof}
  Siano $f,g\in R[X]$ e siano $X_f, X_g\subseteq X$ finiti tali che $f\in R[X_f]$ e $g\in R[X_g]$. 
  Osserviamo innanzitutto che $X_f\cup X_g$ è un sottoinsieme finito di $X$ e $f\cdot g\in R[X_f\cup X_g]$. 
  Dunque, detto $X_f\cup X_g=\{x_1,...\,,x_n\}$, per dimostrare che $R[X]$ è un dominio di integrità~è sufficiente provare che 
  $R[x_1,...\,,x_n]$ è un dominio di integrità.\footnote{Se il polinomio $f\cdot g$ si annulla in $R[X]$, 
  allora si annulla anche pensato come polinomio di $R[X_f\cup X_g]$. Dunque, se $R[X_f\cup X_g]$ è un dominio di integrità per ogni $f,g\in R[X]$, 
  allora anche $R[X]$ deve essere un dominio di integrità. Infatti, se esistessero $f,g\in R[X]$ divisori dello zero, 
  per quanto appena detto essi sarebbero divisori dello zero anche in $R[X_f\cup X_g]$, il che contraddice la definizione di dominio di integrità.}
  Per fare ciò, procediamo~per induzione sul numero di variabili. Se $n=1$, per la \emph{Proposizione 1.1.1 ref da sistemare} 
  sappiamo che $R[y_1]$ è un dominio di integrità. Supponiamo quindi che la tesi valga per un certo intero $n\geq 1$. Allora, per il 
  \emph{Corollario 1.3.3 ref da sistemare} si ha che $R[y_1,...\,,y_{n+1}]\simeq (R[y_1,...\,,y_n])[y_{n+1}]$, 
  ed essendo $R[y_1,...\,,y_n]$ un dominio di integrità per ipotesi induttiva, per la 
  \emph{Proposizione 1.1.1 ref da sistemare} anche $(R[y_1,...\,,y_n])[y_{n+1}]$ è un dominio di integrità, da cui lo è pure $R[y_1,...\,,y_{n+1}]$. 
  Dunque,~$R[Y]$ è un dominio di integrità per ogni insieme finito $Y$, ed in particolare lo è per $Y=X_f\cup X_g$. 
  Per l'arbitrarietà di $f,g\in R[X]$, possiamo concludere che $R[X]$ è un dominio di integrità.
\end{proof}

\noindent Concludiamo con un'osservazione che acquisirà importanza quando passeremo allo studio dell'estensione di campi. 
Preso un anello commutativo $R$ e un qualunque oggetto $x\not\in R$, l'anello dei polinomi $R[x]$ è il più piccolo anello contenente $R$ e $x$. 
Infatti, se $S$ è un anello contenente $R$ e $x$, per la chiusura di $S$ rispetto a somma e prodotto esso conterrà tutte le potenze non negative 
$\{x^0, x^1, x^2, ...\}$ di $x$ e tutte le combinazioni lineari tra potenze di $x$ ed elementi di $R$, cioè tutti gli elementi della forma 
$a_nx^n + ... + a_1x+a_0$ con $a_0,...\,,a_n\in R$. In generale, se $X$ è un insieme non vuoto, 
possiamo quindi vedere $R[X]$ come la più piccola ``estensione'' di $R$ contenente $X$, cioè come il più piccolo anello contenente sia $R$ che $X$.
