\subsection{Domini a valutazione discreta}

Introduciamo subito la definizione dell'oggetto in questione, il cui nome viene spesso abbrevviato in 
\emph{d.v.d} (\emph{discrete valutation domain}).
(Magari inserire un introduzione dell'argomento)

\begin{defn}[Dominio a valutazione discreta]{def dvd}
  Sia $R$ un dominio di integrità. Allora $R$ si dice \emph{dominio a valutazione discreta} se: \\
  \noindent ($i$) \ $R$ e' un dominio a ideali principali ($P.I.D$) \\
  \noindent ($ii$) $R$ e' un anello locale
\end{defn}

\noindent Vediamone subito un esempio

\begin{exm}
  Sia $p\in \Z$, $p$ primo. Allora 

  \[ \mathcal{O}_p = \bigg\{ \frac{r}{s} \mid r \in \Z \land s \in \Z \setminus p\Z \bigg\} \]

  \noindent e' un dominio a valutazione discreta (dimostrarne il perche' magari). Si noti che tale condizione
  e' equivalente ad $\operatorname{mcd}(s, p) = 1$.
\end{exm}

\noindent Come possiamo intuire dal nome di questa struttura algebrica, e' presente un funzione di valutazione. \\
\noindent Consideriamo $(\N_0 \cup \{\infty\}, +)$, che e' un monoide commutativo (ovvero un semigruppo con $1$ ?) con
l'operazione '+' definita nel capitolo dei polinomi in una variabile. Possiamo quindi definire:

\begin{defn}[Valutazione discreta di $R$]{def valutazione discreta}
  Siano $R$ un dominio di integrità, Si dice \emph{valutazione discreta di R} una mappa $\omega : R \rightarrow \N_0 \cup \{\infty\}$ che, 
  dati $a,b \in R$, soddisfa
  \begin{tabbing}
      ($i$)\quad\quad \=$\omega(a) = \infty \iff a = 0_R$, \\
      ($ii$)     \>$\omega(a\cdot b) = \omega(a) + \omega(b)$, \\
      ($iii$)    \>$\omega(a+b) \geq \min\{ \omega(a), \omega(b) \}$, \\
      ($iv$)     \>$\omega$ e' suriettivo.
  \end{tabbing}
\end{defn}

\noindent L'ipotesi di $R$ dominio di integrità nella definizione precedente fa intuire una relazione tra tale struttura algebrica e quella di
dominio a valutazione discreta che viene chiarita nella seguente proposizione.

\begin{prop}[]{R dom integrita dvd se ha valutazione discreta}
  Sia $(R, \omega)$ un dominio di integrità dotato di una valutazione discreta $\omega$. Supponiamo inoltre che

  \[ R^{\times} = \big\{ r\in R \mid \omega(r) = 0 \big\} =: R_0,  \] 

  \noindent allora $R$ e' un dominio a valutazione discreta.
\end{prop}
\begin{proof}
  ASPETTARE APPUNTI WEIGEL.
\end{proof}

\begin{example}[Valutazione discreta]{esempio valutazione discreta}
  Siano $R = \Z, p\in \Z$ un numero primo. Sappiamo allora che (algebra I?) per ogni $z$ in $\Z \setminus \{\pm 1, 0\}$ esiste $\varepsilon_p(z) 
  \in \N_0$ tale che $p^{\varepsilon_p}$ e' la massima potenza che divide $z$, cioe' $p^{\varepsilon_p} ||\ z$, \emph{se e solo se} 
  $z \in p^{\varepsilon_p}\Z \setminus p^{\varepsilon_p + 1}\Z$.\\

  \noindent Definiamo allora $\tilde{\varepsilon}_p : \Z \rightarrow \N_0 \cup \{\infty\}$ ponendo $\tilde{\varepsilon}_p(0) := \infty,\ 
  \tilde{\varepsilon}_p(z) := \varepsilon_p(z)$ per ogni $z \in \Z\setminus \{0\}$. Proviamo allora che $\tilde{\varepsilon}_p$ soddisfa i quattro
  assiomi di valutazione discreta. \\
  ($i$) E' dato dalla definizione di $\tilde{\varepsilon}_p$. \\
  \noindent ($ii$) Siano $x,y \in \Z,\ \tilde{\varepsilon}_p(x) = j,\ \tilde{\varepsilon}_p(y) = k.$ Allora esistono $a,b \in \Z$ tali che
  \mbox{$x = p^ja$}, $y = p^kb$ e $p \nmid a,b$. Segue che $xy = p^{j+k}ab \in p^{j+k}\Z$, ma $p \nmid ab$, altrimenti per definizione di primo dev'essere che
  $p \mid a$ oppure $p \mid b$, che e' assurdo. Dunque, $xy \in p^{j+k} \Z \setminus p^{j+k+1} \Z \Longrightarrow \tilde{\varepsilon}_p(xy) = 
  \tilde{\varepsilon}_p(x) + \tilde{\varepsilon}_p(y)$. \\
  \noindent ($iii$) Considerando $x, y$ come nel punto precedentemente, abbiamo che \\
  $xy \in p^{\min\{ \tilde{\varepsilon}_p(x), \tilde{\varepsilon}_p(y) \}}\Z
  \Longrightarrow \tilde{\varepsilon}_p(x+y) \geq \min{\tilde{\varepsilon}_p(x), \tilde{\varepsilon}_p(y)}$. \\
  \noindent ($iv$) Sia $n \in \N_0$. Allora $\tilde{\varepsilon}_p(p^n) = n$, quindi $\tilde{\varepsilon}_p(y)$ e' suriettivo.\\

  \noindent Pertanto la nostra funzione e' una valutazione discreta anche $\Z$ non e' un anello locale. Pertanto tale costruzione non e' una prerogativa
  ristretta solamente ai domini a valutazione discreta.
\end{example}

\noindent Nel caso in cui avessimo a che fare con un campo, possiamo intuire che la funzione di valutazione discreta vada leggermente modificata
(anche se ho i miei dubbi, dopotutto un campo e' un dominio di integrita')

\begin{defn}[Valutazione discreta di un campo]{def valutazione discreta campo}
  Sia $\mathbb{b}K$ un campo. Si dice \emph{valutazione discreta di un campo} una mappa 
  $\omega : \mathbb{K} \rightarrow \Z \setminus \{ 0_{\mathbb{K}} \}$ che soddisfa gli assiomi ($i$), ... ,($iv$) di 
  \emph{Definzione \ref{def valutazione discreta}} 
\end{defn}

\begin{prop}[]{prop campo con valutazione discreta}
  Sia 
\end{prop}