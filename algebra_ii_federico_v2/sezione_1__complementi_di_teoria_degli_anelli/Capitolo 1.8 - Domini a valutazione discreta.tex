\clearpage

\subsection{Domini a valutazione discreta}

Introduciamo subito la definizione dell'oggetto in questione, il cui nome viene spesso abbreviato in 
\emph{d.v.d} (\emph{discrete valuation domain}). La definizione seguente e' un po' misteriosa ma verra'
chiarita successivamente con una alternativa che rispecchia davvero il suo nome.


\begin{defn}[Dominio a valutazione discreta]{def dvd}
  Sia $R$ un dominio di integrità. Allora $R$ si dice \emph{dominio a valutazione discreta} se: \\
  \noindent ($i$) \ \ $R$ e' un dominio a ideali principali ($P.I.D$) \\
  \noindent ($ii$) \ $R$ e' un anello locale\\
  \noindent ($iii$) $R$ non e' un campo.
\end{defn}

\begin{exm}
  Sia $p\in \Z$, $p$ primo. Allora 

  \[ \mathcal{O}_p = \bigg\{ \frac{r}{s} \mid r \in \Z \land s \in \Z \setminus p\Z\footnotemark \bigg\} \]

  \noindent e' un dominio a valutazione discreta (dimostrarne il perche' magari).  
\end{exm}
\footnotetext{Si noti che tale condizione e' equivalente ad $\operatorname{mcd}(s, p) = 1$.}

\noindent Introduciamo ora formalmente il concetto di \emph{valutazione di un campo} che risultera' chiave per questa sezione.
\begin{defn}[Valutazione di un campo]{def valutazione}
  Siano $K$ un campo, $(G, +, \leq)$ un gruppo abeliano parzialmente ordinato 
  (ovvero dotato di relazione d'ordine parziale che preserva l'operazione di gruppo). \\
  \noindent Si dice \emph{valutazione} una mappa $\omega : K \setminus \{0\} \rightarrow G$ tale che, per ogni $a,b \in G$,
  \begin{tabbing}
    ($i$)\quad\quad \=$\omega$ e' suriettivo, \\
    ($ii$)     \>$\omega(a\cdot b) = \omega(a) + \omega(b)$, \\
    ($iii$)    \>$\omega(a+b) \geq \min\{ \omega(a), \omega(b) \}$, 
  \end{tabbing}

  \noindent Se $G$ e' discreto (cioe' numerabile finito o infinito) si dice \emph{valutazione discreta}.
\end{defn}

\noindent Se vogliamo includere lo $0$ nella nostra valutazione, otteniamo la seguente:

\begin{defn}[Valutazione di un campo (estesa)]{def valutazione estesa}
  Siano $K$ un campo, $(G, +, \leq)$ un gruppo abeliano parzialmente ordinato e consideriamo $G \cup \{\infty\}$ dove, per ogni $a \in G$,
  $a+\infty = \infty + a=\infty,$ e $a \leq \infty$. Allora possiamo estendere la precedente definendo 
  $\omega : K \rightarrow G$ dove vale inoltre

  \[ \omega(a) = \infty \iff a=0 \]
\end{defn}

\begin{obs}
  Come nel \emph{Capitolo 1} per $\N_0$, consideriamo l'insieme $\overline{\Z} := \Z \cup \{ \infty \}$ 
  dotato della somma usuale tra interi dove inoltre, per ogni $a \in \overline{\Z}$

  \[ a + \infty = \infty + a = \infty. \]

  \noindent Allora $(\overline{\Z}, +)$ e' un monoide commutativo, ovvero un semigruppo abeliano con $0$.
\end{obs}

\noindent Il seguente teorema descrive il legame tra i domini a valutazione discreta e i campi con una valutazione discreta.

\begin{teo}[]{teo dvd e campi}
  Sia $R$ un dominio di integrità, con campo quoziente $K$. Allora\\
  ($i$) \ Se $(K,\omega)$ e' un campo con una valutazione discreta, l'insieme
    \[ \mathcal{O}_{\omega} = \{ x \in K \mid \omega(x) \geq 0 \} \]
  \noindent e' un dominio a valutazione discreta tale che $K = \quot(\mathcal{O}_{\omega})$\\

  \noindent ($ii$) Se $R$ e' un dominio a valutazione discreta, allora esiste una valutazione discreta $\omega : K \rightarrow \overline{\Z}$
  tale che $R = \mathcal{O}_{\omega}$.
\end{teo}
\begin{proof}
  Da fare.
\end{proof}

\begin{example}[Valutazione discreta]{esempio valutazione discreta}
  Siano $R = \Z, p\in \Z$ un numero primo. Sappiamo allora che (algebra I?) per ogni $z$ in $\Z \setminus \{\pm 1, 0\}$ esiste $\varepsilon_p(z) 
  \in \N_0$ tale che $p^{\varepsilon_p}$ e' la massima potenza che divide $z$, cioe' $p^{\varepsilon_p} ||\ z$, \emph{se e solo se} 
  $z \in p^{\varepsilon_p}\Z \setminus p^{\varepsilon_p + 1}\Z$.\\

  \noindent Definiamo allora $\tilde{\varepsilon}_p : \Z \rightarrow \N_0 \cup \{\infty\}$ ponendo $\tilde{\varepsilon}_p(0) := \infty,\ 
  \tilde{\varepsilon}_p(z) := \varepsilon_p(z)$ per ogni $z \in \Z\setminus \{0\}$. Proviamo allora che $\tilde{\varepsilon}_p$ soddisfa i quattro
  assiomi di valutazione discreta. \\
  ($i$) E' dato dalla definizione di $\tilde{\varepsilon}_p$. \\
  \noindent ($ii$) Siano $x,y \in \Z,\ \tilde{\varepsilon}_p(x) = j,\ \tilde{\varepsilon}_p(y) = k.$ Allora esistono $a,b \in \Z$ tali che
  \mbox{$x = p^ja$}, $y = p^kb$ e $p \nmid a,b$. Segue che $xy = p^{j+k}ab \in p^{j+k}\Z$, ma $p \nmid ab$, altrimenti per definizione di primo dev'essere che
  $p \mid a$ oppure $p \mid b$, che e' assurdo. Dunque, $xy \in p^{j+k} \Z \setminus p^{j+k+1} \Z \Longrightarrow \tilde{\varepsilon}_p(xy) = 
  \tilde{\varepsilon}_p(x) + \tilde{\varepsilon}_p(y)$. \\
  \noindent ($iii$) Considerando $x, y$ come nel punto precedente, abbiamo che \\
  $xy \in p^{\min\{ \tilde{\varepsilon}_p(x), \tilde{\varepsilon}_p(y) \}}\Z
  \Longrightarrow \tilde{\varepsilon}_p(x+y) \geq \min{\tilde{\varepsilon}_p(x), \tilde{\varepsilon}_p(y)}$. \\
  \noindent ($iv$) Sia $n \in \N_0$. Allora $\tilde{\varepsilon}_p(p^n) = n$, quindi $\tilde{\varepsilon}_p(y)$ e' suriettivo.\\

  \noindent Pertanto la nostra funzione e' una valutazione discreta anche $\Z$ non e' un anello locale. Pertanto tale costruzione non e' una prerogativa
  ristretta solamente ai domini a valutazione discreta.
\end{example}

\noindent Nel caso in cui avessimo a che fare con un campo, possiamo intuire che la funzione di valutazione discreta vada leggermente modificata
(anche se ho i miei dubbi, dopotutto un campo e' un dominio di integrita')

\begin{example}[Valutazione discreta su un campo]{esempio valutazione discreta su campo}
  Sia $\tilde{\varepsilon}_p$ definita come nell' \emph{Esempio \ref{esempio valutazione discreta}}, e proviamo a definire
  $\ul{\tilde{\varepsilon}}_p : \Q \rightarrow \Z \cup \{ \infty \}$ basandoci su $\tilde{\varepsilon}_p$. Dovremo in qualche modo gestire i vari quozienti, 
  in quanto la condizione critica da soddisfare e' la ($ii$). Infatti, sia $\frac{1}{s} \in \Q$, allora vorremmo che 
  $\ul{\tilde{\varepsilon}}_p(\frac{1}{s}) + \ul{\tilde{\varepsilon}}_p(s) = \tilde{\varepsilon}_p(1) = 0$. Dato $\frac{r}{s} \in \Q$, poniamo allora
  \mbox{$\ul{\tilde{\varepsilon}}_p(\frac{r}{s}) := \tilde{\varepsilon}_p(r) - \tilde{\varepsilon}_p(s)$}. Cosi' facendo gli assiomi sono 
  soddisfatti e otteniamo una valutazione discreta.
\end{example}