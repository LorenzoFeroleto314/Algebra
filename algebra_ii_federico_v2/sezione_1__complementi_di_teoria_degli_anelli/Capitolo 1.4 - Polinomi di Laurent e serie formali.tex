
\subsection{Polinomi di Laurent e serie formali}

\noindent Vogliamo ora introdurre alcune generalizzazioni del concetto di anello di polinomi molto usate nell'analisi reale e complessa, quali i polinomi di Laurent e le serie di potenze. 

\

\begin{defn}[Polinomi di Laurent e operazioni]{def polinomi Laurent}
  Sia $R$ un anello commutativo e sia $R[x,x^{-1}]=\left\{\sum\limits_{i=-p}^n a_ix^i : a_i\in R,\, n,p\in \mathbb{N}\right\}$. 
  Presi due elementi $f=\sum\limits_{i=-p}^m a_ix^i$ e $g=\sum\limits_{j=-q}^n b_jx^j$ di $R[x,x^{-1}]$, definiamo le operazioni di \emph{somma} 
  \[ f+g=\sum\limits_{i=-r}^s (a_i+b_i)x^i \] 
  \noindent dove $s=\max\{m,n\}$, $r=\max\{p,q\}$ e $a_i=b_j=0$ per $i\not\in [-p,m]$ e $j\not\in [-q,n]$, e \emph{prodotto} 
  \[ f\cdot g=\sum\limits_{k=-p-q}^{m+n}c_k x^k\] 
  \noindent dove abbiamo posto $c_k=\sum\limits_{i+j=k} a_i b_j$. 
\end{defn}

\begin{obs}
  Possiamo pensare $R[x,x^{-1}]$ come l'anello dei polinomi $R[x]$ dove però l'esponente della variabile $x$ può essere anche un intero negativo.
\end{obs}

\begin{lem}[]{R[x,x-1] e' un anello commutativo}
  $R[x, x^{-1}]$ e' un anello commutativo.
\end{lem}
\begin{proof}
  Da fare.
\end{proof}

\noindent Come per i poliniomi in una variabile, andiamo a definire il grado dei polinomi di Laurent.

\begin{defn}[Grado di un polinomio di Laurent]{def grado di un polinomio di Laurent}
  Siano $R$ un anello commutativo unitario, $R[x, x^{-1}]$ come descritto in precedenza e $f = \sum_{i\geq -q}^{n}a^i x^i \in R[x, x^{-1}]$. 
  Definiamo allora la funzione \emph{grado}

    \[ \omega : R[x,x^{-1}] \rightarrow \Z \cup \{\infty\},\ \omega(f) = 
      \begin{cases}
        \min\{ z \in \Z \mid a_n \neq 0 \},\ &f \not\equiv 0_R\\
        \infty, &f \equiv 0_R
      \end{cases}
    \]
  
  Dove la quantita' $\omega(f)$ e' chiamata \emph{grado del polinomio di Laurent f}.
\end{defn}

\noindent Introduciamo ora un fatto abbastanza intuitivo su questa struttura algebrica:

\begin{lem}[]{R[x] sottoanello di R[x,x-1]}
  Sia $R$ un anello commutativo unitario. Allora $R[x] \leq R[x, x^{-1}]$, cioe' $R[x]$ e' un sottoanello di $R[x,x^{-1}]$.
  Inoltre, per ogni $f\in R[x,x^{-1}], f \not\equiv 0_R$ vale che

  \[ x^{-\omega(f)} \cdot f \in R[x] \]
\end{lem}
\begin{proof}
  Da fare.
\end{proof}

\noindent Definiamo ora la seconda struttura algebrica di cui vogliamo parlare.

\begin{defn}[Serie di potenze formali, operazioni e grado]{def serie di potenze formali}
  Sia $\F$ un campo. Definiamo l'insieme delle \emph{serie di potenze formali} come 

  \[ \F\llbracket x \rrbracket = \left\{ \sum\limits_{i \in \N_0} a_i x^i \mid a_i \in \F \right\} \]

  \noindent Siano $f = \sum_{i\in \N_0}a_i x^i,\ g = \sum_{i\in N_0}b_i x^i$ e definiamo l'operazione di \emph{somma} e \emph{prodotto}

  \begin{align*}
    +\ &: \F\llbracket x \rrbracket \times \F\llbracket x \rrbracket \rightarrow \F\llbracket x \rrbracket,\ f + g = \sum\limits_{i\in \N_0}(a_i + b_i)x^i \\ 
    \cdot\ &: \F\llbracket x \rrbracket \times \F\llbracket x \rrbracket \rightarrow \F\llbracket x \rrbracket,\ f \cdot g = \sum\limits_{k\in \N_0} \big(\sum_{\substack{i,j\in \N_0\\i+j = k}} (a_i \cdot b_j)x^k \big)
  \end{align*}

  \noindent dove tale prodotto viene spesso denotato piu' brevemente con $\sum_{k\in \N_0} c_k x^k$. \vspace{0.5mm}\\
  \noindent Su $\F\llbracket x \rrbracket$ e' definita $\omega$, ovvero la funzione \emph{grado}, esattamente come nella \emph{Definizione 
  \ref{def grado di un polinomio di Laurent}} dove pero' ovviamente i nostri indici appartengono a $\N_0$.
\end{defn}

\begin{lem}[]{F[x] dominio di integrita}
  $\F\llbracket x \rrbracket$ e' un dominio di integrità.
\end{lem}
\begin{proof}
  Da fare.
\end{proof}

\noindent Elenchiamo ora alcuni fatti su tale struttura algebrica, uno dei quali richiede conoscenze su anelli locali e massimali, 
che vedremo nei prossimi sottocapitoli.

\begin{lem}[]{fatti su F[x]}
  ($i$) La funzione grado $\omega$ soddisfa che, per ogni $f, g \in \F\llbracket x \rrbracket, \omega(f\cdot g) = \omega(f) + \omega(g)$. \\
  \noindent ($ii$) Sia $f = \sum_{i\in \N_0} a_i x^i \in \F\llbracket x \rrbracket$. La valutazione in 0, ovvero
      \[ \phi_0 : \F\llbracket x \rrbracket \rightarrow \F,\ \phi_0(f) = a_0 \]
    \noindent e' l'unico omomorfismo di anelli da $\F\llbracket x \rrbracket$ a $\F$.\\
  \noindent ($iii$) $\F\llbracket x \rrbracket^{\times} = \{ f \in \F\llbracket x \rrbracket \mid \phi_0(f) \neq 0 \} 
    = \{ f \in \F\llbracket x \rrbracket \mid \omega(f) = 0 \}$.
 
\end{lem}
\begin{proof}
  i e iii da fare. ii e' per \emph{Proposizione \ref{prop su ideale massimale}}
\end{proof}

AGGIUNGERE ANCHE ANELLO CROSS OVER: ANELLO DELLE SERIE DI POTENZE DI LAURENT.


\noindent Sketch del capitolo: Polinomi di Laurent ma le dim sono trivial per il capitolo 1.3, serie formali (qui dimostro cose), 
serie formali di Laurent (c'è una sola dim)

\clearpage