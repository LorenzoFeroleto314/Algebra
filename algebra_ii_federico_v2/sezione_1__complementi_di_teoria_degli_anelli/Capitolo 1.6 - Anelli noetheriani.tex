
\subsection{Anelli noetheriani}

\begin{lem}[]{lem per ideale generato}
  \vspace{-3mm}
  Sia $R$ un anello e siano $a_1,...\,,a_n\in R.$ Allora, 
  $I=\sum\limits_{i=1}^n Ra_i=\left\{ \sum\limits_{i=1}^n r_ia_i:r_i\in R \right\}$ è un ideale\footnotemark di $R.$ 
\end{lem}
\begin{proof}
  Infatti, presi $x=\sum\limits_{i=1}^n r_ia_i$ e $y=\sum\limits_{i=1}^n s_ia_i$ in $I,$ si ha che 
  \vspace{-4mm}
  \begin{align*}
    x+y &=\sum\limits_{i=1}^n (r_i+s_i)a_i\in I,\\
    tx &=t\sum\limits_{i=1}^n r_ia_i=\sum\limits_{i=1}^n (tr_i)a_i\in I,
  \end{align*}
  per ogni $t\in R,$ da cui $I\lhd R.$
\end{proof}


\footnotetext{Ricordiamo che $I$ è un ideale di un anello commutativo $R$ se $a-b\in I$ per ogni $a,b \in I$ e se $ra\in I$ per ogni $a\in I$ e $r\in R$. 
Per ragioni estetiche, si preferisce spesso mostrare equivalentemente che $a+b\in I$ per ogni $a,b \in I$ e non che $a-b\in I$. 
Infatti, se l'opposto esiste, detto $c=-b$ si ha che $a-b\in I\Leftrightarrow a+c\in I$.} 

\vspace{-1mm}

\begin{defn}[Ideale generato]{def ideale generato}
  Sia $R$ un anello e siano $a_1,...\,,a_n\in R.$ Allora, l'ideale $I=\sum\limits_{i=1}^n Ra_i$ è detto \underline{ideale generato} 
  \underline{da $a_1,...\,,a_n$} e si denota con $I=\langle a_1,...\,,a_n\rangle.$
\end{defn}

\noindent L'ideale generato da $a_1,...\,,a_n$ è il più piccolo ideale di $R$ contenente $a_1,...\,,a_n$, 
e possiamo pensarlo come l'insieme delle combinazioni lineari in $R$ di $a_1,...\,,a_n$.

\begin{exm}
  Consideriamo in $\mathbb{Z}$ gli ideali $I=\langle 2\rangle$ e $J=\langle 2,3\rangle$. Allora, $I=\{2n: n\in\mathbb{Z}\}=2\mathbb{Z}$ 
  è l'insieme dei numeri pari e $J=\{2m+3n : m,n\in \mathbb{Z}\}=\mathbb{Z}$ perché $1=2\cdot 2+3\cdot (-1)\in J$, 
  da cui $k=2\cdot (2k)+3\cdot (-k)\in J$ per ogni $k\in \mathbb{Z}. \ \square$
\end{exm}

\vspace{-2mm}

\begin{defn}[]{def anello finitamente generato}
  Dato un ideale $I\lhd R,$ definiamo \underline{numero minimo di generatori} $d_R(I)$ il più piccolo $n\in \mathbb{N}$ 
  per cui esistano $a_1,...\,,a_n\in R$ tali che $I=\langle a_1,...\,,a_n\rangle.$ Se tale $n\in \mathbb{N}$ non esiste, 
  poniamo $d_R(I)=\infty.$ Diciamo che $I\lhd R$ è \underline{finitamente generato} se $d_R(I)<\infty.$
\end{defn}

\begin{exm} 
  Sia $I=\langle 2, x\rangle \lhd \mathbb{Z}[x]$ e supponiamo che $d_{\mathbb{Z}[x]}(I)=1$, 
  cioè che $I=\langle f(x)\rangle$ per un certo $f(x)\in \mathbb{Z}[x]$ non nullo. 
  Se $\deg^{\star}(f)=0,$ cioè $f(x)\equiv k$, allora $k$ è pari e $I$ contiene solo polinomi con coefficienti pari, 
  da cui $x\not\in I$, assurdo. Se $\deg^{\star}(f)\geq 1$, allora $I$ contiene solo polinomi di grado almeno $1$, 
  cioè $2\not\in I$, assurdo. Dunque $d_{\mathbb{Z}[x]}(I)=2. \ \square$
\end{exm}

\noindent Esiste un'importante famiglia di anelli in cui ogni ideale è finitamente generato.

\begin{defn}[Anello noetheriano]{def anello noetheriano}
  Un anello commutativo $R$ si dice \underline{noetheriano} se ogni suo ideale è finitamente generato, 
  cioè se ogni ideale $I\lhd R$ soddisfa $d_R(I)<\infty.$
\end{defn}

\begin{exm}Sia $R$ un dominio ad ideali principali.\footnote{Ricordiamo che un dominio ad ideali principali (spesso abbreviato PID) è un dominio di integrità in cui ogni ideale è principale, cioè generato da un solo elemento. Esempi di PID sono $\mathbb{Z}$ e ogni campo $\mathbb{K}.$} Allora, $R$ è un anello noetheriano poiché per definizione di PID si ha che $d_R(I)=1$ per ogni $I\lhd R$ non banale e $d_R(\{0_R\})=0.$ In particolare, ogni campo $\mathbb{K}$ è noetheriano perché i suoi unici ideali sono $\{0_{\mathbb{K}}\}$ e $\mathbb{K}=\langle 1_{\mathbb{K}}\rangle. \ \square$\end{exm}

\noindent Nelle dimostrazioni è spesso utile considerare una caratterizzazione equivalente degli anelli noetheriani in termini di successioni ascendenti di ideali, cioè successioni di ideali $(I_k)_{k\in \mathbb{N}}$ tali che $I_k\subseteq I_{k+1}$ per ogni $k\in \mathbb{N}$.

\begin{prop}[]{se e solo noetheriani}
  Sia $R$ un anello commutativo. Allora, $R$ è noetheriano se e solo se per ogni successione ascendente di ideali $(I_k)_{k\in \mathbb{N}}$ 
  esiste $N\in\mathbb{N}$ tale che $I_{N+j}=I_N$ per ogni $j\in\mathbb{N}.$
\end{prop}
\vspace{-4mm}
\begin{proof}
  Supponiamo che $R$ sia noetheriano, e sia $(I_k)_{k\in \mathbb{N}}$ una successione ascendente di ideali. 
  Poiché $I_{\infty}=\bigcup\limits_{k\in \mathbb{N}} I_k$ è un ideale di $R,$\footnote{Siano $a,b\in I_{\infty}$ 
  con $a\in I_{s}$ e $b\in I_{t}$, dove $s\leq t$, cioè $I_s\subseteq I_t$. Poiché $a,b\in I_t$, anche $a+b\in I_t\subseteq I_{\infty}$, 
  da cui $a+b\in I_{\infty}$. Inoltre, preso $r\in R$, si ha che $ra\in I_s\subseteq I_{\infty},$ cioè $ra\in I_{\infty}$, da cui $I_{\infty}\lhd R$.} 
  essendo $R$ noetheriano $d_R(I_{\infty})=n<\infty.$ Siano quindi $a_1,...\,,a_n\in I_{\infty}$ tali che $I_{\infty}=\langle a_1,...\,,a_n \rangle$, 
  e siano $k_1,...\,,k_n\in \mathbb{N}$ tali che $a_i\in I_{k_i}$. Detto $N=\max\{k_i : 1\leq i\leq n\}$, 
  essendo $(I_k)_{k\in \mathbb{N}}$ ascendente si ha che $a_1,...\,,a_n\in I_N.$ Dunque, essendo $I_N$ un ideale, 
  $\sum\limits_{i=1}^n r_ia_i\in I_N$ per ogni $r_1,...\,,r_n\in R,$ cioè $\sum\limits_{i=1}^n Ra_i=I_{\infty}\subseteq I_N,$ 
  da cui $I_{N+j}\subseteq I_N$ $\forall j\in \mathbb{N}.$ Poiché $(I_k)_{k\in \mathbb{N}}$ è ascendente, è anche vero che 
  $I_N\subseteq I_{N+j}$ $\forall j\in N$. Combinando le doppie inclusioni, si ha quindi che $I_{N+j}=I_N$ $\forall j\in \mathbb{N}.$\\

  \noindent Viceversa, supponiamo per assurdo che esista $J\lhd R$ con $d_R(J)=\infty.$ Preso $a_0\in J,$ costruiamo la successione 
  $(a_k)_{k\in \mathbb{N}}$ di elementi di $J$ tale che $a_{k+1}\in J\setminus \langle a_0,...\,,a_k\rangle$ $\forall k\in \mathbb{N}.$ 
  Tale successione esiste poiché $J$ non è finitamente generato, quindi $J\setminus \langle a_0,...\,,a_k\rangle \neq \emptyset$ 
  per ogni $k\in \mathbb{N}.$ Si consideri la successione di ideali $(I_k)_{k\in \mathbb{N}},$ $I_k=\langle a_0,...\,,a_k\rangle.$ 
  Allora, è evidente che $I_k\subseteq I_{k+1}$ $\forall k\in \mathbb{N},$ ma essendo $a_{k+1}\not\in I_{k}$ per come abbiamo definito 
  $(a_k)_{k\in \mathbb{N}},$ risulta essere $I_k\subset I_{k+1}.$ Abbiamo quindi costruito una successione ascendente di ideali che viola le ipotesi, 
  perché non esiste $N\in \mathbb{N}$ tale che $I_{N+j}=I_{N}$ $\forall j\in \mathbb{N},$ assurdo. Dunque $d_R(J)<\infty$, 
  e per l'arbitrarietà di $J$ concludiamo che $R$ è noetheriano.
\end{proof}

\noindent Dimostriamo ora un risultato fondamentale nello studio degli anelli noetheriani.

\begin{teo}[Teorema della base di Hilbert]{Teorema della base di Hilbert}
  Sia $R$ un anello noetheriano. Allora, anche l'anello dei polinomi $R[x]$ è noetheriano.
\end{teo}
\vspace{-4mm}
\begin{proof}
  Supponiamo per assurdo che $R[x]$ non sia noetheriano, e sia quindi $J\lhd R[x]$ tale che $d_{R[x]}(J)=\infty.$ 
  Preso $f_0\in J$ non nullo di grado minimo, costruiamo la successione di polinomi $(f_k)_{k\in \mathbb{N}}$ tale che $f_{k+1}$ 
  sia il polinomio di grado minimo in $J\setminus \langle f_0, ...\,,f_k \rangle$ $\forall k\in \mathbb{N}.$ 
  Tale successione esiste poiché $J$ non è finitamente generato, quindi $J\setminus \langle f_0, ...\,,f_k \rangle\neq \emptyset$ 
  per ogni $k\in \mathbb{N}.$ Sia $d_k=\deg^{\star}(f_k)$ e sia $a_k\neq 0_R$ il coefficiente direttore di $f_k.$ 
  Allora, detta $(I_k)_{k\in \mathbb{N}}$ la successione ascendente di ideali di $R$ definita come $I_k=\langle a_0, ...\,,a_k\rangle$, 
  per la \emph{Proposizione \ref{se e solo noetheriani}} esiste $N\in \mathbb{N}$ tale che $I_{N+j}=I_{N}$ $\forall j\in \mathbb{N}.$ 
  In particolare $I_{N+1}=I_N$, ed esistono $r_0,...\,,r_N\in R$ tali che $a_{N+1}=\sum\limits_{i=0}^N r_ia_i.$ 
  Consideriamo ora il polinomio $h=f_{N+1}-\sum\limits_{i=0}^{N}r_i\, x^{d_{N+1}-d_i} f_i \in J.$\footnotemark\newline Se 
  $h\in \langle f_0,...\,,f_N\rangle$, allora anche $f_{N+1}=h+\sum\limits_{i=0}^{N}r_i\, x^{d_{N+1}-d_i} f_i\in \langle f_0,...\,,f_N\rangle$, 
  il che è assurdo per come abbiamo definito $(f_k)_{k\in \mathbb{N}}$. Poiché il coefficiente del termine di grado 
  $d_{N+1}$ in $h$ è $a_{N+1}-\sum\limits_{i=0}^N r_ia_i=0$, si ha che $h\in J\setminus \langle f_0,...\,,f_N\rangle$ è un polinomio di grado 
  $\deg^{\star}(h)<d_{N+1}$, e questo viola la minimalità del grado nella scelta di $f_{N+1}$. Dunque $d_{R[x]}(J)<\infty$, 
  da cui per l'arbitrarietà di $J$ concludiamo che $R[x]$ è noetheriano.
\end{proof}

\footnotetext{Vogliamo sfruttare la relazione tra $a_{N+1}$ e $a_1,...\,,a_N$ che abbiamo appena trovato per costruire un polinomio
 $h\in J\setminus \langle f_0, ...\,,f_N\rangle$ di grado minore di $d_{N+1}$, giungendo quindi ad un assurdo.}

\begin{cor}[]{Se R noetheriano anche anello dei polinomi n variabili noetheriano}
  Sia $n\in \mathbb{N}^+$ e sia $R$ un anello noetheriano. Allora, anche $R[x_1,...\,,x_n]$ è noetheriano.
\end{cor}
\vspace{-4mm}
\begin{proof}
  Essendo $R$ noetheriano, per il \emph{Teorema \ref{Teorema della base di Hilbert}} anche $R[x_1]$ è noetheriano, 
  ed induttivamente sono noetheriani pure $(R[x_1])[x_2],...\,, (\cdots((R[x_1])[x_2])\cdots )[x_n]$. 
  Poiché per il \emph{Corollario SISTEMARE REF} si ha che $R[x_1,...\,,x_n]\simeq (\cdots((R[x_1])[x_2])\cdots )[x_n]$, 
  possiamo concludere che anche $R[x_1,...\,,x_n]$ è noetheriano.
\end{proof}

\noindent Questo risultato non è più valido quando l'insieme delle variabili $X$ è un insieme infinito, ed in particolare, 
esistono domini di integrità che non sono noetheriani.

\begin{exm}
  Sia $R$ un domimio di integrità noetheriano e sia $X=\{x_n : n\in \mathbb{N}\}$ un insieme numerabile di variabili. 
  Per la \emph{Proposizione SISTEMARE REF} sappiamo già che $R[X]$ è un dominio di integrità, quindi è sufficiente mostrare che esso non è noetheriano. 
  Sia $(I_k)_{k\in \mathbb{N}}$ la successione di ideali di $R[X]$ definita come $I_k=\langle x_0,...\,,x_k \rangle$. 
  Allora $I_k\varsubsetneq I_{k+1}$, poiché $I_k\subseteq I_{k+1}$ ma $x_{k+1}\not\in \langle x_0, ...\,,x_k\rangle=I_k$. 
  Dunque, $(I_k)_{k\in \mathbb{N}}$ è una successione ascendente di ideali che viola la \emph{Proposizione \ref{se e solo noetheriani}}, 
  da cui concludiamo che $R[X]$ non è noetheriano$. \ \square$
\end{exm}

\noindent La proposizione seguente, molto utile negli esercizi, permette di dimostrare che un anello è noetheriano semplicemente 
esibendo un omomorfismo suriettivo.

\begin{prop}[1.6.4]{noetheriano si passa con epimorfismo}
  Sia $R$ un anello noetheriano e sia $\phi\colon R\to S$ un omomorfismo di anelli suriettivo. Allora, anche $S$ è un anello noetheriano.
\end{prop}
\vspace{-4mm}
\begin{proof} 
  Siano $J\lhd S$ e $I=\phi^{-1}(J)=\{r\in R : \phi(r)\in J\}.$ Poiché $I$ è un ideale~di~$R$,\footnote{In generale, 
  se $\varphi \colon A\to B$ è un omomorfismo e $J\lhd B$, allora $I=\varphi^{-1}(J)=\{a\in A: \varphi(a)\in J\}\lhd A$. 
  Infatti, presi $a,\, b\in I$, per definizione $\varphi(a),\,\varphi(b)\in J$. Dunque, essendo $J$ un ideale e $\varphi$ un omomorfismo, 
  $\varphi(a+b)=\varphi(a)+\varphi(b)\in J\Rightarrow a+b\in I$ e $\varphi(ra)=\varphi(r)\varphi(a)\in J\Rightarrow ra\in I$ 
  per ogni $r\in A$, da cui $I\lhd A$.} che per ipotesi è noetheriano, esistono $a_1,...\,,a_n \in R$ tali che $I=\langle a_1,...\,,a_n \rangle$. 
  Allora, essendo $\phi$ suriettivo, sappiamo che $J=\phi(I)=\langle \phi(a_1),...\,,\phi(a_n) \rangle$, da cui $d_S(J)\leq d_R(I)=n<\infty$. 
  Dunque, per l'arbitrarietà di $J$ concludiamo che $S$ è noetheriano. 
\end{proof}

\begin{exm} 
  Sia $\mathbb{Z}[\sqrt{2}\,]=\{a+b\sqrt{2} : a,b\in \mathbb{Z}\}\subseteq \mathbb{R}$. 
  Poiché $\mathbb{Z}$ è un PID, esso è noetheriano, dunque per il \emph{Teorema REF DI CAPITOLO 1.4} anche $\mathbb{Z}[x]$ è noetheriano. 
  Sia $\phi_{\sqrt{2}}\colon \mathbb{Z}[x]\to \mathbb{Z}[\sqrt{2}\,]$ la valutazione in $\sqrt{2}$. 
  Poiché per ogni $a+b\sqrt{2}\in \mathbb{Z}[\sqrt{2}\,]$ si ha che $\phi_{\sqrt{2}}(a+bx)=a+b\sqrt{2}$, 
  tale $\phi_{\sqrt{2}}$ è un omomorfismo suriettivo, quindi per la \emph{Proposizione \ref{noetheriano si passa con epimorfismo}} 
  anche $\mathbb{Z}[\sqrt{2}\,]$ è noetheriano$. \ \square$
\end{exm}

\noindent Un caso particolare della \emph{Proposizione \ref{noetheriano si passa con epimorfismo}} vale per gli anelli quoziente.

\begin{cor}[]{il noetheriano passa ai quozienti}
  Sia $R$ un anello noetheriano e sia $I\lhd R$. Allora, anche $R/I$ è noetheriano.
\end{cor}
\vspace{-4mm}
\begin{proof}
  Sia $\pi\colon R\to R/I$, $\pi(r)=r+I$ la proiezione canonica sul quoziente. 
  Poiché $\pi$ è un omomorfismo suriettivo, per la \emph{Proposizione \ref{noetheriano si passa con epimorfismo}} 
  anche $R/I$ è noetheriano.
\end{proof}

\noindent Possiamo quindi mostrare che esistono anelli noetheriani che non sono domini di integrità.

\begin{exm}
  Poiché $4\mathbb{Z}$ è un ideale di $\mathbb{Z}$, per il \emph{Corollario \ref{il noetheriano passa ai quozienti}} 
  anche $\mathbb{Z}/4\mathbb{Z}$ è noetheriano.\footnote{In realtà basta osservare che ogni anello finito è noetheriano 
  poiché $d_R(I)\leq |R|<\infty$ per ogni $I\lhd R$.} 
  Tuttavia, esso non è dominio di integrità perché ha divisori dello zero: infatti, $2\cdot 2=0. \ \square$
\end{exm}

\noindent Proposizione che anello noetheriano ha un ideale massimale, discussione sulla noetherianità che gratuitamente 
permette la dimostrazione senza il lemma di zorn, dimostrazione che ogni anello ha un ideale massimale usando zorn.

\begin{obs}
prova
\end{obs}
