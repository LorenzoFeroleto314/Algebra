\subsection{Localizzazione}

\noindent Introduciamo un metodo per aumentare la struttura di un dominio di integrità.

\begin{defn}[Sistema moltiplicativo]{def sistema moltiplicativo}
  Sia $R$ un dominio di integrità. Diciamo che $S\subseteq R$ è un \underline{sistema moltiplicativo} se:

  \noindent \ ($i$) $1_R\in S$ e $0_R\not\in S;$
  \noindent \ ($ii$) per ogni $a,b\in S$ anche $ab\in S$.
\end{defn}

\noindent Osserviamo che $S$ è un monoide commutativo rispetto all'operazione binaria di prodotto. 
Infatti, esso eredita l'associatività e la commutatività da $R$, la (ii) garantisce che $S$ è chiuso rispetto al prodotto e 
per la (i) sappiamo che $S$ contiene l'elemento neutro $1_R$. Tuttavia, $S$ non è sempre un gruppo, 
in quanto non richiediamo l'esistenza degli inversi moltiplicativi.

\begin{exm}
  L'insieme $S=\{2^n : n\in \mathbb{N}\}\subseteq \mathbb{Q}$ è un sistema moltiplicativo. 
  Infatti, $2^0=1\in S,$ $0\not\in S$ per le proprietà dell'esponenziale, e presi $2^a,\,2^b \in S$ anche $2^a\cdot 2^b=2^{a+b}\in S$. 
  Tuttavia, tale $S$ non è un sottogruppo di $\mathbb{Q}$ poiché ad esempio $2^{-1}=\frac{1}{2}\not\in S. \ \square$
\end{exm}

\noindent Definiamo ora una relazione di equivalenza necessaria per l'argomento.

\noindent Sull'insieme delle coppie $(r,s)\in R\times S$ definiamo la relazione $(r,s)\sim (t,u)\Leftrightarrow ru=st$.

\begin{prop}[1.7.1]{rel equivalenza su sistemi moltiplicativi}
  Sia $R$ un dominio di integrità e sia $S \subseteq R$ un suo sistema moltiplicativo. Allora 
  \vspace{-1.5mm}
  \[\sim\colon (R\times S)\times (R\times S)\to \{\operatorname{v},\operatorname{f}\}\]
  \vspace{-1.5mm}
  \noindent definita sulle coppie $(r, s)$ è una relazione di equivalenza.\footnotemark
\end{prop}
\vspace{-4mm}
\begin{proof}
  Chiaramente $(r,s)\sim (r,s)$ perché $rs=sr$, dunque $\sim$ è riflessiva. Inoltre, se $(r,s)\sim (t,u)$, allora $ru=st$, cioè $ts=ur$, 
  da cui $(t,u)\sim (r,s)$ e $\sim$ è simmetrica. Siano $(r,s)\sim (t,u)$ e $(t,u)\sim (v,w)$. Allora, $ru=st$ e $tw=uv$, cioè, 
  moltiplicando per $w$ entrambi i membri della prima uguaglianza, $ruw=s(tw)=s(uv) \Rightarrow ruw-suv=(rw-sv)u=0_R$. 
  Poiché $R$ è un dominio di integrità e $u\neq 0_R$ essendo $u\in S$, si ha che $rw-sv=0_R$, 
  cioè $rw=sv$, da cui $(r,s)\sim (v,w)$ e dunque $\sim$ è transitiva.
\end{proof}
\footnotetext{La relazione $\sim$ restituisce vero ($\operatorname{v}$) o falso ($\operatorname{f}$) a seconda che le due coppie siano o meno in relazione. 
Ricordiamo che una relazione di equivalenza è R-S-T, cioè riflessiva, simmetrica e transitiva.}

\vspace{-2mm}
\begin{obs}
  ($i$) Denotiamo con $\frac{r}{s}=[(r,s)]_{\sim}$ la classe di equivalenza di $(r,s)$ rispetto a $\sim$, 
    e sia $S^{-1}R=\{\frac{r}{s}: r\in R,s\in S\}=(R\times S)/{\sim}$ il quoziente di $R\times S$ rispetto a $\sim$.   \vspace{1.5mm}\\
  ($ii$) Nella prossima pagina andremo a definire la ``localizzazione'', che deve il suo nome alla geometria 
  algebrica\footnote{Se $R$ è un anello di funzioni definito su un 
  oggetto geometrico (come una varietà algebrica, cioè l'insieme delle soluzioni di un sistema di equazioni polinomiali) e 
  vogliamo studiare tale varietà in un certo punto $x_0$, definiamo $S$ come l'insieme delle funzioni che non si annullano in $x_0$ e 
  localizziamo $R$ a $S$. Allora, $S^{-1}R$ è un anello generalmente più semplice di $R$ che contiene informazioni solo sul comportamento 
  della varietà in un intorno di $x_0$, da cui l'origine del termine ``locale''.}. Dal punto di vista dell'algebra astratta, 
  l'idea della localizzazione è quella di aggiungere ad un anello gli inversi moltiplicativi di alcuni suoi elementi 
  introducendo delle ``frazioni'', in modo simile a quanto si fa nel passare dai numeri interi ai numeri razionali.
\end{obs}

\begin{defn}[Localizzazione e operazioni]{def localizzazione}
  Sia $R$ un dominio di integrità e sia $S$ un sistema moltipicativo di $R$. Allora, l'insieme $S^{-1}R=\{\frac{r}{s}: r\in R,s\in S\}$ 
  è detto \underline{localizzazione di $R$ a $S$}.
  Siano $\frac{r}{s}, \frac{t}{u} \in S^{-1}R$. Definiamo \emph{somma} e \emph{prodotto}:
  \begin{align*}
  &\oplus\colon S^{-1}R\times S^{-1}R\to S^{-1}R,\ \frac{r}{s}\oplus\frac{t}{u}=\frac{ru+st}{su} \\
  &\odot \colon S^{-1}R\times S^{-1}R\to S^{-1}R,\ \frac{r}{s}\odot \frac{t}{u}=\frac{rt}{su}.
  \end{align*}
\end{defn}
\begin{obs}
  Poiché $S^{-1}R$ è un insieme quoziente, per dimostrare che tali operazioni sono ben poste è necessario 
  mostrare che il loro risultato non dipende dai rappresentanti delle classi di equivalenza. 
  Per fare ciò, dimostriamo prima il seguente lemma.
\end{obs}

\begin{lem}[1.7.2: Lemma della forbice]{lemma della forbice}
  Siano $X$ e $Y$ insiemi non vuoti e sia $f\colon X\to Y$ una mappa. Sia $\sim$ una relazione di equivalenza su $X$ e $\tau\colon X\to X/{\sim}$ 
  la proiezione canonica\footnotemark. Allora, esiste una mappa $\overline{f}\colon X/{\sim}\to Y$ tale che 
  $f=\overline{f}\circ \tau$ se e solo se $f(x)=f(y)$ per ogni $x,y\in X$ con $x\sim y$.
\end{lem}
\footnotetext{Cioè la mappa che manda ogni elmento $x\in X$ nella sua classe di equivalenza $[x]_{\sim}$, 
che per comodità di notazione denoteremo di qui in seguito semplicemente con $[x]$.}
\vspace{-4mm}
\begin{proof}
  Supponiamo che $f(x)=f(y)$ per ogni $x,y\in X$ con $x\sim y$. Per l'assioma della scelta,\footnote{L'assioma della scelta afferma che data 
  una famiglia non vuota di insiemi non vuoti, esiste una funzione che ad ogni insieme della famiglia fa corrispondere un suo elemento. 
  Per poter dimostrare questo lemma è necessario assumere tale assioma, di cui si fa uso nel definire la funzione $\sigma$, 
  che altrimenti a priori non esisterebbe. Infatti, $X/{\sim}$ è la famiglia delle classi di equivalenza di $X$, 
  ognuna delle quali è non vuota poiché $x\in [x]$, e $\sigma$ è la funzione che ad ogni classe $[x]\in X/{\sim}$ 
  fa corrispondere un suo rappresentante $\sigma([x])\in X$.} esiste $\sigma\colon X/{\sim}\to X$ tale che $\sigma([x])\sim x$ 
  per ogni $x\in X$ e $\tau \circ \sigma = \operatorname{id}_{X/{\sim}}$. Si consideri ora la funzione $\overline{f}=f\circ \sigma \colon X/{\sim}\to Y$.
  \[
    \begin{tikzcd}[column sep=small]
  X \arrow[rd, "\tau"] \arrow[rr, "f"] &                                                                               & Y \\
                                      & X/{\sim} \arrow[ru, "\overline{f}"'] \arrow[lu, "\sigma", dashed, bend left=40] &  
  \end{tikzcd}
  \]
  Osserviamo innanzitutto che $\overline{f}$ è ben definita, poiché se $[x]=[y]$, 
  allora $\overline{f}([x])=\overline{f}([y])$ perché $\sigma([x])\sim x\sim y\sim \sigma([y])$ e $f(\sigma([x]))=f(\sigma([y]))$ 
  essendo per ipotesi $f$ costante sulle classi di equivalenza. Inoltre, 
  $\left(\,\overline{f}\circ \tau\right) (x)=f(\sigma(\tau(x)))=f(\sigma([x]))=f(x)$ per ogni $x\in X$ poiché $\sigma([x])\sim x$, 
  dunque è effettivamente vero che $f=\overline{f}\circ \tau$.
  \vspace{1.5mm}
  \noindent Viceversa, sia $\overline{f}\colon X/{\sim}\to Y$ tale che $f=\overline{f}\circ \tau$. 
  Allora, per ogni $x,y \in X$ con $x\sim y$, cioè $[x]=[y]$, si ha che 
  $f(x)=\overline{f}(\tau(x))=\overline{f}([x])=\overline{f}([y])=\overline{f}(\tau(y))=f(y)$ come desiderato.
\end{proof}

\vspace{2mm}
\noindent Dimostriamo quindi che le operazioni sono precedentemente discusse sono ben poste.

\clearpage


\begin{lem}[]{operazioni ben poste localizz}
Le operazioni $\oplus, \odot$ definite in precedenza su $S^{-1}R$ sono ben poste.
\end{lem}
\begin{proof}
  \noindent Sia $\widetilde{+}\colon (R,S)\times (R,S)\to S^{-1}R$ l'operazione binaria definita come $(r,s)\,\widetilde{+}\,(t,u)=\frac{ru+st}{su}$.
  \[
    \begin{tikzcd}[column sep=small]
  {(R,S)\times(R,S)} \arrow[rd, "\tau"'] \arrow[rr, "\widetilde{+}"] &                                                     & S^{-1}R \\
                                                                    & S^{-1}R\times S^{-1}R \arrow[ru, "\oplus"', dashed] &        
  \end{tikzcd}
  \]
  Per verificare che l'operazione $\oplus$ esiste ed è ben posta, per il \emph{Lemma della forbice} è sufficiente mostrare che se 
  $(r,s)\sim (r',s')$ e $(t,u)\sim (t',u')$, allora $(r,s)\,\widetilde{+}\,(t,u)=(r',s')\,\widetilde{+}\,(t',u')$, 
  cioè $\frac{ru+st}{su}=\frac{r'u'+s't'}{s'u'}$. Poiché per definizione di $\sim$ si ha che $rs'=sr'$ e $tu'=ut'$, 
  osserviamo che $(ru+st)s'u'=(rs')uu'+(tu')ss'=(sr')uu'+(ut')ss'=(r'u'+s't')su$. 
  Dunque, vale $(ru+st,su)\sim (r'u'+s't',s'u')$, da cui $\frac{ru+st}{su}=\frac{r'u'+s't'}{s'u'}$.
  \noindent Analogamente, sia $\widetilde{\cdot}\,\colon (R,S)\times (R,S)\to S^{-1}R$ l'operazione definita come 
  $(r,s)\,\widetilde{\cdot}\,(t,u)=\frac{rt}{su}$.
  \[
    \begin{tikzcd}[column sep=small]
  {(R,S)\times(R,S)} \arrow[rd, "\tau"'] \arrow[rr, "\widetilde{\cdot}"] &                                                    & S^{-1}R \\
                                                                        & S^{-1}R\times S^{-1}R \arrow[ru, "\odot"', dashed] &        
  \end{tikzcd}
  \]
  Se $(r,s)\sim (r',s')$ e $(t,u)\sim (t',u')$, osserviamo che $rts'u'=(rs')(tu')=(sr')(ut')=sur't'$, dunque $(rt,su)\sim(r't',s'u')$. 
  Allora, $(r,s)\,\widetilde{\cdot}\,(t,u)=\frac{rt}{su}=\frac{r't'}{s'u'}=(r',s')\,\widetilde{\cdot}\,(t',u')$, da cui per il 
  \emph{Lemma della forbice} l'operazione $\odot$ esiste ed è ben posta.
\end{proof}

\begin{obs} Per comodità di notazione, denoteremo di qui in seguito le due operazioni $\oplus$ e $\odot$ di $S^{-1}R$ semplicemente con $+$ e $\cdot$\,, 
  rispettivamente.\footnote{Per quanto appena provato, possiamo effettivamente vedere tali operazioni come somma e prodotto di 
  ``frazioni'' con le usuali regole di calcolo delle frazioni.}
\end{obs}

\begin{prop}[1.7.3. La localizzazione di un anello e' un dominio di integrità]{localiz di anello e' un dominio di integrità'}
  Sia $R$ un dominio di integrità e sia $S$ un sistema moltiplicativo di $R$. 
  Allora, $S^{-1}R$ dotato di tali operazioni di somma e prodotto è un dominio di integrità.
\end{prop}
\vspace{-4mm}
\begin{proof}
  Siano $\frac{r}{s}, \frac{t}{u}$ e $\frac{v}{w}$ elementi di $S^{-1}R$. Osserviamo innanzitutto che 
  
  \[ \left(\frac{r}{s}+\frac{t}{u}\right)+\frac{v}{w}=\frac{ru+st}{su}+\frac{v}{w}=\frac{ruw+stw+suv}{suw}=
  \frac{r}{s}+\frac{tw+uv}{uw}=\frac{r}{s}+\left(\frac{t}{u}+\frac{v}{w}\right) \] 
  
  \noindent da cui la somma è associativa. Inoltre, $\frac{r}{s}+\frac{t}{u}=\frac{ru+st}{su}=\frac{ts+ur}{us}=\frac{t}{u}+\frac{r}{s}$, 
  dunque $(S^{-1}R,+)$ è un gruppo abeliano con elemento neutro $0_{S^{-1}R}=\frac{0_R}{1_R}$ e opposto $-\frac{r}{s}=\frac{-r}{s}$. Essendo 
  
  \[ \left(\frac{r}{s}\cdot\frac{t}{u}\right)\cdot\frac{v}{w}=\frac{rt}{su}\cdot\frac{v}{w}=\frac{rtv}{suw}=\frac{r}{s}\cdot\frac{tv}{uw}=
  \frac{r}{s}\cdot\left(\frac{t}{u}\cdot\frac{v}{w}\right) \] 
  
  \noindent e $\frac{r}{s}\cdot\frac{t}{u}=\frac{rt}{su}=\frac{tr}{us}=\frac{t}{u}\cdot\frac{r}{s}$, il prodotto è associativo e commutativo. Infine, 
  
  \[ \left(\frac{r}{s}+\frac{t}{u}\right)\cdot \frac{v}{w}=\frac{ru+st}{su}\cdot \frac{v}{w}=\frac{ruv+stv}{suw}=
  \frac{rv}{sw}+\frac{tv}{uw}=\frac{r}{s}\cdot \frac{v}{w}+\frac{t}{u}\cdot \frac{v}{w} \] 
  
  \noindent perché $\frac{rv}{sw}+\frac{tv}{uw}=\frac{ruvw+stvw}{suww}=\frac{ruv+stv}{suw}$ essendo $(ruvw+stvw)suw=(ruv+stv)suww$. \vspace{0.5mm}\\
  Dunque, vale la proprietà distributiva e $(S^{-1}R,+,\cdot)$ è un anello commutativo con unità $1_{S^{-1}R}=\frac{1_R}{1_R}$. 
  Resta da mostrare che $S^{-1}R$ non ha divisori dello zero. Siano $\frac{r}{s}, \frac{t}{u}\in S^{-1}R$ tali che 
  $\frac{r}{s}\cdot \frac{t}{u}=0_{S^{-1}R}=\frac{0_R}{1_R}$. Allora $\frac{rt}{su}=\frac{0_R}{1_R}$, cioè $rt=(rt)1_R=(su)0_R=0_R$, da cui, 
  essendo $R$ un dominio di integrità, $r=0$ oppure $t=0$, quindi $\frac{r}{s}=\frac{0_R}{s}=\frac{0_R}{1_R}=0_{S^{-1}R}$ oppure 
  $\frac{t}{u}=\frac{0_R}{u}=\frac{0_R}{1_R}=0_{S^{-1}R}$. Dunque, $S^{-1}R$ è effettivamente un dominio di integrità.
\end{proof}

\begin{obs}
  Sia $\iota_R\colon R\rightarrow S^{-1}R$ definita come $\iota_R(r)=\frac{r}{1_R}$ l'inclusione da $R$ a $S^{-1}R$. 
  Allora $\iota_R$ è un omomorfismo di anelli iniettivo. Infatti, presi $x,y\in R$, si ha che 
  
  \[\iota_R(x+y)=\frac{x+y}{1_R}=\frac{x}{1_R}+\frac{y}{1_R}=\iota_R(x)+\iota_R(y)\] 
  \[\iota_R(xy)=\frac{xy}{1_R}=\frac{x}{1_R}\cdot \frac{y}{1_R}=\iota_R(x)\cdot \iota_R(y)\] 
  
  \noindent e $\iota_R(0_R)=\frac{0_R}{1_R}=0_{S^{-1}R}$, $\iota_R(1_R)=\frac{1_R}{1_R}=1_{S^{-1}R}$. 
  Inoltre, $\iota_R(r)=\iota_R(r')$ se e solo se $\frac{r}{1_R}=\frac{r'}{1_R}$, cioè $r=r'$. 
  Dunque, $\iota_R$ è effettivamente un omomorfismo di anelli iniettivo. 
\end{obs}

\noindent Vediamo ora alcuni esempi di sistemi moltiplicativi con le relative localizzazioni.

\begin{example}[Sistemi moltiplicativi e relative localizzazioni]{exmp sistemi moltiplicativi e relative localizzazioni}
  ($i$) Sia $R$ un dominio di integrità e sia $S=\{1_R\}$. Allora, $S$ è il più piccolo sistema moltiplicativo di 
  $R$ e $S^{-1}R\simeq R$. Infatti, in questo caso l'inclusione $\iota_R\colon R\hookrightarrow S^{-1}R$ è anche suriettiva, 
  perché preso $\frac{r}{1_R}\in S^{-1}R$ si ha che $\iota_R(r)=\frac{r}{1_R}$, ed è quindi un isomorfismo. \\

  \noindent ($ii$) Sia $R$ un dominio di integrità e sia $S=R^{\times}$. Poiché $R^{\times}$ è un gruppo rispetto al prodotto e 
  $0_R\not\in R^{\times}$, tale $S$ è un sistema moltiplicativo di $R$ e $S^{-1}R\simeq R$ perché anche in questo caso l'inclusione 
  $\iota_R\colon R\hookrightarrow S^{-1}R$ risulta essere suriettiva. Infatti, preso $\frac{r}{s}\in S^{-1}R$, 
  poiché $s\in R^{\times}$, per definizione esiste $t\in R$ tale che $st=1_R$. Dunque, $\iota_R(rt)=\frac{rt}{1_R}=\frac{r}{s}$ essendo 
  $(rt)s=r(1_R)$, da cui $\iota_R$ è un isomorfismo e $S^{-1}R\simeq R$. \\
  
  \noindent ($iii$) Sia $R$ un dominio di integrità e sia $\mathfrak{p}\lhd R$ un ideale primo.\footnotemark Detto $S=R\setminus \mathfrak{p}$, 
  osserviamo che $0_R\in \mathfrak{p}$, cioè $0_R\not\in S$, e se fosse $1_R \in\mathfrak{p}$, 
  allora $\mathfrak{p}=\langle 1_R\rangle =R$ non sarebbe proprio,\footnotemark da cui $1_R\in R\setminus \mathfrak{p}=S$. 
  Inoltre, presi $a,b\in S$, se fosse $ab\in \mathfrak{p}$, essendo $\mathfrak{p}$ primo si avrebbe che $a\in \mathfrak{p}$ o $b\in\mathfrak{p}$, assurdo. 
  Dunque, $ab\in S$ e $S$ è un sistema moltiplicativo di $R.$ Mostriamo ora che $S^{-1}R=(S^{-1}R)^{\times}\sqcup S^{-1}\mathfrak{p}$. 
  Osserviamo che $S^{-1}R=S^{-1}(R\setminus \mathfrak{p}) \sqcup S^{-1}\mathfrak{p}$, dove tale unione è disgiunta poiché 
  $S^{-1}(R\setminus \mathfrak{p}) \cap S^{-1}\mathfrak{p}=\emptyset$.\footnotemark Sia ora $\frac{r}{s}\in S^{-1}(R\setminus\mathfrak{p});$ allora, anche 
  $\frac{s}{r}\in S^{-1}(R\setminus\mathfrak{p})$ e $\frac{r}{s}\cdot \frac{s}{r}=\frac{1_R}{1_R}=1_{S^{-1}R}$, cioè $\frac{r}{s}$ è invertibile, 
  da cui $S^{-1}(R\setminus\mathfrak{p})\subseteq (S^{-1}R)^{\times}$. D'altra parte, se esistesse $\frac{r}{s}\in S^{-1}\mathfrak{p}$ invertibile,
  detto $\frac{t}{u}\in S^{-1}R$ il suo inverso si avrebbe $rt=su\in \mathfrak{p}$, il che è assurdo poiché $s,u \in S=R\setminus \mathfrak{p}$ 
  violando la definizione di ideale primo. Dunque $(S^{-1}R)^{\times}\subseteq S^{-1}(R\setminus\mathfrak{p})$, 
  da cui $S^{-1}R=S^{-1}(R\setminus \mathfrak{p}) \sqcup S^{-1}\mathfrak{p}=(S^{-1}R)^{\times}\sqcup S^{-1}\mathfrak{p}$.
\end{example}

\footnotetext{Un ideale proprio $\mathfrak{p}\lhd R$ si dice primo se, presi $a,b\in R$, si ha che $ab\in \mathfrak{p}$ se e solo se 
  $a\in \mathfrak{p}$ o $b\in \mathfrak{p}$.}
\footnotetext{In generale, se un ideale $I\lhd R$ contiene l'unità $1_R$, allora $r=r1_R\in I$ per ogni $r\in R$, cioè $I=R$.}
\footnotetext{Sia $\frac{r}{s}\in S^{-1}\mathfrak{p}$; se fosse $\frac{r}{s}=\frac{t}{u}\in S^{-1}(R\setminus\mathfrak{p})$, 
  essendo $r\in \mathfrak{p}$, si avrebbe che $ru=st\in \mathfrak{p}$. 
  Dunque, essendo $\mathfrak{p}$ primo, dovrebbe essere $s\in \mathfrak{p}$ o $t\in\mathfrak{p}$, il che è assurdo essendo 
  $s,t \in S=R\setminus\mathfrak{p}$.}

\begin{obs} 
  Se $R$ è un dominio di integrità, $\{0_R\}\lhd R$ è un ideale primo perché $R$ non ha divisori dello zero, 
  cioè $ab=0_R$ se e solo se $a=0_R$ oppure $b=0_R$. Dunque, per quanto visto nell'ultimo esempio, 
  $S=R\setminus \{0_R\}$ è un sistema moltiplicativo di $R$ e $S^{-1}R=(S^{-1}R)^{\times} \sqcup \left\{ \frac{0_R}{1_R} \right\}$ 
  è un dominio di integrità in cui ogni elemento non nullo è invertibile, cioè un campo.
\end{obs}

\begin{defn}[Campo dei quozienti]{def campo dei quozienti}
  Sia $R$ un dominio di integrità e sia $S=R\setminus \{0_R\}$. Allora, $S^{-1}R$ è un campo detto \underline{campo dei quozienti di $R$} e 
  si denota con $\quot(R)$.
\end{defn}

\begin{exm}
  Se consideriamo $\mathbb{Z}$, si ha che $\quot(\mathbb{Z})=\left\{\frac{m}{n}: m\in \mathbb{Z}, n\in\mathbb{Z}\setminus\{0\}\right\}=\mathbb{Q}.$
\end{exm}

\noindent I numeri razionali sono denotati con il simbolo $\mathbb{Q}$ proprio perché essi sono il ``quoziente''  dei numeri interi. Inoltre, $\mathbb{Z}$ è un sottoanello del campo $\mathbb{Q}=\quot(\mathbb{Z})$. Quest'ultimo è un fatto generale, come dimostrato dalla proposizione seguente.

\begin{prop}[1.7.4]{Dom integrita isomorfo a sottoanello campo quoz}
  Ogni dominio di integrità è isomorfo a un sottoanello del suo campo dei quozienti.
\end{prop}
\vspace{-4mm}
\begin{proof}
  Sia $R$ un dominio di integrità e sia $\iota_R\colon R\rightarrow \quot(R)$ l'inclusione. 
  Poiché $\iota_R$ è un omomorfismo iniettivo, $\ker(\iota_R)=\{0_R\}$ e $\operatorname{Im}(\iota_R)$ è un sottoanello del campo $\quot(R)$. 
  Dunque, per il \emph{primo teorema d'isomorfismo} si ha che $R=R/\ker(\iota_R)\simeq \operatorname{Im}(\iota_R)$.
\end{proof}

\begin{obs} 
  In particolare, $\mathbb{Q}=\quot(\mathbb{Z})$ non solo contiene $\mathbb{Z}$ come sottoanello, 
  ma è proprio il più piccolo campo contenente $\mathbb{Z}$. Infatti, se $\mathbb{K}$ è un campo contenente $\mathbb{Z}$, 
  allora $n^{-1}=\frac{1}{n}\in \mathbb{K}$ per ogni $n\in \mathbb{Z}\setminus\{0\}$ e $m\cdot \frac{1}{n}\in \mathbb{K}$ 
  per ogni $m\in \mathbb{Z}$, da cui $\mathbb{Q}\subseteq \mathbb{K}$. Anche questo è un fatto generale che caratterizza 
  il campo dei quozienti di ogni dominio di integrità.
\end{obs}

\begin{prop}[1.7.5]{campo quoz piccolo sottoanello isomorfo a R}
  Sia $R$ un dominio di integrità. Allora, il campo dei quozienti $\quot(R)$ è il più piccolo campo contenente un sottoanello isomorfo a $R$.
\end{prop}
\vspace{-4mm}
\begin{proof}
  Osserviamo innanzitutto che per la \emph{proposizione \ref{Dom integrita isomorfo a sottoanello campo quoz}} sappiamo che $R$ è isomorfo al sottoanello 
  $\operatorname{Im}(\iota_R)$ del campo $\quot(R)$. Sia quindi $\mathbb{K}$ un campo contenente $R$ e sia 
  $\phi\colon \operatorname{quot(R)}\to \mathbb{K}$ la mappa definita come $\phi(\frac{r}{s})=rs^{-1}$. 
  Tale mappa è ben definita: infatti, $r\in \mathbb{K}$ perché $r\in R\subseteq \mathbb{K}$, e in quanto campo $\mathbb{K}$ contiene 
  anche tutti gli inversi $s^{-1}$ degli elementi $s\in R\setminus\{0_R\}$, da cui $rs^{-1}\in \mathbb{K}$. 
  Inoltre, se $\frac{r}{s}=\frac{r'}{s'}$ per definizione vale $rs'=r's$, quindi $\phi(\frac{r}{s})=rs^{-1}=r's'^{-1}=\phi(\frac{r'}{s'})$. 
  Siano ora $\frac{r}{s}, \frac{t}{u}\in \quot(R)$. Allora, si ha che 
  
  \[\phi\left(\frac{r}{s}+\frac{t}{u}\right)=\phi\left(\frac{ru+st}{su}\right)=(ru+st)(su)^{-1}=rs^{-1}+tu^{-1}=
  \phi\left(\frac{r}{s}\right)+\phi\left(\frac{t}{u}\right)\] 
  
  \[\phi\left(\frac{r}{s}\cdot \frac{t}{u}\right)=\phi\left(\frac{rt}{su}\right)=rt(su)^{-1}=rs^{-1}tu^{-1}=
  \phi\left(\frac{r}{s}\right)\cdot \phi\left(\frac{t}{u}\right)\] 
  
  \noindent e $\phi(\frac{r}{s})=rs^{-1}=0_{\mathbb{K}}$ se e solo se $r=0_R$, da cui $\phi$ è un omomorfismo di campi iniettivo. 
  Poiché $\operatorname{Im}(\phi)$ è un sottoanello del campo $\mathbb{K}$ e per il \emph{primo teorema d'isomorfismo} si ha che 
  $\operatorname{quot(R)}=\operatorname{quot(R)}/\ker(\phi)\simeq \operatorname{Im}(\phi)$, concludiamo che $\mathbb{K}$ 
  contiene un sottoanello isomorfo a $\quot(R)$ ed è quindi un campo più grande del campo dei quozienti $\quot(R)$. 
\end{proof}

\noindent Sia $R$ un dominio di integrità e sia $S$ un sistema moltiplicativo di $R$. Vogliamo ora studiare le eventuali relazioni tra gli ideali di $R$ e quelli di $S^{-1}R$. Presi gli ideali $I\lhd R$ e $J\lhd S^{-1}R$, definiamo $S^{-1}I=\{\frac{i}{s}: i\in I, s\in S\}$ e denotiamo con $\overline{J}=\iota_R^{-1}(J)=\{r\in R: \frac{r}{1_R}\in J\}$.

\begin{prop}[1.7.6]{su rel tra ideali di R e di localizzazione}
  Sia $S$ un sistema moltiplicativo di un dominio di integrità $R$. Presi $I\lhd R$ e $J\lhd S^{-1}R$,\\
  \ ($a$) $S^{-1}I\lhd S^{-1}R$ e $S^{-1}I=S^{-1}R$ se e solo se $I\cap S\neq \emptyset$;
  \ ($b$) $\overline{J}\lhd R$ e $S^{-1}\overline{J}=J$.
\end{prop}
\vspace{-4mm}
\begin{proof}
  ($a$) Siano $\frac{i}{s},\frac{j}{t}\in S^{-1}I$. Allora, $\frac{i}{s}+\frac{j}{t}=\frac{it+js}{st}\in S^{-1}I$ perché $it+js\in I$ 
  per definizione di ideale e $st\in S$ per definizione di sistema moltiplicativo. Analogamente, $\frac{i}{s}\cdot \frac{j}{t}=\frac{ij}{st}\in S^{-1}I$ 
  perché $ij\in I$ e $st\in S$, da cui $S^{-1}I$ è effettivamente un ideale di $S^{-1}R$. 
  Osserviamo ora che se $S^{-1}I=S^{-1}R$, in particolare esiste un elemento $\frac{i}{s}\in S^{-1}I$ tale che $\frac{i}{s}=1_{S^{-1}R}=\frac{1_R}{1_R}$. 
  Dunque, $i=i(1_R)=s(1_R)=s$, cioè $i=s\in I\cap S$, da cui $I\cap S\neq \emptyset$. 
  Viceversa, supponiamo che $I\cap S\neq \emptyset$. Preso $t\in I\cap S$, si ha che $\frac{t}{t}=\frac{1_R}{1_R}=1_{S^{-1}R}\in S^{-1}I$, da cui, 
  essendo $S^{-1}I$ un ideale, $\frac{r}{s}\cdot 1_{S^{-1}R}=\frac{r}{s}\in S^{-1}I$ per ogni $\frac{r}{s}\in S^{-1}R$, cioè $S^{-1}I=S^{-1}R$.
  \vspace{1mm}
  \noindent ($b$) Poiché $\iota_R\colon R\hookrightarrow S^{-1}R$ è un omomorfismo, la preimmagine $\iota_R^{-1}(J)\subseteq R$ di un ideale 
  $J\lhd S^{-1}R$ è un ideale di $R$, cioè $\overline{J}\lhd R$. Preso $\frac{j}{s}\in S^{-1}\overline{J}$, 
  per definizione si ha che $\frac{j}{1_R}\in J$. Quindi, essendo $J$ un ideale, $\frac{j}{s}=\frac{1_R}{s}\cdot \frac{j}{1_R}\in J$, 
  da cui $S^{-1}\overline{J}\subseteq J$. D'altra parte, preso $\frac{r}{s}\in J$, per definizione di ideale si ha che 
  $\frac{s}{1_R}\cdot \frac{r}{s}=\frac{r}{1_R}\in J$, cioè $r\in \overline{J}$. Dunque risulta $\frac{r}{s}\in S^{-1}\overline{J}$, 
  da cui $J\subseteq S^{-1}\overline{J}$. Combinando le doppie inclusioni, si ha quindi che $S^{-1}\overline{J}=J$.
\end{proof}

\noindent Vi è quindi un legame tra la noetherianità di $R$ e quella di una sua localizzazione $S^{-1}R$.

\begin{cor}[1.7.7]{localizzazione dom integrita noetheriano}
  Sia $R$ un dominio di integrità noetheriano e sia $S$ un sistema moltiplicativo di $R$. Allora, anche $S^{-1}R$ è un dominio di integrità noetheriano.
\end{cor}
\vspace{-4mm}
\begin{proof}
  Per la \emph{proposizione \ref{localiz di anello e' un dominio di integrità'}} sappiamo che $S^{-1}R$ è un dominio di integrità, 
  quindi è sufficiente provare che esso è anche noetheriano. Siano $J\lhd S^{-1}R$ e $\overline{J}=\iota_R^{-1}(J)\lhd R$. 
  Essendo $R$ noetheriano, esistono $a_1,...\,,a_n\in R$ tali che $\overline{J}=\langle a_1,...\,,a_n \rangle$. 
  Dunque, per la \emph{Proposizione \ref{su rel tra ideali di R e di localizzazione}} si ha che 
  $J=S^{-1}\overline{J}=\{\frac{j}{s}: j\in\overline{J}, s\in S\}=\langle \frac{a_1}{1_R},...\,,\frac{a_n}{1_R} \rangle$ è finitamente generato, 
  da cui per l'arbitrarietà di $J$ concludiamo che $S^{-1}R$ è noetheriano.
\end{proof}

\noindent Esiste un'importante famiglia di anelli strettamente legata al concetto di localizzazione.

\begin{defn}[Anello locale]{def anello locale}
  Un anello commutativo $R$ si dice \underline{locale} se $\mathfrak{m}=R\setminus R^{\times}$ è un ideale di $R$.
\end{defn}

\noindent Un anello locale è quindi un anello i cui elementi non invertibili costituiscono un ideale.

\begin{example}[]{esempi anello locale}
  ($i$) Ogni campo $\mathbb{K}$ è un anello locale. Infatti, $\mathbb{K}^{\times}=\mathbb{K}\setminus \{0_{\mathbb{K}}\}$ 
  poiché per definizione di campo ogni elemento non nullo è invertibile, dunque 
  $\mathfrak{m}=\mathbb{K}\setminus\mathbb{K}^{\times}=\{0_{\mathbb{K}}\}\lhd \mathbb{K}.$\\

  \noindent ($ii$) Sia $R$ un dominio di integrità e sia $\mathfrak{p}\lhd R$ un ideale primo. Detto $S=R\setminus \mathfrak{p}$, $S^{-1}R$ 
  è un anello locale perché abbiamo mostrato che $S^{-1}R \setminus (S^{-1}R)^{\times}=S^{-1}\mathfrak{p}\lhd S^{-1}R.$\\

  \noindent ($iii$) Sia $\mathbb{K}$ un campo. Allora, l'anello $\mathbb{K}\llbracket x\rrbracket$ delle serie formali è un anello locale poiché 
  per la \emph{Proposizione 1.4.X} si ha che $\mathfrak{m}=\mathbb{K}\llbracket x\rrbracket\setminus\mathbb{K}\llbracket x\rrbracket^{\times}=\langle x 
  \rangle\lhd \mathbb{K}\llbracket x\rrbracket.$\\

  \noindent Osserviamo che non tutti i domini di integrità sono anche anelli locali.\\

  \noindent ($iv$) Sia $\mathbb{K}$ un campo. Allora, $\mathbb{K}[x]$ non è un anello locale. Infatti, sappiamo che per la 
  \emph{Proposizione \ref{se R dom R[x] dom}} vale $\mathbb{K}[x]^{\times}=\mathbb{K}^{\times}$, 
  da cui $\mathfrak{m}=\mathbb{K}[x]\setminus\mathbb{K}[x]^{\times}=\{f(x)\in \mathbb{K}[x]: \deg^{\star}(f)\geq 1\}$. 
  Tuttavia, $\mathfrak{m}$ non è un ideale di $\mathbb{K}[x]$ poiché $f(x)=x+1_{\mathbb{K}}$ e $g(x)=x$ sono elementi 
  di $\mathfrak{m}$ ma $h(x)=f(x)-g(x)=1_{\mathbb{K}}\not\in \mathfrak{m}$ perché $\deg^{\star}(h)=0.$\\

  \noindent D'altra parte, esistono esempi di anelli locali che non sono domini di integrità.\\

  \noindent ($v$) Esempio$. \ \square$
\end{example}

\noindent Esiste una caratterizzazione equivalente degli anelli locali in termini di ideali massimali.

\begin{prop}[1.7.8]{prop su ideale massimale}
  Sia $R$ un anello locale. Allora, $\mathfrak{m}=R\setminus R^{\times}$ è l'unico ideale massimale di $R$.
\end{prop}
\vspace{-4mm}
\begin{proof}
  Osserviamo innanzitutto che $\mathfrak{m}\lhd R$ è massimale perché, preso $I\lhd R$ tale che $\mathfrak{m} \varsubsetneq I$, 
  si ha che $I\setminus \mathfrak{m}\neq \emptyset$, cioè $I\cap R^{\times}\neq \emptyset$, da cui $I=R$ poiché $I$ contiene un elemento 
  invertibile.\footnote{Infatti, se $I$ contiene $r\in R^{\times}$, detto $r^{-1}$ il suo inverso si ha che $r^{-1}r=1_R\in I$, da cui $I=R$.} 
  D'altra parte, se $J\lhd R$ è un ideale massimale, per quanto appena visto deve essere $J\cap R^{\times}=\emptyset$, 
  cioè $J\subseteq R\setminus R^{\times}=\mathfrak{m}$ che è già massimale, da cui $J=\mathfrak{m}$ e $\mathfrak{m}$ è unico.
\end{proof}

\noindent Possiamo quindi caratterizzare tutti e soli gli interi $n$ per cui $\mathbb{Z}/n\mathbb{Z}$ è un anello locale.

\begin{prop}[1.7.9]{quando z nz e' locale}
  L'anello $\mathbb{Z}/n\mathbb{Z}$ è locale se e solo se $n$ è la potenza di un primo.
\end{prop}
\vspace{-4mm}
\begin{proof}
  Sia $n=p^k$ con $p$ primo e $k\geq 1$ intero. Osserviamo che $a+p^k\mathbb{Z}\in (\mathbb{Z}/p^k\mathbb{Z})^{\times}$ se e solo se esiste 
  $b\in \mathbb{Z}$ tale che $ab\in 1+p^k\mathbb{Z}$, cioè $ab\equiv 1 \pmod{p^k}$. In particolare, vale $ab\equiv 1\pmod{p}$, 
  da cui per l'\emph{Identità di Bézout} si ha che $\operatorname{MCD}(a,p)=1$.\footnote{Secondo l'\emph{Identità di Bézout}, 
  dati due interi $a,b$ non entrambi nulli e detto $d=\operatorname{MCD}(a,b)$, esistono $x,y\in \mathbb{Z}$ tali che $ax+by=d$, 
  e $d$ è il più piccolo intero che può essere scritto in questa forma. In questo caso, essendo $ab\equiv 1\pmod{p}$, 
  esiste $t\in \mathbb{Z}$ tale che $ab+pt=1$, dunque $\operatorname{MCD}(a,p)\leq 1$, cioè $\operatorname{MCD}(a,p)=1$.}\,Si ha quindi che 
  $(\mathbb{Z}/p^k\mathbb{Z})^{\times}=\{a+p^k\mathbb{Z}: \operatorname{MCD}(a,p)=1\}$, da cui 
  $\mathfrak{m}=\mathbb{Z}/p^k\mathbb{Z}\setminus (\mathbb{Z}/p^k\mathbb{Z})^{\times}=p\mathbb{Z}/p^k\mathbb{Z}$ 
  è un ideale di $\mathbb{Z}/p^k\mathbb{Z}$.\footnote{Il complementare di $(\mathbb{Z}/p^k\mathbb{Z})^{\times}$ in $\mathbb{Z}/p^k\mathbb{Z}$ 
  è costituito da tutte le classi di equivalenza $a+p^k\mathbb{Z}$ per cui $\operatorname{MCD}(a,p)>1$, cioè, essendo $p$ primo, 
  $\operatorname{MCD}(a,p)=p$. Dunque, $\mathfrak{m}=\{a+p^k\mathbb{Z}: p\mid a\}=p\mathbb{Z}/p^k\mathbb{Z}\lhd \mathbb{Z}/p^k\mathbb{Z}$.} 
  Dunque, $\mathbb{Z}/p^k\mathbb{Z}$ è effettivamente un anello locale.\\

  \noindent Viceversa, supponiamo per assurdo che esistano $p\neq q$ primi con $p\mid n$ e $q \mid n$. Poiché per il \emph{terzo teorema d'isomorfismo} 
  sappiamo che $(\mathbb{Z}/n\mathbb{Z})/(p\mathbb{Z}/n\mathbb{Z})\simeq \mathbb{Z}/p\mathbb{Z}$ che è un campo, 
  $p\mathbb{Z}/n\mathbb{Z}\lhd \mathbb{Z}/n\mathbb{Z}$ è un ideale massimale.\footnote{Ricordiamo che preso un ideale $I\lhd R$, 
  l'anello quoziente $R/I$ è un campo se e solo se $I$ è massimale.} Analogamente, anche $q\mathbb{Z}/n\mathbb{Z}\lhd \mathbb{Z}/n\mathbb{Z}$ è massimale, 
  ed essendo $p\neq q$ tali ideali sono distinti.\footnote{Infatti, sono ideali finiti contenenti un numero diverso di elementi, 
  essendo $|p\mathbb{Z}/n\mathbb{Z}|=\frac{n}{p}\neq \frac{n}{q}=|q\mathbb{Z}/n\mathbb{Z}|$.} 
  Abbiamo quindi trovato due ideali massimali distinti di $\mathbb{Z}/n\mathbb{Z}$, da cui per la \emph{Proposizione \ref{prop su ideale massimale}} 
  concludiamo che $\mathbb{Z}/n\mathbb{Z}$ non è locale.
\end{proof}

\noindent Sia $R$ un anello locale e sia $\mathfrak{m}=R\setminus R^{\times}$. Poiché per la \emph{Proposizione \ref{prop su ideale massimale}} 
l'ideale $\mathfrak{m}\lhd R$ è massimale, l'anello quoziente $R/\mathfrak{m}$ risulta essere un campo.

\begin{defn}[Campo dei residui]{def campo dei residui}
Sia $R$ un anello locale e sia $\mathfrak{m}=R\setminus R^{\times}$ il suo unico ideale massimale. Allora, il campo $\operatorname{res}(R)=R/\mathfrak{m}$ è detto \underline{campo dei residui di $R$}.
\end{defn}

\begin{example}[]{exmp campo dei residui}
  ($i$) Sia $\mathbb{K}$ un campo. Poiché $\mathfrak{m}=\mathbb{K}\setminus \mathbb{K}^{\times}=\{0_{\mathbb{K}}\}$, si ha che 
  $\operatorname{res}(\mathbb{K})=\mathbb{K}/\{0_{\mathbb{K}}\}\simeq \mathbb{K}.$\\

  \noindent ($ii$) Si consideri $\mathbb{Z}/p^k\mathbb{Z}$ con $p$ primo e $k\geq 1$ intero. Per quanto appena provato nella 
  \emph{Proposizione \ref{quando z nz e' locale}}, si ha che $\mathfrak{m}=\mathbb{Z}/p^k\mathbb{Z}\setminus 
  (\mathbb{Z}/p^k\mathbb{Z})^{\times}=p\mathbb{Z}/p^k\mathbb{Z}$, da cui per il \emph{terzo teorema d'isomorfismo} otteniamo che 
  $\operatorname{res}(\mathbb{Z}/p^k\mathbb{Z})=(\mathbb{Z}/p^k\mathbb{Z})/(p\mathbb{Z}/p^k\mathbb{Z})\simeq \mathbb{Z}/p\mathbb{Z}.$ \\

  \noindent ($iii$) Sia $\mathbb{K}$ un campo. Poiché per la \emph{Proposizione 1.4.X} vale 
  $\mathfrak{m}=\mathbb{K}\llbracket x\rrbracket\setminus\mathbb{K}\llbracket x\rrbracket^{\times}=\langle x \rangle$, si ha che 
  $\operatorname{res}(\mathbb{K}\llbracket x\rrbracket)=\mathbb{K}\llbracket x\rrbracket/\langle x\rangle = 
  \{a+\langle x\rangle : a\in \mathbb{K}\}\simeq \mathbb{K}.$
\end{example}

\noindent Nel caso in cui $\mathfrak{p}\lhd R$ sia un ideale primo del dominio di integrità $R$ e $S=R\setminus \mathfrak{p}$, 
la struttura del campo dei residui dell'anello locale $S^{-1}R$ risulta essere particolarmente interessante.

\begin{prop}[1.7.10]{prop su dom integrita e campo dei residui}
Sia $R$ un dominio di integrità, $\mathfrak{p}\lhd R$ primo e $S=R\setminus \mathfrak{p}$. Allora, $\operatorname{res}(S^{-1}R)\simeq \quot(R/\mathfrak{p})$.
\end{prop}
\vspace{-4mm}
\begin{proof}
Dim
\end{proof}

\noindent Cose, sarebbe carino avere 1.7.10 con il relativo esempio in una pagina a sè stante.

\begin{exm}Esempio di mostrare che $S=\mathbb{Z}\setminus 13\mathbb{Z}$ è locale e calcolare $\operatorname{res}(S^{-1}\mathbb{Z})$.\end{exm}
asdfasdf