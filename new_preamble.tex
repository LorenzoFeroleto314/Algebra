% math packages
\usepackage{mathtools}
\usepackage{amsmath}
\usepackage{amssymb}    % Math symbols such as \mathbb
\usepackage{amsthm}

% other packages
\usepackage{lipsum}
\usepackage[a4paper, total={6in, 8in}]{geometry} % Per margini meno larghi

% proper inline math display, adjust height for symbols like \sum
\everymath{\displaystyle}

% Fancy boxes
\usepackage[framemethod=TikZ]{mdframed}


%%%%%%%%%%%%%%%%%%%%%%%%%%%%%%
% Ricordati di mettere sempre le label
%%%%%%%%%%%%%%%%%%%%%%%%%%%%%%
%Theorem
\newcounter{theo}[section] \setcounter{theo}{0}
\renewcommand{\thetheo}{\arabic{section}.\arabic{theo}}
\newenvironment{theo}[2][]{%
  \refstepcounter{theo}%
  \ifstrempty{#1}%
  % if condition (without title)
  {\mdfsetup{%
      frametitle={%
          \tikz[baseline=(current bounding box.east),outer sep=0pt]
          \node[anchor=east,rectangle,fill=cyan!20]
          {\strut Teorema~\thetheo};}
      }%
  % else condition (with title)
  }{\mdfsetup{%
      frametitle={%
          \tikz[baseline=(current bounding box.east),outer sep=0pt]
          \node[anchor=east,rectangle,fill=cyan!20]
          {\strut Teorema~\thetheo:~#1};}%
      }%
  }%
  % Both conditions
  \mdfsetup{%
      innertopmargin=5pt,linecolor=cyan!20,%
      linewidth=2pt,topline=true,%
      frametitleaboveskip=\dimexpr-\ht\strutbox\relax%
  }
\begin{mdframed}[]\relax%
\label{#2}}{\end{mdframed}}
%%%%%%%%%%%%%%%%%%%%%%%%%%%%%%

%%%%%%%%%%%%%%%%%%%%%%%%%%%%%%
% Lemma
\newenvironment{lem}[2][]{%
  \refstepcounter{theo}%
  \ifstrempty{#1}%
  % if condition (without title)
  {\mdfsetup{%
      frametitle={%
          \tikz[baseline=(current bounding box.east),outer sep=0pt]
          \node[anchor=east,rectangle,fill=cyan!20]
          {\strut Lemma~\thetheo};}
      }%
  % else condition (with title)
  }{\mdfsetup{%
      frametitle={%
          \tikz[baseline=(current bounding box.east),outer sep=0pt]
          \node[anchor=east,rectangle,fill=cyan!20]
          {\strut Lemma~\thetheo:~#1};}%
      }%
  }%
  % Both conditions
  \mdfsetup{%
      innertopmargin=5pt,linecolor=cyan!20,%
      linewidth=2pt,topline=true,%
      frametitleaboveskip=\dimexpr-\ht\strutbox\relax%
  }
\begin{mdframed}[]\relax%
\label{#2}}{\end{mdframed}}
%%%%%%%%%%%%%%%%%%%%%%%%%%%%%%

%%%%%%%%%%%%%%%%%%%%%%%%%%%%%%
%Definition
\newenvironment{defn}[2][]{%
  \refstepcounter{theo}%
  \ifstrempty{#1}%
  % if condition (without title)
  {\mdfsetup{%
      frametitle={%
          \tikz[baseline=(current bounding box.east),outer sep=0pt]
          \node[anchor=east,rectangle,fill=yellow!40]
          {\strut Definizione~\thetheo};}
      }%
  % else condition (with title)
  }{\mdfsetup{%
      frametitle={%
          \tikz[baseline=(current bounding box.east),outer sep=0pt]
          \node[anchor=east,rectangle,fill=yellow!40]
          {\strut Definizione~\thetheo:~#1};}%
      }%
  }%
  % Both conditions
  \mdfsetup{%
      innertopmargin=5pt,linecolor=yellow!40,%
      linewidth=2pt,topline=true,%
      frametitleaboveskip=\dimexpr-\ht\strutbox\relax%
  }
\begin{mdframed}[]\relax%
\label{#2}}{\end{mdframed}}
%%%%%%%%%%%%%%%%%%%%%%%%%%%%%%

%shortcut to mathbb
\newcommand{\N}{\mathbb{N}}
\newcommand{\Z}{\mathbb{Z}}
\newcommand{\Q}{\mathbb{Q}}
\newcommand{\R}{\mathbb{R}}
\newcommand{\I}{\mathbb{I}}
\newcommand{\C}{\mathbb{C}}

% Gives begin{dimostrazione} same formating as \begin{proof}
\newenvironment{dimostrazione}
  {\begin{proof}[Dimostrazione]}
  {\end{proof}}