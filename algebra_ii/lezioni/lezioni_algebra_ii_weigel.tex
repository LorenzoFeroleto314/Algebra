\documentclass{article}
\usepackage[utf8]{inputenc}
\usepackage{import}

\import{../}{new_preamble.tex}

\title{
  \Large{\bf{Appunti di Algebra II}} \\
  Dipartimento di Matematica e Applicazioni,\\
  Univerisita' degli studi di Milano-Bicocca. \\
  A.A 2020/2021
}
\author{Lorenzo Feroleto}
\date{September 2020}
 
%%

\begin{document}
\maketitle
\thispagestyle{empty}

\newpage
\tableofcontents
\thispagestyle{empty}

\newpage
\pagenumbering{arabic}

\section{Introduzione}

\subsection{Sottosezione}

\lipsum[2]

\begin{defn}[La definizione magica]{n}
BAAAAAAAAAAAAAAAAAAAAA
\end{defn}

\lipsum[3]

\begin{theo}[Il teorema grosso piu' grosso]{o}
CAAAAAAAAAAAAAAAAAAAAAAAAAA
\[3x + 5y = 5z\]
\end{theo}
\begin{dimostrazione}
prova
\end{dimostrazione}

\begin{lem}[Il lemmone]{l}
Il lemma potente
\end{lem}

\section{Altro capitolo}
\begin{defn}[La definizione magica]{magica}
  BAAAAAAAAAAAAAAAAAAAAA
\end{defn}

\begin{dimostrazione} 
asdf
\end{dimostrazione}

\begin{prop}[fatto]{fatto}
prova
\end{prop}
\begin{dimostrazione}
\lipsum[3]
\end{dimostrazione}

\begin{coroll}[corollo]{corollo}
  provare
\end{coroll}

\end{document}