\section{Complementi di teoria degli anelli}

\subsection{Anelli di polinomi in una variabile}

Andiamo a definire l'anello dei polinomi senza il concetto di successione normalmente 
utilizzato in approcci piu' rigorosi.


\begin{defn}[Anello di polinomi a coefficienti in R nella variabile x]{anel_pol_1}
Sia $R$ un anello commutativo e definiamo l'\emph{anello di polinomi in una variabile}
come la seguente struttura algebrica

\begin{equation}
R[x] = \{ f:= \sum\limits_{i=0}^{n}a_i x^i \mid a_i \in R, n\in \N\} 
\end{equation}
\end{defn}
Notiamo come $x^i$ in questo contesto non e' nient'altro che una indeterminata che
obbedisce alle proprieta' degli esponenti di una potenza. \\
Procediamo ora a definire le operazioni di somma e prodotto di polinomi in una variabile.

\begin{defn}[Operazioni tra polinomi in una variabile]{op tra pol.1var}
Siano $f = \sum_{i=0}^{n}r_i \cdot x^i,\ g = \sum_{i=0}^{n}s_i \cdot x^i \in R[x]$. 
Definiamo la \emph{somma}

\begin{equation}
+ : R[x] \times R[x] \to R[x],\ f + g = \sum\limits_{i=0}^{max\{n,m\}} (r_i + s_i) \cdot x^i
\end{equation}
ponendo $r_{n+1} = \dots = r_m = 0$ se $m > n$ e $s_{m+1} = \dots = s_n = 0$ se $n > m$.\\
Definiamo il \emph{prodotto}

\begin{equation}
f\cdot g = \sum\limits_{k=0}^{n+m} (\sum\limits_{i=0}^{k} r_i \cdot s_{k-i}) \cdot x^i
\end{equation}
Tale scrittura e' la normale moltiplicazione tra i polinomi ma scritta formalmente.
\end{defn}
Vediamo alcuni semplici esempi: 

\begin{example}[]{polinomi example}
Aggiungere esempio
\end{example}
Come visto nel corso di Algebra I, si verifica facilmente che $R[x]$ dotato di 
tali operazioni di somma e prodotto e' un anello commutativo con elemento neutro il polinomio
identicamente nullo $0_{R[x]} = 0_R$ e unita' il polinomio costante $1_{R[x]} = 1_R$.\\
Definiamo ora un imporante funzione che descrive un polinomio.

\newpage
\begin{defn}[Funzione grado; grado di un polinomio]{grado polinomio}
Sia $R$ un anello commutativo e sia $f(x) = f \in R[x]$ definita come in precedenza. Allora definiamo
la funzione grado:
\[
  deg^*(f) := \left\{\begin{array}{lr}
      \max\{k \in \N : a_k \neq 0_R\}, & \text{ se } f(x) \not\equiv 0_R\\
      -\infty, & \text{ se } f(x) \equiv 0_R 
      \end{array}\right\} 
\]
e il risultato di $deg^*(f)$ come il \emph{grado} del polinomio $f$.
\end{defn}
