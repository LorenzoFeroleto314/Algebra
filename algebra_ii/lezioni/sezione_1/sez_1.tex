\section{Complementi di teoria degli anelli}

\subsection{Anelli di polinomi in una variabile}

Andiamo a definire l'anello dei polinomi senza il concetto di successione normalmente 
utilizzato in approcci piu' rigorosi.


\begin{defn}[Anello di polinomi a coefficienti in R nella variabile x]{anel_pol_1}
Sia $R$ un anello commutativo e definiamo l'\emph{anello di polinomi in una variabile}
come la seguente struttura algebrica

\begin{equation}
R[x] = \{ f:= \sum\limits_{i=0}^{n}a_i x^i \mid a_i \in R, n\in \N\} 
\end{equation}
Sia $f$ un polinomio come quello sopracitato; allora il coefficiente $a_n$ viene chiamato 
\emph{coefficiente direttivo} di $f$. Se $a_n=1$, allora il polinomio viene detto \emph{monico}.
\end{defn}
Notiamo come $x^i$ in questo contesto non e' nient'altro che una indeterminata che
obbedisce alle proprieta' degli esponenti di una potenza. \\
Procediamo ora a definire le operazioni di somma e prodotto di polinomi in una variabile.

\begin{defn}[Operazioni tra polinomi in una variabile]{op tra pol.1var}
Siano $f = \sum_{i=0}^{n}r_i \cdot x^i,\ g = \sum_{i=0}^{n}s_i \cdot x^i \in R[x]$. 
Definiamo la \emph{somma}

\begin{equation}
+ : R[x] \times R[x] \to R[x],\ f + g = \sum\limits_{i=0}^{\max\{n,m\}} (r_i + s_i) \cdot x^i
\end{equation}
ponendo $r_{n+1} = \dots = r_m = 0$ se $m > n$ e $s_{m+1} = \dots = s_n = 0$ se $n > m$.\\
Definiamo il \emph{prodotto}

\begin{equation}
  \cdot : R[x] \times R[x] \to R[x],\ f\cdot g = \sum\limits_{k=0}^{n+m} (\sum\limits_{i=0}^{k} r_i \cdot s_{k-i}) \cdot x^i
\end{equation}
Tale scrittura e' la normale moltiplicazione tra i polinomi ma scritta formalmente.
\end{defn}
Vediamo alcuni semplici esempi: 

\begin{example}[]{polinomi example}
Aggiungere esempio
\end{example}
Come visto nel corso di Algebra I, si verifica facilmente che $R[x]$ dotato di 
tali operazioni di somma e prodotto e' un anello commutativo con elemento neutro il polinomio
identicamente nullo $0_{R[x]} = 0_R$ e unita' il polinomio costante $1_{R[x]} = 1_R$.\\
Definiamo ora un imporante funzione che descrive un polinomio.

%%%%%%%%%%%%%%%
\newpage
%%%%%%%%%%%%%%%
\begin{defn}[Funzione grado; grado di un polinomio]{grado polinomio}
Sia $R$ un anello commutativo e sia $f(x) = f \in R[x]$ definita come in precedenza. Allora definiamo
la funzione grado:
\begin{equation}
  deg^*(f) := \left\{\begin{array}{lr}
      \max\{k \in \N : a_k \neq 0_R\}, & \text{ se } f(x) \not\equiv 0_R\\
      -\infty, & \text{ se } f(x) \equiv 0_R 
      \end{array}\right\} 
\end{equation}
e il risultato di $deg^*(f)$ come il \emph{grado} del polinomio $f$. Se $deg^*(f) = 0$ si dice che $f$
e' un polinomio \emph{costante}.
\end{defn}
Tale definizinone coincide con quella classica di grado di un polinomio tranne
nel caso in cui $f(x)$ sia identicamente nullo. 

\begin{example}[Grado di un polinomio]{exmp grad pol}
Se consideriamo i polinomi $f(x) = x^2 + 1, g(x) = 1$ e $h(x) = 0, f,g,h \in \Z[x]$, si ha che
\[deg*(f) = 2,\ deg*(g) = 0,\ deg*(h) = -\infty.\]
\end{example}
Per calcolare il grado di un prodotto di un polinomio e' necessario aritmetizzare
alcuni simboli.

\begin{defn}[$\N_k$, somma in $\N_k \cup \{-\infty\}$]{N_k}
Poniamo $\N_k := \{z \in \Z \mid z \geq k \in \Z\}$; ad esempio, $\N_0 = \{0, 1, 2, \dots\}$.\\
Definiamo la \emph{somma} in $\N_k \cup \{-\infty\}$

\[+ : \N_k \cup \{-\infty\} \times \N_k \cup \{-\infty\} \to \N_k \cup \{-\infty\},\]
dove per $n, m \in N_k$, $n+m$ coincide con la somma definita su $\Z$, e

\begin{equation}\label{sum_N_k}
n + (-\infty) = -\infty + n := -\infty
\end{equation}
per ogni $n \in N_k$
\end{defn}
Riportiamo ora una definizione precedente discussa nel corso di Algebra I

\begin{defn}[Dominio d'integrita']{dom integrita'}
Un \emph{dominio d'integrita'} e' un anello commutativo $R$ che soddisfa:
\begin{enumerate}
  \item $1_R \neq 0_R$,
  \item $\forall a,b\ (a, b \in R \land a\cdot b = 0 \Rightarrow (a=0 \lor b=0))$.
\end{enumerate}
La seconda condizione e' equivalente ad affermare che $R$ e' privo di divisori dello zero.
\end{defn}
Riportiamo alcuni fatti che mettono in relazioni i domini di integrita con i polinomi.

\begin{prop}[]{R dom R[x] e grado}
Sia $R$ un dominio di integrita'. Allora per ogni $f,g \in R[x]$ vale

\begin{equation}
deg^*(f\cdot g) = deg^*(f) + deg^*(g).
\end{equation}

\end{prop}
\begin{dimostrazione}
Se $f$ oppure $g$ e' il poliniomio nullo, allora l'uguaglianza e' verificata in seguito a (\ref{sum_N_k}). 
Supponendo che
\[f = \sum_{i=0}^{n}r_ix^i,\ g = \sum_{i=0}^{m}s_ix^i \neq 0_{R[x]}\]
dove $r_n,s_m \neq 0_R$, e poiche' $R$ e' un dominio di integrita', allora
$r_ns_m \cdot x^{n+m}$ e' il monomio di grado massimo presente nel prodotto $f\cdot g$, quindi

\[deg^*(f \cdot g) = n + m = deg^*(f) + deg^*(g)\]
\end{dimostrazione}

\begin{prop}[]{R dom allora R[x] dom}
Se $R$ un dominio di integrita', allora lo e' anche $R[x]$.
\end{prop}
\begin{dimostrazione}
Devono essere soddisfatti i due assiomi di dominio di integrita'.
Il primo deriva da $1_{R[x]} \equiv 1_R \neq 0_R \equiv 0_{R[x]}$.
Per il secondo assioma siano $f,g \in R[x]$ tali che $f\cdot g = 0$. Allora 
\[deg^*(f\cdot g) = -\infty \Longrightarrow (deg^*(f) = -\infty \land f=0_{R[x]}) \lor (deg^*(g) = -\infty \land g=0_{R[x]}) \]
\end{dimostrazione}
Grazie alle precedenti proposizioni, denotiamo il gruppo dei polinomi invertibili

\begin{defn}[Gruppo dei polinomi invertibili]{R[x]^x}
Sia $R$ un dominio di integrita'. Allora possiamo definire
\[R[x]^{\times} = \{ f \in R[x] \mid \exists g\ (g\in R[x] \land f\cdot g = 1_{R[x]}) \} ,\]
ovvero il \emph{gruppo degli elementi invertibili} in $R[x]$.
\end{defn}

\begin{prop}[]{R[x]^x = R^x}
Sia $R$ un dominio di integrita'. Allora $R[x]^{\times} = R^{\times}$
\end{prop}
\begin{dimostrazione}
Siano $f \in R[x]^{\times}, g \in R[x]$ tali che $f\cdot g = 1_{R[x]}$. 
Allora dalla proposizione \ref{R dom R[x] e grado} segue che
\[ deg^*(f) + deg^*(g) = 0 \land deg^*(g) = 0 \Longrightarrow deg^*(f) = 0 \Longrightarrow f\in R[x]\]
\end{dimostrazione}



