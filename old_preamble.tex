% math packages
\usepackage{mathtools}
\usepackage{amsmath}
\usepackage{amssymb}    % Math symbols such as \mathbb
\usepackage{amsthm}


% other packages
\usepackage{lipsum}
\usepackage{graphicx}
\graphicspath{ {../assets/} }
\usepackage{enumitem}
\usepackage{color}
\usepackage[a4paper, total={6in, 8in}]{geometry}
\usepackage{hyperref}
\hypersetup{
    colorlinks=true,
    linktoc=all,     %set to all if you want both sections and subsections linked
    linkcolor=blue,
}

\usepackage{microtype}
\usepackage[framemethod=TikZ]{mdframed}

% proper inline math display, adjust height for symbols like \sum
\everymath{\displaystyle}

% define tags for math use..
\theoremstyle{plain}% default
\newtheorem{theorem}{Teorema}[subsection]
\newtheorem{corollary}{Corollario}
\newtheorem{exercise}[theorem]{Esercizio}


\theoremstyle{definition}
\newtheorem{defn}{Definizione}[subsection]
\newtheorem{proposition}{Proposizione}[subsection]
\newtheorem{example}[theorem]{Example}
\theoremstyle{remark}
\newtheorem*{rem}{Remark}
\newtheorem*{note}{Note}
\newtheorem{case}{Case}

% Gives begin{solution} same formating as \begin{proof}
\newenvironment{solution}
  {\begin{proof}[Solution]}
  {\end{proof}}


%running fraction with slash - requires math mode.
\newcommand*\rfrac[2]{{}^{#1}\!/_{#2}}
%shortcut to mathbb
\newcommand{\N}{\mathbb{N}}
\newcommand{\Z}{\mathbb{Z}}
\newcommand{\Q}{\mathbb{Q}}
\newcommand{\R}{\mathbb{R}}
\newcommand{\I}{\mathbb{I}}
\newcommand{\C}{\mathbb{C}}
% color highlighting
\newcommand{\hilight}[1]{\colorbox{yellow}{#1}}



\newcommand{\icol}[1]{% inline column vector
  \left(\begin{smallmatrix}#1\end{smallmatrix}\right)%
}

\newcommand{\irow}[1]{% inline row vector
  \begin{smallmatrix}(#1)\end{smallmatrix}%
}