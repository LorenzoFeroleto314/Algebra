\documentclass{article}

\usepackage[utf8]{inputenc}
\usepackage[english]{babel}
\usepackage{amsmath}
\usepackage{amssymb}
\usepackage{amsthm}
\usepackage{yhmath}
\usepackage{gensymb}
\usepackage{graphicx}
\usepackage{siunitx}
\usepackage{amscd}
\usepackage{sectsty}
\usepackage{stmaryrd}
\usepackage{tikz-cd}
\usepackage{wrapfig}
\usepackage{xcolor}
\usepackage[margin=1.5in]{geometry}
\usepackage[framemethod=TikZ]{mdframed}

\theoremstyle{definition}
\newtheorem{thm}{Teorema}[subsection]
\newtheorem*{exm}{Esempio}
\renewcommand\qedsymbol{$\blacksquare$}
\addto\captionsenglish{\renewcommand*{\proofname}{Dimostrazione}}
\addto\captionsenglish{\renewcommand{\contentsname}{Indice}}

\sectionfont{\fontsize{20}{15}\selectfont}
\subsectionfont{\fontsize{14}{15}\selectfont}

\renewcommand\thefootnote{\textcolor{red}{\arabic{footnote}}}

\newcommand{\quot}{\operatorname{quot}}
\newcommand{\tor}{\operatorname{tor}}
\newcommand{\sat}{\operatorname{sat}}
\newcommand{\Ann}{\operatorname{Ann}}
\newcommand{\spec}{\operatorname{spec}}
\newcommand{\id}{\operatorname{id}}

\newenvironment{teo}[2][]{%
\ifstrempty{#1}%
{\mdfsetup{%
frametitle={%
\tikz[baseline=(current bounding box.east),outer sep=0pt]
\node[anchor=east,rectangle,fill=blue!25]
{\strut Teorema};}}
}%
{\mdfsetup{%
frametitle={%
\tikz[baseline=(current bounding box.east),outer sep=0pt]
\node[anchor=east,rectangle,fill=blue!25]
{\strut Teorema~#1};}}%
}%
\mdfsetup{innertopmargin=1.5pt,linecolor=blue!25,%
linewidth=1.75pt,topline=true,%
frametitleaboveskip=\dimexpr-\ht\strutbox\relax
}
\begin{mdframed}[]\relax%
}{\end{mdframed}}

\newenvironment{prop}[2][]{%
\ifstrempty{#1}%
{\mdfsetup{%
frametitle={%
\tikz[baseline=(current bounding box.east),outer sep=0pt]
\node[anchor=east,rectangle,fill=blue!25]
{\strut Proposizione};}}
}%
{\mdfsetup{%
frametitle={%
\tikz[baseline=(current bounding box.east),outer sep=0pt]
\node[anchor=east,rectangle,fill=blue!25]
{\strut Proposizione~#1};}}%
}%
\mdfsetup{innertopmargin=1.5pt,linecolor=blue!25,%
linewidth=1.75pt,topline=true,%
frametitleaboveskip=\dimexpr-\ht\strutbox\relax
}
\begin{mdframed}[]\relax%
}{\end{mdframed}}

\newenvironment{cor}[2][]{%
\ifstrempty{#1}%
{\mdfsetup{%
frametitle={%
\tikz[baseline=(current bounding box.east),outer sep=0pt]
\node[anchor=east,rectangle,fill=blue!25]
{\strut Corollario};}}
}%
{\mdfsetup{%
frametitle={%
\tikz[baseline=(current bounding box.east),outer sep=0pt]
\node[anchor=east,rectangle,fill=blue!25]
{\strut Corollario~#1};}}%
}%
\mdfsetup{innertopmargin=1.5pt,linecolor=blue!25,%
linewidth=1.75pt,topline=true,%
frametitleaboveskip=\dimexpr-\ht\strutbox\relax
}
\begin{mdframed}[]\relax%
}{\end{mdframed}}

\newenvironment{lem}[2][]{%
\ifstrempty{#1}%
{\mdfsetup{%
frametitle={%
\tikz[baseline=(current bounding box.east),outer sep=0pt]
\node[anchor=east,rectangle,fill=blue!25]
{\strut Lemma};}}
}%
{\mdfsetup{%
frametitle={%
\tikz[baseline=(current bounding box.east),outer sep=0pt]
\node[anchor=east,rectangle,fill=blue!25]
{\strut Lemma~#1};}}%
}%
\mdfsetup{innertopmargin=1.5pt,linecolor=blue!25,%
linewidth=1.75pt,topline=true,%
frametitleaboveskip=\dimexpr-\ht\strutbox\relax
}
\begin{mdframed}[]\relax%
}{\end{mdframed}}

\newenvironment{defn}[2][]{%
\ifstrempty{#1}%
{\mdfsetup{%
frametitle={%
\tikz[baseline=(current bounding box.east),outer sep=0pt]
\node[anchor=east,rectangle,fill=green!35]
{\strut Definizione};}}
}%
{\mdfsetup{%
frametitle={%
\tikz[baseline=(current bounding box.east),outer sep=0pt]
\node[anchor=east,rectangle,fill=green!35]
{\strut Definizione:~#1};}}%
}%
\mdfsetup{innertopmargin=1.5pt,linecolor=green!35,%
linewidth=1.75pt,topline=true,%
frametitleaboveskip=\dimexpr-\ht\strutbox\relax
}
\begin{mdframed}[]\relax%
}{\end{mdframed}}

%%%%%%%%%%%%%%%%%%%%%%%%%%%%%%%%%%%%%%%%%%%%%%%%%%%%%%%%%%%%%%%%%%%%%

\begin{document}

\subsection{Estensioni finite}

\noindent \textbf{Lezioni del 05-06/11/2019} (appunti grezzi)

\begin{defn}[]{}
Un'estensione di campi $\mathbb{L}/\mathbb{K}$ si dice finita se $|\mathbb{L}:\mathbb{K}|<\infty$.
\end{defn}

\begin{prop}[]{}
Sia $\mathbb{L}/\mathbb{K}$ un'estensione finita. Allora, ogni elemento $a\in \mathbb{L}$ è algebrico su $\mathbb{K}$.
\end{prop}
\vspace{-4mm}
\begin{proof}
Sia $\phi_a\colon \mathbb{K}[x]\to \mathbb{L}$ la valutazione in $a$. Poiché $\dim_{\mathbb{K}}\mathbb{K}[x]=\infty$ (una base sono tutti i monomi $1, x, x^2, ...$) e $|\mathbb{L}:\mathbb{K}|=\dim_{K}(L)<\infty$, $\phi_a$ non è iniettiva, cioè $\ker(\phi_a)\neq \{0_K\}$. Dunque, per $f\in \ker(\phi_a)\setminus \{0_K\}$, si ha $f(a)=\phi_a(f)=0$.
\end{proof}

\begin{prop}[]{}
Sia $\mathbb{L}/\mathbb{K}$ un'estensione di campi, e sia $a\in \mathbb{L}$. Allora, sono equivalenti

\noindent (i) $a$ è algebrico su $\mathbb{K}$

\noindent (ii) $|\mathbb{K}(a)\colon \mathbb{K}|<\infty$

\noindent (iii) esiste un'estensione finita $M/K$, $M\subseteq L$ tale che $a\in M$
\end{prop}
\vspace{-4mm}
\begin{proof}
Per la proposizione precedente, sappiamo già che (iii) implica (i). Vediamo che (i) implica (ii). Infatti, $K[a]=\operatorname{Im}(\phi_a)\simeq K[x]/\ker(\phi_a)$ è un campo, e $K(a)=K[a]$ implica che $|K[a]\colon K|=|K[a] \colon K|=\deg^{\star}(\min_{a,K})<\infty$. Mostriamo ora che non (i) implica non (ii). Infatti, non (i) sse a è trascendente su $K$. Quindi, $\phi_a\colon K[x]\to L$ è iniettiva, e $K(a)\supseteq im(\phi_a)\simeq K[x]$ perché $K(a)$ contiene il sottospazio vettoriale $im(\phi_a)$ e $\dim_{K}(K(a))=\infty$. Dunque, il fatto che (i) implica (ii) e non (i) implica non (ii), sappiamo che (i) se e solo se (ii). Ma (ii) implica (iii) è banale: infatti prendo $M=K(a)$.
\end{proof}

\begin{defn}[]{}
Sia $\mathbb{L}/\mathbb{K}$ un'estensione di campi. Denotiamo con $\operatorname{alg}_{\mathbb{K}}(\mathbb{L})$ l'insieme degli elementi $a\in L$ algebrici su $K$.
\end{defn}

\begin{prop}[]{}
Sia $\mathbb{L}/\mathbb{K}$ un'estensione di campi. Allora, $\operatorname{alg}_{\mathbb{K}}(\mathbb{L})$ è un sottocampo di $\mathbb{L}$.
\end{prop}
\vspace{-4mm}
\begin{proof}
Siano $a,b\in \operatorname{alg}_{\mathbb{K}}(\mathbb{L})$. Basta dimostrare che $a+b, ab$ e $a^{-1}$ stanno in $\operatorname{alg}_{\mathbb{K}}(\mathbb{L})$. Poiché $a\in \operatorname{alg}_{\mathbb{K}}(\mathbb{L})$, per la proposizione 2 sappiamo che $|K(a):K|<\infty$. Poiché $b\in \operatorname{alg}_{\mathbb{K}}(\mathbb{L})$, esiste $\min_{b,K}(x)\in K[x]\subseteq K(a)[x]$. Dunque, $b$ è algebrico su $K(a)$, da cui \[|K(\{a,b\}): K|=|K(a)(b): K(a)|\cdot |K(a): K|<\infty\] per la Formula del grado. Poiché $a+b, ab, a^{-1}\in K(\{a,b\})$,per la Proposizione 2 sappiamo che $a+b, ab, a^{-1}\in \operatorname{alg}_{\mathbb{K}}(\mathbb{L})$.
\end{proof}

\noindent Trovare esplicitamente i polinomi che annullano $a+b, ab$ e $a^{-1}$ sarebbe stato un incubo!

\begin{defn}[]{}
Un campo $\mathbb{K}$ si dice algebricamente chiuso se ogni polinomio $f\in \mathbb{K}[x]$ con $\deg^{\star}(f)\geq 1$ ammette una radice.
\end{defn}

\begin{exm}Per il Teorema Fondamentale dell'Algebra (lui dice Teorema di Gauss) sappiamo che $\mathbb{C}$ è un campo algebricamente chiuso. La dimostrazione è tutt'altro che banale e richiede o l'analisi complessa o la Teoria di Galois$. \ \square$\end{exm}

\noindent Denotiamo con $\overline{\mathbb{Q}}=\operatorname{alg}_{\mathbb{Q}}(\mathbb{C})$. 

\begin{prop}[]{}
$\overline{\mathbb{Q}}$ è un campo algebricamente chiuso.
\end{prop}
\vspace{-4mm}
\begin{proof}
Sia $f=\sum\limits_{i=0}^n a_i x^i\in \overline{\mathbb{Q}}[x]$ con $\deg^{\star}(f)\geq 1$. Poiché $f\in C[x]$ essendo $\overline{\mathbb{Q}}\subseteq \mathbb{C}$, per il Teorema Fondamentale dell'Algebra esiste $c\in \mathbb{C}$ tale che $f(c)=0$. Definiamo $M=\overline{\mathbb{Q}}(\{a_0, a_1, ...\,,a_n\})$. Allora, per la formula del grado \[|M:Q|=|M:Q(\{a_0, ...\,,a_{n-1}\})|\cdot |Q(\{a_0, ...\,,a_{n-1}\}): Q(\{a_0, ...\,,a_{n-2}\})|\cdot ...\] Ma sappiamo che $|M:Q(\{a_0, ...\,,a_{n-1}\})|\leq \deg^{\star}(\min_{a_n,Q})$ e induttivamente $|M:Q|\leq \prod\limits_{i=0}^n \deg^{\star}(\min_{a_i, Q})$. Quindi, $|M(c):Q|=|M(c):M|\cdot |M:Q|$, dove $|M(c):M|\leq n$ e $|M:Q|\leq \infty$. Dunque, per la Proposizione 2 concludiamo che $c\in \overline{\mathbb{Q}}$.
\end{proof}

\begin{prop}[2.X.Y: Costruzione di Kronecker]{}
Sia $\mathbb{K}$ un campo e sia $f\in\mathbb{K}[x]$ con $\deg^{\star}(f)\geq 1$. Allora, esiste un'estensione $\mathbb{L}/\mathbb{K}_0$ finita e un elemento $a\in \mathbb{L}$ tale che $f(a)=0$, dove $\mathbb{K}_0\simeq \mathbb{K}$.
\end{prop}
\vspace{-4mm}
\begin{proof}
Poiché $K[x]$ è un dominio principale, possiamo scrivere $f=h\cdot f_0$ dove $h\in K[x]$ è primo e dunque irriducibile. Definiamo $L=K[x]/\langle h\rangle$. Poiché $\langle h\rangle \lhd K[x]$ è un ideale massimale, tale $L$ è un campo. Definiamo $K_0=\{b+\langle h\rangle : b\in K\}$, cioè $K_0=\pi(K)$, dove $\pi\colon K[x]\to L$ è la proiezione canonica. Poiché $\langle h\rangle$ è un ideale primo, $K\cap \langle h\rangle \{0_K\}$, quindi la restrizione $\pi_K\colon K\to K_0$ è un isomorfismo. Inoltre, $|L:K_0|=\deg^{\star}(h)<\deg^{\star}(f)<\infty$, quindi abbiamo trovato un'estensione finita. Sia $h=\sum\limits_{k=0}^n a_k x^k$, e sia $I=\langle h \rangle$. Detto $a=x+I\in L$, si ha che \[h(a)=\sum\limits_{k=0}^n a_k (x+I)^k=\sum\limits_{k=0}^n a_k (x^k+I)=\left(\sum\limits_{k=0}^n a_k x^k\right)+I=h+I=I=O_{L}.\] Questo mostra che per ogni polinomio troviamo $a\in L$ tale che $f(a)=0$, come desiderato.\end{proof}

\end{document}