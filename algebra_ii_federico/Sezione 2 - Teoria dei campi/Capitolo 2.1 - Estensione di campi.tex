\documentclass{article}

\usepackage[utf8]{inputenc}
\usepackage[english]{babel}
\usepackage{amsmath}
\usepackage{amssymb}
\usepackage{amsthm}
\usepackage{yhmath}
\usepackage{gensymb}
\usepackage{graphicx}
\usepackage{siunitx}
\usepackage{amscd}
\usepackage{sectsty}
\usepackage{stmaryrd}
\usepackage{tikz-cd}
\usepackage{wrapfig}
\usepackage{xcolor}
\usepackage[margin=1.5in]{geometry}
\usepackage[framemethod=TikZ]{mdframed}

\theoremstyle{definition}
\newtheorem{thm}{Teorema}[subsection]
\newtheorem*{exm}{Esempio}
\renewcommand\qedsymbol{$\blacksquare$}
\addto\captionsenglish{\renewcommand*{\proofname}{Dimostrazione}}
\addto\captionsenglish{\renewcommand{\contentsname}{Indice}}

\sectionfont{\fontsize{20}{15}\selectfont}
\subsectionfont{\fontsize{14}{15}\selectfont}

\renewcommand\thefootnote{\textcolor{red}{\arabic{footnote}}}

\newcommand{\quot}{\operatorname{quot}}
\newcommand{\tor}{\operatorname{tor}}
\newcommand{\sat}{\operatorname{sat}}
\newcommand{\Ann}{\operatorname{Ann}}
\newcommand{\spec}{\operatorname{spec}}
\newcommand{\id}{\operatorname{id}}

\newenvironment{teo}[2][]{%
\ifstrempty{#1}%
{\mdfsetup{%
frametitle={%
\tikz[baseline=(current bounding box.east),outer sep=0pt]
\node[anchor=east,rectangle,fill=blue!25]
{\strut Teorema};}}
}%
{\mdfsetup{%
frametitle={%
\tikz[baseline=(current bounding box.east),outer sep=0pt]
\node[anchor=east,rectangle,fill=blue!25]
{\strut Teorema~#1};}}%
}%
\mdfsetup{innertopmargin=1.5pt,linecolor=blue!25,%
linewidth=1.75pt,topline=true,%
frametitleaboveskip=\dimexpr-\ht\strutbox\relax
}
\begin{mdframed}[]\relax%
}{\end{mdframed}}

\newenvironment{prop}[2][]{%
\ifstrempty{#1}%
{\mdfsetup{%
frametitle={%
\tikz[baseline=(current bounding box.east),outer sep=0pt]
\node[anchor=east,rectangle,fill=blue!25]
{\strut Proposizione};}}
}%
{\mdfsetup{%
frametitle={%
\tikz[baseline=(current bounding box.east),outer sep=0pt]
\node[anchor=east,rectangle,fill=blue!25]
{\strut Proposizione~#1};}}%
}%
\mdfsetup{innertopmargin=1.5pt,linecolor=blue!25,%
linewidth=1.75pt,topline=true,%
frametitleaboveskip=\dimexpr-\ht\strutbox\relax
}
\begin{mdframed}[]\relax%
}{\end{mdframed}}

\newenvironment{cor}[2][]{%
\ifstrempty{#1}%
{\mdfsetup{%
frametitle={%
\tikz[baseline=(current bounding box.east),outer sep=0pt]
\node[anchor=east,rectangle,fill=blue!25]
{\strut Corollario};}}
}%
{\mdfsetup{%
frametitle={%
\tikz[baseline=(current bounding box.east),outer sep=0pt]
\node[anchor=east,rectangle,fill=blue!25]
{\strut Corollario~#1};}}%
}%
\mdfsetup{innertopmargin=1.5pt,linecolor=blue!25,%
linewidth=1.75pt,topline=true,%
frametitleaboveskip=\dimexpr-\ht\strutbox\relax
}
\begin{mdframed}[]\relax%
}{\end{mdframed}}

\newenvironment{lem}[2][]{%
\ifstrempty{#1}%
{\mdfsetup{%
frametitle={%
\tikz[baseline=(current bounding box.east),outer sep=0pt]
\node[anchor=east,rectangle,fill=blue!25]
{\strut Lemma};}}
}%
{\mdfsetup{%
frametitle={%
\tikz[baseline=(current bounding box.east),outer sep=0pt]
\node[anchor=east,rectangle,fill=blue!25]
{\strut Lemma~#1};}}%
}%
\mdfsetup{innertopmargin=1.5pt,linecolor=blue!25,%
linewidth=1.75pt,topline=true,%
frametitleaboveskip=\dimexpr-\ht\strutbox\relax
}
\begin{mdframed}[]\relax%
}{\end{mdframed}}

\newenvironment{defn}[2][]{%
\ifstrempty{#1}%
{\mdfsetup{%
frametitle={%
\tikz[baseline=(current bounding box.east),outer sep=0pt]
\node[anchor=east,rectangle,fill=green!35]
{\strut Definizione};}}
}%
{\mdfsetup{%
frametitle={%
\tikz[baseline=(current bounding box.east),outer sep=0pt]
\node[anchor=east,rectangle,fill=green!35]
{\strut Definizione:~#1};}}%
}%
\mdfsetup{innertopmargin=1.5pt,linecolor=green!35,%
linewidth=1.75pt,topline=true,%
frametitleaboveskip=\dimexpr-\ht\strutbox\relax
}
\begin{mdframed}[]\relax%
}{\end{mdframed}}

%%%%%%%%%%%%%%%%%%%%%%%%%%%%%%%%%%%%%%%%%%%%%%%%%%%%%%%%%%%%%%%%%%%%%

\begin{document}

\section{Teoria dei campi}
\vspace{1.75mm}
\subsection{Estensione di campi}

\noindent Introduciamo ora un concetto fondamentale nella teoria algebrica dei numeri e nello studio delle radici polinomiali, che costituirà la base della teoria di Galois.

\begin{defn}[]{}
Una coppia di campi $\mathbb{K}$ e $\mathbb{L}$ con $\mathbb{K}\subseteq \mathbb{L}$ si dice \underline{estensione di campi} e si denota con $\mathbb{L}/\mathbb{K}$.
\end{defn}

\noindent Resta inteso che $\mathbb{K}$ ha le stesse operazioni binarie di $\mathbb{L}$, cioè che $\mathbb{K}$ è un sottocampo di $\mathbb{L}$. Inoltre, in questo caso la notazione $\mathbb{L}/\mathbb{K}$ non ha nulla a che vedere con il quoziente di campi.

\begin{exm}Se consideriamo $\mathbb{R}$ e $\mathbb{C}$ con le usuali operazioni di somma e prodotto, $\mathbb{R}$ è un sottocampo di $\mathbb{C}$, dunque $\mathbb{C}/\mathbb{R}$ è un'estensione di campi$. \ \square$\end{exm}

\noindent Se $\mathbb{L}/\mathbb{K}$ è un'estensione di campi, sia $\cdot \raisebox{-.5em}{$\vert_{\mathbb{K}\times\mathbb{L}}$}$ la restrizione a $\mathbb{K}$ della prima componente del prodotto $\cdot \, \colon \mathbb{L}\times\mathbb{L}\to\mathbb{L}$ del campo $\mathbb{L}$. Considerando tale moltiplicazione per gli elementi di $\mathbb{K}$ e la usuale somma di $\mathbb{L}$, si ha che $(\mathbb{L},+,\cdot)$ ha la struttura di uno spazio vettoriale su $\mathbb{K}$. Infatti, possiamo pensare gli elementi di $\mathbb{K}$ come scalari e quelli di $\mathbb{L}$ come vettori.

\begin{defn}[]{}
Sia $\mathbb{L}/\mathbb{K}$ un'estensione di campi. Definiamo \underline{grado dell'estensione} $\mathbb{L}/\mathbb{K}$ la dimensione\footnotemark \ $\operatorname{dim}_{\mathbb{K}}(\mathbb{L})\in \mathbb{N}\cup \{\infty\}$ dello spazio vettoriale $\mathbb{L}$ sul campo $\mathbb{K}$, e si denota con $|\mathbb{L} : \mathbb{K}|$.
\end{defn}\footnotetext{Ricordiamo che la dimensione di uno spazio vettoriale è la cardinalità di una sua base, cioè un insieme di vettori linearmente indipendenti che generano tutto lo spazio.} 

\noindent La scelta del termine ``grado'', che richiama il concetto di grado di un polinomio, sarà più chiara in seguito, quando approfondiremo i legami tra estensione di campi e polinomi.
 
\begin{exm}Se consideriamo $\mathbb{Q}$, $\mathbb{R}$ e $\mathbb{C}$ con le usuali operazioni di somma e prodotto, si ha che $|\mathbb{C}: \mathbb{R}|=2$ perché $\mathcal{B}=\{1, i\}$ è una base per $\mathbb{C}$, e $|\mathbb{R}:\mathbb{Q}|=\infty$ perché $\mathbb{R}$ non è numerabile, quindi non ammette una base finita su $\mathbb{Q}$, che invece è numerabile$. \ \square$\end{exm}

\vspace{-2mm}

\begin{defn}[]{}
Sia $\mathbb{L}/\mathbb{K}$ un'estensione di campi. Un elemento $a\in \mathbb{L}$ si dice:

\noindent (i) \underline{algebrico} su $\mathbb{K}$ se esiste un polinomio non nullo $f(x)\in \mathbb{K}[x]$ tale che $f(a)=0$;

\noindent (ii) \underline{trascendente} su $\mathbb{K}$ se non è algebrico.
\end{defn}

\begin{exm}Se consideriamo $\mathbb{R}/\mathbb{Q}$, l'elemento $a=\sqrt{2}$ è algebrico perché $f(x)=x^2-2\in \mathbb{Q}[x]$ e $f(a)=0$, mentre $e$ e $\pi$ sono entrambi elementi trascendenti.\footnote{La dimostrazione è tutt'altro che elementare e prende il nome di \emph{Teorema di Lindemann-Weierstrass}.}$\, \square$\end{exm}

\noindent Sia $\mathbb{L}/\mathbb{K}$ un'estensione di campi e sia $a\in \mathbb{L}$. Detta $\phi_a\colon \mathbb{K}[x]\to \mathbb{L}$ la valutazione in $a$, essendo $\phi_a$ un omomorfismo si ha che $\ker(\phi_a)\lhd \mathbb{K}[x]$. SISTEMARE TUTTO.

\clearpage

\begin{defn}[]{}
Sia $\mathbb{L}/\mathbb{K}$ un'estensione di campi e sia $a\in \mathbb{L}$ un elemento algebrico su $\mathbb{K}$. Il generatore monico di $\ker(\phi_a)$ è detto \underline{polinomio minimo di $a$} e si denota con $\min_{a,\mathbb{K}}(x)\in \mathbb{K}[x]$.
\end{defn}

\begin{prop}[]{}
Sia $\mathbb{L}/\mathbb{K}$ un'estensione di campi e sia $a\in \mathbb{L}$ algebrico su $\mathbb{K}$. Sia $f(x)\in \mathbb{K}[x]$ tale che:

\noindent (i) $f(a)=0$;

\noindent (ii) $f(x)$ è monico;

\noindent (iii) $f(x)$ è irrudicibile.

\noindent Allora, $f(x)$ è il polinomio minimo di $a$, cioè $f(x)=\min_{a,\mathbb{K}}(x)$.
\end{prop}
\vspace{-4mm}
\begin{proof}
Per (i) si ha che $f(x)\in\ker(\phi_a)$, dunque esiste un polinomio $q(x)\in \mathbb{K}[x]$ tale che $f(x)=q(x)\cdot \min_{a,\mathbb{K}}(x)$. Essendo $f(x)$ irriducibile per (iii), almeno uno fra $q(x)$ e $\min_{a,\mathbb{K}}(x)$ è invertibile; tuttavia, $\min_{a,\mathbb{K}}(x)\not\in\mathbb{K}[x]^{\times}$ e quindi CONCLUDERE
\end{proof}

\noindent Sia $\mathbb{L}/\mathbb{K}$ un'estensione di campi e sia $S\subseteq \mathbb{L}$ un sottoinsieme.

\begin{prop}[]{}
Sia $\mathbb{L}/\mathbb{K}$ un'estensione di campi e sia $a\in \mathbb{L}$ algebrico su $\mathbb{K}$. Allora, $\mathbb{K}(a)=\mathbb{K}[a]$.
\end{prop}
\vspace{-4mm}
\begin{proof}
Sia $f(x)=\sum\limits_{i=0}^n c_i x^i\in \mathbb{K}[x]$. Poiché $c_i \in \mathbb{K}\subseteq \mathbb{K}(a)$ e $a\in \mathbb{K}(a)\Rightarrow a^k\in \mathbb{K}(a)$ essendo $\mathbb{K}(a)$ chiuso rispetto al prodotto, $f(a)\in \mathbb{K}(a)$. Dunque, per l'arbitrarietà di $f(x)$ concludiamo che $\operatorname{Im}(\phi_a)=\mathbb{K}[a]\subseteq \mathbb{K}(a)$. FINIRE, ESERCIZIO PER CASA XD COME SEI SIMPATICO
\end{proof}

\

\noindent Manca anche la lezione del 30/10/2019, al momento è solo cartacea, e contiene cose davvero molto importanti tipo la formula del grado.

\end{document}