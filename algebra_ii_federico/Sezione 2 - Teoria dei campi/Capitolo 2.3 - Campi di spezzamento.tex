\documentclass{article}

\usepackage[utf8]{inputenc}
\usepackage[english]{babel}
\usepackage{amsmath}
\usepackage{amssymb}
\usepackage{amsthm}
\usepackage{yhmath}
\usepackage{gensymb}
\usepackage{graphicx}
\usepackage{siunitx}
\usepackage{amscd}
\usepackage{sectsty}
\usepackage{stmaryrd}
\usepackage{tikz-cd}
\usepackage{wrapfig}
\usepackage{xcolor}
\usepackage[margin=1.5in]{geometry}
\usepackage[framemethod=TikZ]{mdframed}

\theoremstyle{definition}
\newtheorem{thm}{Teorema}[subsection]
\newtheorem*{exm}{Esempio}
\renewcommand\qedsymbol{$\blacksquare$}
\addto\captionsenglish{\renewcommand*{\proofname}{Dimostrazione}}
\addto\captionsenglish{\renewcommand{\contentsname}{Indice}}

\sectionfont{\fontsize{20}{15}\selectfont}
\subsectionfont{\fontsize{14}{15}\selectfont}

\renewcommand\thefootnote{\textcolor{red}{\arabic{footnote}}}

\newcommand{\quot}{\operatorname{quot}}
\newcommand{\tor}{\operatorname{tor}}
\newcommand{\sat}{\operatorname{sat}}
\newcommand{\Ann}{\operatorname{Ann}}
\newcommand{\spec}{\operatorname{spec}}
\newcommand{\id}{\operatorname{id}}

\newenvironment{teo}[2][]{%
\ifstrempty{#1}%
{\mdfsetup{%
frametitle={%
\tikz[baseline=(current bounding box.east),outer sep=0pt]
\node[anchor=east,rectangle,fill=blue!25]
{\strut Teorema};}}
}%
{\mdfsetup{%
frametitle={%
\tikz[baseline=(current bounding box.east),outer sep=0pt]
\node[anchor=east,rectangle,fill=blue!25]
{\strut Teorema~#1};}}%
}%
\mdfsetup{innertopmargin=1.5pt,linecolor=blue!25,%
linewidth=1.75pt,topline=true,%
frametitleaboveskip=\dimexpr-\ht\strutbox\relax
}
\begin{mdframed}[]\relax%
}{\end{mdframed}}

\newenvironment{prop}[2][]{%
\ifstrempty{#1}%
{\mdfsetup{%
frametitle={%
\tikz[baseline=(current bounding box.east),outer sep=0pt]
\node[anchor=east,rectangle,fill=blue!25]
{\strut Proposizione};}}
}%
{\mdfsetup{%
frametitle={%
\tikz[baseline=(current bounding box.east),outer sep=0pt]
\node[anchor=east,rectangle,fill=blue!25]
{\strut Proposizione~#1};}}%
}%
\mdfsetup{innertopmargin=1.5pt,linecolor=blue!25,%
linewidth=1.75pt,topline=true,%
frametitleaboveskip=\dimexpr-\ht\strutbox\relax
}
\begin{mdframed}[]\relax%
}{\end{mdframed}}

\newenvironment{cor}[2][]{%
\ifstrempty{#1}%
{\mdfsetup{%
frametitle={%
\tikz[baseline=(current bounding box.east),outer sep=0pt]
\node[anchor=east,rectangle,fill=blue!25]
{\strut Corollario};}}
}%
{\mdfsetup{%
frametitle={%
\tikz[baseline=(current bounding box.east),outer sep=0pt]
\node[anchor=east,rectangle,fill=blue!25]
{\strut Corollario~#1};}}%
}%
\mdfsetup{innertopmargin=1.5pt,linecolor=blue!25,%
linewidth=1.75pt,topline=true,%
frametitleaboveskip=\dimexpr-\ht\strutbox\relax
}
\begin{mdframed}[]\relax%
}{\end{mdframed}}

\newenvironment{lem}[2][]{%
\ifstrempty{#1}%
{\mdfsetup{%
frametitle={%
\tikz[baseline=(current bounding box.east),outer sep=0pt]
\node[anchor=east,rectangle,fill=blue!25]
{\strut Lemma};}}
}%
{\mdfsetup{%
frametitle={%
\tikz[baseline=(current bounding box.east),outer sep=0pt]
\node[anchor=east,rectangle,fill=blue!25]
{\strut Lemma~#1};}}%
}%
\mdfsetup{innertopmargin=1.5pt,linecolor=blue!25,%
linewidth=1.75pt,topline=true,%
frametitleaboveskip=\dimexpr-\ht\strutbox\relax
}
\begin{mdframed}[]\relax%
}{\end{mdframed}}

\newenvironment{defn}[2][]{%
\ifstrempty{#1}%
{\mdfsetup{%
frametitle={%
\tikz[baseline=(current bounding box.east),outer sep=0pt]
\node[anchor=east,rectangle,fill=green!35]
{\strut Definizione};}}
}%
{\mdfsetup{%
frametitle={%
\tikz[baseline=(current bounding box.east),outer sep=0pt]
\node[anchor=east,rectangle,fill=green!35]
{\strut Definizione:~#1};}}%
}%
\mdfsetup{innertopmargin=1.5pt,linecolor=green!35,%
linewidth=1.75pt,topline=true,%
frametitleaboveskip=\dimexpr-\ht\strutbox\relax
}
\begin{mdframed}[]\relax%
}{\end{mdframed}}

%%%%%%%%%%%%%%%%%%%%%%%%%%%%%%%%%%%%%%%%%%%%%%%%%%%%%%%%%%%%%%%%%%%%%

\begin{document}

\subsection{Campi di spezzamento}

\begin{defn}[]{}
Sia $\mathbb{K}$ un campo e sia $f\in\mathbb{K}[x]$ con $\deg^{\star}(f)=n\geq 1$. Un'estensione di campi $\mathbb{L}/\mathbb{K}$ si dice campo di spezzamento di $f\in \mathbb{K}[x]$ se esistono $c_1,...\,,c_n\in \mathbb{L}$ e $c_f\in \mathbb{K}$ tali che:

\noindent (i) $f(x)=c_f \prod\limits_{i=1}^n (x-c_i)\in \mathbb{L}[x]$;

\noindent (ii) $\mathbb{L}=\mathbb{K}(\{c_1, ...\,,c_n\})$.
\end{defn}

\noindent La condizione (i) ci dice che $f$ si spezza in fattori lineari su $\mathbb{L}[x]$, e la condizione (ii) serve a limitare la grandezza di $\mathbb{L}$.

\noindent Il Teorema seguente è il teorema di esistenza e unicità del campo di spezzamento.

\begin{teo}[2.3.1]{}
Sia $\mathbb{K}$ un campo e sia $f\in\mathbb{K}[x]$ non nullo con $\deg^{\star}(f)=n\geq 1$. Allora, 

\noindent (a) esiste $\mathbb{L}/\mathbb{K}$ campo di spezzamento di $f$;

\noindent (b) se $\mathbb{L}_1/\mathbb{K}$ e $\mathbb{L}_2/\mathbb{K}$ sono campi di spezzamento di $f$, esiste $\alpha\colon \mathbb{L}_1\to \mathbb{L}_2$ isomorfismo di campi tale che la restrizione $\alpha_{\mathbb{K}}=\operatorname{id}_{\mathbb{K}}$.
\end{teo}
\vspace{-4mm}
\begin{proof}
Parte (a). Idea: esiste $a\in \mathbb{L}$ tale che $f(a)=0$, quindi $f(x)=(x-a)\cdot f_0$, dove $\deg^{\star}(f_0)=\deg^{\star}(f)-1$, e uso induzione forte. Come faccio? Definisco $\mathcal{P}(n)$ l'affermazione seguente: 
\begin{itemize}
\item $\mathcal{P}(n)$: Sia $\mathbb{K}$ un campo e sia $f\in\mathbb{K}[x]$ con $\deg^{\star}(f)=n$. Allora, esiste un'estensione di campi $\mathbb{L}/\mathbb{K}$ e esistono $c_1,...\,,c_n\in \mathbb{L}$ e $c_f\in \mathbb{K}$ tali che:

\noindent (i) $f(x)=c_f \prod\limits_{i=1}^n (x-c_i)\in \mathbb{L}[x]$;

\noindent (ii) $\mathbb{L}=\mathbb{K}(\{c_1, ...\,,c_n\})$.
\end{itemize}
Osserviamo che $\mathcal{P}(1)$ è vera: detto $f=a_1 x+a_0$, basta prendere $L=K$ e $c_1=-\frac{a_0}{a_1}\in \mathbb{K}$. Infatti, $f=a_1 \cdot (x-c_1)$ e $L=K=K(c_1)$. Assumiamo ora che $\mathcal{P}(k)$ sia vera per ogni $k< n$. Per la Costruzione di Kronecker, esiste $\mathbb{L}/\mathbb{K}$ e $a\in L$ tale che $f(a)=0$. Detto $K_1=K(a)\subseteq L$, $f=(x-a)\cdot f_1$, dove $f_1\in K_1[x]$, poiché per ipotesi induttiva vale $\mathcal{P}(n-1)$, esiste $L/K_1$ e esistono $c_1, ...\,,c_{n-1}\in L$ tali che $f_1(x)=c_f \prod\limits_{i=1}^{n-1}(x-c_i)$ e $L=K_1(\{c_1,...\,,c_{n-1}\})$. Allora, considerando $L/K$ e $c_0=a$, si ha che $f=c_f\prod\limits_{i=0}^{n-1}(x-c_i)\in L[x]$ e $L=K_1(\{c_1,...\,,c_{n-1}\})=K(\{c_0,...\,,c_{n-1}\})$ , cioè $\mathcal{P}(n)$ è effettivamente vera.

\

\noindent Parte (b). Sia $\alpha\colon K_1\to K_2$ isomorfismo di campi. Definiamo $(-)^{\alpha}\colon K_1[x]\to K_2[x]$ che preso $f=\sum\limits_{k=0}^n a_k x^k$ lo manda in $f^{\alpha}=\sum\limits_{k=0}^n \alpha(a_k) x^k$. Allora, tale funzione è un isomorfismo di anelli. Definisco $\mathcal{P}(n)$ l'affermazione seguente: 
\begin{itemize}
\item $\mathcal{P}(n)$: Siano $\mathbb{K}_1, K_2$ campi e sia $f\in\mathbb{K}_1[x]$ non nullo con $\deg^{\star}(f)=n$. Allora, detto $\alpha\colon K_1\to K_2$ isomorfismo di campi, $L_1/K_1$ il campo di spezzamento di $f\in K_1[x]$ e $L_2/K_2$ il campo di spezzamento di $f^{\alpha}\in K_2[x]$, esiste $\alpha_{\star}\colon L_1\to L_2$ isomorfismo di campi tale che la restrizione $\alpha_{\star} \raisebox{-.5em}{$\vert_{K_1}$}=\alpha$.
\end{itemize}
Osserviamo che $\mathcal{P}(1)$ è vera, perché preso $f=a_1x+a_0=a_1(x-c_1)$, dove $c_1=-\frac{a_0}{a_1}\in K_1$, e scelgo $L_1=K_1$; inoltre, $f^{\alpha}=\alpha(a_1)x+\alpha(a_0)$ e $\alpha(c_1)=\frac{\alpha(a_0)}{\alpha(a_1)}$, cioè $f^{\alpha}=\alpha(a_1)(x-\alpha(c_1))$ e prendo $L_2=K_2$, quindi $\alpha\colon L_1\to L_2$ è l'isomorfismo tra campi richiesto. Supponiamo ora che $\mathcal{P}(k)$ sia vera per ogni $k<n$. Sia $f=h\cdot f_0$ di grado $n$ con $h\in K_1[x]$ irriducibile, $\deg^{\star}(h)>0$. Allora, perché $L_1/K_1$ è campo di spezzamento per $f_1$, esiste $c\in L_1$ tale che $h(c)=0$. Dunque, $h(x)=(x-c)\cdot h_0(x)\in L_1[x]$. Sia $M_1=K_1[c]$ (cioè, $K_1$ con l'aggiunta dell'elemento algebrico $c$), così che abbiamo $h(x)=(x-c)h_0(x)\in M_1[x]$. Allora, detto $f_1=h_0(x)\cdot f_0(x)\in M_1[x]$, si ha $\deg^{\star}(f_1)=n-1$. Da $f=h\cdot f_0$, deduciamo che $f^{\alpha}=h^{\alpha}\cdot f_0^{\alpha}$, dove $h^{\alpha}$ è anch'esso irriducibile (se fosse $h^{\alpha}=h_1\cdot h_2$, avremmo $h=h_1^{\alpha^{-1}}\cdot h_2^{\alpha^{-1}}$ dove $\alpha^{-1}\colon K_2\to K_1$ è la funzione inversa di $\alpha$). Poiché $L_2/K_2$ è campo di spezzamento di $f^{\alpha}$, esiste $d\in L_2$ tale che $h^{\alpha}(d)=0$. Detto $M_2=K_2[d]$, $L_1/M_1$ è campo di spezzamento di $f_1$ e $L_2/M_2$ è campo di spezzamento di $f_2=f_1^{\alpha}/(x-d)$. Resta da mostrare che esiste un isomorfismo di campi $\beta\colon M_1\to M_2$ tale che $\beta(c)=d$, e $\beta \raisebox{-.5em}{$\vert_{K_1}$}=\alpha$, perché in questo modo $f_1^{\beta}=f_2$. Infatti $f^{\beta}=(x-c)^{\beta}f_1^{\beta}=(x-d)\cdot f_2=f^{\alpha}$, da cui effettivamente $f_1^{\beta}=f_2$. Per ipotesi induttiva, poiché vale $\mathcal{P}(n-1)$, sappiamo che esiste un isomorfismo tra campi $\beta_{\star}\colon L_1\to L_2$ tale che $\beta_{\star} \raisebox{-.5em}{$\vert_{M_1}$}=\beta$, da cui $\beta_{\star} \raisebox{-.5em}{$\vert_{K_1}$}=\beta \raisebox{-.5em}{$\vert_{K_1}$}=\alpha$, e abbiamo concluso perché ora prendiamo $\alpha_{\star}=\beta_{\star}$, quindi vale $\mathcal{P}(n)$. A quanto pare serve un pezzo del Teorema seguente per concludere.
\end{proof}

\begin{prop}[2.3.2]{}
Sia $\alpha\colon K_1\to K_2$ isomorfismo di campi e siano $h \in K_1[x]$ irriducibile, $L_1/K_1$ estensione di campi, $c\in L_1$ tale che $h(c)=0$, $L_2/K_2$ estensione di campi, $d\in L_2$ tale che $h^{\alpha}(d)=0$. Allora, esiste un isomorfismo di campi $\beta\colon K_1[c]\to K_2[d]$ tale che $\beta \raisebox{-.5em}{$\vert_{K_1}$}=\alpha$.
\end{prop}
\vspace{-4mm}
\begin{proof}
Vedi appunti cartacei per diagramma commutativo da aggiungere; lui ha anche messo un asterisco a sx nelle funzioni ma non so come metterlo adesso. Senza perdita di generalità siano $h$ monico, cioè $h=\min_{c,K_1}$, e $h^{\alpha}$ monico (so già che è irriducibile), $h^{\alpha}(d)=0$, quindi prendo $h^{\alpha}=\min_{d, K_2}$. Siano $\phi_c\colon K_1[x]/\langle h\rangle \to K_1[c]$, $\phi_d\colon K_2[x]/\langle h^{\alpha}\rangle \to K_2[d]$ le mappe indotte dalle valutazioni, $\alpha\colon K_1[x]/\langle h\rangle \to K_2[x]/\langle h^{\alpha}\rangle$ la mappa indotta da $\alpha$ del Teorema 2.3.1 punto (b). Definisco $\beta=\phi_d\circ \alpha \phi_c^{-1}\colon K_1[c]\to K_2[d]$. Questa è un isomorfismo perché composizione di isomorfismi, in quanto tutte le funzioni definite precedentemente sono isomorfismi per il \emph{Primo teorema d'isomorfismo}. La verifica che $\beta \raisebox{-.5em}{$\vert_{K_1}$}=\alpha$ è banale. Boh, sta cosa è completamente delirante.
\end{proof}

\end{document}