\documentclass{article}

\usepackage[utf8]{inputenc}
\usepackage[english]{babel}
\usepackage{amsmath}
\usepackage{amssymb}
\usepackage{amsthm}
\usepackage{yhmath}
\usepackage{gensymb}
\usepackage{graphicx}
\usepackage{siunitx}
\usepackage{amscd}
\usepackage{sectsty}
\usepackage{stmaryrd}
\usepackage{tikz-cd}
\usepackage{wrapfig}
\usepackage{xcolor}
\usepackage[margin=1.5in]{geometry}
\usepackage[framemethod=TikZ]{mdframed}

\theoremstyle{definition}
\newtheorem{thm}{Teorema}[subsection]
\newtheorem*{exm}{Esempio}
\renewcommand\qedsymbol{$\blacksquare$}
\addto\captionsenglish{\renewcommand*{\proofname}{Dimostrazione}}
\addto\captionsenglish{\renewcommand{\contentsname}{Indice}}

\sectionfont{\fontsize{20}{15}\selectfont}
\subsectionfont{\fontsize{14}{15}\selectfont}

\renewcommand\thefootnote{\textcolor{red}{\arabic{footnote}}}

\newcommand{\quot}{\operatorname{quot}}
\newcommand{\tor}{\operatorname{tor}}
\newcommand{\sat}{\operatorname{sat}}
\newcommand{\Ann}{\operatorname{Ann}}
\newcommand{\spec}{\operatorname{spec}}
\newcommand{\id}{\operatorname{id}}

\newenvironment{teo}[2][]{%
\ifstrempty{#1}%
{\mdfsetup{%
frametitle={%
\tikz[baseline=(current bounding box.east),outer sep=0pt]
\node[anchor=east,rectangle,fill=blue!25]
{\strut Teorema};}}
}%
{\mdfsetup{%
frametitle={%
\tikz[baseline=(current bounding box.east),outer sep=0pt]
\node[anchor=east,rectangle,fill=blue!25]
{\strut Teorema~#1};}}%
}%
\mdfsetup{innertopmargin=1.5pt,linecolor=blue!25,%
linewidth=1.75pt,topline=true,%
frametitleaboveskip=\dimexpr-\ht\strutbox\relax
}
\begin{mdframed}[]\relax%
}{\end{mdframed}}

\newenvironment{prop}[2][]{%
\ifstrempty{#1}%
{\mdfsetup{%
frametitle={%
\tikz[baseline=(current bounding box.east),outer sep=0pt]
\node[anchor=east,rectangle,fill=blue!25]
{\strut Proposizione};}}
}%
{\mdfsetup{%
frametitle={%
\tikz[baseline=(current bounding box.east),outer sep=0pt]
\node[anchor=east,rectangle,fill=blue!25]
{\strut Proposizione~#1};}}%
}%
\mdfsetup{innertopmargin=1.5pt,linecolor=blue!25,%
linewidth=1.75pt,topline=true,%
frametitleaboveskip=\dimexpr-\ht\strutbox\relax
}
\begin{mdframed}[]\relax%
}{\end{mdframed}}

\newenvironment{cor}[2][]{%
\ifstrempty{#1}%
{\mdfsetup{%
frametitle={%
\tikz[baseline=(current bounding box.east),outer sep=0pt]
\node[anchor=east,rectangle,fill=blue!25]
{\strut Corollario};}}
}%
{\mdfsetup{%
frametitle={%
\tikz[baseline=(current bounding box.east),outer sep=0pt]
\node[anchor=east,rectangle,fill=blue!25]
{\strut Corollario~#1};}}%
}%
\mdfsetup{innertopmargin=1.5pt,linecolor=blue!25,%
linewidth=1.75pt,topline=true,%
frametitleaboveskip=\dimexpr-\ht\strutbox\relax
}
\begin{mdframed}[]\relax%
}{\end{mdframed}}

\newenvironment{lem}[2][]{%
\ifstrempty{#1}%
{\mdfsetup{%
frametitle={%
\tikz[baseline=(current bounding box.east),outer sep=0pt]
\node[anchor=east,rectangle,fill=blue!25]
{\strut Lemma};}}
}%
{\mdfsetup{%
frametitle={%
\tikz[baseline=(current bounding box.east),outer sep=0pt]
\node[anchor=east,rectangle,fill=blue!25]
{\strut Lemma~#1};}}%
}%
\mdfsetup{innertopmargin=1.5pt,linecolor=blue!25,%
linewidth=1.75pt,topline=true,%
frametitleaboveskip=\dimexpr-\ht\strutbox\relax
}
\begin{mdframed}[]\relax%
}{\end{mdframed}}

\newenvironment{defn}[2][]{%
\ifstrempty{#1}%
{\mdfsetup{%
frametitle={%
\tikz[baseline=(current bounding box.east),outer sep=0pt]
\node[anchor=east,rectangle,fill=green!35]
{\strut Definizione};}}
}%
{\mdfsetup{%
frametitle={%
\tikz[baseline=(current bounding box.east),outer sep=0pt]
\node[anchor=east,rectangle,fill=green!35]
{\strut Definizione:~#1};}}%
}%
\mdfsetup{innertopmargin=1.5pt,linecolor=green!35,%
linewidth=1.75pt,topline=true,%
frametitleaboveskip=\dimexpr-\ht\strutbox\relax
}
\begin{mdframed}[]\relax%
}{\end{mdframed}}

%%%%%%%%%%%%%%%%%%%%%%%%%%%%%%%%%%%%%%%%%%%%%%%%%%%%%%%%%%%%%%%%%%%%%

\begin{document}

\subsection{Campi finiti}

\noindent Sia $\mathbb{K}$ un campo e sia $\chi_{\mathbb{K}}\colon \mathbb{Z}\to \mathbb{K}$ definita come $\chi_{\mathbb{K}}(n)=\sum\limits_{i=1}^{n} 1_{\mathbb{K}}$ per $n\geq 0$ (si intende che $\chi_{\mathbb{K}}(0)=0_{\mathbb{K}}$) e $\chi_{\mathbb{K}}(-n)=-\chi_{\mathbb{K}}(n)$. Allora, $\chi_{\mathbb{K}}$ è un omomorfismo di anelli, quindi $\operatorname{Im}(\chi_{\mathbb{K}})\subseteq \mathbb{K}$ è un dominio di integrità, da cui $\ker(\chi_{\mathbb{K}})\lhd \mathbb{Z}$ è un ideale primo. 

\begin{defn}[]{}
Se $\ker(\chi_{\mathbb{K}})=\{0\}$, allora $\mathbb{K}$ si dice di caratteristica $0$, e si scrive $\operatorname{char} (\mathbb{K})=0$. Se $\ker(\chi_{\mathbb{K}})\neq \{0\}$, esiste un primo $p$ tale che $\ker(\chi_{\mathbb{K}})=p\mathbb{Z}$; in questo caso, $\mathbb{K}$ si dice di caratteristica $p$, e si scrive $\operatorname{char} (\mathbb{K})=p$.
\end{defn}

\noindent Fatto (lo chiamerò Lemma 2.3.3): Sia $K$ un campo finito. Allora $\chi_K$ non può essere iniettivo, quindi $K$ è di caratteristica $p$ per un primo $p$. 

\

\noindent (In realtà è una definizione) Se $K$ è un campo di caratteristica $p\neq 0$, $K_0=\operatorname{Im}(\chi_k)\subseteq K$ è un campo detto campo primo di $K$, ed è isomorfo a $Z/pZ$.

\begin{teo}[2.3.4]{}
Siano $E, F$ campi finiti tali che $|E|=|F|$. Allora, $E\simeq F$
\end{teo}
\vspace{-4mm}
\begin{proof}
Sia $|E|=|F|=q$. Per il Fatto, $\operatorname{char}(E)=p_1$ e $\operatorname{char}(F)=p_2$ per $p_1, p_2$ primi. Poiché $E_0\subseteq E$, detto $n_1=|E:E_0|$, si ha che $|E|=p_1^{n_1}$. Analogamente, poiché $F_0\subseteq F$, detto $n_2=|F:F_0|$ si ha che $|F|=p_2^{n_2}$. Dunque $p_1=p_2=p$ e $n_1=n_2=n$. La dimostrazione dell'isomorfismo continua dopo (dannazione è un sacco disorganizzato negli appunti
\end{proof}

\noindent Questo teorema non è valido per i gruppi e per gli anelli. Infatti, nei gruppi $|S_3|=|\mathbb{Z}/6\mathbb{Z}|=6$, ma uno è abeliano e l'altro no, quindi $S_3\not\simeq \mathbb{Z}/6\mathbb{Z}$. Negli anelli, $\mathbb{Z}/4\mathbb{Z}\not\simeq \mathbb{F}_2[x]/\langle x^2\rangle$ perché uno ha gruppo additivo ciclico e l'altro no.

\

\noindent Osservare come questo dice che ogni campo finito ha cardinalità $p^n$ per un primo $p$ e $n>1$. In realtà questo è un se e solo se, cioè, per ogni $p^n$ esiste un campo di ordine $p^n$. Due strade: considerare lo splitting field $E$ di $x^{p^n}-x$ su $\mathbb{Z}/p\mathbb{Z}$ e mostrare che $|E|=p^n$, oppure costruire un polinomio irriducibile $f(x)$ di grado $n$ in $\mathbb{F}_p[x]$ e considerare il quoziente $\mathbb{F}_p[x]/\langle f \rangle$.

\	
	
\noindent Da questo segue anche che se $E,F$ sono campi finiti tali che $|E|=|F|=q$, allora $E^{\times}\simeq F^{\times}\simeq \mathbb{Z}/(q-1)\mathbb{Z}$ (questo perché se $G$ è un gruppo abeliano il cui esponente è $\exp(G)=|G|$, significa che $G$ è ciclico).

\begin{prop}[2.3.5]{}
Sia $A\subseteq K^{\times}$ sottoanello di $K$ campo, $|A|<\infty$. Allora, $A$ è un gruppo ciclico.
\end{prop}
\vspace{-4mm}
\begin{proof}
Sia $n=\exp(A)$, e sia $f=x^n-1$. Allora, $A\subseteq Z_f(K)$ (sta indicando con $Z_f(K)$ qualcosa che ha a che fare con gli zeri...) da cui $|A|\leq \deg^{\star}(f)=n=\exp(A)$. Poiché $\exp(A)\mid |A|$, deve essere $|A|=\exp(A)$, cioè $A$ è ciclico.
\end{proof}

\noindent Conclusione dimostrazione 2.3.4: Sia $\psi_q=(x^{q-1}-1)x$. Allora $Z_{\psi_q}(E)=E$ e $Z_{\psi_q}(F)=F$, quindi $\psi_q(x)=\prod\limits_{\lambda\in E}(x-\lambda)=\prod\limits_{\mu \in F}(x-\mu)$. Dunque, $E/E_0$ e $F/F_0$ sono campi di spezzamento di $\psi_q\in \mathbb{F}_p[x]$. Dunque, per il punto (b) del Teorema 2.3.1 sappiamo che $E\simeq F$.

\

\noindent Notare come questo dimostra che non è ambiguo denotare con $\mathbb{F}_p$ il campo di cardinalità $p$ primo, perché esso è effettivamente l'unico!

\

\noindent\textbf{Lezione del 12/11/2019} (appunti grezzi)

\begin{defn}[]{}
Sia $\mathbb{K}$ un campo con $\operatorname{char}(\mathbb{K})=p$. Allora, la mappa $F\colon \mathbb{K}\to \mathbb{K}$ definita come $F(x)=x^p$ è un omomorfismo di campi detto omomorfismo di Frobenius.
\end{defn}

\noindent Osserviamo che tale mappa è effettivamente un omomorfismo. Infatti, $F(0_K)=0_K$, $F(1_K)=1_K$ e $F(x+y)=\sum\limits_{k=0}^p \binom{p}{k}x^ky^{p-k}=x^p+y^p=F(x)+F(y)$ perché $\binom{p}{k}$ è divisibile per $p$ se $0<k<p$. Infine, è evidente che $F(xy)=(xy)^p=x^py^p=F(x)F(y)$. 

\

\noindent (Diventerà un Lemma) Osserviamo anche che $F$ è iniettiva, e se $|\mathbb{K}|<\infty$ è pure un automorfismo. Infatti, sappiamo che $\ker(F)\lhd \mathbb{K}$, ed essendo $1\not\in \ker(F)$, $\ker(F)=\{0_K\}$ perché gli unici ideali di un campo sono $\{0_K\}$ e $K$. Dunque $F$ è iniettiva. Se vale anche $|K|<\infty$, essendo $F$ iniettiva su un insieme finito, è chiaramente anche suriettiva, da cui è biettiva e quindi un automorfismo.

\begin{prop}[]{}
Sia $K$ un campo finito, $p=\operatorname{char}(K)$ e $K_0=\operatorname{Im}\chi_K$. Se $|K|=p^n$, $n=|K:K_0|$, allora $\operatorname{ord}(F)=n$, dove $\operatorname{ord}(F)=n$ è l'ordine di $F$ pensata come elemento di $\operatorname{Sym}(K)$.
\end{prop}
\vspace{-4mm}
\begin{proof}
Sappiamo che $K/K_0$ è campo di spezzamento di $x^{p^n}-x$, cioè $K$ è l'insieme degli zeri di $x^{p^n}-x$, il che è vero se e solo se $z^{p^n}=F^n(z)=z$ per ogni $z\in \mathbb{K}$. Dunque, $F^n=\operatorname{id}_K$. Resta da verificare che $n$ è effettivamente il minimo intero positivo per cui $F^k$ sia l'identità. Sia quindi $k\in \mathbb{N}^+$ con $k<n$ tale che $F^k=\operatorname{id}_K$. Allora, $z^{p^k}=F^k(z)=z$ per ogni $z\in K$, cioè $K$ è l'insieme degli zeri di $x^{p^k}-x$. Essendo $\deg^{\star}(x^{p^k}-x)=p^k$, tale polinomio ha al più $p^k$ radici, cioè $|K|=|Z_K(x^{p^k}-p)|\leq p^k<p^n$, assurdo. Dunque $F^k\neq \operatorname{id}_K$.
\end{proof}

\noindent Il grado di un'estensione si può chiamare ordine perché è l'ordine di un automorfismo (nel caso dei campi finiti).

\end{document}