\documentclass{article}

\usepackage[utf8]{inputenc}
\usepackage[english]{babel}
\usepackage{amsmath}
\usepackage{amssymb}
\usepackage{amsthm}
\usepackage{yhmath}
\usepackage{gensymb}
\usepackage{graphicx}
\usepackage{siunitx}
\usepackage{amscd}
\usepackage{sectsty}
\usepackage{stmaryrd}
\usepackage{tikz-cd}
\usepackage{wrapfig}
\usepackage{xcolor}
\usepackage[margin=1.5in]{geometry}
\usepackage[framemethod=TikZ]{mdframed}
\usepackage{eso-pic}

\theoremstyle{definition}
\newtheorem{thm}{Teorema}[subsection]
\newtheorem*{exm}{Esempio}
\renewcommand\qedsymbol{$\blacksquare$}
\addto\captionsenglish{\renewcommand*{\proofname}{Dimostrazione}}
\addto\captionsenglish{\renewcommand{\contentsname}{Indice}}

\sectionfont{\fontsize{20}{15}\selectfont}
\subsectionfont{\fontsize{14}{15}\selectfont}

\renewcommand\thefootnote{\textcolor{red}{\arabic{footnote}}}

\newcommand{\id}{\operatorname{id}}
\newcommand{\Mat}{\operatorname{Mat}}
\newcommand{\supp}{\operatorname{supp}}
\newcommand{\quot}{\operatorname{quot}}
\newcommand{\tor}{\operatorname{tor}}
\newcommand{\sat}{\operatorname{sat}}
\newcommand{\Ann}{\operatorname{Ann}}
\newcommand{\spec}{\operatorname{spec}}
\newcommand{\con}{\operatorname{con}}
\newcommand{\valpha}{\underline{\alpha}}
\newcommand{\vbeta}{\underline{\beta}}
\newcommand{\vgamma}{\underline{\gamma}}
\newcommand{\vdelta}{\underline{\delta}}
\newcommand{\vepsilon}{\underline{\varepsilon}}
\newcommand{\rbar}{r\underline{\, \, \,}}
\newcommand{\sbar}{s\underline{\, \, \,}}
\newcommand{\tbar}{t\underline{\, \, \,}}

\newcommand\BackgroundPic{%
\put(0,0){%
\parbox[b][\paperheight]{\paperwidth}{%
\vfill
\centering
\includegraphics[width=\paperwidth,height=\paperheight]{Fractal.png}%
\vfill
}}}

\newenvironment{teo}[2][]{%
\ifstrempty{#1}%
{\mdfsetup{%
frametitle={%
\tikz[baseline=(current bounding box.east),outer sep=0pt]
\node[anchor=east,rectangle,fill=blue!25]
{\strut Teorema};}}
}%
{\mdfsetup{%
frametitle={%
\tikz[baseline=(current bounding box.east),outer sep=0pt]
\node[anchor=east,rectangle,fill=blue!25]
{\strut Teorema~#1};}}%
}%
\mdfsetup{innertopmargin=1.5pt,linecolor=blue!25,%
linewidth=1.75pt,topline=true,%
frametitleaboveskip=\dimexpr-\ht\strutbox\relax
}
\begin{mdframed}[]\relax%
}{\end{mdframed}}

\newenvironment{prop}[2][]{%
\ifstrempty{#1}%
{\mdfsetup{%
frametitle={%
\tikz[baseline=(current bounding box.east),outer sep=0pt]
\node[anchor=east,rectangle,fill=blue!25]
{\strut Proposizione};}}
}%
{\mdfsetup{%
frametitle={%
\tikz[baseline=(current bounding box.east),outer sep=0pt]
\node[anchor=east,rectangle,fill=blue!25]
{\strut Proposizione~#1};}}%
}%
\mdfsetup{innertopmargin=1.5pt,linecolor=blue!25,%
linewidth=1.75pt,topline=true,%
frametitleaboveskip=\dimexpr-\ht\strutbox\relax
}
\begin{mdframed}[]\relax%
}{\end{mdframed}}

\newenvironment{cor}[2][]{%
\ifstrempty{#1}%
{\mdfsetup{%
frametitle={%
\tikz[baseline=(current bounding box.east),outer sep=0pt]
\node[anchor=east,rectangle,fill=blue!25]
{\strut Corollario};}}
}%
{\mdfsetup{%
frametitle={%
\tikz[baseline=(current bounding box.east),outer sep=0pt]
\node[anchor=east,rectangle,fill=blue!25]
{\strut Corollario~#1};}}%
}%
\mdfsetup{innertopmargin=1.5pt,linecolor=blue!25,%
linewidth=1.75pt,topline=true,%
frametitleaboveskip=\dimexpr-\ht\strutbox\relax
}
\begin{mdframed}[]\relax%
}{\end{mdframed}}

\newenvironment{lem}[2][]{%
\ifstrempty{#1}%
{\mdfsetup{%
frametitle={%
\tikz[baseline=(current bounding box.east),outer sep=0pt]
\node[anchor=east,rectangle,fill=blue!25]
{\strut Lemma};}}
}%
{\mdfsetup{%
frametitle={%
\tikz[baseline=(current bounding box.east),outer sep=0pt]
\node[anchor=east,rectangle,fill=blue!25]
{\strut Lemma~#1};}}%
}%
\mdfsetup{innertopmargin=1.5pt,linecolor=blue!25,%
linewidth=1.75pt,topline=true,%
frametitleaboveskip=\dimexpr-\ht\strutbox\relax
}
\begin{mdframed}[]\relax%
}{\end{mdframed}}

\newenvironment{defn}[2][]{%
\ifstrempty{#1}%
{\mdfsetup{%
frametitle={%
\tikz[baseline=(current bounding box.east),outer sep=0pt]
\node[anchor=east,rectangle,fill=green!35]
{\strut Definizione};}}
}%
{\mdfsetup{%
frametitle={%
\tikz[baseline=(current bounding box.east),outer sep=0pt]
\node[anchor=east,rectangle,fill=green!35]
{\strut Definizione:~#1};}}%
}%
\mdfsetup{innertopmargin=1.5pt,linecolor=green!35,%
linewidth=1.75pt,topline=true,%
frametitleaboveskip=\dimexpr-\ht\strutbox\relax
}
\begin{mdframed}[]\relax%
}{\end{mdframed}}

\begin{document}
\AddToShipoutPicture*{\BackgroundPic}

\title{\Huge{\textbf{Appunti del corso di Algebra II}} \\ \vspace{3mm}
	   \LARGE{\textbf{Dipartimento di Matematica e Applicazioni,}} \\ 
	   \LARGE{\textbf{Università di Milano-Bicocca}} \\ \vspace{3mm}
	   \Large{\textbf{A.A. 2019/2020}} % \\ \vspace{2.5mm} \large{\textbf{A cura di F. Clerici}}
	   }
\author{Versione del 7 Ottobre 2020}
\date{}

\maketitle
\thispagestyle{empty}
\clearpage

\tableofcontents

\

\

\noindent \textbf{Changelog (versione del 7 Ottobre 2020):}
\begin{itemize}
\item Reworking completo di varie cose
\end{itemize}

\noindent \textbf{To do (in ordine di importanza):}
\begin{itemize}
\item Teoria dei moduli (lezioni dal 06/11/2019 fino alla fine del corso)
\item Estensione di campi (lezioni del 25-30/10/19)
\item Campi di spezzamento e campi finiti (lezioni del 05-06/11/2019)
\item Domini a valutazione discreta (lezioni del 22-23/10/19)
\item Capitolo 1.7: sistemare spacing, anello locale che non è dominio, proposizione 1.7.10
\item Capitolo 1.5: riduzione mod p, Eisenstein, ciclotomici $x^{p-1}+...+x+1$
\item Capitolo 1.4: polinomi di Laurent e serie formali (fix i due rif in anelli locali)
\item Introduzione?
\end{itemize}
\clearpage

\section{Complementi di teoria degli anelli}
\vspace{1.75mm}
\subsection{Anelli di polinomi in una variabile}

Sia $R$ un anello e sia $R[x]=\left\{\sum\limits_{i=0}^n a_i x^i : a_i\in R,\, n\in \mathbb{N}_0\right\}$.\footnote{Useremo la convenzione secondo cui gli anelli sono commutativi unitari e $\mathbb{N}=\{1,2,...\}$, $\mathbb{N}_0=\mathbb{N}\cup \{0\}$.} Presi due elementi $f(x)=\sum\limits_{i=0}^m a_i x^i$ e $g(x)=\sum\limits_{j=0}^n b_j x^j$ di $R[x],$ definiamo le operazioni binarie di somma {\setlength{\belowdisplayskip}{4.5pt} \setlength{\abovedisplayskip}{-3.5pt} \[ f(x)+g(x)=\sum\limits_{i=0}^s (a_i+b_i)x^i \] dove abbiamo posto $s=\max\{m,n\}$ e $a_i=b_j=0_R$ per $i>m$ e $j>n$, e prodotto} {\setlength{\belowdisplayskip}{-3.5pt} \setlength{\abovedisplayskip}{3pt} \[ f(x)\cdot g(x)=\sum\limits_{k=0}^{m+n}c_k x^k\] dove abbiamo posto $c_k=\sum\limits_{i=0}^{k}a_i b_{k-i}$.\footnote{È solo un modo formale per definire il classico prodotto tra polinomi, come chiarificato dall'esempio.}}

\begin{exm}Se prendiamo $R=\mathbb{Z},$ $f(x)=x^2+2x+3$ e $g(x)=4x+5,$ si ha che \[ f(x)+g(x)=(1+0)x^2+(2+4)x+(3+5)=x^2+6x+8, \] {\setlength{\belowdisplayskip}{-2pt}\setlength{\abovedisplayskip}{0pt}\begin{align*}f(x)\cdot g(x) &= (3\cdot 0+2\cdot 0+1\cdot 4+0\cdot 5)x^3+(3\cdot 0+2\cdot 4+1\cdot 5)x^2+(3\cdot 4+2\cdot 5)x+3\cdot 5 \\ &= 4x^3+13x^2+22x+15. \ \square\end{align*}}\end{exm}

\noindent Come visto nel corso di Algebra I, si verifica facilmente che $R[x]$ dotato di tali operazioni di somma e prodotto è un anello commutativo\footnote{Infatti $a_ib_{k-i}=b_{k-i}a_i$ essendo $R$ un anello commutativo per ipotesi, da cui $f(x)\cdot g(x)=g(x)\cdot f(x)$.} con elemento neutro il polinomio identicamente nullo $0_{R[x]}=0_{R}$ e unità il polinomio costante $1_{R[x]}=1_{R}$. 

\vspace{1.5mm}

\noindent Di qui in seguito, denoteremo il prodotto di polinomi semplicemente come $f(x)g(x)$ o $f\cdot g$.

\begin{defn}[]{}
Tale insieme $R[x]$ è detto \underline{anello dei polinomi a coefficienti in $R$ nella variabile $x$}.
\end{defn}

\noindent Possiamo quindi definire su $R[x]$ il concetto di ``grado'' di un polinomio.

\begin{defn}[]{}
Sia $R$ un anello e sia $f(x)=\sum\limits_{i=0}^n a_i x^i\in R[x]$. La funzione $\deg^{\star}\colon R[x]\to \mathbb{N}_0\cup \{\infty\}$ definita come $\deg^{\star}(f)=\begin{cases} \max\{k\in \mathbb{N}_0: a_k\neq 0_R\} \text{ \ se }f(x)\not\equiv 0_R \\ \infty \text{ \ \ \ \ \ \ \ \ \ \ \ \ \ \ \ \ \ \ \ \ \ \ \ \ \ \ \ se } f(x)\equiv 0_R\end{cases}$è detta \underline{grado}.\footnotemark
\end{defn}
\footnotetext{Sarebbe più corretto scrivere $\deg^{\star}(f(x)),$ ma si preferisce evitare l'uso di troppe parentesi. Ricordiamo che con $f(x)\equiv k$ si intende il polinomio costante uguale a $k$. Tale notazione serve per non confondere un polinomio costante $p(x)\equiv 0$ con l'equazione algebrica $p(x)=0.$}

\noindent Tale definizione coincide con quella classica di grado di un polinomio tranne nel caso in cui $f(x)$ sia identicamente nullo. Infatti, per questa definizione $f(x)\equiv 0_R$ è l'unico polinomio di grado infinito, mentre secondo quella classica anch'esso ha grado $0$ in quanto costante. 

\begin{exm}Se consideriamo i polinomi $f(x)=x^2+1,$ $g(x)\equiv 1$ e $h(x)\equiv 0$ in $\mathbb{Z}[x]$, si ha che $\deg^{\star}(f)=2$ e $\deg^{\star}(g)=0,$ ma $\deg^{\star}(h)=\infty. \ \square$\end{exm}

\noindent Possiamo ora dimostrare un risultato che mette in relazione l'anello dei polinomi con quello dei suoi coefficienti, nel caso in cui quest'ultimo sia un dominio di integrità.\footnote{Ricordiamo che un dominio di integrità è un anello commutativo unitario $R\neq \{0_R\}$ senza divisori dello zero, cioè in cui $ab=0_R$ se e solo se $a=0_R$ o $b=0_R$. Esempi di domini di integrità sono $\mathbb{Z},$ le classi di resto $\mathbb{F}_p=\mathbb{Z}/p\mathbb{Z}$ con $p$ primo, gli interi gaussiani $\mathbb{Z}[i]=\left\{a+bi: a,b\in\mathbb{Z}\right\}$ e $\mathbb{Z}[\sqrt{2}\,]=\{a+b\sqrt{2}: a,b\in \mathbb{Z}\}$.}

\begin{prop}[1.1.1]{}
Sia $R$ un dominio di integrità. Allora, per ogni $f(x),\,g(x)\in R[x]$ vale \[\deg^{\star}(f\cdot g)=\deg^{\star}(f)+\deg^{\star}(g). \ \bf{(\star)}\] In particolare, $R[x]$ è un dominio di integrità se e solo se $R$ è un dominio di integrità.
\end{prop}
\vspace{-4mm}
\begin{proof} Osserviamo innanzitutto che se almeno uno tra $f(x)$ e $g(x)$ è identicamente nullo, allora $\bf{(\star)}$ è vera perché $f(x)g(x)\equiv 0_R$ e quindi \[\deg^{\star}(f\cdot g)=\infty=\deg^{\star}(f)+\deg^{\star}(g).\] D'altra parte, siano $f(x)=\sum\limits_{i=0}^m a_i x^i$ e $g(x)=\sum\limits_{j=0}^n b_j x^j$ non nulli con $a_m\neq 0_R$ e $b_n\neq 0_R$. Poiché $R$ è un dominio di integrità, $a_m b_n\neq 0_R$, cioè $a_m b_n x^{m+n}$ è il monomio di grado massimo nel prodotto $f(x)g(x)$. Per definizione di grado, concludiamo quindi che \[\deg^{\star}(f\cdot g)=m+n=\deg^{\star}(f)+\deg^{\star}(g).\]

\noindent Sia ora $R$ un dominio di integrità, e mostriamo che lo è anche $R[x]$. Osserviamo innanzitutto che $R[x]$ è un anello commutativo unitario, in quanto eredita tali proprietà da $R$. Inoltre, presi $f(x),\,g(x)\in R[x]$ tali che $f(x)g(x)\equiv 0_R$, per quanto appena mostrato vale \[\deg^{\star}(f)+\deg^{\star}(g)=\deg^{\star}(f\cdot g)=\deg^{\star}(0_R)=\infty.\] Dunque, almeno uno fra $f(x)$ e $g(x)$ ha grado infinito ed è quindi il polinomio nullo, cioè $R[x]$ non ha divisori dello zero ed è effettivamente un dominio di integrità.

\vspace{1mm}

\noindent Viceversa, sia $R[x]$ un dominio di integrità. Allora, $R\subseteq R[x]$ è commutativo e unitario in quanto sottoanello, e presi $a,b\in R,$ possiamo vedere $a$ e $b$ come polinomi costanti in $R[x]$. Essendo $R[x]$ un dominio di integrità, $ab=0_R$ se e solo se $a=0_R$ o $b=0_R$, da cui anche $R$ non ha divisori dello zero ed è quindi un dominio di integrità.\end{proof}

\noindent Osserviamo che $\bf{(\star)}$ non vale quando l'anello $R$ non è un dominio di integrità.

\begin{exm}Siano $f(x)=2x+1$ e $g(x)=3x+2$ in $\mathbb{Z}/6\mathbb{Z}\,[x]$. Allora, $\deg^{\star}(f)=\deg^{\star}(g)=1$, ma $f(x)g(x)=6x^2+7x+2\equiv_6 x+2$, da cui $\deg^{\star}(f\cdot g)=1\neq 2=\deg^{\star}(f)+\deg^{\star}(g). \ \square$\end{exm}

\noindent Più in generale, se $R$ non è un dominio di integrità, per definizione esistono $a,b\in R$ non nulli tali che $ab=0_R$. Allora, detti $f(x)=ax$ e $g(x)=bx$, si ha $f(x)g(x)=abx^2=0_Rx^2=0_R$, da cui, essendo $\deg^{\star}(f)=\deg^{\star}(g)=1$, l'uguaglianza $\bf{(\star)}$ non vale perché \[\deg^{\star}(f\cdot g)=\deg^{\star}(0_R)=\infty\neq 2=\deg^{\star}(f)+\deg^{\star}(g).\] Dunque, per la \emph{Proposizione 1.1.1} segue che $\bf{(\star)}$ vale se e solo se $R$ è un dominio di integrità.

\vspace{1.5mm}

\noindent Prima di procedere nello studio degli anelli di polinomi, richiamiamo il concetto di elemento invertibile di un anello. Preso un anello $R$, sia $R^{\times}$ l'insieme degli elementi di $R$ che hanno inverso moltiplicativo, cioè l'insieme degli $a\in R$ per cui esiste $b\in R$ tale che $ab=1_R$. Se esiste, denotiamo l'inverso moltiplicativo di $a$ con $a^{-1}$. Allora, vale la proposizione seguente.

\begin{prop}[1.1.2]{}
Sia $R$ un anello. Allora, $R^{\times}$ è un gruppo rispetto al prodotto.
\end{prop}
\vspace{-4mm}
\begin{proof}
Osserviamo innanzitutto che il prodotto è associativo essendo $R$ un anello, e in particolare $1_R$ è l'unità anche di $R^{\times}$. Inoltre, presi $a,b\in R^{\times}$, per definizione esistono $c,d\in R$ tali che $ac=1_R$ e $bd=1_R$, dunque \[(ab)(dc)=a(bd)c=a1_Rc=ac=1_R,\] cioè $ab\in R^{\times}$ è invertibile con inverso $dc$, da cui $R^{\times}$ è chiuso rispetto al prodotto. Infine, se $ab=1_R$ è evidente che anche $a^{-1}=b\in R^{\times}$, dunque $(R^{\times},\cdot)$ è effettivamente un gruppo.
\end{proof}

\noindent Grazie a tale proposizione, la definizione seguente risulta quindi ben posta.

\begin{defn}[]{}
Sia $R$ un anello. L'insieme $R^{\times}$ degli elementi di $R$ che ammettono inverso moltiplicativo è un gruppo detto \underline{gruppo moltiplicativo di $R$}.\footnotemark
\end{defn}\footnotetext{Tale gruppo viene spesso indicato anche con $\mathcal{U}(R)$ o $R^{\star}$ ed è anche detto ``gruppo delle unità di $R$''.}

\noindent Se da una parte la \emph{Proposizione 1.1.1} mostra che $R[x]$ può avere la struttura di un dominio di integrità, l'anello dei polinomi $R[x]$ non è mai un campo, nemmeno se lo è $R$ stesso.\footnote{Vedremo nel \emph{Capitolo 1.4} una generalizzazione degli anelli di polinomi con la struttura di un campo.} Infatti, $x\in R[x]$ non è un elemento invertibile perché il suo inverso $1/x$ non è un polinomio.\footnote{Più rigorosamente, se $f(x)=x$ fosse invertibile, esisterebbe $g(x)\in R[x]$ tale che $f(x)g(x)=1_R$, da cui $\deg^{\star}(f\cdot g)=\deg^{\star}(1_R)=0=\deg^{\star}(f)+\deg^{\star}(g)$, cioè $\deg^{\star}(g)=-\deg^{\star}(f)=-1<0$, assurdo.} Risulta quindi naturale chiedersi quali elementi di $R[x]$ siano effettivamente invertibili.

\begin{prop}[1.1.3]{}
Sia $R$ un dominio di integrità. Allora, $R[x]^{\times}=R^{\times}$.
\end{prop}
\vspace{-4mm}
\begin{proof}
Poiché ogni elemento di $R^{\times}$ può essere visto come polinomio costante di~$R[x]$, è evidente che $R^{\times}\subseteq R[x]^{\times}$. D'altra parte, siano $f(x),\,g(x)\in R[x]^{\times}$ tali che $f(x)g(x)=1_R$. Allora, per la \emph{Proposizione 1.1.1} si ha che \[\deg^{\star}(f\cdot g)=\deg^{\star}(1_R)=0=\deg^{\star}(f)+\deg^{\star}(g),\] quindi $\deg^{\star}(f)=\deg^{\star}(g)=0$ essendo il grado non negativo. Questo prova che ogni elemento di $R[x]^{\times}$ è in realtà una costante invertibile, cioè $R[x]^{\times}\subseteq R^{\times}$, dunque $R[x]^{\times}=R^{\times}$.
\end{proof}

\clearpage

\noindent Sia $R$ un anello, e supponiamo di voler aggiungere a $R$ un certo elemento $x\notin R$ senza alcuna relazione con gli altri elementi di $R$, in modo che la struttura algebrica risultante sia ancora un anello e sia la più piccola possibile. Come possiamo fare? 

\vspace{1.5mm}

\noindent Poiché ogni anello è chiuso rispetto a somma e prodotto, tale struttura conterrà anche tutte le potenze non negative $\{x^0, x^1, x^2, ...\}$ di $x$ e tutte le combinazioni lineari tra potenze di $x$ ed elementi di $R$, cioè tutti gli elementi della forma $a_nx^n + ... + a_1x+a_0$ con $a_0,...\,,a_n\in R$. Dunque, l'anello dei polinomi $R[x]$ sembra essere la struttura che soddisfa le nostre richieste, cioè il più piccolo anello contenente sia $R$ che $x$. Resta solo da formalizzare meglio il concetto di ``più piccolo anello'', cioè chiarire cosa significa che un anello ne contiene un altro.

\vspace{1.5mm}

\noindent A questo scopo, potremmo considerare sull'insieme degli anelli la relazione d'ordine data dall'inclusione, cioè dire che un anello $R$ è più piccolo di un altro anello $S$ se e solo se $R\subseteq S$. Tuttavia, questo non terrebbe conto dell'importanza algebrica degli isomorfismi: infatti, la struttura che stiamo cercando di costruire è definita a meno di isomorfismi, e anelli isomorfi potrebbero essere non confrontabili secondo l'inclusione.\footnote{Ad esempio, si verifica facilmente che la mappa $\varphi\colon \mathbb{C}\to \Mat_{2\times 2}(\mathbb{R})$, $a+bi \mapsto \begin{pmatrix} a & b \\ -b & a \end{pmatrix}$ è un isomorfismo di anelli, ma $\mathbb{C}\not\subseteq \operatorname{Mat}_{2\times 2}(\mathbb{R})$ e $\operatorname{Mat}_{2\times 2}(\mathbb{R})\not\subseteq \mathbb{C}$, cioè tali anelli non sono confrontabili secondo l'inclusione.} Per risolvere tale problema, ha quindi più senso definire che $R$ è più piccolo di $S$ se e solo se $S$ contiene una copia isomorfa dell'anello $R$, cioè se e solo se esiste un sottanello di $S$ isomorfo a $R$.

\begin{defn}[]{}
Siano $R$ e $S$ due anelli. Diciamo che $R$ è \underline{più piccolo} di $S$ (o anche che $S$ contiene $R$) se e solo se esiste un omomorfismo di anelli iniettivo $\varphi \colon R\to S$.
\end{defn} 

\noindent Si osservi che tale definizione è equivalente a quanto detto sopra: se esiste un monomorfismo (cioè un omomorfismo iniettivo) $\varphi\colon R\to S$, la restrizione $\varphi\colon R\to \varphi(R)$ è un isomorfismo, dunque l'immagine $\varphi(R)\subseteq S$ è un sottoanello di $S$ isomorfo a $R$.

\begin{exm}Chiaramente $\mathbb{R}$ non è un sottoanello di $\mathbb{R}^2$, in quanto $\mathbb{R}\not\subseteq \mathbb{R}^2$. D'altra parte, la mappa $\varphi\colon \mathbb{R}\to \mathbb{R}^2$, $x\mapsto (x,x)$ è un omomorfismo iniettivo, quindi $\mathbb{R}^2$ contiene una copia isomorfa di $\mathbb{R}$, che geometricamente corrisponde alla bisettrice $y=x. \ \square$\end{exm}

\noindent Tornando al problema iniziale, sia $X$ la struttura algebrica che stiamo cercando di costruire. Allora, possiamo riformulare le condizioni su $X$ come segue:
\begin{itemize}
\item $X$ contiene $R\Rightarrow$ esiste un monomorfismo $\iota\colon R\to X$;
\item $X$ è il più piccolo anello contenente sia $R$ che $x\notin R\Rightarrow$ per ogni altro anello $S$ con tali proprietà (cioè tale che esista un monomorfismo $\varphi\colon R\to S$ e contenente un $s\notin R$), abbiamo che $X$ è più piccolo di $S$, ossia esiste un monomorfismo $\phi\colon X\to S$. 
\end{itemize}In particolare, richiediamo che tale mappa $\phi$ soddisfi $\phi(x)=s$ e $\phi(\iota(R))=\varphi(R)$, cioè che mandi l'elemento aggiunto $x$ nell'elemento aggiunto $s$ e la copia isomorfa $\iota(R)$ di $R$ in $X$ nella copia isomorfa $\varphi(R)$ di $R$ in $S$.
\[
\begin{tikzcd}
R \arrow[rr, "\varphi", hook] \arrow[dd, "\iota"', hook] &  & S \\
                                                         &  &   \\
X \arrow[rruu, "\phi"', dashed, hook]                    &  &  
\end{tikzcd}
\]

\noindent Osserviamo ora che l'anello dei polinomi $R[x]$ soddisfa effettivamente tali proprietà. Infatti, detta $\iota\colon R\to R[x]$ la mappa di inclusione che manda ogni elemento $r\in R$ nel corrispondente polinomio costante $r\in R[x]$, è evidente che $\iota$ sia un monomorfismo, e preso un qualunque monomorfismo $\varphi\colon R\to S$, basta definire $\phi\colon R[x]\to S$ ponendo $\phi(x)=s$ e $\phi(\iota(r))=\varphi(r)$ per ogni $r\in R$. Tale mappa si estende per linearità su tutto $R[x]$ ponendo $$\phi\left( \sum\limits_{i=0}^n r_ix^i \right)=\sum\limits_{i=0}^n \varphi(r_i)s^i$$ ed è facile verificare che $\phi$ sia un monomorfismo.\footnote{Approfondiremo meglio questa questione nel \emph{Capitolo 2.1} quando tratteremo le estensioni di campi.} Più in generale, vale il teorema seguente.

\begin{teo}[1.1.4: Proprietà universale]{}
Siano $R$ e $S$ due anelli e sia $\varphi\colon R\to S$ un omomorfismo. Allora, per ogni $s\in S$ esiste un unico omomorfismo di anelli $\phi\colon R[x]\to S$ tale che $\phi(x)=s$ e $\phi \raisebox{-.5em}{$\vert_{R}$}= \varphi$.
\end{teo}
\vspace{-4mm}
\begin{proof}Siano $f(x)=\sum\limits_{i=0}^m a_i x^i$ e $g(x)=\sum\limits_{j=0}^n b_j x^j$ in $R[x]$ e sia $\phi(f)=\sum\limits_{i=0}^m \varphi(a_i) s^i$. Osserviamo innanzitutto che $\phi(f)$ è ben definita. Infatti, $\varphi(a_i)\in S$ e $\phi(f)\in S$ perché somma di prodotti di elementi dell'anello $S$, che è chiuso rispetto a somma e prodotto. Inoltre, $\phi(x)=\varphi(1_R)s^1=s$ e $\phi(r)=\varphi(r)s^0=\varphi(r)$ per ogni $r\in R$, quindi $\phi$ soddisfa le condizioni richieste. Mostriamo ora che $\phi$ preserva le operazioni. Infatti, \[ \phi(f+g)=\sum\limits_{i=0}^{\max\{m,n\}}\varphi(a_i+b_i)s^i=\sum\limits_{i=0}^{m}\varphi(a_i)s^i+\sum\limits_{j=0}^{n}\varphi(b_j)s^j=\phi(f)+\phi(g) \] per la distributività del prodotto rispetto alla somma e perché $\varphi(a_i+b_i)=\varphi(a_i)+\varphi(b_i)$, e \[ \phi(f\cdot g)=\sum\limits_{k=0}^{m+n}\left(\sum\limits_{i=0}^{k}\varphi(a_ib_{k-i})\right)s^k=\left(\sum\limits_{i=0}^{m}\varphi(a_i)s^i\right)\left(\sum\limits_{j=0}^{n}\varphi(b_j)s^j\right)=\phi(f)\cdot \phi(g) \] per come è definito il prodotto tra polinomi e perché $\varphi(a_ib_{k-i})=\varphi(a_i)\varphi(b_{k-i})$ essendo $\varphi$ un omomorfismo. Poiché $\phi(0_{R[x]})=\varphi(0_R)=0_S$ e $\phi(1_{R[x]})=\varphi(1_R)=1_S,$ concludiamo che tale mappa $\phi$ è effettivamente un omomorfismo di anelli.

\vspace{1mm}

\noindent Mostriamo ora che $\phi$ è unico. Sia $\psi\colon R[x]\to S$ un altro omomorfismo di anelli tale che $\psi(x)=s$ e $\psi \raisebox{-.5em}{$\vert_{R}$}= \varphi$. Poiché $\psi$ preserva le operazioni, per ogni $f(x)=\sum\limits_{i=0}^m a_i x^i\in R[x]$ vale \[ \psi(f) = \psi\left( \sum\limits_{i=0}^m a_i x^i \right) = \sum\limits_{i=0}^m \psi(a_i) \psi(x^i)=\sum\limits_{i=0}^m \varphi(a_i) \psi(x)^i=\sum\limits_{i=0}^m \varphi(a_i) s^i=\phi(f) \] essendo $\psi(a_i)=\varphi(a_i)$ perché $a_i\in R$ e $\psi(x^i)=\psi(x)^i=s^i$. Dunque, $\psi$ coincide con $\phi$ per ogni polinomio $f(x)\in R[x]$, da cui $\phi$ è unico.\end{proof}

\noindent Nel caso particolare in cui $\varphi=\id_R$ e quindi $R\subseteq S$, la mappa $\phi$ di cui sopra viene spesso denotata con $\phi_s$. In questo caso, $\phi_s(f)$ non è altro che il polinomio $f(x)$ calcolato in $x=s$, cioè $\phi_s(f)=f(s)$, il che spiega l'origine del nome ``valutazione in $s$'' per tale mappa.

\begin{defn}[]{}
Tale omomorfismo di anelli $\phi_s$ è detto \underline{valutazione in $s$}.
\end{defn}

\begin{exm}Se $R=\mathbb{Z}$, $S=\mathbb{Z}[\sqrt{2}\,]$ e $f(x)=x^2+2x+3\in \mathbb{Z}[x]$, detta $\phi_{\sqrt{2}}\colon \mathbb{Z}[x]\to \mathbb{Z}[\sqrt{2}\,]$ la valutazione in $\sqrt{2}$, abbiamo che $\phi_{\sqrt{2}}(f)=(\sqrt{2})^2+2\sqrt{2}+3=5+2\sqrt{2}\in \mathbb{Z}[\sqrt{2}\,]. \ \square$\end{exm}

\noindent Vogliamo ora dimostrare che la \emph{Proprietà universale} è una caratteristica propria degli anelli di polinomi, cioè che se $T$ è un anello contenente sia $R$ che un elemento $t\notin R$ e dotato della \emph{Proprietà universale}, allora $T\cong R[x]$. Nella dimostrazione ci limiteremo al caso in cui $R\subseteq T$ e $\varphi=\id_R$ (e quindi $R\subseteq S$), ma il caso generale è del tutto analogo.
\begin{teo}[1.1.5]{}
Sia $R$ un anello e sia $T\supseteq R$ un anello contenente un elemento $t\notin R$ e tale che per ogni anello $S\supseteq R$ e per ogni $s\in S$ esista un unico omomorfismo di anelli $\psi \colon T\to S$ con $\psi(t)=s$ e $\psi \raisebox{-.5em}{$\vert_{R}$}= \id_R$. Allora, $T\cong R[x]$.
\end{teo}
\vspace{-4mm}
\begin{proof}Poiché per ipotesi tale proprietà vale per ogni anello $S\supseteq R$, in particolare scegliamo $S=R[s]$ e siano $\phi_t\colon R[s]\to T$ la valutazione in $t$\footnote{Ricordiamo che per il \emph{Teorema 1.1.4} tale omomorfismo è l'unico che soddisfa $\phi_t(s)=t$ e $\phi_t \raisebox{-.5em}{$\vert_{R}$}= \operatorname{id}_R$.} e $\alpha=\phi_t\circ \psi\colon T\to T$. 
\vspace{-2mm}
\[
  \begin{tikzcd}
    T \arrow{r}{\psi} \arrow[swap]{dr}{\alpha} & R[s] \arrow{d}{\phi_t} \\
     & T
  \end{tikzcd}
\]
Osserviamo innanzitutto che $\alpha$ è ben definito ed è un omomorfismo in quanto composizione di omomorfismi. Inoltre, $\alpha(t) = \phi_t(\psi(t)) = \phi_t(s)=t$ e $\alpha(r)=\phi_t(\psi(r))=\phi_t(r)=r$ per ogni $r\in R$, cioè $\alpha \raisebox{-.5em}{$\vert_{R}$}= \operatorname{id}_R$. D'altra parte, poiché $T\supseteq R$, possiamo scegliere $S=T$ e $s=t$ nell'enunciato del teorema, così sappiamo che esiste un unico omomorfismo $\psi'\colon T\to T$ tale che $\psi'(t)=t$ e $\psi' \raisebox{-.5em}{$\vert_{R}$}= \id_R$. Poiché anche l'identità $\id_T\colon T\to T$ soddisfa tali proprietà, per l'unicità di $\psi'$ deve essere $\alpha=\id_T$. Sia ora $\beta=\psi\circ \phi_t\colon R[s]\to R[s]$.
\vspace{-2mm}
\[
  \begin{tikzcd}
    R[s] \arrow{r}{\phi_t} \arrow[swap]{dr}{\beta} & T \arrow{d}{\psi} \\
     & R[s]
  \end{tikzcd}
\]
Come sopra, osserviamo che $\beta$ è ben definito ed è un omomorfismo in quanto composizione di omomorfismi. Inoltre, $\beta(s) = \psi(\phi_t(s)) = \psi(t)=s$ e $\beta(r)=\psi(\phi_t(r))=\psi(r)=r$ per ogni $r\in R$, cioè $\beta \raisebox{-.5em}{$\vert_{R}$}= \operatorname{id}_R$. Poiché anche l'identità $\id_{R[s]}\colon R[s]\to R[s]$ soddisfa $\id_{R[s]}(s)=s$ e $\id_{R[s]} \raisebox{-.5em}{$\vert_{R}$}= \id_R$, e per il \emph{Teorema 1.1.4} esiste un unico omomorfismo con queste proprietà, deve essere $\beta=\operatorname{id}_{R[s]}$. Dunque, essendo $\phi_t\circ \psi=\operatorname{id}_T$ e $\psi\circ \phi_t=\operatorname{id}_{R[s]}$ isomorfismi, lo sono anche $\psi$ e $\phi_t$,\footnote{In generale, se $f\colon X\to Y$ e $g\colon Y\to X$ sono omomorfismi tali che $g\circ f=\operatorname{id}_X$ e $f\circ g=\operatorname{id}_Y,$ allora $f$ e $g$ sono isomorfismi. Infatti, $f$ è iniettivo perché $f(x)=f(x')\Rightarrow x=g(f(x))=g(f(x'))=x',$ ed è suriettivo perché preso $y\in Y,$ si ha che $g(y)\in X$ e $f(g(y))=y.$ In modo del tutto analogo si dimostra che anche $g$ è un isomorfismo, e in particolare risulta quindi che $g=f^{-1}.$} da cui concludiamo che $T\cong R[s]\cong R[x].\footnote{Infatti $s$ è solo un nome qualunque per la variabile dei polinomi a coefficienti in $R$.}$
\end{proof}

\subsection{Anelli di polinomi in $n$ variabili}

\noindent Vogliamo ora estendere il concetto di anello di polinomi ad un numero finito di variabili.

\begin{defn}[]{}
Sia $n$ un intero positivo. Denotiamo con $M=\operatorname{mon}\{x_1,...\,,x_n\}$ l'\underline{insieme dei monomi} \underline{nelle variabili $x_1,...\,,x_n$}, cioè $M=\{x_1^{\alpha_1}\cdot ...\cdot x_n^{\alpha_n}: \alpha_i\in\mathbb{N}_0\}$.
\end{defn} 
\vspace{-2mm}
\noindent Presi due elementi $u=x_1^{\alpha_1}\cdot ...\cdot x_n^{\alpha_n}$ e $v=x_1^{\beta_1}\cdot ...\cdot x_n^{\beta_n}$ di $M$, è possibile definire su $M$ un'operazione binaria corrispondente al prodotto di monomi: \[ u\cdot v=x_1^{\alpha_1+\beta_1}\cdot ...\cdot x_n^{\alpha_n+\beta_n}.\] Osserviamo che $M$ dotato di tale operazione è un monoide commutativo. Infatti,
\begin{itemize}
\item $u\cdot v=x_1^{\alpha_1+\beta_1}\cdot ...\cdot x_n^{\alpha_n+\beta_n}\in M$ perché $\alpha_i+\beta_i\in \mathbb{N}_0$, cioè $M$ è chiuso rispetto a $\cdot$
\item tale operazione agisce sugli esponenti delle variabili $x_1, ...\,,x_n$ mediante la somma, ed essendo tali esponenti in $\mathbb{N}_0$ e la somma associativa su $\mathbb{N}_0$, anche $\cdot$ è associativo
\item esiste un elemento neutro $1_{M}=x_1^0\cdot ... \cdot x_n^0\in M$
\item $u\cdot v=x_1^{\alpha_1+\beta_1}\cdot ...\cdot x_n^{\alpha_n+\beta_n}=x_1^{\beta_1+\alpha_1}\cdot ...\cdot x_n^{\beta_n+\alpha_n}=v\cdot u$, cioè $M$ è commutativo.
\end{itemize}

\noindent Per semplicità di notazione, sia $I_n=\{1,...\,,n\}$ e sia $\valpha \colon I_n\to\mathbb{N}_0$ la funzione che associa alla $i$-esima variabile $x_i$ l'esponente $\underline{\alpha}(i)=\alpha_i$. Denotiamo con $x^{\valpha}=x_1^{\alpha_1}\cdot ... \cdot x_n^{\alpha_n}\in M$. 

\begin{exm}Se $M=\operatorname{mon}\{x_1,x_2,x_3,x_4\}$ e $\valpha\colon \{1,2,3,4\}\to \mathbb{N}_0$ è la funzione definita come $\valpha(1)=2$, $\valpha(2)=\valpha(3)=1$ e $\valpha(4)=0$, abbiamo che $x^{\valpha}=x_1^2 x_2^1 x_3^1 x_4^0=x_1^2 x_2 x_3\in M. \ \square$\end{exm}

\noindent Detto $\mathcal{F}=\mathcal{F}(I_n, \mathbb{N}_0)$ l'insieme delle funzioni $\valpha \colon I_n\to\mathbb{N}_0$,\footnote{In generale, dati due insiemi $X$ e $Y$, si denota con $\mathcal{F}(X,Y)$ l'insieme di tutte le funzioni $f\colon X\to Y$.} vi è una corrispondenza biunivoca tra $\mathcal{F}$ e l'insieme dei monomi $M$. Infatti, ogni monomio $x_1^{\alpha_1}\cdot ...\cdot x_n^{\alpha_n}$ corrisponde in modo naturale all'unica funzione $\valpha\in\mathcal{F}$ tale che $\valpha(i)=\alpha_i$ per ogni $i\in I_n$, e ogni funzione $\vbeta \in\mathcal{F}$ rappresenta univocamente il monomio $x_1^{\vbeta(1)}\cdot ... \cdot x_n^{\vbeta(n)}\in M$.

\vspace{1.5mm}

\noindent Prima di procedere nella costruzione dei polinomi nelle variabili $x_1,...\,,x_n$, richiamiamo un importante concetto derivante dalla topologia e alcune sue proprietà.

\begin{defn}[]{}
Siano $X$ e $Y$ insiemi non vuoti e sia $f\colon X\to Y$ una funzione. Si definisce \underline{supporto di $f$} l'insieme $\supp(f)=\left\{x\in X: f(x)\neq 0_Y \right\}$. 
\end{defn}

\begin{exm}Sia $f\colon \mathbb{Z}\to\mathbb{R}$ la funzione $f(x)=x^2-1$. Allora, $\supp(f)=\mathbb{Z}\setminus\{\pm 1\}. \ \square$\end{exm}

\noindent Se $|\supp(f)|<\infty$, diciamo che $f$ ha supporto finito. Si osservi che tale definizione ha senso solo se l'insieme $Y$ contiene un elemento neutro $0_Y$: nel nostro caso, avendo a che fare con anelli, è naturale identificare tale elemento con l'elemento neutro dell'addizione.

\begin{exm}Se $f\colon \operatorname{Mat}_{2\times 2}(\mathbb{F}_2)\to \mathbb{F}_2$ è il determinante, allora $f$ ha supporto finito perché $\supp(f)=\mathrm{GL}(2,\mathbb{F}_2) = \left\{ \begin{pmatrix} 0 & 1 \\ 1 & 0 \end{pmatrix}, \begin{pmatrix} 0 & 1 \\ 1 & 1 \end{pmatrix}, \begin{pmatrix} 1 & 0 \\ 0 & 1 \end{pmatrix}, \begin{pmatrix} 1 & 0 \\ 1 & 1 \end{pmatrix}, \begin{pmatrix} 1 & 1 \\ 0 & 1 \end{pmatrix}, \begin{pmatrix} 1 & 1 \\ 1 & 0 \end{pmatrix} \right\}.\footnotemark \ \square$\end{exm}\footnotetext{Ricordiamo che $\mathrm{GL}(n,\mathbb{K})$ è il gruppo delle matrici $n\times n$ invertibili con entrate nel campo $\mathbb{K}$.}

\begin{prop}[1.2.1]{} 
Siano $f,\,g\colon X\to Y$ funzioni e siano $(f+g)(x)=f(x)+g(x)$ e $(f\cdot g)(x)=f(x)\cdot g(x)$. Allora, $\supp(f+g)\subseteq [\supp(f)\cup \supp(g)]$ e $\supp(f\cdot g)\subseteq [\supp(f)\cap \supp(g)]$. \end{prop}
\vspace{-4mm}
\begin{proof}
Osserviamo che se $x\in \supp(f+g)$, allora per definizione $f(x)+g(x)\neq 0_Y$, cioè almeno uno tra $f(x)$ e $g(x)$ è non nullo e quindi $x\in [\supp(f)\cup \supp(g)]$. 

\noindent Analogamente, se $x\in \supp(f\cdot g)$, per definizione abbiamo che $f(x)\cdot g(x)\neq 0_Y$, dunque $f(x)\neq 0_Y$ e $g(x)\neq 0_Y$, ossia $x\in [\supp(f)\cap \supp(g)]$.
\end{proof}

\begin{exm}Siano $f,\,g\colon \mathbb{Z}\to\mathbb{R}$ le funzioni $f(x)=x^2-3x+2$ e $g(x)=x^2+x-2$. Allora, è evidente che $\supp(f)=\mathbb{Z}\setminus\{ 1,2\}$ e $\supp(g)=\mathbb{Z}\setminus \{ 1,-2 \}$, ed essendo $(f+g)(x)=2x^2-2x$ e $(f\cdot g)(x)=(x-1)^2(x-2)(x+2)$, abbiamo che $$\supp(f+g)=\mathbb{Z}\setminus\{0,1\}\subseteq \mathbb{Z}\setminus\{1\} = \supp(f)\cup\supp(g)$$ $$\supp(f\cdot g)=\mathbb{Z}\setminus \{1,\pm 2\} = \supp(f)\cap\supp(g)$$ in accordo con la \emph{Proposizione 1.2.1}. $\square$\end{exm}

\noindent Sia $R$ un anello e sia $\mathcal{F}^{\times}(\mathcal{F},R)=\left\{ \rbar\,\colon \mathcal{F}\to R : |\operatorname{supp}(\rbar\,)|<\infty \right\}$, cioè l'insieme di tutte le funzioni $\rbar$ che associano ad ogni funzione $\valpha\in\mathcal{F}$ un elemento $r_{\valpha}\in R$ e che sono diverse dall'elemento neutro $0_R$ solo per un numero finito di elementi di $\mathcal{F}$. Possiamo quindi definire un polinomio nelle variabili $x_1,...\,,x_n$ ponendo \[ f(x_1,...\,,x_n)=\sum\limits_{\valpha\in \mathcal{F}}r_{\valpha} x^{\valpha}.\] Infatti, $f(x_1,...\,,x_n)$ risulta essere la somma di un numero finito di monomi non nulli, ognuno con il relativo coefficiente $r_{\valpha}$. Questo punto è fondamentale: abbiamo scelto $\rbar \in \mathcal{F}^{\times}(\mathcal{F},R)$ con supporto finito così che soltanto un numero finito degli infiniti monomi di $M$ abbia un coefficiente $r_{\valpha}\neq 0_R$. Così facendo, nella sommatoria vi è solo un numero finito di elementi perché tutti gli infiniti altri sono nulli, dunque $f$ è effettivamente un polinomio.

\begin{exm}Siano $M=\operatorname{mon}\{x,y\}$ e $R=\mathbb{Z}$. Detta $\rbar\,\colon \mathcal{F}\to \mathbb{Z}$ la funzione \[r_{\valpha} = \begin{cases}2\valpha(1)-\valpha(2) \text{ \ se } \valpha(1)+\valpha(2)=3 \\ 0 \text{ \ \ \ \ \ \ \ \ \ \ \ \ \ \ \ \ altrimenti}\end{cases}\] al variare di $\valpha\in\mathcal{F}= \mathcal{F}(I_2,\mathbb{N}_0)$, essendo $\valpha(1)\geq 0$ e $\valpha(2)\geq 0$, è evidente che esista solo un numero finito di funzioni $\valpha\in \mathcal{F}$ per cui $\valpha(1)+\valpha(2)=3$. In tutti gli altri casi abbiamo che $r_{\valpha}=0$, quindi $\rbar$ ha supporto finito, cioè $\rbar\in\mathcal{F}^{\times}(\mathcal{F},R)$. Se identifichiamo $\valpha$ con la coppia $(\alpha_1,\alpha_2)=(\valpha(1),\,\valpha(2))$,\footnote{Infatti $\mathcal{F}(I_n,\mathbb{N}_0)\cong \mathbb{N}_0^n$ mediante l'isomorfismo $\varphi\colon \mathcal{F}(I_n,\mathbb{N}_0)\to \mathbb{N}_0^n$, $\valpha \mapsto (\valpha(1),...\,,\valpha(n))$.} possiamo quindi definire il polinomio \begin{align*}f(x,y) &= \sum\limits_{\valpha\in \mathcal{F}}r_{(\alpha_1,\alpha_2)} x^{\alpha_1}y^{\alpha_2} \\ &= r_{(3,0)}x^3y^0+r_{(2,1)}x^2y^1+r_{(1,2)}x^1y^2+r_{(3,0)}x^3y^0+...\,\footnotemark \\ &= (2\cdot 3-0)x^3+(2\cdot 2-1)x^2y+(2\cdot 1-2)xy^2+(2\cdot 0-3)y^3 \\ &= 6x^3+3x^2y-3y^3.\ \square\end{align*}\end{exm}\footnotetext{Tutti gli altri termini della sommatoria sono nulli perché $\valpha(1)+\valpha(2)\neq 3$ e quindi, per come abbiamo definito $\rbar\,$, il coefficiente del monomio $x^{\alpha_1}y^{\alpha_2}$ è $r_{\valpha}=0$.}

\clearpage

\noindent Possiamo procedere nella costruzione dell'anello dei polinomi nelle variabili $x_1,...\,,x_n$. Detto $$R[x_1,...\,,x_n]=\left\{ \sum\limits_{\valpha\in \mathcal{F}} r_{\valpha} x^{\valpha} : \rbar \in \mathcal{F}^{\times}(\mathcal{F},R) \right\}$$ vogliamo quindi introdurre su tale insieme delle operazioni binarie di somma e prodotto così che esso sia effettivamente un anello. Presi due elementi $$f(x_1,...\,,x_n) = \sum\limits_{\valpha\in \mathcal{F}}r_{\valpha} x^{\valpha} \ \ \text{ e } \ \ g(x_1,...\,,x_n)=\sum\limits_{\vbeta\in \mathcal{F}}s_{\vbeta} x^{\vbeta}$$ di $R[x_1,...\,,x_n]$, definiamo le operazioni di somma e prodotto $$f(x_1,...\,,x_n)+g(x_1,...\,,x_n)=\sum\limits_{\valpha\in \mathcal{F}}(r_{\valpha}+s_{\valpha}) x^{\valpha}$$ $$f(x_1,...\,,x_n)\cdot g(x_1,...\,,x_n)=\sum\limits_{\vgamma \in \mathcal{F}}t_{\vgamma}x^{\vgamma}$$ dove abbiamo posto $t_{\vgamma}=\sum\limits_{\valpha+\vbeta=\vgamma} r_{\valpha}s_{\vbeta}$. Anche in questo caso, tali operazioni non sono altro che la formalizzazione delle usuali operazioni di somma e prodotto tra polinomi. 

\begin{prop}[1.2.2]{} 
Tali operazioni di somma e prodotto su $R[x_1,...\,,x_n]$ sono ben poste.
\end{prop}
\vspace{-4mm}
\begin{proof}
Nel caso della somma, è sufficiente mostrare che $(\rbar + \sbar\,) \in \mathcal{F}^{\times}(\mathcal{F},R)$, cioè che la somma di due funzioni in $\mathcal{F}^{\times}(\mathcal{F},R)$ è ancora una funzione in $\mathcal{F}^{\times}(\mathcal{F},R)$. Osserviamo che $(\rbar + \sbar\,)(\valpha)=r_{\valpha}+s_{\valpha}\in R$ per ogni $\valpha\in \mathcal{F}$ essendo $r_{\valpha}, s_{\valpha}\in R$ e $R$ chiuso rispetto alla somma in quanto anello, quindi $\rbar + \sbar$ è effettivamente una funzione da $\mathcal{F}$ in $R$. Inoltre, per la \emph{Proposizione 1.2.1} si ha che $$\supp(\rbar+\sbar\,)\subseteq [\supp(\rbar\,) \cup \supp(\sbar\,)]$$ e tale insieme è finito poiché unione di insiemi finiti. Dunque, $\rbar + \sbar$ ha supporto finito, da cui concludiamo che $(\rbar + \sbar\,) \in \mathcal{F}^{\times}(\mathcal{F},R)$, cioè che $\mathcal{F}^{\times}(\mathcal{F},R)$ è chiuso rispetto alla somma.

\vspace{1.5mm}

\noindent Nel caso del prodotto, dobbiamo mostrare che $\tbar\in \mathcal{F}^{\times}(\mathcal{F},R)$. Osserviamo innanzitutto che per ogni $\vgamma\in \mathcal{F}$ fissato, la somma $$t_{\vgamma} = \sum\limits_{\valpha+\vbeta=\vgamma} r_{\valpha}s_{\vbeta}$$ contiene un numero finito di addendi. Infatti, la condizione $\vgamma=\valpha+\vbeta\Rightarrow \vgamma(i)=\valpha(i)+\vbeta(i)$ per ogni $i\in I_n$ implica che $0\leq \valpha(i) \leq \vgamma(i)$, dunque abbiamo un numero finito di scelte per ogni $\valpha(i)$ e quindi anche per $\valpha$. Essendo $t_{\vgamma}$ la somma di un numero finito di prodotti $r_{\valpha}s_{\vbeta}\in R$, anche $t_{\vgamma}\in R$ per ogni $\vgamma\in \mathcal{F}$, cioè $\tbar$ è effettivamente una funzione da $\mathcal{F}$ in $R$. Infine, osserviamo che sempre per la \emph{Proposizione 1.2.1} si ha che $$\supp(\rbar\cdot \sbar\,)\subseteq [\supp(\rbar\,) \cap \supp(\sbar\,)]$$ dove tale insieme è finito poiché intersezione di insiemi finiti, quindi $(\rbar\cdot \sbar\,) \in \mathcal{F}^{\times}(\mathcal{F},R)$. Dunque, $\tbar$ è la somma di un numero finito di funzioni in $\mathcal{F}^{\times}(\mathcal{F},R)$, e avendo mostrato sopra che $\mathcal{F}^{\times}(\mathcal{F},R)$ è chiuso rispetto alla somma, concludiamo che $\tbar\in \mathcal{F}^{\times}(\mathcal{F},R)$.
\end{proof}

\clearpage

\noindent Per semplicità di notazione denoteremo di qui in seguito gli elementi di $R[x_1,...\,,x_n]$ come $f$, $g$, eccetera, dove si intende che $f=f(x_1,...\,,x_n),$ $g=g(x_1,...\,,x_n)$ e così via. Possiamo quindi finalmente dimostrare la proposizione seguente.

\begin{prop}[1.2.3]{}
Sia $R$ un anello commutativo. Allora, $R[x_1,...\,,x_n]$ dotato di tali operazioni di somma e prodotto è un anello commutativo.
\end{prop}
\vspace{-3mm}
\begin{proof}Siano $f=\sum\limits_{\valpha\in \mathcal{F}} r_{\valpha}x^{\valpha}\,, \ g=\sum\limits_{\vbeta\in \mathcal{F}} s_{\vbeta}x^{\vbeta}\,$ e $h=\sum\limits_{\vgamma\in \mathcal{F}} t_{\vgamma}x^{\vgamma}\,$ elementi di $R[x_1,...\,,x_n].$ Osserviamo innanzitutto che \begin{align*}\left(f+g\right)+h &= \sum\limits_{\valpha\in \mathcal{F}}(r_{\valpha}+s_{\valpha}) x^{\valpha} + \sum\limits_{\vgamma\in \mathcal{F}} t_{\vgamma}x^{\vgamma} = \sum\limits_{\valpha\in \mathcal{F}}(r_{\valpha}+s_{\valpha}+t_{\valpha}) x^{\valpha} \\ &= \sum\limits_{\valpha\in \mathcal{F}} r_{\valpha}x^{\valpha}+ \sum\limits_{\vbeta\in \mathcal{F}}(s_{\vbeta}+t_{\vbeta}) x^{\vbeta}=f+\left(g+h\right)\end{align*} da cui la somma è associativa. Poiché $(R,+)$ è abeliano, $r_{\valpha}+s_{\valpha}=s_{\valpha}+r_{\valpha},$ quindi \[ f+g = \sum\limits_{\valpha\in \mathcal{F}}(r_{\valpha}+s_{\valpha}) x^{\valpha}=\sum\limits_{\valpha\in \mathcal{F}}(s_{\valpha}+r_{\valpha}) x^{\valpha}=g+f\] da cui anche $(R[x_1,...\,,x_n],+)$ è un gruppo abeliano con elemento neutro $\sum\limits_{\valpha\in \mathcal{F}} 0_{\valpha}x^{\valpha}=0_R,$ dove $0_{\valpha}=0_R$ $\forall \valpha\in \mathcal{F}$ è la funzione nulla, e opposto $-f=\sum\limits_{\valpha\in \mathcal{F}} -r_{\valpha}x^{\valpha}$. Inoltre, \begin{align*}\left(f\cdot g\right) \cdot h &= \sum\limits_{\vdelta\in \mathcal{F}}\, \sum\limits_{\valpha+\vbeta=\vdelta} r_{\valpha}s_{\vbeta} \ x^{\vdelta}\,\cdot\,\sum\limits_{\vgamma\in \mathcal{F}} t_{\vgamma}x^{\vgamma}=\sum\limits_{\vepsilon\in \mathcal{F}} \, \sum\limits_{\vdelta+\vgamma=\vepsilon} \, \sum\limits_{\valpha+\vbeta=\vdelta} r_{\valpha}s_{\vbeta}t_{\vgamma} \ x^{\vepsilon} \\ &= \sum\limits_{\vepsilon\in \mathcal{F}} \,\sum\limits_{\valpha+\vbeta+\vgamma=\vepsilon} r_{\valpha}s_{\vbeta}t_{\vgamma} \ x^{\vepsilon} = \sum\limits_{\valpha\in \mathcal{F}} r_{\valpha}x^{\valpha} \, \cdot \, \sum\limits_{\vdelta\in \mathcal{F}}\, \sum\limits_{\vbeta+\vgamma=\vdelta} s_{\vbeta}t_{\vgamma} \ x^{\vdelta}=f\cdot \left(g\cdot h\right)\end{align*} da cui il prodotto è associativo. Essendo $R$ commutativo, $r_{\valpha}s_{\vbeta}=s_{\vbeta}r_{\valpha}$ e quindi \[ f\cdot g=\sum\limits_{\vdelta\in \mathcal{F}}\, \sum\limits_{\valpha+\vbeta=\vdelta} r_{\valpha}s_{\vbeta} \ x^{\vdelta}=\sum\limits_{\vdelta\in \mathcal{F}}\, \sum\limits_{\vbeta+\valpha=\vdelta} s_{\vbeta}r_{\valpha} \ x^{\vdelta}=g\cdot f\] da cui anche $R[x_1,...\,,x_n]$ è commutativo con unità $\sum\limits_{\valpha\in\mathcal{F}} 1_{\valpha}x^{\valpha}=1_R$ dove $1_{\valpha}$ è la funzione che vale $1_R$ per $\valpha=\underline{0}$ e $0_R$ per ogni altro $\valpha \in \mathcal{F}$.\footnote{Chiaramente si intende che $x^{\underline{0}}=x_1^0\cdot ...\cdot x_n^0=1_R\cdot ...\cdot 1_R=1_R$.} Infine, \begin{align*}\left(f+g\right) \cdot h &= \sum\limits_{\valpha\in \mathcal{F}}(r_{\valpha}+s_{\valpha}) x^{\valpha} \, \cdot \, \sum\limits_{\vgamma\in \mathcal{F}} t_{\vgamma}x^{\vgamma}= \sum\limits_{\vdelta\in \mathcal{F}}\, \sum\limits_{\valpha+\vgamma=\vdelta} (r_{\valpha}+s_{\valpha})t_{\vgamma} \ x^{\vdelta} \\ &= \sum\limits_{\vdelta\in \mathcal{F}}\, \sum\limits_{\valpha+\vgamma=\vdelta} r_{\valpha}t_{\vgamma} \ x^{\vdelta}\,+\,\sum\limits_{\vdelta\in \mathcal{F}}\, \sum\limits_{\valpha+\vgamma=\vdelta} s_{\valpha}t_{\vgamma} \ x^{\vdelta} = f\cdot h+g\cdot h\end{align*} dunque vale la proprietà distributiva e $(R[x_1,...\,,x_n],+,\cdot)$ è un anello commutativo.\end{proof}

\begin{defn}[]{}
Sia $R$ un anello commutativo e sia $n$ un intero positivo. Allora, l'insieme $R[x_1,...\,,x_n]$ è detto \underline{anello dei polinomi a coefficienti in $R$ nelle variabili $x_1,...\,,x_n$}.
\end{defn}

\noindent Anche per gli anelli di polinomi in $n$ variabili vale il corrispondente della \emph{Proprietà universale}, che per semplicità ci limiteremo a dimostrare nel caso in cui $R\subseteq S$.

\begin{teo}[1.2.4: Proprietà universale]{}
Sia $R$ un anello commutativo. Allora, per ogni anello commutativo $S\supseteq R$ e per ogni $\underline{s}=(s_1,...\,,s_n)\in S^n$ esiste un unico omomorfismo di anelli $\phi_{\underline{s}}\colon R[x_1, ...\, ,x_n]\to S$ tale che $\phi_{\underline{s}} (x_i)=s_i$ per ogni $i=1,...\,,n$ e $\phi_{\underline{s}} \raisebox{-.5em}{$\vert_{R}$}= \operatorname{id}_R$.
\end{teo}
\vspace{-3.5mm}
\begin{proof}Siano $f=\sum\limits_{\valpha\in \mathcal{F}} r_{\valpha}x^{\valpha}\,$ e $g=\sum\limits_{\vbeta\in \mathcal{F}} t_{\vbeta}x^{\vbeta}\,$ due elementi di $R[x_1,...\,,x_n].$ Per ogni monomio $x^{\valpha}\in M$ definiamo $\phi_{\underline{s}}(x^{\valpha})=\prod\limits_{i=1}^n s_i^{\alpha_i}$, e sia quindi $\phi_{\underline{s}}(f)=\sum\limits_{\valpha\in \mathcal{F}} r_{\valpha}\phi_{\underline{s}}(x^{\valpha}).$ Osserviamo innanzitutto che $\phi_{\underline{s}}(f)$ è ben definita. Infatti, $r_{\valpha}\in R\subseteq S$ e $\phi_{\underline{s}}(f)\in S$ perché somma di prodotti di elementi dell'anello $S$, che è chiuso rispetto a somma e prodotto. Inoltre, $\phi_{\underline{s}}(x_i)=s_i$ e $\phi_{\underline{s}}(\rho)=\rho$ per ogni $\rho\in R,$ quindi $\phi_{\underline{s}}$ soddisfa le condizioni richieste. Mostriamo ora che $\phi_{\underline{s}}$ preserva le operazioni. Infatti, $$\phi_{\underline{s}}\left(f+g\right)=\sum\limits_{\valpha\in \mathcal{F}}(r_{\valpha}+t_{\valpha}) \phi_{\underline{s}}(x^{\valpha})=\sum\limits_{\valpha\in \mathcal{F}}r_{\valpha}\phi_{\underline{s}}(x^{\valpha})+\sum\limits_{\valpha\in \mathcal{F}}t_{\valpha} \phi_{\underline{s}}(x^{\valpha})=\phi_{\underline{s}}(f)+\phi_{\underline{s}}(g)$$ per la proprietà distributiva del prodotto rispetto alla somma, essendo $S$ un anello, e $$\phi_{\underline{s}}(f\cdot g)=\sum\limits_{\vgamma\in \mathcal{F}} \sum\limits_{\valpha+\vbeta=\vgamma} r_{\valpha}t_{\vbeta} \ \phi_{\underline{s}}(x^{\vgamma})=\left(\sum\limits_{\valpha\in \mathcal{F}} r_{\valpha}\phi_{\underline{s}}(x^{\valpha})\right)\cdot \left(\sum\limits_{\vbeta\in \mathcal{F}} t_{\vbeta}\phi_{\underline{s}}(x^{\vbeta})\right)=\phi_{\underline{s}}(f)\cdot \phi_{\underline{s}}(g)$$ perché $\phi_{\underline{s}}(x^{\vgamma})=\prod\limits_{i=1}^n s_i^{\gamma_i}=\prod\limits_{i=1}^n s_i^{\alpha_i+\beta_i}=\prod\limits_{i=1}^n s_i^{\alpha_i}\cdot \prod\limits_{i=1}^n s_i^{\beta_i}=\phi_{\underline{s}}(x^{\valpha})\cdot \phi_{\underline{s}}(x^{\vbeta})$. Poiché $\phi_{\underline{s}}(0_R)=0_S$ e $ \phi_{\underline{s}}(1_R)=1_S$, concludiamo che tale mappa $\phi_{\underline{s}}$ è effettivamente un omomorfismo di anelli.

\vspace{2mm}

\noindent Mostriamo ora che $\phi_{\underline{s}}$ è unico. Sia $\psi\colon R[x_1,...\,,x_n]\to S$ un altro omomorfismo di anelli tale che $\psi(x_i)=s_i$ per ogni $i=1,...\,,n$ e $\psi \raisebox{-.5em}{$\vert_{R}$}= \operatorname{id}_R.$ Allora, per ogni monomio $x^{\valpha}\in M$ vale $$\psi(x^{\valpha})=\psi\left(\prod\limits_{i=1}^n x_i^{\alpha_i}\right)=\prod\limits_{i=1}^n \psi\left(x_i^{\alpha_i}\right)=\prod\limits_{i=1}^n \psi(x_i)^{\alpha_i}=\prod\limits_{i=1}^n s_i^{\alpha_i}=\phi_{\underline{s}}(x^{\valpha}).$$ Poiché $\psi$ preserva le operazioni, per ogni $f=\sum\limits_{\valpha\in \mathcal{F}} r_{\valpha}x^{\valpha}\in R[x_1,...\,,x_n]$ si ha quindi che \[ \psi(f)=\psi\left(\sum\limits_{\valpha\in \mathcal{F}} r_{\valpha}x^{\valpha}\right)=\sum\limits_{\valpha\in \mathcal{F}} \psi(r_{\valpha}x^{\valpha})=\sum\limits_{\valpha\in \mathcal{F}} \psi(r_{\valpha})\psi(x^{\valpha})=\sum\limits_{\valpha\in \mathcal{F}} r_{\valpha}\phi_{\underline{s}}(x^{\valpha})=\phi_{\underline{s}}(f)\] essendo $\psi(r_{\valpha})=r_{\valpha}$ perché $r_{\valpha}\in R$ e $\psi(x^{\valpha})=\phi_{\underline{s}}(x^{\valpha})$ per quanto provato sopra. Dunque, $\psi$ coincide con $\phi_{\underline{s}}$ per ogni polinomio $f\in R[x_1,...\,,x_n]$, da cui $\phi_{\underline{s}}$ è unico.\end{proof}

\subsection{Anelli di polinomi in più variabili}

\noindent Vogliamo ora generalizzare il concetto di anello di polinomi ad un numero qualsiasi di variabili, anche infinito. Sia $X$ un insieme non vuoto e sia $\mathcal{F}^{\times}=\mathcal{F}^{\times}(X,\mathbb{N})$ l'insieme delle funzioni $\underline{\alpha}\colon X\to \mathbb{N}$ che hanno supporto finito.\footnote{Notare come a differenza dei polinomi in $n$ variabili, ora richiediamo esplicitamente che tali funzioni $\underline{\alpha}$ abbiano supporto finito. Infatti, nel caso dei polinomi in $n$ variabili, $X$ è un insieme finito con $n$ elementi, quindi ogni funzione $\underline{\alpha}\colon X\to \mathbb{N}$ ha in realtà supporto finito perché $\operatorname{supp}(\underline{\alpha})\subseteq X,$ che è finito. Dunque, se $|X|<\infty$, non vi è differenza tra $\mathcal{F}^{\times}(X, \mathbb{N})=\mathcal{F}(X,\mathbb{N})$.}
\vspace{-0.75mm}
\begin{defn}[]{}
Sia $X$ un insieme. Denotiamo con $M=\operatorname{mon}\{X\}$ l'\underline{insieme dei monomi di $X$}, cioè $M=\left\{X^{\underline{\alpha}}: \underline{\alpha}\in \mathcal{F}^{\times}\right\}$ dove $X^{\underline{\alpha}}=\prod\limits_{x\in X} x^{\underline{\alpha}(x)}$.
\end{defn}
\vspace{-1.5mm}
\noindent Poiché abbiamo scelto $\underline{\alpha}$ con supporto finito, osserviamo che ogni monomio di $X$ è il prodotto di un numero finito di elementi di $X$, anche nel caso in cui $X$ sia un insieme infinito. Inoltre, analogamente al caso dei polinomi in $n$ variabili, $M$ è un monoide commutativo ed esiste una corrispondenza biunivoca tra i monomi di $M$ e le funzioni di $\mathcal{F}^{\times}$.

\vspace{1.5mm}

\noindent Sia $R$ un anello commutativo e sia $\mathcal{F}^{\times}(\mathcal{F}^{\times}, R)=\left\{ f\colon \mathcal{F}^{\times}\to R : |\operatorname{supp}(f)| <\infty \right\}$, cioè l'insieme delle funzioni che associano ad ogni funzione di $\mathcal{F}^{\times}$ un elemento dell'anello $R$, e che sono diverse da $0_R$ solo per un numero finito elementi di $\mathcal{F}^{\times}$. Al variare di $\underline{\alpha}\in \mathcal{F}^{\times}$, sia $r\underline{\, \, \,}\in \mathcal{F}^{\times}(\mathcal{F}^{\times},R)$ la funzione che associa ad ogni $\underline{\alpha}\in \mathcal{F}^{\times}$ l'elemento $r_{\underline{\alpha}}\in R.$ Osserviamo che possiamo definire un polinomio a variabili in $X$ ponendo {\setlength{\belowdisplayskip}{2.75pt} \setlength{\abovedisplayskip}{4.75pt} \[ f(X)=\sum\limits_{\underline{\alpha}\in \mathcal{F}^{\times}}r_{\underline{\alpha}} X^{\underline{\alpha}}.\]}\noindent Infatti, $f(X)$ è la somma di un numero finito di monomi non nulli, ognuno con un numero finito di variabili e preceduto dal relativo coefficiente $r_{\underline{\alpha}}.$ 

\vspace{0.6mm}

\noindent Sia $R[X]=\left\{ \sum\limits_{\underline{\alpha}\in \mathcal{F}^{\times}} r_{\underline{\alpha}} X^{\underline{\alpha}} : r\underline{\, \, \,}\in \mathcal{F}^{\times}(\mathcal{F}^{\times},R) \right\}$. Presi due elementi $f(X)=\sum\limits_{\underline{\alpha}\in \mathcal{F}^{\times}}r_{\underline{\alpha}} X^{\underline{\alpha}}\,$ e $g(X)=\sum\limits_{\underline{\beta}\in \mathcal{F}^{\times}}s_{\underline{\beta}} X^{\underline{\beta}}\,$ di $R[X],$ definiamo su $R[X]$ le operazioni binarie di somma e prodotto {\setlength{\belowdisplayskip}{2pt} \setlength{\abovedisplayskip}{-4.5pt}\[f(X)+g(X)=\sum\limits_{\underline{\alpha}\in \mathcal{F}^{\times}}(r_{\underline{\alpha}}+s_{\underline{\alpha}}) X^{\underline{\alpha}}\]} \[f(X)\cdot g(X)=\sum\limits_{\underline{\gamma}\in \mathcal{F}^{\times}}t_{\underline{\gamma}}X^{\underline{\gamma}}\] dove abbiamo posto $\underline{\gamma}=\underline{\alpha}+\underline{\beta}$ e $t_{\underline{\gamma}}=\sum\limits_{\underline{\alpha}+\underline{\beta}=\underline{\gamma}}r_{\underline{\alpha}}s_{\underline{\beta}}$. In modo del tutto analogo a quanto visto nel caso di $R[x_1,...\,,x_n]$, si dimostra che tali operazioni sono ben poste e che $R[X]$ dotato di tali operazioni di somma e prodotto è un anello commutativo con elemento neutro il polinomio nullo $\sum\limits_{\underline{\alpha}\in \mathcal{F}^{\times}} 0_{\underline{\alpha}}X^{\underline{\alpha}}=0_R$ e unità il monomio banale $X^{\underline{0}}=1_R$.
\vspace{-3mm}
\begin{defn}[]{}
Sia $R$ un anello commutativo e sia $X$ un insieme non vuoto. Allora, l'insieme $R[X]$ è detto \underline{anello dei polinomi a coefficienti in $R$ e a variabili in $X$}.
\end{defn}
\clearpage
\noindent Anche per gli anelli di polinomi in più variabili vale la \emph{Proprietà universale}.

\begin{teo}[1.3.1: Proprietà universale]{}
Sia $X$ un insieme e sia $R$ un anello commutativo. Allora, per ogni anello commutativo $S\supseteq R$ e per ogni mappa $\varphi\colon X\to S$ esiste un unico omomorfismo di anelli $\phi\colon R[X]\to S$ tale che $\phi(X^{\underline{\delta}_x})=\varphi(x) \ \forall x\in X \text{ e } \phi \raisebox{-.5em}{$\vert_{R}$}= \operatorname{id}_R$, dove $\underline{\delta}_x\colon X\to \mathbb{N}$, $\underline{\delta}_x(y)=\begin{cases}1 \text{ \ se }y=x\\0 \text{ \ se } y\neq x\end{cases}$
\end{teo}
\vspace{-3mm}
\begin{proof}Siano $f=\sum\limits_{\underline{\alpha}\in \mathcal{F}^{\times}} r_{\underline{\alpha}}X^{\underline{\alpha}}\,$ e $g=\sum\limits_{\underline{\beta}\in \mathcal{F}^{\times}} s_{\underline{\beta}}X^{\underline{\beta}}\,$ due elementi di $R[X].$ Per ogni monomio $X^{\underline{\alpha}}\in M$, sia $\phi(X^{\underline{\alpha}})=\prod\limits_{x\in X} \varphi(x)^{\underline{\alpha}(x)}$, e sia quindi $\phi(f)=\sum\limits_{\underline{\alpha}\in \mathcal{F}^{\times}} r_{\underline{\alpha}}\phi(X^{\underline{\alpha}}).$ Poiché $r_{\underline{\alpha}}\in R\subseteq S$ per ipotesi e $\phi(f)\in S$ perché somma di prodotti di elementi di $S,$ che in quanto anello è chiuso rispetto a somma e prodotto, $\phi$ è ben definita. Inoltre, $\phi(X^{\underline{\delta}_x})=\varphi(x)$ e $\phi(\rho)=\rho$ per ogni $\rho\in R,$ quindi $\phi$ soddisfa le condizioni richieste.\footnote{$\underline{\delta}_x$ è la funzione tale che per ogni $x\in X$ si abbia $X^{\underline{\delta}_x}=x$. Infatti, $X^{\underline{\delta}_x}=\prod\limits_{y\in X}y^{\underline{\delta}_x(y)}=x^{\underline{\delta}_x(x)}=x^1=x$ perché tutti gli altri termini del prodotto hanno esponente $0$, essendo per definizione $\underline{\delta}_x(y)=0$ se $y\neq x$.} Mostriamo ora che è un omomorfismo di anelli. Infatti, \[\phi\left(f+g\right)=\sum\limits_{\underline{\alpha}\in \mathcal{F}^{\times}}(r_{\underline{\alpha}}+s_{\underline{\alpha}}) \phi(X^{\underline{\alpha}})=\sum\limits_{\underline{\alpha}\in \mathcal{F}^{\times}}r_{\underline{\alpha}}\phi(X^{\underline{\alpha}})+\sum\limits_{\underline{\alpha}\in \mathcal{F}^{\times}}s_{\underline{\alpha}} \phi(X^{\underline{\alpha}})=\phi(f)+\phi(g)\] per la proprietà distributiva del prodotto rispetto alla somma, essendo $S$ un anello, e \[\phi(f\cdot g)=\sum\limits_{\underline{\gamma}\in \mathcal{F}^{\times}} \sum\limits_{\underline{\alpha}+\underline{\beta}=\underline{\gamma}} r_{\underline{\alpha}}s_{\underline{\beta}} \ \phi(X^{\underline{\gamma}})=\left(\sum\limits_{\underline{\alpha}\in \mathcal{F}^{\times}} r_{\underline{\alpha}}\phi(X^{\underline{\alpha}})\right)\cdot \left(\sum\limits_{\underline{\beta}\in \mathcal{F}^{\times}} s_{\underline{\beta}}\phi(X^{\underline{\beta}})\right)=\phi(f)\cdot \phi(g)\] perché $\phi(X^{\underline{\gamma}})=\prod\limits_{x\in X} \varphi(x)^{\underline{\gamma}(x)}=\prod\limits_{x\in X} \varphi(x)^{\underline{\alpha}(x)}\cdot \prod\limits_{x\in X} \varphi(x)^{\underline{\beta}(x)}=\phi(X^{\underline{\alpha}})\cdot \phi(X^{\underline{\beta}})$. Poiché $\phi(0_R)=0_S$ e $ \phi(1_R)=1_S$, concludiamo che $\phi$ è un omomorfismo di anelli.

\vspace{1mm}

Mostriamo ora che $\phi$ è unico. Sia $\psi\colon R[X]\to S$ un omomorfismo di anelli tale che $\psi(X^{\underline{\delta}_x})=\varphi(x)$ e $\psi \raisebox{-.5em}{$\vert_{R}$}= \operatorname{id}_R.$ Allora, per ogni monomio $X^{\underline{\alpha}}\in M$ vale \[ \psi(X^{\underline{\alpha}})=\psi\left(\prod\limits_{x\in X} x^{\underline{\alpha}(x)}\right)=\prod\limits_{x\in X} \psi\left(x^{\underline{\alpha}(x)}\right)=\prod\limits_{x\in X} \psi(X^{\underline{\delta}_x})^{\underline{\alpha}(x)}=\prod\limits_{x\in X} \varphi(x)^{\underline{\alpha}(x)}=\phi(X^{\underline{\alpha}}). \] Poiché $\psi$ è un omomorfismo, per ogni $f=\sum\limits_{\underline{\alpha}\in \mathcal{F}^{\times}} r_{\underline{\alpha}}X^{\underline{\alpha}}\in R[X]$ si ha che \[ \psi(f)=\psi\left(\sum\limits_{\underline{\alpha}\in \mathcal{F}^{\times}} r_{\underline{\alpha}}X^{\underline{\alpha}}\right)=\sum\limits_{\underline{\alpha}\in \mathcal{F}^{\times}} \psi(r_{\underline{\alpha}}X^{\underline{\alpha}})=\sum\limits_{\underline{\alpha}\in \mathcal{F}^{\times}} \psi(r_{\underline{\alpha}})\psi(X^{\underline{\alpha}})=\sum\limits_{\underline{\alpha}\in \mathcal{F}^{\times}} r_{\underline{\alpha}}\phi(X^{\underline{\alpha}})=\phi(f)\] essendo $\psi(r_{\underline{\alpha}})=r_{\underline{\alpha}}$ perché $r_{\underline{\alpha}}\in R$ e $\psi(X^{\underline{\alpha}})=\phi(X^{\underline{\alpha}})$ per quanto appena mostrato. Dunque, $\psi$ coincide con $\phi,$ che risulta quindi essere unico.\end{proof}

\noindent In modo del tutto analogo al \emph{Teorema 1.1.5} è possibile mostrare che, a meno di isomorfismi, $R[X]$ è l'unico anello contenente $R$ avente questa proprietà. \clearpage

\noindent Sia $R[x]$ l'anello dei polinomi a coefficienti in $R$ nella variabile $x.$ Possiamo considerare $R[x]$ stesso come anello dei coefficienti per l'anello dei polinomi nella variabile $y$, cioè \[(R[x])[y]=\left\{\sum\limits_{i=0}^n f_i\,y^i : f_i\in R[x],\, n\in \mathbb{N}\right\}.\] Poiché ogni polinomio di $(R[x])[y]$ può essere visto come un polinomio in due variabili di $R[x,y]$ e ogni polinomio di $R[x,y]$ può essere pensato come un polinomio di $(R[x])[y]$ raccogliendo i termini dello stesso grado in $y$, questo suggerisce che $(R[x])[y]\simeq R[x,y]$.

\begin{exm}Sia $f(y)=(x^2+1)y^2+(2x)y+3 \in (\mathbb{Z}[x])[y]$. Allora, possiamo vedere $f(y)$ come un polinomio in due variabili $g(x,y)=x^2y^2+y^2+2xy+3 \in \mathbb{Z}[x,y]$. Viceversa, preso $p(x,y)=xy^2+2xy+3y+4\in \mathbb{Z}[x,y]$, raccogliendo i termini dello stesso grado in $y$ possiamo pensare $p(x,y)$ come un polinomio $q(y)=(x)y^2+(2x+3)y+4 \in (\mathbb{Z}[x])[y]. \ \square$\end{exm}

\noindent In generale, se $X$ e $Y$ sono insiemi non vuoti e $(R[X])[Y]$ è l'anello dei polinomi a coefficienti in $R[X]$ e a variabili in $Y$, detta $X \sqcup Y$ l'unione disgiunta,\footnote{Ricordiamo che l'unione disgiunta di una famiglia di insiemi $\{A_i\}_{i\in I}$ è l'insieme $\bigsqcup\limits_{i\in I} A_i = \bigcup\limits_{i\in I} (A \times \{i\})$. Ad esempio, presi $A_0=\{3,4,5\}$ e $A_1=\{5,6\}$, si ha che $A_0 \sqcup A_1=\{(3,0), (4,0), (5,0), (5,1), (6,1)\}$.} vale il teorema seguente.

\begin{teo}[1.3.2]{}
Sia $R$ un anello commutativo e siano $X$ e $Y$ non vuoti. Allora, $R[X\sqcup Y]\simeq (R[X])[Y]$.
\end{teo}
\vspace{-4mm}
\begin{proof}
Sia $S$ un anello commutativo tale che $R\subseteq R[X]\subseteq S$ e sia $\varphi_X\colon X\to S$ definita come $\varphi_X(x)=X^{\underline{\delta}_x}$. Presa una qualunque funzione $\varphi_Y\colon Y\to S$, sia $\widetilde{\varphi}\colon X\sqcup Y\to S$ l'unica mappa tale che $\widetilde{\varphi} \raisebox{-.5em}{$\vert_{X}$}=\varphi_X$ e $\widetilde{\varphi} \raisebox{-.5em}{$\vert_{Y}$}=\varphi_Y$. Allora, per il \emph{Teorema 1.3.1} esiste un unico omomorfismo $\widetilde{\phi}\colon R[X\sqcup Y]\to S$ tale che $\widetilde{\phi}(Z^{\underline{\delta}_z})=\widetilde{\varphi}(z)$ per ogni $z\in X\sqcup Y$ e $\widetilde{\phi} \raisebox{-.5em}{$\vert_{R}$}= \operatorname{id}_R$. Per ogni $\underline{\alpha}\in \mathcal{F}^{\times}(X,\mathbb{N})$, sia $\underline{\widetilde{\alpha}}\in \mathcal{F}^{\times}(X \sqcup Y,\mathbb{N})$ l'unica funzione tale che $\underline{\widetilde{\alpha}} \raisebox{-.5em}{$\vert_{X}$} = \underline{\alpha}$ e $\underline{\widetilde{\alpha}} \raisebox{-.5em}{$\vert_{Y}$} = \underline{0}$. Allora, possiamo pensare ogni monomio $X^{\underline{\alpha}}$ di $R[X]$ come monomio $Z^{\widetilde{\underline{\alpha}}}$ di $R[X\sqcup Y]$, da~cui 
{\setlength{\belowdisplayskip}{2.5pt} \setlength{\abovedisplayskip}{5pt}\begin{align*} \widetilde{\phi}(Z^{\widetilde{\underline{\alpha}}}) &= \widetilde{\phi}\left( \prod\limits_{z\in X\sqcup Y} z^{\widetilde{\underline{\alpha}}(z)} \right) = \prod\limits_{z\in X\sqcup Y} \widetilde{\phi}\left(z^{\widetilde{\underline{\alpha}}(z)}\right) = \prod\limits_{z\in X\sqcup Y} \widetilde{\phi}\left(Z^{\underline{\delta}_z}\right)^{\widetilde{\underline{\alpha}}(z)} = \prod\limits_{z\in X\sqcup Y} \widetilde{\varphi}(z)^{\widetilde{\underline{\alpha}}(z)} \\ &= \prod\limits_{x\in X} \varphi_X(x)^{\underline{\alpha}(x)}\cdot \prod\limits_{y\in Y} \varphi_Y(y)^{\underline{0}} = \prod\limits_{x\in X} (X^{\underline{\delta}_x})^{\underline{\alpha}(x)}\cdot 1_R = X^{\underline{\alpha}} \end{align*}}\noindent per come abbiamo definito $\widetilde{\varphi}$ e $\underline{\widetilde{\alpha}}$ ed usando il fatto che $\widetilde{\phi}$ è un omomorfismo. Quindi, preso $f=\sum\limits_{\underline{\alpha}\in \mathcal{F}^{\times}} r_{\underline{\alpha}} X^{\underline{\alpha}}\in R[X]$, pensando $f$ come elemento $\widetilde{f}=\sum\limits_{\widetilde{\underline{\alpha}}\in \mathcal{F}^{\times}} r_{\widetilde{\underline{\alpha}}} Z^{\widetilde{\underline{\alpha}}}\in R[X\sqcup Y]$ si ha che \[ \widetilde{\phi}(\widetilde{f}\,) = \sum\limits_{\widetilde{\underline{\alpha}}\in \mathcal{F}^{\times}} \widetilde{\phi}(r_{\widetilde{\underline{\alpha}}}) \widetilde{\phi}(Z^{\widetilde{\underline{\alpha}}}) = \sum\limits_{\widetilde{\underline{\alpha}}\in \mathcal{F}^{\times}} r_{\widetilde{\underline{\alpha}}}\,\widetilde{\phi}(Z^{\widetilde{\underline{\alpha}}}) = \sum\limits_{\underline{\alpha}\in \mathcal{F}^{\times}} r_{\underline{\alpha}} X^{\underline{\alpha}} = f \] perché $\widetilde{\phi}(r_{\widetilde{\underline{\alpha}}})=r_{\widetilde{\underline{\alpha}}}$ essendo $\widetilde{\phi} \raisebox{-.5em}{$\vert_{R}$}= \operatorname{id}_R$, da cui $\widetilde{\phi} \raisebox{-.5em}{$\vert_{R[X]}$}=\operatorname{id}_{R[X]}$. Inoltre, per ogni $y\in Y$ si ha che $\widetilde{\phi}(Z^{\underline{\delta}_y})=\widetilde{\varphi}(y)=\varphi_Y(y)$. Poiché $R[X\sqcup Y]$ è un anello commutativo contenente $R[X]$ che soddisfa la proprietà universale di $(R[X])[Y]$,\footnote{Infatti, abbiamo appena mostrato che per ogni anello $S\supseteq R[X]$ e per ogni mappa $\varphi_Y\colon Y\to S$, esiste un unico omomorfismo $\widetilde{\phi}\colon R[X\sqcup Y]\to S$ tale che $\widetilde{\phi}(Z^{\underline{\delta}_y})=\varphi_Y(y)$ per ogni $y\in Y$ e $\widetilde{\phi} \raisebox{-.5em}{$\vert_{R[X]}$}=\operatorname{id}_{R[X]}$.} per la generalizzazione del \emph{Teorema 1.1.5} possiamo effettivamente concludere che $R[X \sqcup Y]\simeq (R[X])[Y]$.
\end{proof}
\clearpage

\noindent Nel caso in cui l'insieme delle variabili sia finito, vale il corollario seguente.

\begin{cor}[1.3.3]{}
Sia $n$ un intero positivo. Allora, $R[x_1,...\,,x_n]\simeq (\cdots((R[x_1])[x_2])\cdots )[x_n]$.
\end{cor}
\vspace{-4mm}
\begin{proof}
Procediamo per induzione sul numero $n$ di variabili. Chiaramente, se $n=1$ allora $R[x_1]\simeq R[x_1]$. Supponiamo quindi che la tesi sia vera per un certo intero $n\geq 1.$ Detti $X=\{x_1,...\,,x_n\}$ e $Y=\{x_{n+1}\}$, per il \emph{Teorema 1.3.2} si ha che $R[X\sqcup Y]\simeq (R[X])[Y]$ da cui $R[x_1,...\,,x_{n+1}]\simeq (R[x_1,...\,,x_n])[x_{n+1}]\simeq ((\cdots((R[x_1])[x_2])\cdots )[x_n])[x_{n+1}]$.\end{proof}

\noindent Possiamo quindi estendere agli anelli di polinomi in più variabili anche la \emph{Proposizione 1.1.1}. Per fare ciò, osserviamo innanzitutto che ogni polinomio di $R[X]$ è la somma di un numero finito di monomi non nulli, ognuno con un numero finito di variabili. Dunque, ogni polinomio di $R[X]$ può essere pensato come un polinomio in un numero finito di variabili, o meglio, per ogni $f\in R[X]$ esiste un sottoinsieme delle variabili $X_f\subseteq X$ finito tale che $f\in R[X_f]$.\footnote{Più formalmente, preso $f=\sum\limits_{\underline{\alpha}\in \mathcal{F}^{\times}}r_{\underline{\alpha}} X^{\underline{\alpha}}\in R[X]$ sappiamo che $\Omega_f=\operatorname{supp}(r\underline{\, \, \,}\,)\subseteq \mathcal{F}^{\times}$ è finito, quindi esiste solo un numero finito di funzioni $\underline{\alpha}\in \mathcal{F}^{\times}$ per cui il monomio $X^{\underline{\alpha}}$ ha un coefficiente $r_{\underline{\alpha}}$ non nullo. Poiché ogni $\underline{\alpha}\in \mathcal{F}^{\times}$ ha supporto finito, $X_f=\bigcup\limits_{\underline{\alpha}\in \Omega_f}\operatorname{supp}(\underline{\alpha})$ è finito in quanto unione finita di insiemi finiti.}

\begin{prop}[1.3.4]{}
Sia $X$ un insieme non vuoto e sia $R$ un dominio di integrità. Allora, anche l'anello dei polinomi $R[X]$ è un dominio di integrità.
\end{prop}
\vspace{-4mm}
\begin{proof}
Siano $f,g\in R[X]$ e siano $X_f, X_g\subseteq X$ finiti tali che $f\in R[X_f]$ e $g\in R[X_g]$. Osserviamo innanzitutto che $X_f\cup X_g$ è un sottoinsieme finito di $X$ e $f\cdot g\in R[X_f\cup X_g]$. Dunque, detto $X_f\cup X_g=\{x_1,...\,,x_n\}$, per dimostrare che $R[X]$ è un dominio di integrità~è sufficiente provare che $R[x_1,...\,,x_n]$ è un dominio di integrità.\footnote{Se il polinomio $f\cdot g$ si annulla in $R[X]$, allora si annulla anche pensato come polinomio di $R[X_f\cup X_g]$. Dunque, se $R[X_f\cup X_g]$ è un dominio di integrità per ogni $f,g\in R[X]$, allora anche $R[X]$ deve essere un dominio di integrità. Infatti, se esistessero $f,g\in R[X]$ divisori dello zero, per quanto appena detto essi sarebbero divisori dello zero anche in $R[X_f\cup X_g]$, il che contraddice la definizione di dominio di integrità.}\,Per fare ciò, procediamo~per induzione sul numero di variabili. Se $n=1$, per la \emph{Proposizione 1.1.1} sappiamo che $R[y_1]$ è un dominio di integrità. Supponiamo quindi che la tesi valga per un certo intero $n\geq 1$. Allora, per il \emph{Corollario 1.3.3} si ha che $R[y_1,...\,,y_{n+1}]\simeq (R[y_1,...\,,y_n])[y_{n+1}]$, ed essendo $R[y_1,...\,,y_n]$ un dominio di integrità per ipotesi induttiva, per la \emph{Proposizione 1.1.1} anche $(R[y_1,...\,,y_n])[y_{n+1}]$ è un dominio di integrità, da cui lo è pure $R[y_1,...\,,y_{n+1}]$. Dunque,~$R[Y]$ è un dominio di integrità per ogni insieme finito $Y$, ed in particolare lo è per $Y=X_f\cup X_g$. Per l'arbitrarietà di $f,g\in R[X]$, possiamo concludere che $R[X]$ è un dominio di integrità.
\end{proof}

\clearpage

\subsection{Polinomi di Laurent e serie formali}

\noindent Vogliamo ora introdurre alcune generalizzazioni del concetto di anello di polinomi molto usate nell'analisi reale e complessa, quali i polinomi di Laurent e le serie di potenze. 

\

\noindent Sia $R$ un anello commutativo e sia $R[x,x^{-1}]=\left\{\sum\limits_{i=-p}^n a_ix^i : a_i\in R,\, n,p\in \mathbb{N}\right\}$. Presi due elementi $f=\sum\limits_{i=-p}^m a_ix^i$ e $g=\sum\limits_{j=-q}^n b_jx^j$ di $R[x,x^{-1}]$, definiamo le operazioni di somma {\setlength{\belowdisplayskip}{4pt} \setlength{\abovedisplayskip}{-2.5pt} \[ f+g=\sum\limits_{i=-r}^s (a_i+b_i)x^i \] dove $s=\max\{m,n\}$, $r=\max\{p,q\}$ e $a_i=b_j=0$ per $i\not\in [-p,m]$ e $j\not\in [-q,n]$, e prodotto} {\setlength{\belowdisplayskip}{2.5pt} \setlength{\abovedisplayskip}{4pt} \[ f\cdot g=\sum\limits_{k=-p-q}^{m+n}c_k x^k\] dove abbiamo posto $c_k=\sum\limits_{i+j=k} a_i b_j$. Possiamo pensare $R[x,x^{-1}]$ come l'anello dei polinomi} \noindent $R[x]$ dove però l'esponente della variabile $x$ può essere anche un intero negativo.

\

\

\noindent Sketch del capitolo: Polinomi di Laurent ma le dim sono trivial per il capitolo 1.3, serie formali (qui dimostro cose), serie formali di Laurent (c'è una sola dim)

\clearpage

\subsection{Riducibilità di polinomi}

Concludiamo lo studio degli anelli di polinomi affrontandone il problema della riducibilità.

\begin{defn}[]{}
Sia $R$ un dominio di integrità e sia $f(x)\in R[x]$ un polinomio non invertibile\footnotemark\, e non nullo. Allora, $f(x)$ si dice \underline{irriducibile in $R[x]$} se ogni volta che esprimiamo $f(x)$ come un prodotto $f(x)=g(x)h(x)$ di polinomi $g(x),h(x)\in R[x]$, almeno uno fra $g(x)$ e $h(x)$ è invertibile. Se $f(x)$ non è irriducibile in $R[x]$, diciamo che $f(x)$ è \underline{riducibile in $R[x]$}.
\end{defn}\footnotetext{Si intende rispetto al prodotto, cioè per la \emph{Proposizione 1.1.2} prendiamo $f(x)\not\in R^{\times}$.}

\noindent La riducibilità di un polinomio non è un fatto generale, ma dipende dal particolare dominio di integrità preso in esame: non ha alcun senso parlare di ``polinomio irriducibile'' senza specificare quale sia il dominio d'integrità considerato.

\begin{exm}Il polinomio $f(x)=2x+4$ è irriducibile in $\mathbb{Q}[x]$ ma riducibile in $\mathbb{Z}[x]$. Infatti, se fosse $f(x)=g(x)h(x)$, per la \emph{Proposizione 1.1.1} si avrebbe $\deg^{\star}(f)=1=\deg^{\star}(g)+\deg^{\star}(h)$. Dunque, almeno uno fra $g(x)$ e $h(x)$ ha grado $0$ e risulta quindi invertibile essendo $\mathbb{Q}$ un campo, da cui $f(x)$ è irriducibile in $\mathbb{Q}[x]$. D'altra parte, $2x+4=2(x+2)$ e né $2$ né $x+2$ sono elementi invertibili in $\mathbb{Z}[x]$, quindi $f(x)$ è riducibile in $\mathbb{Z}[x]. \ \square$\end{exm}

\noindent Nel caso in cui il dominio di integrità sia un campo $\mathbb{K}$, poiché ogni elemento non nullo di $\mathbb{K}$ è invertibile, un polinomio non costante $f(x)\in \mathbb{K}[x]$ è riducibile in $\mathbb{K}[x]$ se e solo se può essere espresso come prodotto di due polinomi non costanti di grado minore di $\deg^{\star}(f)$.

\begin{exm}Il polinomio $f(x)=x^2+1$ è irriducibile in $\mathbb{R}[x]$ ma riducibile in $\mathbb{C}[x]$. Infatti, se $f(x)$ fosse riducibile in $\mathbb{R}[x]$, per quanto appena detto esso sarebbe il prodotto di due termini di grado $1$, il che è impossibile poiché $f(x)$ non ha radici reali. D'altra parte, sappiamo che $x^2+1=(x+i)(x-i)$, dunque $f(x)$ è riducibile in $\mathbb{C}[x]. \ \square$\end{exm}

\noindent In generale, stabilire se un polinomio sia o meno irriducibile in un certo dominio di integrità è un problema complesso. Tuttavia, esistono alcuni casi particolari in cui ciò è molto semplice.

\begin{teo}[1.5.1: Criterio del grado]{}
Sia $\mathbb{K}$ un campo e sia $f(x)\in \mathbb{K}[x]$ un polinomio di grado $2$ o $3$. Allora, $f(x)$ è riducibile in $\mathbb{K}[x]$ se e solo se $f(x)$ ha una radice in $\mathbb{K}$.
\end{teo}
\vspace{-4mm}
\begin{proof}
Supponiamo che $f(x)$ sia riducibile in $\mathbb{K}[x]$. Allora, per definizione esistono $g(x),h(x)\in \mathbb{K}[x]$ non costanti di grado minore di $\deg^{\star}(f)$ tali che $f(x)=g(x)h(x)$. Poiché per ipotesi $\deg^{\star}(g)+\deg^{\star}(h)=\deg^{\star}(f)\leq 3$, almeno uno fra $g(x)$ e $h(x)$ ha grado $1$, e senza perdita di generalità sia esso $g(x)=ax+b$. Essendo $\mathbb{K}$ un campo, $\alpha=-a^{-1}b\in \mathbb{K}$, da cui $g(\alpha)=a(-a^{-1}b)+b=0_{\mathbb{K}}$. Dunque, $f(\alpha)=g(\alpha)h(\alpha)=0_{\mathbb{K}}$, cioè $\alpha$ è una radice di $f(x)$.

Viceversa, supponiamo che esista $\alpha\in \mathbb{K}$ tale che $f(\alpha)=0_{\mathbb{K}}$. Per il \emph{Teorema di Ruffini} sappiamo che $(x-\alpha)$ divide $f(x)$, cioè $f(x)=(x-\alpha)q(x)$ per un opportuno $q(x)\in \mathbb{K}[x]$. Poiché $\deg^{\star}(q)=\deg^{\star}(f)-\deg^{\star}(x-\alpha)\geq 2-1=1$, si ha che $f(x)$ è riducibile in $\mathbb{K}[x]$.
\end{proof}

\noindent Tale teorema è particolarmente comodo nel caso dei campi finiti, poiché per stabilire la riducibilità di $f(x)\in \mathbb{F}_p[x]$ è sufficiente verificare se $f(n)\equiv 0\pmod{p}$ per $n=0,1,...\,,p-1$.

\begin{exm}Il polinomio $f(x)=x^3+x+1$ è irriducibile in $\mathbb{F}_2[x]$ ma riducibile in $\mathbb{F}_3[x]$. Infatti, $f(0)\equiv f(1)\equiv 1\not\equiv 0\pmod{2}$ in $\mathbb{F}_2$, ma $f(1)=3\equiv 0\pmod{3}$ in $\mathbb{F}_3. \ \square$\end{exm} \clearpage

\noindent Osserviamo che il \emph{Teorema 1.5.1} vale solo nei campi, dunque non è applicabile in $\mathbb{Z}$. Inoltre, esistono polinomi riducibili di grado maggiore o uguale a $4$ che non hanno radici.

\begin{exm}Entrambi i polinomi $f(x)=x^4+1$ e $g(x)=x^6+1$ non ammettono chiaramente radici reali. Tuttavia, osserviamo che $x^4+1=(x^2+\sqrt{2}x+1)(x^2-\sqrt{2}x+1)$ e possiamo scomporre $x^6+1=(x^2+1)(x^4-x^2+1)$, dunque $f(x)$ e $g(x)$ sono riducibili in $\mathbb{R}[x]. \ \square$\end{exm}

\noindent Di qui in seguito ci concentreremo principalmente sul problema della riducibilità in $\mathbb{Z}[x]$.

\begin{defn}[]{}
Sia $f(x)=\sum\limits_{i=0}^n a_i x^i \in \mathbb{Z}[x]$ un polinomio non nullo. Si definisce \underline{contenuto} di $f$ il valore di $\operatorname{MCD}(a_0,...\,,a_n)$. Un polinomio si dice \underline{primitivo} se il suo contenuto è $1$.
\end{defn}

\begin{exm}Il polinomio $f(x)=2x^2+3x+4$ è primitivo perché $\operatorname{MCD}(2,3,4)=1$. D'altra parte, il polinomio $g(x)=2x^2+4$ non è primitivo poiché $\operatorname{MCD}(2,0,4)=2\neq 1. \ \square$\end{exm}

\noindent Osserviamo che presi i due polinomi primitivi $f(x)=x+1$ e $g(x)=2x+3$, anche il loro prodotto $f(x)g(x)=2x^2+5x+3$ è primitivo, poiché il suo contenuto è $\operatorname{MCD}(2,5,3)=1$. Questo è un fatto generale, come dimostrato dal lemma seguente.

\begin{lem}[1.5.2: Lemma di Gauss]{}
Il prodotto di due polinomi primitivi è un polinomio primitivo.
\end{lem}
\vspace{-4mm}
\begin{proof}
Siano $f(x),\,g(x)\in \mathbb{Z}[x]$ polinomi primitivi, e supponiamo per assurdo che $f(x)g(x)$ non sia primitivo. Allora, esiste $p$ primo che divide tutti i coefficienti di $f(x)g(x)$, cioè $f(x)g(x)\equiv 0$ in $\mathbb{F}_p[x]$. Poiché $\mathbb{F}_p[x]$ è un dominio di integrità, deve essere $f(x)\equiv 0$ oppure $g(x)\equiv 0$, da cui $p$ divide tutti i coefficienti di almeno uno fra $f(x)$ e $g(x)$, e tale polinomio risulta quindi non primitivo, assurdo. Dunque, $f(x)g(x)$ è primitivo.
\end{proof}

\noindent Esiste una stretta relazione tra la riducibilità in $\mathbb{Z}[x]$ e quella in $\mathbb{Q}[x]$.

\begin{teo}[1.5.3]{}
Sia $f(x)\in \mathbb{Z}[x]$ un polinomio irriducibile in $\mathbb{Z}[x]$. Allora, $f(x)$ è irriducibile in $\mathbb{Q}[x]$.
\end{teo}
\vspace{-4mm}
\begin{proof}
Supponiamo per assurdo che $f(x)$ sia riducibile in $\mathbb{Q}[x]$. Allora, esistono $g(x),\,h(x)\in \mathbb{Q}[x]$ non costanti tali che $f(x)=g(x)h(x)$, dove, a meno di dividere $g(x)$ per il contenuto di $f$, possiamo assumere senza perdita di generalità che $f(x)$ sia primitivo. Siano $a$ e $b$ il minimo comune multiplo dei denominatori dei coefficienti di $g(x)$ e $h(x)$, rispettivamente, così che $ag(x)$ e $bh(x)$ siano polinomi a coefficienti interi. Detti $c_1$ e $c_2$ il contenuto di $ag(x)$ e $bh(x)$, rispettivamente, si ha che $ag(x)=c_1 g'(x)$ e $bh(x)=c_2h'(x)$, dove $g'(x)$ e $h'(x)$ sono polinomi primitivi. Poiché $abf(x)=ag(x)\,bh(x)=c_1c_2\,g'(x)h'(x)$ e per il \emph{Lemma 1.5.2} anche $g'(x)h'(x)$ è primitivo, deve essere $ab=c_1c_2$. Dunque, si ha che $f(x)=g'(x)h'(x)$ dove $g'(x),\,h'(x)\in \mathbb{Z}[x]$, cioè $f(x)$ è riducibile in $\mathbb{Z}[x]$, assurdo.
\end{proof}

\noindent Sebbene $\mathbb{Q}$ sia un campo più grande di $\mathbb{Z}$, tale teorema mostra che esso non è abbastanza grande per permettere di scomporre in $\mathbb{Q}[x]$ un polinomio irriducibile in $\mathbb{Z}[x]$, ed è quindi necessario passare a campi ancora più grandi quali $\mathbb{R}$ e $\mathbb{C}$. Inoltre, la dimostrazione mostra che se un polinomio $f(x)\in \mathbb{Z}[x]$ è riducibile in $\mathbb{Q}[x]$, allora esso è riducibile anche in $\mathbb{Z}[x]$.

\begin{exm}Sia $f(x)=6x^2-5x+1=(3x-\frac{3}{2})(2x-\frac{2}{3})$ un polinomio riducibile in $\mathbb{Q}[x]$. Utilizzando la notazione del \emph{Teorema 1.5.3}, definiamo $g(x)=(3x-\frac{3}{2})$ e $h(x)=(2x-\frac{2}{3})$. Allora, $a=2$ e $b=3$, da cui $ag(x)=6x-3$ e $bh(x)=6x-2$. Dunque, $c_1=\operatorname{MCD}(6,3)=3$ e $c_2=\operatorname{MCD}(6,2)=2$, da cui $g'(x)=2x-1$ e $h'(x)=3x-1$ sono polinomi primitivi e $f(x)=g'(x)h'(x)=(2x-1)(3x-1)$ risulta quindi riducibile in $\mathbb{Z}[x]. \ \square$\end{exm}

\

\noindent Sketch del capitolo: riduzione mod p, Eisenstein, polinomi ciclotomici, tanti esempi, e tutto quello che Weigel dà per scontato sia stato fatto ad Algebra 1.

\clearpage

\subsection{Anelli noetheriani}

Sia $R$ un anello e siano $a_1,...\,,a_n\in R.$ Allora, $I=\sum\limits_{i=1}^n Ra_i=\left\{ \sum\limits_{i=1}^n r_ia_i:r_i\in R \right\}$ è un ideale\footnote{Ricordiamo che $I$ è un ideale di un anello commutativo $R$ se $a-b\in I$ per ogni $a,b \in I$ e se $ra\in I$ per ogni $a\in I$ e $r\in R$. Per ragioni estetiche, si preferisce spesso mostrare equivalentemente che $a+b\in I$ per ogni $a,b \in I$ e non che $a-b\in I$. Infatti, se l'opposto esiste, detto $c=-b$ si ha che $a-b\in I\Leftrightarrow a+c\in I$.} di $R.$ Infatti, presi $x=\sum\limits_{i=1}^n r_ia_i$ e $y=\sum\limits_{i=1}^n s_ia_i$ in $I,$ si ha che $x+y=\sum\limits_{i=1}^n (r_i+s_i)a_i\in I$ e $tx=t\sum\limits_{i=1}^n r_ia_i=\sum\limits_{i=1}^n (tr_i)a_i\in I$ per ogni $t\in R,$ da cui $I\lhd R.$

\vspace{-1mm}

\begin{defn}[]{}
Sia $R$ un anello e siano $a_1,...\,,a_n\in R.$ Allora, l'ideale $I=\sum\limits_{i=1}^n Ra_i$ è detto \underline{ideale generato} \underline{da $a_1,...\,,a_n$} e si denota con $I=\langle a_1,...\,,a_n\rangle.$
\end{defn}

\noindent L'ideale generato da $a_1,...\,,a_n$ è il più piccolo ideale di $R$ contenente $a_1,...\,,a_n$, e possiamo pensarlo come l'insieme delle combinazioni lineari in $R$ di $a_1,...\,,a_n$.

\begin{exm}Consideriamo in $\mathbb{Z}$ gli ideali $I=\langle 2\rangle$ e $J=\langle 2,3\rangle$. Allora, $I=\{2n: n\in\mathbb{Z}\}=2\mathbb{Z}$ è l'insieme dei numeri pari e $J=\{2m+3n : m,n\in \mathbb{Z}\}=\mathbb{Z}$ perché $1=2\cdot 2+3\cdot (-1)\in J$, da cui $k=2\cdot (2k)+3\cdot (-k)\in J$ per ogni $k\in \mathbb{Z}. \ \square$\end{exm}

\vspace{-2mm}

\begin{defn}[]{}
Dato un ideale $I\lhd R,$ definiamo \underline{numero minimo di generatori} $d_R(I)$ il più piccolo $n\in \mathbb{N}$ per cui esistano $a_1,...\,,a_n\in R$ tali che $I=\langle a_1,...\,,a_n\rangle.$ Se tale $n\in \mathbb{N}$ non esiste, poniamo $d_R(I)=\infty.$ Diciamo che $I\lhd R$ è \underline{finitamente generato} se $d_R(I)<\infty.$
\end{defn}

\begin{exm} Sia $I=\langle 2, x\rangle \lhd \mathbb{Z}[x]$ e supponiamo che $d_{\mathbb{Z}[x]}(I)=1$, cioè che $I=\langle f(x)\rangle$ per un certo $f(x)\in \mathbb{Z}[x]$ non nullo. Se $\deg^{\star}(f)=0,$ cioè $f(x)\equiv k$, allora $k$ è pari e $I$ contiene solo polinomi con coefficienti pari, da cui $x\not\in I$, assurdo. Se $\deg^{\star}(f)\geq 1$, allora $I$ contiene solo polinomi di grado almeno $1$, cioè $2\not\in I$, assurdo. Dunque $d_{\mathbb{Z}[x]}(I)=2. \ \square$\end{exm}

\noindent Esiste un'importante famiglia di anelli in cui ogni ideale è finitamente generato.

\begin{defn}[]{}
Un anello commutativo $R$ si dice \underline{noetheriano} se ogni suo ideale è finitamente generato, cioè se ogni ideale $I\lhd R$ soddisfa $d_R(I)<\infty.$
\end{defn}

\begin{exm}Sia $R$ un dominio ad ideali principali.\footnote{Ricordiamo che un dominio ad ideali principali (spesso abbreviato PID) è un dominio di integrità in cui ogni ideale è principale, cioè generato da un solo elemento. Esempi di PID sono $\mathbb{Z}$ e ogni campo $\mathbb{K}.$} Allora, $R$ è un anello noetheriano poiché per definizione di PID si ha che $d_R(I)=1$ per ogni $I\lhd R$ non banale e $d_R(\{0_R\})=0.$ In particolare, ogni campo $\mathbb{K}$ è noetheriano perché i suoi unici ideali sono $\{0_{\mathbb{K}}\}$ e $\mathbb{K}=\langle 1_{\mathbb{K}}\rangle. \ \square$\end{exm}

\noindent Nelle dimostrazioni è spesso utile considerare una caratterizzazione equivalente degli anelli noetheriani in termini di successioni ascendenti di ideali, cioè successioni di ideali $(I_k)_{k\in \mathbb{N}}$ tali che $I_k\subseteq I_{k+1}$ per ogni $k\in \mathbb{N}$.

\begin{prop}[1.6.1]{}
Sia $R$ un anello commutativo. Allora, $R$ è noetheriano se e solo se per ogni successione ascendente di ideali $(I_k)_{k\in \mathbb{N}}$ esiste $N\in\mathbb{N}$ tale che $I_{N+j}=I_N$ per ogni $j\in\mathbb{N}.$
\end{prop}
\vspace{-4mm}
\begin{proof}Supponiamo che $R$ sia noetheriano, e sia $(I_k)_{k\in \mathbb{N}}$ una successione ascendente di ideali. Poiché $I_{\infty}=\bigcup\limits_{k\in \mathbb{N}} I_k$ è un ideale di $R,$\footnote{Siano $a,b\in I_{\infty}$ con $a\in I_{s}$ e $b\in I_{t}$, dove $s\leq t$, cioè $I_s\subseteq I_t$. Poiché $a,b\in I_t$, anche $a+b\in I_t\subseteq I_{\infty}$, da cui $a+b\in I_{\infty}$. Inoltre, preso $r\in R$, si ha che $ra\in I_s\subseteq I_{\infty},$ cioè $ra\in I_{\infty}$, da cui $I_{\infty}\lhd R$.} essendo $R$ noetheriano $d_R(I_{\infty})=n<\infty.$ Siano quindi $a_1,...\,,a_n\in I_{\infty}$ tali che $I_{\infty}=\langle a_1,...\,,a_n \rangle$, e siano $k_1,...\,,k_n\in \mathbb{N}$ tali che $a_i\in I_{k_i}$. Detto $N=\max\{k_i : 1\leq i\leq n\}$, essendo $(I_k)_{k\in \mathbb{N}}$ ascendente si ha che $a_1,...\,,a_n\in I_N.$ Dunque, essendo $I_N$ un ideale, $\sum\limits_{i=1}^n r_ia_i\in I_N$ per ogni $r_1,...\,,r_n\in R,$ cioè $\sum\limits_{i=1}^n Ra_i=I_{\infty}\subseteq I_N,$ da cui $I_{N+j}\subseteq I_N$ $\forall j\in \mathbb{N}.$ Poiché $(I_k)_{k\in \mathbb{N}}$ è ascendente, è anche vero che $I_N\subseteq I_{N+j}$ $\forall j\in N$. Combinando le doppie inclusioni, si ha quindi che $I_{N+j}=I_N$ $\forall j\in \mathbb{N}.$

Viceversa, supponiamo per assurdo che esista $J\lhd R$ con $d_R(J)=\infty.$ Preso $a_0\in J,$ costruiamo la successione $(a_k)_{k\in \mathbb{N}}$ di elementi di $J$ tale che $a_{k+1}\in J\setminus \langle a_0,...\,,a_k\rangle$ $\forall k\in \mathbb{N}.$ Tale successione esiste poiché $J$ non è finitamente generato, quindi $J\setminus \langle a_0,...\,,a_k\rangle \neq \emptyset$ per ogni $k\in \mathbb{N}.$ Si consideri la successione di ideali $(I_k)_{k\in \mathbb{N}},$ $I_k=\langle a_0,...\,,a_k\rangle.$ Allora, è evidente che $I_k\subseteq I_{k+1}$ $\forall k\in \mathbb{N},$ ma essendo $a_{k+1}\not\in I_{k}$ per come abbiamo definito $(a_k)_{k\in \mathbb{N}},$ risulta essere $I_k\varsubsetneq I_{k+1}.$ Abbiamo quindi costruito una successione ascendente di ideali che viola le ipotesi, perché non esiste $N\in \mathbb{N}$ tale che $I_{N+j}=I_{N}$ $\forall j\in \mathbb{N},$ assurdo. Dunque $d_R(J)<\infty$, e per l'arbitrarietà di $J$ concludiamo che $R$ è noetheriano.\end{proof}

\noindent Dimostriamo ora un risultato fondamentale nello studio degli anelli noetheriani.

\begin{teo}[1.6.2: Teorema della base di Hilbert]{}
Sia $R$ un anello noetheriano. Allora, anche l'anello dei polinomi $R[x]$ è noetheriano.
\end{teo}
\vspace{-4mm}
\begin{proof}
Supponiamo per assurdo che $R[x]$ non sia noetheriano, e sia quindi $J\lhd R[x]$ tale che $d_{R[x]}(J)=\infty.$ Preso $f_0\in J$ non nullo di grado minimo, costruiamo la successione di polinomi $(f_k)_{k\in \mathbb{N}}$ tale che $f_{k+1}$ sia il polinomio di grado minimo in $J\setminus \langle f_0, ...\,,f_k \rangle$ $\forall k\in \mathbb{N}.$ Tale successione esiste poiché $J$ non è finitamente generato, quindi $J\setminus \langle f_0, ...\,,f_k \rangle\neq \emptyset$ per ogni $k\in \mathbb{N}.$ Sia $d_k=\deg^{\star}(f_k)$ e sia $a_k\neq 0_R$ il coefficiente direttore di $f_k.$ Allora, detta $(I_k)_{k\in \mathbb{N}}$ la successione ascendente di ideali di $R$ definita come $I_k=\langle a_0, ...\,,a_k\rangle$, per la \emph{Proposizione 1.4.1} esiste $N\in \mathbb{N}$ tale che $I_{N+j}=I_{N}$ $\forall j\in \mathbb{N}.$ In particolare $I_{N+1}=I_N$, ed esistono $r_0,...\,,r_N\in R$ tali che $a_{N+1}=\sum\limits_{i=0}^N r_ia_i.$ Consideriamo ora il polinomio $h=f_{N+1}-\sum\limits_{i=0}^{N}r_i\, x^{d_{N+1}-d_i} f_i \in J.$\footnotemark\newline Se $h\in \langle f_0,...\,,f_N\rangle$, allora anche $f_{N+1}=h+\sum\limits_{i=0}^{N}r_i\, x^{d_{N+1}-d_i} f_i\in \langle f_0,...\,,f_N\rangle$, il che è assurdo per come abbiamo definito $(f_k)_{k\in \mathbb{N}}$. Poiché il coefficiente del termine di grado $d_{N+1}$ in $h$ è $a_{N+1}-\sum\limits_{i=0}^N r_ia_i=0$, si ha che $h\in J\setminus \langle f_0,...\,,f_N\rangle$ è un polinomio di grado $\deg^{\star}(h)<d_{N+1}$, e questo viola la minimalità del grado nella scelta di $f_{N+1}$. Dunque $d_{R[x]}(J)<\infty$, da cui per l'arbitrarietà di $J$ concludiamo che $R[x]$ è noetheriano.
\end{proof}\footnotetext{Vogliamo sfruttare la relazione tra $a_{N+1}$ e $a_1,...\,,a_N$ che abbiamo appena trovato per costruire un polinomio $h\in J\setminus \langle f_0, ...\,,f_N\rangle$ di grado minore di $d_{N+1}$, giungendo quindi ad un assurdo.}

\begin{cor}[1.6.3]{}
Sia $n\in \mathbb{N}^+$ e sia $R$ un anello noetheriano. Allora, anche $R[x_1,...\,,x_n]$ è noetheriano.
\end{cor}
\vspace{-4mm}
\begin{proof}Essendo $R$ noetheriano, per il \emph{Teorema 1.6.2} anche $R[x_1]$ è noetheriano, ed induttivamente sono noetheriani pure $(R[x_1])[x_2],...\,, (\cdots((R[x_1])[x_2])\cdots )[x_n]$. Poiché per il \emph{Corollario 1.3.3} si ha che $R[x_1,...\,,x_n]\simeq (\cdots((R[x_1])[x_2])\cdots )[x_n]$, possiamo concludere che anche $R[x_1,...\,,x_n]$ è noetheriano.\end{proof}

\noindent Questo risultato non è più valido quando l'insieme delle variabili $X$ è un insieme infinito, ed in particolare, esistono domini di integrità che non sono noetheriani.

\begin{exm}Sia $R$ un domimio di integrità noetheriano e sia $X=\{x_n : n\in \mathbb{N}\}$ un insieme numerabile di variabili. Per la \emph{Proposizione 1.3.4} sappiamo già che $R[X]$ è un dominio di integrità, quindi è sufficiente mostrare che esso non è noetheriano. Sia $(I_k)_{k\in \mathbb{N}}$ la successione di ideali di $R[X]$ definita come $I_k=\langle x_0,...\,,x_k \rangle$. Allora $I_k\varsubsetneq I_{k+1}$, poiché $I_k\subseteq I_{k+1}$ ma $x_{k+1}\not\in \langle x_0, ...\,,x_k\rangle=I_k$. Dunque, $(I_k)_{k\in \mathbb{N}}$ è una successione ascendente di ideali che viola la \emph{Proposizione 1.6.1}, da cui concludiamo che $R[X]$ non è noetheriano$. \ \square$\end{exm}

\noindent La proposizione seguente, molto utile negli esercizi, permette di dimostrare che un anello è noetheriano semplicemente esibendo un omomorfismo suriettivo.

\begin{prop}[1.6.4]{}
Sia $R$ un anello noetheriano e sia $\phi\colon R\to S$ un omomorfismo di anelli suriettivo. Allora, anche $S$ è un anello noetheriano.
\end{prop}
\vspace{-4mm}
\begin{proof} Siano $J\lhd S$ e $I=\phi^{-1}(J)=\{r\in R : \phi(r)\in J\}.$ Poiché $I$ è un ideale~di~$R$,\footnote{In generale, se $\varphi \colon A\to B$ è un omomorfismo e $J\lhd B$, allora $I=\varphi^{-1}(J)=\{a\in A: \varphi(a)\in J\}\lhd A$. Infatti, presi $a,\, b\in I$, per definizione $\varphi(a),\,\varphi(b)\in J$. Dunque, essendo $J$ un ideale e $\varphi$ un omomorfismo, $\varphi(a+b)=\varphi(a)+\varphi(b)\in J\Rightarrow a+b\in I$ e $\varphi(ra)=\varphi(r)\varphi(a)\in J\Rightarrow ra\in I$ per ogni $r\in A$, da cui $I\lhd A$.} che per ipotesi è noetheriano, esistono $a_1,...\,,a_n \in R$ tali che $I=\langle a_1,...\,,a_n \rangle$. Allora, essendo $\phi$ suriettivo, sappiamo che $J=\phi(I)=\langle \phi(a_1),...\,,\phi(a_n) \rangle$, da cui $d_S(J)\leq d_R(I)=n<\infty$. Dunque, per l'arbitrarietà di $J$ concludiamo che $S$ è noetheriano. \end{proof}

\begin{exm} Sia $\mathbb{Z}[\sqrt{2}\,]=\{a+b\sqrt{2} : a,b\in \mathbb{Z}\}\subseteq \mathbb{R}$. Poiché $\mathbb{Z}$ è un PID, esso è noetheriano, dunque per il \emph{Teorema 1.4.2} anche $\mathbb{Z}[x]$ è noetheriano. Sia $\phi_{\sqrt{2}}\colon \mathbb{Z}[x]\to \mathbb{Z}[\sqrt{2}\,]$ la valutazione in $\sqrt{2}$. Poiché per ogni $a+b\sqrt{2}\in \mathbb{Z}[\sqrt{2}\,]$ si ha che $\phi_{\sqrt{2}}(a+bx)=a+b\sqrt{2}$, tale $\phi_{\sqrt{2}}$ è un omomorfismo suriettivo, quindi per la \emph{Proposizione 1.6.4} anche $\mathbb{Z}[\sqrt{2}\,]$ è noetheriano$. \ \square$\end{exm}

\noindent Un caso particolare della \emph{Proposizione 1.6.4} vale per gli anelli quoziente.

\begin{cor}[1.6.5]{}
Sia $R$ un anello noetheriano e sia $I\lhd R$. Allora, anche $R/I$ è noetheriano.
\end{cor}
\vspace{-4mm}
\begin{proof}
Sia $\pi\colon R\to R/I$, $\pi(r)=r+I$ la proiezione canonica sul quoziente. Poiché $\pi$ è un omomorfismo suriettivo, per la \emph{Proposizione 1.6.4} anche $R/I$ è noetheriano.
\end{proof}

\noindent Possiamo quindi mostrare che esistono anelli noetheriani che non sono domini di integrità.

\begin{exm}Poiché $4\mathbb{Z}$ è un ideale di $\mathbb{Z}$, per il \emph{Corollario 1.6.5} anche $\mathbb{Z}/4\mathbb{Z}$ è noetheriano.\footnote{In realtà basta osservare che ogni anello finito è noetheriano poiché $d_R(I)\leq |R|<\infty$ per ogni $I\lhd R$.} Tuttavia, esso non è dominio di integrità perché ha divisori dello zero: infatti, $2\cdot 2=0. \ \square$\end{exm}

\clearpage

\noindent Proposizione che anello noetheriano ha un ideale massimale, discussione sulla noetherianità che gratuitamente permette la dimostrazione senza il lemma di zorn, dimostrazione che ogni anello ha un ideale massimale usando zorn.

\clearpage

\subsection{Localizzazione}

\noindent Introduciamo un metodo per aumentare la struttura di un dominio di integrità.

\begin{defn}[]{}
Sia $R$ un dominio di integrità. Diciamo che $S\subseteq R$ è un \underline{sistema moltiplicativo} se:

\noindent (i) $1_R\in S$ e $0_R\not\in S;$

\noindent (ii) per ogni $a,b\in S$ anche $ab\in S$.
\end{defn}

\noindent Osserviamo che $S$ è un monoide commutativo rispetto all'operazione binaria di prodotto. Infatti, esso eredita l'associatività e la commutatività da $R$, la (ii) garantisce che $S$ è chiuso rispetto al prodotto e per la (i) sappiamo che $S$ contiene l'elemento neutro $1_R$. Tuttavia, $S$ non è sempre un gruppo, in quanto non richiediamo l'esistenza degli inversi moltiplicativi.

\begin{exm}L'insieme $S=\{2^n : n\in \mathbb{N}\}\subseteq \mathbb{Q}$ è un sistema moltiplicativo. Infatti, $2^0=1\in S,$ $0\not\in S$ per le proprietà dell'esponenziale, e presi $2^a,\,2^b \in S$ anche $2^a\cdot 2^b=2^{a+b}\in S$. Tuttavia, tale $S$ non è un sottogruppo di $\mathbb{Q}$ poiché ad esempio $2^{-1}=\frac{1}{2}\not\in S. \ \square$\end{exm}

\noindent Sull'insieme delle coppie $(r,s)\in R\times S$ definiamo la relazione $(r,s)\sim (t,u)\Leftrightarrow ru=st$.

\begin{prop}[1.7.1]{}
La relazione $\sim\colon (R\times S)\times (R\times S)\to \{\operatorname{v},\operatorname{f}\}$ è una relazione di equivalenza.\footnotemark
\end{prop}
\vspace{-4mm}
\begin{proof}
Chiaramente $(r,s)\sim (r,s)$ perché $rs=sr$, dunque $\sim$ è riflessiva. Inoltre, se $(r,s)\sim (t,u)$, allora $ru=st$, cioè $ts=ur$, da cui $(t,u)\sim (r,s)$ e $\sim$ è simmetrica. Siano $(r,s)\sim (t,u)$ e $(t,u)\sim (v,w)$. Allora, $ru=st$ e $tw=uv$, cioè, moltiplicando per $w$ entrambi i membri della prima uguaglianza, $ruw=s(tw)=s(uv) \Rightarrow ruw-suv=(rw-sv)u=0_R$. Poiché $R$ è un dominio di integrità e $u\neq 0_R$ essendo $u\in S$, si ha che $rw-sv=0_R$, cioè $rw=sv$, da cui $(r,s)\sim (v,w)$ e dunque $\sim$ è transitiva.
\end{proof}\footnotetext{La relazione $\sim$ restituisce vero ($\operatorname{v}$) o falso ($\operatorname{f}$) a seconda che le due coppie siano o meno in relazione. Ricordiamo che una relazione di equivalenza è R-S-T, cioè riflessiva, simmetrica e transitiva.}

\noindent Denotiamo con $\frac{r}{s}=[(r,s)]_{\sim}$ la classe di equivalenza dell'elemento $(r,s)$ rispetto a $\sim$, e sia $S^{-1}R=\{\frac{r}{s}: r\in R,s\in S\}=(R\times S)/{\sim}$ il quoziente di $R\times S$ rispetto a $\sim$.

\begin{defn}[]{}
Sia $R$ un dominio di integrità e sia $S$ un sistema moltipicativo di $R$. Allora, l'insieme $S^{-1}R=\{\frac{r}{s}: r\in R,s\in S\}$ è detto \underline{localizzazione di $R$ a $S$}.
\end{defn}

\noindent Il termine ``localizzazione'' deve il suo nome alla geometria algebrica.\footnote{Se $R$ è un anello di funzioni definito su un oggetto geometrico (come una varietà algebrica, cioè l'insieme delle soluzioni di un sistema di equazioni polinomiali) e vogliamo studiare tale varietà in un certo punto $x_0$, definiamo $S$ come l'insieme delle funzioni che non si annullano in $x_0$ e localizziamo $R$ a $S$. Allora, $S^{-1}R$ è un anello generalmente più semplice di $R$ che contiene informazioni solo sul comportamento della varietà in un intorno di $x_0$, da cui l'origine del termine ``locale''.} Dal punto di vista dell'algebra astratta, l'idea della localizzazione è quella di aggiungere ad un anello gli inversi moltiplicativi di alcuni suoi elementi introducendo delle ``frazioni'', in modo simile a quanto si fa nel passare dai numeri interi ai numeri razionali.

\noindent  Vogliamo ora dotare tale insieme $S^{-1}R$ delle operazioni binarie di somma e prodotto affinché sia un anello. Siano $\oplus\colon S^{-1}R\times S^{-1}R\to S^{-1}R$\, e \,$\odot \colon S^{-1}R\times S^{-1}R\to S^{-1}R$ definite come \[ \frac{r}{s}\oplus\frac{t}{u}=\frac{ru+st}{su} \ \ \, \text{ e } \, \ \ \frac{r}{s}\odot \frac{t}{u}=\frac{rt}{su}. \] Poiché $S^{-1}R$ è un insieme quoziente, per dimostrare che tali operazioni sono ben poste è necessario mostrare che il loro risultato non dipende dai rappresentanti delle classi di equivalenza. Per fare ciò, dimostriamo prima il seguente lemma.

\begin{lem}[1.7.2: Lemma della forbice]{}
Siano $X$ e $Y$ insiemi non vuoti e sia $f\colon X\to Y$ una mappa. Sia $\sim$ una relazione di equivalenza su $X$ e $\tau\colon X\to X/{\sim}$ la proiezione canonica.\footnotemark\,Allora, esiste una mappa $\overline{f}\colon X/{\sim}\to Y$ tale che $f=\overline{f}\circ \tau$ se e solo se $f(x)=f(y)$ per ogni $x,y\in X$ con $x\sim y$.
\end{lem}\footnotetext{Cioè la mappa che manda ogni elmento $x\in X$ nella sua classe di equivalenza $[x]_{\sim}$, che per comodità di notazione denoteremo di qui in seguito semplicemente con $[x]$.}
\vspace{-4mm}
\begin{proof}
Supponiamo che $f(x)=f(y)$ per ogni $x,y\in X$ con $x\sim y$. Per l'assioma della scelta,\footnote{L'assioma della scelta afferma che data una famiglia non vuota di insiemi non vuoti, esiste una funzione che ad ogni insieme della famiglia fa corrispondere un suo elemento. Per poter dimostrare questo lemma è necessario assumere tale assioma, di cui si fa uso nel definire la funzione $\sigma$, che altrimenti a priori non esisterebbe. Infatti, $X/{\sim}$ è la famiglia delle classi di equivalenza di $X$, ognuna delle quali è non vuota poiché $x\in [x]$, e $\sigma$ è la funzione che ad ogni classe $[x]\in X/{\sim}$ fa corrispondere un suo rappresentante $\sigma([x])\in X$.} esiste $\sigma\colon X/{\sim}\to X$ tale che $\sigma([x])\sim x$ per ogni $x\in X$ e $\tau \circ \sigma = \operatorname{id}_{X/{\sim}}$. Si consideri ora la funzione $\overline{f}=f\circ \sigma \colon X/{\sim}\to Y$.
\[
	\begin{tikzcd}[column sep=small]
X \arrow[rd, "\tau"] \arrow[rr, "f"] &                                                                               & Y \\
                                     & X/{\sim} \arrow[ru, "\overline{f}"'] \arrow[lu, "\sigma", dashed, bend left=40] &  
\end{tikzcd}
\]
Osserviamo innanzitutto che $\overline{f}$ è ben definita, poiché se $[x]=[y]$, allora $\overline{f}([x])=\overline{f}([y])$ perché $\sigma([x])\sim x\sim y\sim \sigma([y])$ e $f(\sigma([x]))=f(\sigma([y]))$ essendo per ipotesi $f$ costante sulle classi di equivalenza. Inoltre, $\left(\,\overline{f}\circ \tau\right) (x)=f(\sigma(\tau(x)))=f(\sigma([x]))=f(x)$ per ogni $x\in X$ poiché $\sigma([x])\sim x$, dunque è effettivamente vero che $f=\overline{f}\circ \tau$.

Viceversa, sia $\overline{f}\colon X/{\sim}\to Y$ tale che $f=\overline{f}\circ \tau$. Allora, per ogni $x,y \in X$ con $x\sim y$, cioè $[x]=[y]$, si ha che $f(x)=\overline{f}(\tau(x))=\overline{f}([x])=\overline{f}([y])=\overline{f}(\tau(y))=f(y)$ come desiderato.
\end{proof}

\noindent Sia $\widetilde{+}\colon (R,S)\times (R,S)\to S^{-1}R$ l'operazione binaria definita come $(r,s)\,\widetilde{+}\,(t,u)=\frac{ru+st}{su}$.
\[
	\begin{tikzcd}[column sep=small]
{(R,S)\times(R,S)} \arrow[rd, "\tau"'] \arrow[rr, "\widetilde{+}"] &                                                     & S^{-1}R \\
                                                                   & S^{-1}R\times S^{-1}R \arrow[ru, "\oplus"', dashed] &        
\end{tikzcd}
\]
Per verificare che l'operazione $\oplus$ esiste ed è ben posta, per il \emph{Lemma della forbice} è sufficiente mostrare che se $(r,s)\sim (r',s')$ e $(t,u)\sim (t',u')$, allora $(r,s)\,\widetilde{+}\,(t,u)=(r',s')\,\widetilde{+}\,(t',u')$, cioè $\frac{ru+st}{su}=\frac{r'u'+s't'}{s'u'}$. Poiché per definizione di $\sim$ si ha che $rs'=sr'$ e $tu'=ut'$, osserviamo che $(ru+st)s'u'=(rs')uu'+(tu')ss'=(sr')uu'+(ut')ss'=(r'u'+s't')su$. Dunque, vale $(ru+st,su)\sim (r'u'+s't',s'u')$, da cui $\frac{ru+st}{su}=\frac{r'u'+s't'}{s'u'}$. \clearpage
\noindent Analogamente, sia $\widetilde{\cdot}\,\colon (R,S)\times (R,S)\to S^{-1}R$ l'operazione definita come $(r,s)\,\widetilde{\cdot}\,(t,u)=\frac{rt}{su}$.
\[
	\begin{tikzcd}[column sep=small]
{(R,S)\times(R,S)} \arrow[rd, "\tau"'] \arrow[rr, "\widetilde{\cdot}"] &                                                    & S^{-1}R \\
                                                                       & S^{-1}R\times S^{-1}R \arrow[ru, "\odot"', dashed] &        
\end{tikzcd}
\]
Se $(r,s)\sim (r',s')$ e $(t,u)\sim (t',u')$, osserviamo che $rts'u'=(rs')(tu')=(sr')(ut')=sur't'$, dunque $(rt,su)\sim(r't',s'u')$. Allora, $(r,s)\,\widetilde{\cdot}\,(t,u)=\frac{rt}{su}=\frac{r't'}{s'u'}=(r',s')\,\widetilde{\cdot}\,(t',u')$, da cui per il \emph{Lemma della forbice} l'operazione $\odot$ esiste ed è ben posta.

\

\vspace{-2mm}

\noindent Per comodità di notazione, denoteremo di qui in seguito le due operazioni $\oplus$ e $\odot$ di $S^{-1}R$ semplicemente con $+$ e $\cdot$\,, rispettivamente.\footnote{Per quanto appena provato, possiamo effettivamente vedere tali operazioni come somma e prodotto di ``frazioni'' con le usuali regole di calcolo delle frazioni.}

\begin{prop}[1.7.3]{}
Sia $R$ un dominio di integrità e sia $S$ un sistema moltiplicativo di $R$. Allora, $S^{-1}R$ dotato di tali operazioni di somma e prodotto è un dominio di integrità.
\end{prop}
\vspace{-4mm}
\begin{proof}
Siano $\frac{r}{s}, \frac{t}{u}$ e $\frac{v}{w}$ elementi di $S^{-1}R$. Osserviamo innanzitutto che \[ \left(\frac{r}{s}+\frac{t}{u}\right)+\frac{v}{w}=\frac{ru+st}{su}+\frac{v}{w}=\frac{ruw+stw+suv}{suw}=\frac{r}{s}+\frac{tw+uv}{uw}=\frac{r}{s}+\left(\frac{t}{u}+\frac{v}{w}\right) \] da cui la somma è associativa. Inoltre, $\frac{r}{s}+\frac{t}{u}=\frac{ru+st}{su}=\frac{ts+ur}{us}=\frac{t}{u}+\frac{r}{s}$, dunque $(S^{-1}R,+)$ è un gruppo abeliano con elemento neutro $0_{S^{-1}R}=\frac{0_R}{1_R}$ e opposto $-\frac{r}{s}=\frac{-r}{s}$. Essendo \[ \left(\frac{r}{s}\cdot\frac{t}{u}\right)\cdot\frac{v}{w}=\frac{rt}{su}\cdot\frac{v}{w}=\frac{rtv}{suw}=\frac{r}{s}\cdot\frac{tv}{uw}=\frac{r}{s}\cdot\left(\frac{t}{u}\cdot\frac{v}{w}\right) \] e $\frac{r}{s}\cdot\frac{t}{u}=\frac{rt}{su}=\frac{tr}{us}=\frac{t}{u}\cdot\frac{r}{s}$, il prodotto è associativo e commutativo. Infine, \[ \left(\frac{r}{s}+\frac{t}{u}\right)\cdot \frac{v}{w}=\frac{ru+st}{su}\cdot \frac{v}{w}=\frac{ruv+stv}{suw}=\frac{rv}{sw}+\frac{tv}{uw}=\frac{r}{s}\cdot \frac{v}{w}+\frac{t}{u}\cdot \frac{v}{w} \] perché $\frac{rv}{sw}+\frac{tv}{uw}=\frac{ruvw+stvw}{suww}=\frac{ruv+stv}{suw}$ essendo $(ruvw+stvw)suw=(ruv+stv)suww$. Dunque, vale la proprietà distributiva e $(S^{-1}R,+,\cdot)$ è un anello commutativo con unità $1_{S^{-1}R}=\frac{1_R}{1_R}$. Resta da mostrare che $S^{-1}R$ non ha divisori dello zero. Siano $\frac{r}{s}, \frac{t}{u}\in S^{-1}R$ tali che $\frac{r}{s}\cdot \frac{t}{u}=0_{S^{-1}R}=\frac{0_R}{1_R}$. Allora $\frac{rt}{su}=\frac{0_R}{1_R}$, cioè $rt=(rt)1_R=(su)0_R=0_R$, da cui, essendo $R$ un dominio di integrità, $r=0$ oppure $t=0$, quindi $\frac{r}{s}=\frac{0_R}{s}=\frac{0_R}{1_R}=0_{S^{-1}R}$ oppure $\frac{t}{u}=\frac{0_R}{u}=\frac{0_R}{1_R}=0_{S^{-1}R}$. Dunque, $S^{-1}R$ è effettivamente un dominio di integrità.
\end{proof}

\noindent Sia $\iota_R\colon R\hookrightarrow S^{-1}R$ definita come $\iota_R(r)=\frac{r}{1_R}$ l'inclusione da $R$ a $S^{-1}R$. Osserviamo che $\iota_R$ è un omomorfismo di anelli iniettivo. Infatti, presi $x,y\in R$, si ha che \[\iota_R(x+y)=\frac{x+y}{1_R}=\frac{x}{1_R}+\frac{y}{1_R}=\iota_R(x)+\iota_R(y)\] \[\iota_R(xy)=\frac{xy}{1_R}=\frac{x}{1_R}\cdot \frac{y}{1_R}=\iota_R(x)\cdot \iota_R(y)\] e $\iota_R(0_R)=\frac{0_R}{1_R}=0_{S^{-1}R}$, $\iota_R(1_R)=\frac{1_R}{1_R}=1_{S^{-1}R}$. Inoltre, $\iota_R(r)=\iota_R(r')$ se e solo se $\frac{r}{1_R}=\frac{r'}{1_R}$, cioè $r=r'$. Dunque, $\iota_R$ è effettivamente un omomorfismo di anelli iniettivo. \clearpage

\noindent Vediamo ora alcuni esempi di sistemi moltiplicativi con le relative localizzazioni.

\begin{exm}Sia $R$ un dominio di integrità e sia $S=\{1_R\}$. Allora, $S$ è il più piccolo sistema moltiplicativo di $R$ e $S^{-1}R\simeq R$. Infatti, in questo caso l'inclusione $\iota_R\colon R\hookrightarrow S^{-1}R$ è anche suriettiva, perché preso $\frac{r}{1_R}\in S^{-1}R$ si ha che $\iota_R(r)=\frac{r}{1_R}$, ed è quindi un isomorfismo$. \ \square$\end{exm}

\begin{exm}Sia $R$ un dominio di integrità e sia $S=R^{\times}$. Poiché $R^{\times}$ è un gruppo rispetto al prodotto e $0_R\not\in R^{\times}$, tale $S$ è un sistema moltiplicativo di $R$ e $S^{-1}R\simeq R$ perché anche in questo caso l'inclusione $\iota_R\colon R\hookrightarrow S^{-1}R$ risulta essere suriettiva. Infatti, preso $\frac{r}{s}\in S^{-1}R$, poiché $s\in R^{\times}$, per definizione esiste $t\in R$ tale che $st=1_R$. Dunque, $\iota_R(rt)=\frac{rt}{1_R}=\frac{r}{s}$ essendo $(rt)s=r(1_R)$, da cui $\iota_R$ è un isomorfismo e $S^{-1}R\simeq R. \ \square$\end{exm}

\begin{exm}Sia $R$ un dominio di integrità e sia $\mathfrak{p}\lhd R$ un ideale primo.\footnote{Un ideale proprio $\mathfrak{p}\lhd R$ si dice primo se, presi $a,b\in R$, si ha che $ab\in \mathfrak{p}$ se e solo se $a\in \mathfrak{p}$ o $b\in \mathfrak{p}$.}\,Detto $S=R\setminus \mathfrak{p}$, osserviamo che $0_R\in \mathfrak{p}$, cioè $0_R\not\in S$, e se fosse $1_R \in\mathfrak{p}$, allora $\mathfrak{p}=\langle 1_R\rangle =R$ non sarebbe proprio,\footnote{In generale, se un ideale $I\lhd R$ contiene l'unità $1_R$, allora $r=r1_R\in I$ per ogni $r\in R$, cioè $I=R$.} da cui $1_R\in R\setminus \mathfrak{p}=S$. Inoltre, presi $a,b\in S$, se fosse $ab\in \mathfrak{p}$, essendo $\mathfrak{p}$ primo si avrebbe che $a\in \mathfrak{p}$ o $b\in\mathfrak{p}$, assurdo. Dunque, $ab\in S$ e $S$ è un sistema moltiplicativo di $R.$ Mostriamo ora che $S^{-1}R=(S^{-1}R)^{\times}\sqcup S^{-1}\mathfrak{p}$. Osserviamo che $S^{-1}R=S^{-1}(R\setminus \mathfrak{p}) \sqcup S^{-1}\mathfrak{p}$, dove tale unione è disgiunta poiché $S^{-1}(R\setminus \mathfrak{p}) \cap S^{-1}\mathfrak{p}=\emptyset$.\footnote{Sia $\frac{r}{s}\in S^{-1}\mathfrak{p}$; se fosse $\frac{r}{s}=\frac{t}{u}\in S^{-1}(R\setminus\mathfrak{p})$, essendo $r\in \mathfrak{p}$, si avrebbe che $ru=st\in \mathfrak{p}$. Dunque, essendo $\mathfrak{p}$ primo, dovrebbe essere $s\in \mathfrak{p}$ o $t\in\mathfrak{p}$, il che è assurdo essendo $s,t \in S=R\setminus\mathfrak{p}$.} Sia ora $\frac{r}{s}\in S^{-1}(R\setminus\mathfrak{p});$ allora, anche $\frac{s}{r}\in S^{-1}(R\setminus\mathfrak{p})$ e $\frac{r}{s}\cdot \frac{s}{r}=\frac{1_R}{1_R}=1_{S^{-1}R}$, cioè $\frac{r}{s}$ è invertibile, da cui $S^{-1}(R\setminus\mathfrak{p})\subseteq (S^{-1}R)^{\times}$. D'altra parte, se esistesse $\frac{r}{s}\in S^{-1}\mathfrak{p}$ invertibile, detto $\frac{t}{u}\in S^{-1}R$ il suo inverso si avrebbe $rt=su\in \mathfrak{p}$, il che è assurdo poiché $s,u \in S=R\setminus \mathfrak{p}$ violando la definizione di ideale primo. Dunque $(S^{-1}R)^{\times}\subseteq S^{-1}(R\setminus\mathfrak{p})$, da cui $S^{-1}R=S^{-1}(R\setminus \mathfrak{p}) \sqcup S^{-1}\mathfrak{p}=(S^{-1}R)^{\times}\sqcup S^{-1}\mathfrak{p}. \ \square$\end{exm}

\noindent Se $R$ è un dominio di integrità, $\{0_R\}\lhd R$ è un ideale primo perché $R$ non ha divisori dello zero, cioè $ab=0_R$ se e solo se $a=0_R$ oppure $b=0_R$. Dunque, per quanto visto nell'ultimo esempio, $S=R\setminus \{0_R\}$ è un sistema moltiplicativo di $R$ e $S^{-1}R=(S^{-1}R)^{\times} \sqcup \left\{ \frac{0_R}{1_R} \right\}$ è un dominio di integrità in cui ogni elemento non nullo è invertibile, cioè un campo.

\begin{defn}[]{}
Sia $R$ un dominio di integrità e sia $S=R\setminus \{0_R\}$. Allora, $S^{-1}R$ è un campo detto \underline{campo dei quozienti di $R$} e si denota con $\quot(R)$.
\end{defn}

\begin{exm}Se consideriamo $\mathbb{Z}$, si ha che $\quot(\mathbb{Z})=\left\{\frac{m}{n}: m\in \mathbb{Z}, n\in\mathbb{Z}\setminus\{0\}\right\}=\mathbb{Q}. \ \square$\end{exm}

\noindent I numeri razionali sono denotati con il simbolo $\mathbb{Q}$ proprio perché essi sono il ``quoziente''  dei numeri interi. Inoltre, $\mathbb{Z}$ è un sottoanello del campo $\mathbb{Q}=\quot(\mathbb{Z})$. Quest'ultimo è un fatto generale, come dimostrato dalla proposizione seguente.

\begin{prop}[1.7.4]{}
Ogni dominio di integrità è isomorfo a un sottoanello del suo campo dei quozienti.
\end{prop}
\vspace{-4mm}
\begin{proof}
Sia $R$ un dominio di integrità e sia $\iota_R\colon R\hookrightarrow \quot(R)$ l'inclusione. Poiché $\iota_R$ è un omomorfismo iniettivo, $\ker(\iota_R)=\{0_R\}$ e $\operatorname{Im}(\iota_R)$ è un sottoanello del campo $\quot(R)$. Dunque, per il \emph{Primo teorema d'isomorfismo} si ha che $R=R/\ker(\iota_R)\simeq \operatorname{Im}(\iota_R)$.
\end{proof}

\noindent In particolare, osserviamo che $\mathbb{Q}=\quot(\mathbb{Z})$ non solo contiene $\mathbb{Z}$ come sottoanello, ma è proprio il più piccolo campo contenente $\mathbb{Z}$. Infatti, se $\mathbb{K}$ è un campo contenente $\mathbb{Z}$, allora $n^{-1}=\frac{1}{n}\in \mathbb{K}$ per ogni $n\in \mathbb{Z}\setminus\{0\}$ e $m\cdot \frac{1}{n}\in \mathbb{K}$ per ogni $m\in \mathbb{Z}$, da cui $\mathbb{Q}\subseteq \mathbb{K}$. Anche questo è un fatto generale che caratterizza il campo dei quozienti di ogni dominio di integrità.

\begin{prop}[1.7.5]{}
Sia $R$ un dominio di integrità. Allora, il campo dei quozienti $\quot(R)$ è il più piccolo campo contenente un sottoanello isomorfo a $R$.
\end{prop}
\vspace{-4mm}
\begin{proof}
Osserviamo innanzitutto che per la \emph{Proposizione 1.7.4} sappiamo che $R$ è isomorfo al sottoanello $\operatorname{Im}(\iota_R)$ del campo $\quot(R)$. Sia quindi $\mathbb{K}$ un campo contenente $R$ e sia $\phi\colon \operatorname{quot(R)}\to \mathbb{K}$ la mappa definita come $\phi(\frac{r}{s})=rs^{-1}$. Tale mappa è ben definita: infatti, $r\in \mathbb{K}$ perché $r\in R\subseteq \mathbb{K}$, e in quanto campo $\mathbb{K}$ contiene anche tutti gli inversi $s^{-1}$ degli elementi $s\in R\setminus\{0_R\}$, da cui $rs^{-1}\in \mathbb{K}$. Inoltre, se $\frac{r}{s}=\frac{r'}{s'}$ per definizione vale $rs'=r's$, quindi $\phi(\frac{r}{s})=rs^{-1}=r's'^{-1}=\phi(\frac{r'}{s'})$. Siano ora $\frac{r}{s}, \frac{t}{u}\in \quot(R)$. Allora, si ha che \[\phi\left(\frac{r}{s}+\frac{t}{u}\right)=\phi\left(\frac{ru+st}{su}\right)=(ru+st)(su)^{-1}=rs^{-1}+tu^{-1}=\phi\left(\frac{r}{s}\right)+\phi\left(\frac{t}{u}\right)\] \[\phi\left(\frac{r}{s}\cdot \frac{t}{u}\right)=\phi\left(\frac{rt}{su}\right)=rt(su)^{-1}=rs^{-1}tu^{-1}=\phi\left(\frac{r}{s}\right)\cdot \phi\left(\frac{t}{u}\right)\] e $\phi(\frac{r}{s})=rs^{-1}=0_{\mathbb{K}}$ se e solo se $r=0_R$, da cui $\phi$ è un omomorfismo di campi iniettivo. Poiché $\operatorname{Im}(\phi)$ è un sottoanello del campo $\mathbb{K}$ e per il \emph{Primo teorema d'isomorfismo} si ha che $\operatorname{quot(R)}=\operatorname{quot(R)}/\ker(\phi)\simeq \operatorname{Im}(\phi)$, concludiamo che $\mathbb{K}$ contiene un sottoanello isomorfo a $\quot(R)$ ed è quindi un campo più grande del campo dei quozienti $\quot(R)$. 
\end{proof}

\noindent Sia $R$ un dominio di integrità e sia $S$ un sistema moltiplicativo di $R$. Vogliamo ora studiare le eventuali relazioni tra gli ideali di $R$ e quelli di $S^{-1}R$. Presi gli ideali $I\lhd R$ e $J\lhd S^{-1}R$, definiamo $S^{-1}I=\{\frac{i}{s}: i\in I, s\in S\}$ e denotiamo con $\overline{J}=\iota_R^{-1}(J)=\{r\in R: \frac{r}{1_R}\in J\}$.

\begin{prop}[1.7.6]{}
Sia $S$ un sistema moltiplicativo di un dominio di integrità $R$. Presi $I\lhd R$ e $J\lhd S^{-1}R$,

\noindent (a) $S^{-1}I\lhd S^{-1}R$ e $S^{-1}I=S^{-1}R$ se e solo se $I\cap S\neq \emptyset$;

\noindent (b) $\overline{J}\lhd R$ e $S^{-1}\overline{J}=J$.
\end{prop}
\vspace{-4mm}
\begin{proof}
(a) Siano $\frac{i}{s},\frac{j}{t}\in S^{-1}I$. Allora, $\frac{i}{s}+\frac{j}{t}=\frac{it+js}{st}\in S^{-1}I$ perché $it+js\in I$ per definizione di ideale e $st\in S$ per definizione di sistema moltiplicativo. Analogamente, $\frac{i}{s}\cdot \frac{j}{t}=\frac{ij}{st}\in S^{-1}I$ perché $ij\in I$ e $st\in S$, da cui $S^{-1}I$ è effettivamente un ideale di $S^{-1}R$. Osserviamo ora che se $S^{-1}I=S^{-1}R$, in particolare esiste un elemento $\frac{i}{s}\in S^{-1}I$ tale che $\frac{i}{s}=1_{S^{-1}R}=\frac{1_R}{1_R}$. Dunque, $i=i(1_R)=s(1_R)=s$, cioè $i=s\in I\cap S$, da cui $I\cap S\neq \emptyset$. Viceversa, supponiamo che $I\cap S\neq \emptyset$. Preso $t\in I\cap S$, si ha che $\frac{t}{t}=\frac{1_R}{1_R}=1_{S^{-1}R}\in S^{-1}I$, da cui, essendo $S^{-1}I$ un ideale, $\frac{r}{s}\cdot 1_{S^{-1}R}=\frac{r}{s}\in S^{-1}I$ per ogni $\frac{r}{s}\in S^{-1}R$, cioè $S^{-1}I=S^{-1}R$.

\vspace{1mm}

\noindent (b) Poiché $\iota_R\colon R\hookrightarrow S^{-1}R$ è un omomorfismo, la preimmagine $\iota_R^{-1}(J)\subseteq R$ di un ideale $J\lhd S^{-1}R$ è un ideale di $R$, cioè $\overline{J}\lhd R$. Preso $\frac{j}{s}\in S^{-1}\overline{J}$, per definizione si ha che $\frac{j}{1_R}\in J$. Quindi, essendo $J$ un ideale, $\frac{j}{s}=\frac{1_R}{s}\cdot \frac{j}{1_R}\in J$, da cui $S^{-1}\overline{J}\subseteq J$. D'altra parte, preso $\frac{r}{s}\in J$, per definizione di ideale si ha che $\frac{s}{1_R}\cdot \frac{r}{s}=\frac{r}{1_R}\in J$, cioè $r\in \overline{J}$. Dunque risulta $\frac{r}{s}\in S^{-1}\overline{J}$, da cui $J\subseteq S^{-1}\overline{J}$. Combinando le doppie inclusioni, si ha quindi che $S^{-1}\overline{J}=J$.
\end{proof}

\noindent Vi è quindi un legame tra la noetherianità di $R$ e quella di una sua localizzazione $S^{-1}R$.

\begin{cor}[1.7.7]{}
Sia $R$ un dominio di integrità noetheriano e sia $S$ un sistema moltiplicativo di $R$. Allora, anche $S^{-1}R$ è un dominio di integrità noetheriano.
\end{cor}
\vspace{-4mm}
\begin{proof}
Per la \emph{Proposizione 1.7.3} sappiamo che $S^{-1}R$ è un dominio di integrità, quindi è sufficiente provare che esso è anche noetheriano. Siano $J\lhd S^{-1}R$ e $\overline{J}=\iota_R^{-1}(J)\lhd R$. Essendo $R$ noetheriano, esistono $a_1,...\,,a_n\in R$ tali che $\overline{J}=\langle a_1,...\,,a_n \rangle$. Dunque, per la \emph{Proposizione 1.7.6} si ha che $J=S^{-1}\overline{J}=\{\frac{j}{s}: j\in\overline{J}, s\in S\}=\langle \frac{a_1}{1_R},...\,,\frac{a_n}{1_R} \rangle$ è finitamente generato, da cui per l'arbitrarietà di $J$ concludiamo che $S^{-1}R$ è noetheriano.
\end{proof}

\noindent Esiste un'importante famiglia di anelli strettamente legata al concetto di localizzazione.

\begin{defn}[]{}
Un anello commutativo $R$ si dice \underline{locale} se $\mathfrak{m}=R\setminus R^{\times}$ è un ideale di $R$.
\end{defn}

\noindent Un anello locale è quindi un anello i cui elementi non invertibili costituiscono un ideale.

\begin{exm}Ogni campo $\mathbb{K}$ è un anello locale. Infatti, $\mathbb{K}^{\times}=\mathbb{K}\setminus \{0_{\mathbb{K}}\}$ poiché per definizione di campo ogni elemento non nullo è invertibile, dunque $\mathfrak{m}=\mathbb{K}\setminus\mathbb{K}^{\times}=\{0_{\mathbb{K}}\}\lhd \mathbb{K}. \ \square$\end{exm}

\begin{exm}Sia $R$ un dominio di integrità e sia $\mathfrak{p}\lhd R$ un ideale primo. Detto $S=R\setminus \mathfrak{p}$, $S^{-1}R$ è un anello locale perché abbiamo mostrato che $S^{-1}R \setminus (S^{-1}R)^{\times}=S^{-1}\mathfrak{p}\lhd S^{-1}R. \ \square$\end{exm}

\begin{exm} Sia $\mathbb{K}$ un campo. Allora, l'anello $\mathbb{K}\llbracket x\rrbracket$ delle serie formali è un anello locale poiché per la \emph{Proposizione 1.4.X} si ha che $\mathfrak{m}=\mathbb{K}\llbracket x\rrbracket\setminus\mathbb{K}\llbracket x\rrbracket^{\times}=\langle x \rangle\lhd \mathbb{K}\llbracket x\rrbracket. \ \square$\end{exm}

\noindent Osserviamo che non tutti i domini di integrità sono anche anelli locali.

\begin{exm}Sia $\mathbb{K}$ un campo. Allora, $\mathbb{K}[x]$ non è un anello locale. Infatti, sappiamo che per la \emph{Proposizione 1.1.2} vale $\mathbb{K}[x]^{\times}=\mathbb{K}^{\times}$, da cui $\mathfrak{m}=\mathbb{K}[x]\setminus\mathbb{K}[x]^{\times}=\{f(x)\in \mathbb{K}[x]: \deg^{\star}(f)\geq 1\}$. Tuttavia, $\mathfrak{m}$ non è un ideale di $\mathbb{K}[x]$ poiché $f(x)=x+1_{\mathbb{K}}$ e $g(x)=x$ sono elementi di $\mathfrak{m}$ ma $h(x)=f(x)-g(x)=1_{\mathbb{K}}\not\in \mathfrak{m}$ perché $\deg^{\star}(h)=0. \ \square$\end{exm}

\noindent D'altra parte, esistono esempi di anelli locali che non sono domini di integrità.

\begin{exm}Esempio$. \ \square$\end{exm}

\noindent Esiste una caratterizzazione equivalente degli anelli locali in termini di ideali massimali.

\begin{prop}[1.7.8]{}
Sia $R$ un anello locale. Allora, $\mathfrak{m}=R\setminus R^{\times}$ è l'unico ideale massimale di $R$.
\end{prop}
\vspace{-4mm}
\begin{proof}
Osserviamo innanzitutto che $\mathfrak{m}\lhd R$ è massimale perché, preso $I\lhd R$ tale che $\mathfrak{m} \varsubsetneq I$, si ha che $I\setminus \mathfrak{m}\neq \emptyset$, cioè $I\cap R^{\times}\neq \emptyset$, da cui $I=R$ poiché $I$ contiene un elemento invertibile.\footnote{Infatti, se $I$ contiene $r\in R^{\times}$, detto $r^{-1}$ il suo inverso si ha che $r^{-1}r=1_R\in I$, da cui $I=R$.} D'altra parte, se $J\lhd R$ è un ideale massimale, per quanto appena visto deve essere $J\cap R^{\times}=\emptyset$, cioè $J\subseteq R\setminus R^{\times}=\mathfrak{m}$ che è già massimale, da cui $J=\mathfrak{m}$ e $\mathfrak{m}$ è unico.
\end{proof}

\clearpage

\noindent Possiamo quindi caratterizzare tutti e soli gli interi $n$ per cui $\mathbb{Z}/n\mathbb{Z}$ è un anello locale.

\begin{prop}[1.7.9]{}
L'anello $\mathbb{Z}/n\mathbb{Z}$ è locale se e solo se $n$ è la potenza di un primo.
\end{prop}
\vspace{-4mm}
\begin{proof}
Sia $n=p^k$ con $p$ primo e $k\geq 1$ intero. Osserviamo che $a+p^k\mathbb{Z}\in (\mathbb{Z}/p^k\mathbb{Z})^{\times}$ se e solo se esiste $b\in \mathbb{Z}$ tale che $ab\in 1+p^k\mathbb{Z}$, cioè $ab\equiv 1 \pmod{p^k}$. In particolare, vale $ab\equiv 1\pmod{p}$, da cui per l'\emph{Identità di Bézout} si ha che $\operatorname{MCD}(a,p)=1$.\footnote{Secondo l'\emph{Identità di Bézout}, dati due interi $a,b$ non entrambi nulli e detto $d=\operatorname{MCD}(a,b)$, esistono $x,y\in \mathbb{Z}$ tali che $ax+by=d$, e $d$ è il più piccolo intero che può essere scritto in questa forma. In questo caso, essendo $ab\equiv 1\pmod{p}$, esiste $t\in \mathbb{Z}$ tale che $ab+pt=1$, dunque $\operatorname{MCD}(a,p)\leq 1$, cioè $\operatorname{MCD}(a,p)=1$.}\,Si ha quindi che $(\mathbb{Z}/p^k\mathbb{Z})^{\times}=\{a+p^k\mathbb{Z}: \operatorname{MCD}(a,p)=1\}$, da cui $\mathfrak{m}=\mathbb{Z}/p^k\mathbb{Z}\setminus (\mathbb{Z}/p^k\mathbb{Z})^{\times}=p\mathbb{Z}/p^k\mathbb{Z}$ è un ideale di $\mathbb{Z}/p^k\mathbb{Z}$.\footnote{Il complementare di $(\mathbb{Z}/p^k\mathbb{Z})^{\times}$ in $\mathbb{Z}/p^k\mathbb{Z}$ è costituito da tutte le classi di equivalenza $a+p^k\mathbb{Z}$ per cui $\operatorname{MCD}(a,p)>1$, cioè, essendo $p$ primo, $\operatorname{MCD}(a,p)=p$. Dunque, $\mathfrak{m}=\{a+p^k\mathbb{Z}: p\mid a\}=p\mathbb{Z}/p^k\mathbb{Z}\lhd \mathbb{Z}/p^k\mathbb{Z}$.} Dunque, $\mathbb{Z}/p^k\mathbb{Z}$ è effettivamente un anello locale.

Viceversa, supponiamo per assurdo che esistano $p\neq q$ primi con $p\mid n$ e $q \mid n$. Poiché per il \emph{Terzo teorema d'isomorfismo} sappiamo che $(\mathbb{Z}/n\mathbb{Z})/(p\mathbb{Z}/n\mathbb{Z})\simeq \mathbb{Z}/p\mathbb{Z}$ che è un campo, $p\mathbb{Z}/n\mathbb{Z}\lhd \mathbb{Z}/n\mathbb{Z}$ è un ideale massimale.\footnote{Ricordiamo che preso un ideale $I\lhd R$, l'anello quoziente $R/I$ è un campo se e solo se $I$ è massimale.} Analogamente, anche $q\mathbb{Z}/n\mathbb{Z}\lhd \mathbb{Z}/n\mathbb{Z}$ è massimale, ed essendo $p\neq q$ tali ideali sono distinti.\footnote{Infatti, sono ideali finiti contenenti un numero diverso di elementi, essendo $|p\mathbb{Z}/n\mathbb{Z}|=\frac{n}{p}\neq \frac{n}{q}=|q\mathbb{Z}/n\mathbb{Z}|$.} Abbiamo quindi trovato due ideali massimali distinti di $\mathbb{Z}/n\mathbb{Z}$, da cui per la \emph{Proposizione 1.7.8} concludiamo che $\mathbb{Z}/n\mathbb{Z}$ non è locale.
\end{proof}

\noindent Sia $R$ un anello locale e sia $\mathfrak{m}=R\setminus R^{\times}$. Poiché per la \emph{Proposizione 1.7.8} l'ideale $\mathfrak{m}\lhd R$ è massimale, l'anello quoziente $R/\mathfrak{m}$ risulta essere un campo.

\begin{defn}[]{}
Sia $R$ un anello locale e sia $\mathfrak{m}=R\setminus R^{\times}$ il suo unico ideale massimale. Allora, il campo $\operatorname{res}(R)=R/\mathfrak{m}$ è detto \underline{campo dei residui di $R$}.
\end{defn}

\begin{exm}Sia $\mathbb{K}$ un campo. Poiché $\mathfrak{m}=\mathbb{K}\setminus \mathbb{K}^{\times}=\{0_{\mathbb{K}}\}$, si ha che $\operatorname{res}(\mathbb{K})=\mathbb{K}/\{0_{\mathbb{K}}\}\simeq \mathbb{K}. \ \square$\end{exm}

\begin{exm}Si consideri $\mathbb{Z}/p^k\mathbb{Z}$ con $p$ primo e $k\geq 1$ intero. Per quanto appena provato nella \emph{Proposizione 1.7.9}, si ha che $\mathfrak{m}=\mathbb{Z}/p^k\mathbb{Z}\setminus (\mathbb{Z}/p^k\mathbb{Z})^{\times}=p\mathbb{Z}/p^k\mathbb{Z}$, da cui per il \emph{Terzo teorema d'isomorfismo} otteniamo che $\operatorname{res}(\mathbb{Z}/p^k\mathbb{Z})=(\mathbb{Z}/p^k\mathbb{Z})/(p\mathbb{Z}/p^k\mathbb{Z})\simeq \mathbb{Z}/p\mathbb{Z}. \ \square$\end{exm}

\begin{exm}Sia $\mathbb{K}$ un campo. Poiché per la \emph{Proposizione 1.4.X} vale $\mathfrak{m}=\mathbb{K}\llbracket x\rrbracket\setminus\mathbb{K}\llbracket x\rrbracket^{\times}=\langle x \rangle$, si ha che $\operatorname{res}(\mathbb{K}\llbracket x\rrbracket)=\mathbb{K}\llbracket x\rrbracket/\langle x\rangle = \{a+\langle x\rangle : a\in \mathbb{K}\}\simeq \mathbb{K}. \ \square$\end{exm}

\noindent Nel caso in cui $\mathfrak{p}\lhd R$ sia un ideale primo del dominio di integrità $R$ e $S=R\setminus \mathfrak{p}$, la struttura del campo dei residui dell'anello locale $S^{-1}R$ risulta essere particolarmente interessante.

\begin{prop}[1.7.10]{}
Sia $R$ un dominio di integrità, $\mathfrak{p}\lhd R$ primo e $S=R\setminus \mathfrak{p}$. Allora, $\operatorname{res}(S^{-1}R)\simeq \quot(R/\mathfrak{p})$.
\end{prop}
\vspace{-4mm}
\begin{proof}
Dim
\end{proof}

\noindent Cose, sarebbe carino avere 1.7.10 con il relativo esempio in una pagina a sè stante.

\begin{exm}Esempio di mostrare che $S=\mathbb{Z}\setminus 13\mathbb{Z}$ è locale e calcolare $\operatorname{res}(S^{-1}\mathbb{Z})$.\end{exm}



\clearpage

\subsection{Domini a valutazione discreta}

Controllare la correttezza degli appunti delle lezioni del 22-23/10/2019 prima di copiare qui quella parte (incluso l'esempio sulla metrica p-adica).

\clearpage

\section{Teoria dei campi}
\vspace{1.75mm}
\subsection{Estensione di campi}

\noindent Introduciamo ora un concetto fondamentale nella teoria algebrica dei numeri e nello studio delle radici polinomiali, che costituirà la base della teoria di Galois.

\begin{defn}[]{}
Una coppia di campi $\mathbb{K}$ e $\mathbb{L}$ con $\mathbb{K}\subseteq \mathbb{L}$ si dice \underline{estensione di campi} e si denota con $\mathbb{L}/\mathbb{K}$.
\end{defn}

\noindent Resta inteso che $\mathbb{K}$ ha le stesse operazioni binarie di $\mathbb{L}$, cioè che $\mathbb{K}$ è un sottocampo di $\mathbb{L}$. Inoltre, in questo caso la notazione $\mathbb{L}/\mathbb{K}$ non ha nulla a che vedere con il quoziente di campi.

\begin{exm}Se consideriamo $\mathbb{R}$ e $\mathbb{C}$ con le usuali operazioni di somma e prodotto, $\mathbb{R}$ è un sottocampo di $\mathbb{C}$, dunque $\mathbb{C}/\mathbb{R}$ è un'estensione di campi$. \ \square$\end{exm}

\noindent Se $\mathbb{L}/\mathbb{K}$ è un'estensione di campi, sia $\cdot \raisebox{-.5em}{$\vert_{\mathbb{K}\times\mathbb{L}}$}$ la restrizione a $\mathbb{K}$ della prima componente del prodotto $\cdot \, \colon \mathbb{L}\times\mathbb{L}\to\mathbb{L}$ del campo $\mathbb{L}$. Considerando tale moltiplicazione per gli elementi di $\mathbb{K}$ e la usuale somma di $\mathbb{L}$, si ha che $(\mathbb{L},+,\cdot)$ ha la struttura di uno spazio vettoriale su $\mathbb{K}$. Infatti, possiamo pensare gli elementi di $\mathbb{K}$ come scalari e quelli di $\mathbb{L}$ come vettori.

\begin{defn}[]{}
Sia $\mathbb{L}/\mathbb{K}$ un'estensione di campi. Definiamo \underline{grado dell'estensione} $\mathbb{L}/\mathbb{K}$ la dimensione\footnotemark \ $\operatorname{dim}_{\mathbb{K}}(\mathbb{L})\in \mathbb{N}\cup \{\infty\}$ dello spazio vettoriale $\mathbb{L}$ sul campo $\mathbb{K}$, e si denota con $|\mathbb{L} : \mathbb{K}|$.
\end{defn}\footnotetext{Ricordiamo che la dimensione di uno spazio vettoriale è la cardinalità di una sua base, cioè un insieme di vettori linearmente indipendenti che generano tutto lo spazio.} 

\noindent La scelta del termine ``grado'', che richiama il concetto di grado di un polinomio, sarà più chiara in seguito, quando approfondiremo i legami tra estensione di campi e polinomi.
 
\begin{exm}Se consideriamo $\mathbb{Q}$, $\mathbb{R}$ e $\mathbb{C}$ con le usuali operazioni di somma e prodotto, si ha che $|\mathbb{C}: \mathbb{R}|=2$ perché $\mathcal{B}=\{1, i\}$ è una base per $\mathbb{C}$, e $|\mathbb{R}:\mathbb{Q}|=\infty$ perché $\mathbb{R}$ non è numerabile, quindi non ammette una base finita su $\mathbb{Q}$, che invece è numerabile$. \ \square$\end{exm}

\vspace{-2mm}

\begin{defn}[]{}
Sia $\mathbb{L}/\mathbb{K}$ un'estensione di campi. Un elemento $a\in \mathbb{L}$ si dice:

\noindent (i) \underline{algebrico} su $\mathbb{K}$ se esiste un polinomio non nullo $f(x)\in \mathbb{K}[x]$ tale che $f(a)=0$;

\noindent (ii) \underline{trascendente} su $\mathbb{K}$ se non è algebrico.
\end{defn}

\begin{exm}Se consideriamo $\mathbb{R}/\mathbb{Q}$, l'elemento $a=\sqrt{2}$ è algebrico perché $f(x)=x^2-2\in \mathbb{Q}[x]$ e $f(a)=0$, mentre $e$ e $\pi$ sono entrambi elementi trascendenti.\footnote{La dimostrazione è tutt'altro che elementare e prende il nome di \emph{Teorema di Lindemann-Weierstrass}.}$\, \square$\end{exm}

\noindent Sia $\mathbb{L}/\mathbb{K}$ un'estensione di campi e sia $a\in \mathbb{L}$. Detta $\phi_a\colon \mathbb{K}[x]\to \mathbb{L}$ la valutazione in $a$, essendo $\phi_a$ un omomorfismo si ha che $\ker(\phi_a)\lhd \mathbb{K}[x]$. SISTEMARE TUTTO.

\clearpage

\begin{defn}[]{}
Sia $\mathbb{L}/\mathbb{K}$ un'estensione di campi e sia $a\in \mathbb{L}$ un elemento algebrico su $\mathbb{K}$. Il generatore monico di $\ker(\phi_a)$ è detto \underline{polinomio minimo di $a$} e si denota con $\min_{a,\mathbb{K}}(x)\in \mathbb{K}[x]$.
\end{defn}

\begin{prop}[]{}
Sia $\mathbb{L}/\mathbb{K}$ un'estensione di campi e sia $a\in \mathbb{L}$ algebrico su $\mathbb{K}$. Sia $f(x)\in \mathbb{K}[x]$ tale che:

\noindent (i) $f(a)=0$;

\noindent (ii) $f(x)$ è monico;

\noindent (iii) $f(x)$ è irrudicibile.

\noindent Allora, $f(x)$ è il polinomio minimo di $a$, cioè $f(x)=\min_{a,\mathbb{K}}(x)$.
\end{prop}
\vspace{-4mm}
\begin{proof}
Per (i) si ha che $f(x)\in\ker(\phi_a)$, dunque esiste un polinomio $q(x)\in \mathbb{K}[x]$ tale che $f(x)=q(x)\cdot \min_{a,\mathbb{K}}(x)$. Essendo $f(x)$ irriducibile per (iii), almeno uno fra $q(x)$ e $\min_{a,\mathbb{K}}(x)$ è invertibile; tuttavia, $\min_{a,\mathbb{K}}(x)\not\in\mathbb{K}[x]^{\times}$ e quindi CONCLUDERE
\end{proof}

\noindent Sia $\mathbb{L}/\mathbb{K}$ un'estensione di campi e sia $S\subseteq \mathbb{L}$ un sottoinsieme.

\begin{prop}[]{}
Sia $\mathbb{L}/\mathbb{K}$ un'estensione di campi e sia $a\in \mathbb{L}$ algebrico su $\mathbb{K}$. Allora, $\mathbb{K}(a)=\mathbb{K}[a]$.
\end{prop}
\vspace{-4mm}
\begin{proof}
Sia $f(x)=\sum\limits_{i=0}^n c_i x^i\in \mathbb{K}[x]$. Poiché $c_i \in \mathbb{K}\subseteq \mathbb{K}(a)$ e $a\in \mathbb{K}(a)\Rightarrow a^k\in \mathbb{K}(a)$ essendo $\mathbb{K}(a)$ chiuso rispetto al prodotto, $f(a)\in \mathbb{K}(a)$. Dunque, per l'arbitrarietà di $f(x)$ concludiamo che $\operatorname{Im}(\phi_a)=\mathbb{K}[a]\subseteq \mathbb{K}(a)$. FINIRE, ESERCIZIO PER CASA XD COME SEI SIMPATICO
\end{proof}

\

\noindent Manca anche la lezione del 30/10/2019, al momento è solo cartacea, e contiene cose davvero molto importanti tipo la formula del grado.

\clearpage

\subsection{Estensioni finite}

\noindent \textbf{Lezioni del 05-06/11/2019} (appunti grezzi)

\begin{defn}[]{}
Un'estensione di campi $\mathbb{L}/\mathbb{K}$ si dice finita se $|\mathbb{L}:\mathbb{K}|<\infty$.
\end{defn}

\begin{prop}[]{}
Sia $\mathbb{L}/\mathbb{K}$ un'estensione finita. Allora, ogni elemento $a\in \mathbb{L}$ è algebrico su $\mathbb{K}$.
\end{prop}
\vspace{-4mm}
\begin{proof}
Sia $\phi_a\colon \mathbb{K}[x]\to \mathbb{L}$ la valutazione in $a$. Poiché $\dim_{\mathbb{K}}\mathbb{K}[x]=\infty$ (una base sono tutti i monomi $1, x, x^2, ...$) e $|\mathbb{L}:\mathbb{K}|=\dim_{K}(L)<\infty$, $\phi_a$ non è iniettiva, cioè $\ker(\phi_a)\neq \{0_K\}$. Dunque, per $f\in \ker(\phi_a)\setminus \{0_K\}$, si ha $f(a)=\phi_a(f)=0$.
\end{proof}

\begin{prop}[]{}
Sia $\mathbb{L}/\mathbb{K}$ un'estensione di campi, e sia $a\in \mathbb{L}$. Allora, sono equivalenti

\noindent (i) $a$ è algebrico su $\mathbb{K}$

\noindent (ii) $|\mathbb{K}(a)\colon \mathbb{K}|<\infty$

\noindent (iii) esiste un'estensione finita $M/K$, $M\subseteq L$ tale che $a\in M$
\end{prop}
\vspace{-4mm}
\begin{proof}
Per la proposizione precedente, sappiamo già che (iii) implica (i). Vediamo che (i) implica (ii). Infatti, $K[a]=\operatorname{Im}(\phi_a)\simeq K[x]/\ker(\phi_a)$ è un campo, e $K(a)=K[a]$ implica che $|K[a]\colon K|=|K[a] \colon K|=\deg^{\star}(\min_{a,K})<\infty$. Mostriamo ora che non (i) implica non (ii). Infatti, non (i) sse a è trascendente su $K$. Quindi, $\phi_a\colon K[x]\to L$ è iniettiva, e $K(a)\supseteq im(\phi_a)\simeq K[x]$ perché $K(a)$ contiene il sottospazio vettoriale $im(\phi_a)$ e $\dim_{K}(K(a))=\infty$. Dunque, il fatto che (i) implica (ii) e non (i) implica non (ii), sappiamo che (i) se e solo se (ii). Ma (ii) implica (iii) è banale: infatti prendo $M=K(a)$.
\end{proof}

\begin{defn}[]{}
Sia $\mathbb{L}/\mathbb{K}$ un'estensione di campi. Denotiamo con $\operatorname{alg}_{\mathbb{K}}(\mathbb{L})$ l'insieme degli elementi $a\in L$ algebrici su $K$.
\end{defn}

\begin{prop}[]{}
Sia $\mathbb{L}/\mathbb{K}$ un'estensione di campi. Allora, $\operatorname{alg}_{\mathbb{K}}(\mathbb{L})$ è un sottocampo di $\mathbb{L}$.
\end{prop}
\vspace{-4mm}
\begin{proof}
Siano $a,b\in \operatorname{alg}_{\mathbb{K}}(\mathbb{L})$. Basta dimostrare che $a+b, ab$ e $a^{-1}$ stanno in $\operatorname{alg}_{\mathbb{K}}(\mathbb{L})$. Poiché $a\in \operatorname{alg}_{\mathbb{K}}(\mathbb{L})$, per la proposizione 2 sappiamo che $|K(a):K|<\infty$. Poiché $b\in \operatorname{alg}_{\mathbb{K}}(\mathbb{L})$, esiste $\min_{b,K}(x)\in K[x]\subseteq K(a)[x]$. Dunque, $b$ è algebrico su $K(a)$, da cui \[|K(\{a,b\}): K|=|K(a)(b): K(a)|\cdot |K(a): K|<\infty\] per la Formula del grado. Poiché $a+b, ab, a^{-1}\in K(\{a,b\})$,per la Proposizione 2 sappiamo che $a+b, ab, a^{-1}\in \operatorname{alg}_{\mathbb{K}}(\mathbb{L})$.
\end{proof}

\noindent Trovare esplicitamente i polinomi che annullano $a+b, ab$ e $a^{-1}$ sarebbe stato un incubo!

\begin{defn}[]{}
Un campo $\mathbb{K}$ si dice algebricamente chiuso se ogni polinomio $f\in \mathbb{K}[x]$ con $\deg^{\star}(f)\geq 1$ ammette una radice.
\end{defn}

\begin{exm}Per il Teorema Fondamentale dell'Algebra (lui dice Teorema di Gauss) sappiamo che $\mathbb{C}$ è un campo algebricamente chiuso. La dimostrazione è tutt'altro che banale e richiede o l'analisi complessa o la Teoria di Galois$. \ \square$\end{exm}

\noindent Denotiamo con $\overline{\mathbb{Q}}=\operatorname{alg}_{\mathbb{Q}}(\mathbb{C})$. 

\begin{prop}[]{}
$\overline{\mathbb{Q}}$ è un campo algebricamente chiuso.
\end{prop}
\vspace{-4mm}
\begin{proof}
Sia $f=\sum\limits_{i=0}^n a_i x^i\in \overline{\mathbb{Q}}[x]$ con $\deg^{\star}(f)\geq 1$. Poiché $f\in C[x]$ essendo $\overline{\mathbb{Q}}\subseteq \mathbb{C}$, per il Teorema Fondamentale dell'Algebra esiste $c\in \mathbb{C}$ tale che $f(c)=0$. Definiamo $M=\overline{\mathbb{Q}}(\{a_0, a_1, ...\,,a_n\})$. Allora, per la formula del grado \[|M:Q|=|M:Q(\{a_0, ...\,,a_{n-1}\})|\cdot |Q(\{a_0, ...\,,a_{n-1}\}): Q(\{a_0, ...\,,a_{n-2}\})|\cdot ...\] Ma sappiamo che $|M:Q(\{a_0, ...\,,a_{n-1}\})|\leq \deg^{\star}(\min_{a_n,Q})$ e induttivamente $|M:Q|\leq \prod\limits_{i=0}^n \deg^{\star}(\min_{a_i, Q})$. Quindi, $|M(c):Q|=|M(c):M|\cdot |M:Q|$, dove $|M(c):M|\leq n$ e $|M:Q|\leq \infty$. Dunque, per la Proposizione 2 concludiamo che $c\in \overline{\mathbb{Q}}$.
\end{proof}

\begin{prop}[2.X.Y: Costruzione di Kronecker]{}
Sia $\mathbb{K}$ un campo e sia $f\in\mathbb{K}[x]$ con $\deg^{\star}(f)\geq 1$. Allora, esiste un'estensione $\mathbb{L}/\mathbb{K}_0$ finita e un elemento $a\in \mathbb{L}$ tale che $f(a)=0$, dove $\mathbb{K}_0\simeq \mathbb{K}$.
\end{prop}
\vspace{-4mm}
\begin{proof}
Poiché $K[x]$ è un dominio principale, possiamo scrivere $f=h\cdot f_0$ dove $h\in K[x]$ è primo e dunque irriducibile. Definiamo $L=K[x]/\langle h\rangle$. Poiché $\langle h\rangle \lhd K[x]$ è un ideale massimale, tale $L$ è un campo. Definiamo $K_0=\{b+\langle h\rangle : b\in K\}$, cioè $K_0=\pi(K)$, dove $\pi\colon K[x]\to L$ è la proiezione canonica. Poiché $\langle h\rangle$ è un ideale primo, $K\cap \langle h\rangle \{0_K\}$, quindi la restrizione $\pi_K\colon K\to K_0$ è un isomorfismo. Inoltre, $|L:K_0|=\deg^{\star}(h)<\deg^{\star}(f)<\infty$, quindi abbiamo trovato un'estensione finita. Sia $h=\sum\limits_{k=0}^n a_k x^k$, e sia $I=\langle h \rangle$. Detto $a=x+I\in L$, si ha che \[h(a)=\sum\limits_{k=0}^n a_k (x+I)^k=\sum\limits_{k=0}^n a_k (x^k+I)=\left(\sum\limits_{k=0}^n a_k x^k\right)+I=h+I=I=O_{L}.\] Questo mostra che per ogni polinomio troviamo $a\in L$ tale che $f(a)=0$, come desiderato.\end{proof}

\clearpage

\subsection{Campi di spezzamento}

\begin{defn}[]{}
Sia $\mathbb{K}$ un campo e sia $f\in\mathbb{K}[x]$ con $\deg^{\star}(f)=n\geq 1$. Un'estensione di campi $\mathbb{L}/\mathbb{K}$ si dice campo di spezzamento di $f\in \mathbb{K}[x]$ se esistono $c_1,...\,,c_n\in \mathbb{L}$ e $c_f\in \mathbb{K}$ tali che:

\noindent (i) $f(x)=c_f \prod\limits_{i=1}^n (x-c_i)\in \mathbb{L}[x]$;

\noindent (ii) $\mathbb{L}=\mathbb{K}(\{c_1, ...\,,c_n\})$.
\end{defn}

\noindent La condizione (i) ci dice che $f$ si spezza in fattori lineari su $\mathbb{L}[x]$, e la condizione (ii) serve a limitare la grandezza di $\mathbb{L}$.

\noindent Il Teorema seguente è il teorema di esistenza e unicità del campo di spezzamento.

\begin{teo}[2.3.1]{}
Sia $\mathbb{K}$ un campo e sia $f\in\mathbb{K}[x]$ non nullo con $\deg^{\star}(f)=n\geq 1$. Allora, 

\noindent (a) esiste $\mathbb{L}/\mathbb{K}$ campo di spezzamento di $f$;

\noindent (b) se $\mathbb{L}_1/\mathbb{K}$ e $\mathbb{L}_2/\mathbb{K}$ sono campi di spezzamento di $f$, esiste $\alpha\colon \mathbb{L}_1\to \mathbb{L}_2$ isomorfismo di campi tale che la restrizione $\alpha_{\mathbb{K}}=\operatorname{id}_{\mathbb{K}}$.
\end{teo}
\vspace{-4mm}
\begin{proof}
Parte (a). Idea: esiste $a\in \mathbb{L}$ tale che $f(a)=0$, quindi $f(x)=(x-a)\cdot f_0$, dove $\deg^{\star}(f_0)=\deg^{\star}(f)-1$, e uso induzione forte. Come faccio? Definisco $\mathcal{P}(n)$ l'affermazione seguente: 
\begin{itemize}
\item $\mathcal{P}(n)$: Sia $\mathbb{K}$ un campo e sia $f\in\mathbb{K}[x]$ con $\deg^{\star}(f)=n$. Allora, esiste un'estensione di campi $\mathbb{L}/\mathbb{K}$ e esistono $c_1,...\,,c_n\in \mathbb{L}$ e $c_f\in \mathbb{K}$ tali che:

\noindent (i) $f(x)=c_f \prod\limits_{i=1}^n (x-c_i)\in \mathbb{L}[x]$;

\noindent (ii) $\mathbb{L}=\mathbb{K}(\{c_1, ...\,,c_n\})$.
\end{itemize}
Osserviamo che $\mathcal{P}(1)$ è vera: detto $f=a_1 x+a_0$, basta prendere $L=K$ e $c_1=-\frac{a_0}{a_1}\in \mathbb{K}$. Infatti, $f=a_1 \cdot (x-c_1)$ e $L=K=K(c_1)$. Assumiamo ora che $\mathcal{P}(k)$ sia vera per ogni $k< n$. Per la Costruzione di Kronecker, esiste $\mathbb{L}/\mathbb{K}$ e $a\in L$ tale che $f(a)=0$. Detto $K_1=K(a)\subseteq L$, $f=(x-a)\cdot f_1$, dove $f_1\in K_1[x]$, poiché per ipotesi induttiva vale $\mathcal{P}(n-1)$, esiste $L/K_1$ e esistono $c_1, ...\,,c_{n-1}\in L$ tali che $f_1(x)=c_f \prod\limits_{i=1}^{n-1}(x-c_i)$ e $L=K_1(\{c_1,...\,,c_{n-1}\})$. Allora, considerando $L/K$ e $c_0=a$, si ha che $f=c_f\prod\limits_{i=0}^{n-1}(x-c_i)\in L[x]$ e $L=K_1(\{c_1,...\,,c_{n-1}\})=K(\{c_0,...\,,c_{n-1}\})$ , cioè $\mathcal{P}(n)$ è effettivamente vera.

\

\noindent Parte (b). Sia $\alpha\colon K_1\to K_2$ isomorfismo di campi. Definiamo $(-)^{\alpha}\colon K_1[x]\to K_2[x]$ che preso $f=\sum\limits_{k=0}^n a_k x^k$ lo manda in $f^{\alpha}=\sum\limits_{k=0}^n \alpha(a_k) x^k$. Allora, tale funzione è un isomorfismo di anelli. Definisco $\mathcal{P}(n)$ l'affermazione seguente: 
\begin{itemize}
\item $\mathcal{P}(n)$: Siano $\mathbb{K}_1, K_2$ campi e sia $f\in\mathbb{K}_1[x]$ non nullo con $\deg^{\star}(f)=n$. Allora, detto $\alpha\colon K_1\to K_2$ isomorfismo di campi, $L_1/K_1$ il campo di spezzamento di $f\in K_1[x]$ e $L_2/K_2$ il campo di spezzamento di $f^{\alpha}\in K_2[x]$, esiste $\alpha_{\star}\colon L_1\to L_2$ isomorfismo di campi tale che la restrizione $\alpha_{\star} \raisebox{-.5em}{$\vert_{K_1}$}=\alpha$.
\end{itemize}
Osserviamo che $\mathcal{P}(1)$ è vera, perché preso $f=a_1x+a_0=a_1(x-c_1)$, dove $c_1=-\frac{a_0}{a_1}\in K_1$, e scelgo $L_1=K_1$; inoltre, $f^{\alpha}=\alpha(a_1)x+\alpha(a_0)$ e $\alpha(c_1)=\frac{\alpha(a_0)}{\alpha(a_1)}$, cioè $f^{\alpha}=\alpha(a_1)(x-\alpha(c_1))$ e prendo $L_2=K_2$, quindi $\alpha\colon L_1\to L_2$ è l'isomorfismo tra campi richiesto. Supponiamo ora che $\mathcal{P}(k)$ sia vera per ogni $k<n$. Sia $f=h\cdot f_0$ di grado $n$ con $h\in K_1[x]$ irriducibile, $\deg^{\star}(h)>0$. Allora, perché $L_1/K_1$ è campo di spezzamento per $f_1$, esiste $c\in L_1$ tale che $h(c)=0$. Dunque, $h(x)=(x-c)\cdot h_0(x)\in L_1[x]$. Sia $M_1=K_1[c]$ (cioè, $K_1$ con l'aggiunta dell'elemento algebrico $c$), così che abbiamo $h(x)=(x-c)h_0(x)\in M_1[x]$. Allora, detto $f_1=h_0(x)\cdot f_0(x)\in M_1[x]$, si ha $\deg^{\star}(f_1)=n-1$. Da $f=h\cdot f_0$, deduciamo che $f^{\alpha}=h^{\alpha}\cdot f_0^{\alpha}$, dove $h^{\alpha}$ è anch'esso irriducibile (se fosse $h^{\alpha}=h_1\cdot h_2$, avremmo $h=h_1^{\alpha^{-1}}\cdot h_2^{\alpha^{-1}}$ dove $\alpha^{-1}\colon K_2\to K_1$ è la funzione inversa di $\alpha$). Poiché $L_2/K_2$ è campo di spezzamento di $f^{\alpha}$, esiste $d\in L_2$ tale che $h^{\alpha}(d)=0$. Detto $M_2=K_2[d]$, $L_1/M_1$ è campo di spezzamento di $f_1$ e $L_2/M_2$ è campo di spezzamento di $f_2=f_1^{\alpha}/(x-d)$. Resta da mostrare che esiste un isomorfismo di campi $\beta\colon M_1\to M_2$ tale che $\beta(c)=d$, e $\beta \raisebox{-.5em}{$\vert_{K_1}$}=\alpha$, perché in questo modo $f_1^{\beta}=f_2$. Infatti $f^{\beta}=(x-c)^{\beta}f_1^{\beta}=(x-d)\cdot f_2=f^{\alpha}$, da cui effettivamente $f_1^{\beta}=f_2$. Per ipotesi induttiva, poiché vale $\mathcal{P}(n-1)$, sappiamo che esiste un isomorfismo tra campi $\beta_{\star}\colon L_1\to L_2$ tale che $\beta_{\star} \raisebox{-.5em}{$\vert_{M_1}$}=\beta$, da cui $\beta_{\star} \raisebox{-.5em}{$\vert_{K_1}$}=\beta \raisebox{-.5em}{$\vert_{K_1}$}=\alpha$, e abbiamo concluso perché ora prendiamo $\alpha_{\star}=\beta_{\star}$, quindi vale $\mathcal{P}(n)$. A quanto pare serve un pezzo del Teorema seguente per concludere.
\end{proof}

\begin{prop}[2.3.2]{}
Sia $\alpha\colon K_1\to K_2$ isomorfismo di campi e siano $h \in K_1[x]$ irriducibile, $L_1/K_1$ estensione di campi, $c\in L_1$ tale che $h(c)=0$, $L_2/K_2$ estensione di campi, $d\in L_2$ tale che $h^{\alpha}(d)=0$. Allora, esiste un isomorfismo di campi $\beta\colon K_1[c]\to K_2[d]$ tale che $\beta \raisebox{-.5em}{$\vert_{K_1}$}=\alpha$.
\end{prop}
\vspace{-4mm}
\begin{proof}
Vedi appunti cartacei per diagramma commutativo da aggiungere; lui ha anche messo un asterisco a sx nelle funzioni ma non so come metterlo adesso. Senza perdita di generalità siano $h$ monico, cioè $h=\min_{c,K_1}$, e $h^{\alpha}$ monico (so già che è irriducibile), $h^{\alpha}(d)=0$, quindi prendo $h^{\alpha}=\min_{d, K_2}$. Siano $\phi_c\colon K_1[x]/\langle h\rangle \to K_1[c]$, $\phi_d\colon K_2[x]/\langle h^{\alpha}\rangle \to K_2[d]$ le mappe indotte dalle valutazioni, $\alpha\colon K_1[x]/\langle h\rangle \to K_2[x]/\langle h^{\alpha}\rangle$ la mappa indotta da $\alpha$ del Teorema 2.3.1 punto (b). Definisco $\beta=\phi_d\circ \alpha \phi_c^{-1}\colon K_1[c]\to K_2[d]$. Questa è un isomorfismo perché composizione di isomorfismi, in quanto tutte le funzioni definite precedentemente sono isomorfismi per il \emph{Primo teorema d'isomorfismo}. La verifica che $\beta \raisebox{-.5em}{$\vert_{K_1}$}=\alpha$ è banale. Boh, sta cosa è completamente delirante.
\end{proof}

\clearpage

\subsection{Campi finiti}

\noindent Sia $\mathbb{K}$ un campo e sia $\chi_{\mathbb{K}}\colon \mathbb{Z}\to \mathbb{K}$ definita come $\chi_{\mathbb{K}}(n)=\sum\limits_{i=1}^{n} 1_{\mathbb{K}}$ per $n\geq 0$ (si intende che $\chi_{\mathbb{K}}(0)=0_{\mathbb{K}}$) e $\chi_{\mathbb{K}}(-n)=-\chi_{\mathbb{K}}(n)$. Allora, $\chi_{\mathbb{K}}$ è un omomorfismo di anelli, quindi $\operatorname{Im}(\chi_{\mathbb{K}})\subseteq \mathbb{K}$ è un dominio di integrità, da cui $\ker(\chi_{\mathbb{K}})\lhd \mathbb{Z}$ è un ideale primo. 

\begin{defn}[]{}
Se $\ker(\chi_{\mathbb{K}})=\{0\}$, allora $\mathbb{K}$ si dice di caratteristica $0$, e si scrive $\operatorname{char} (\mathbb{K})=0$. Se $\ker(\chi_{\mathbb{K}})\neq \{0\}$, esiste un primo $p$ tale che $\ker(\chi_{\mathbb{K}})=p\mathbb{Z}$; in questo caso, $\mathbb{K}$ si dice di caratteristica $p$, e si scrive $\operatorname{char} (\mathbb{K})=p$.
\end{defn}

\noindent Fatto (lo chiamerò Lemma 2.3.3): Sia $K$ un campo finito. Allora $\chi_K$ non può essere iniettivo, quindi $K$ è di caratteristica $p$ per un primo $p$. 

\

\noindent (In realtà è una definizione) Se $K$ è un campo di caratteristica $p\neq 0$, $K_0=\operatorname{Im}(\chi_k)\subseteq K$ è un campo detto campo primo di $K$, ed è isomorfo a $Z/pZ$.

\begin{teo}[2.3.4]{}
Siano $E, F$ campi finiti tali che $|E|=|F|$. Allora, $E\simeq F$
\end{teo}
\vspace{-4mm}
\begin{proof}
Sia $|E|=|F|=q$. Per il Fatto, $\operatorname{char}(E)=p_1$ e $\operatorname{char}(F)=p_2$ per $p_1, p_2$ primi. Poiché $E_0\subseteq E$, detto $n_1=|E:E_0|$, si ha che $|E|=p_1^{n_1}$. Analogamente, poiché $F_0\subseteq F$, detto $n_2=|F:F_0|$ si ha che $|F|=p_2^{n_2}$. Dunque $p_1=p_2=p$ e $n_1=n_2=n$. La dimostrazione dell'isomorfismo continua dopo (dannazione è un sacco disorganizzato negli appunti
\end{proof}

\noindent Questo teorema non è valido per i gruppi e per gli anelli. Infatti, nei gruppi $|S_3|=|\mathbb{Z}/6\mathbb{Z}|=6$, ma uno è abeliano e l'altro no, quindi $S_3\not\simeq \mathbb{Z}/6\mathbb{Z}$. Negli anelli, $\mathbb{Z}/4\mathbb{Z}\not\simeq \mathbb{F}_2[x]/\langle x^2\rangle$ perché uno ha gruppo additivo ciclico e l'altro no.

\

\noindent Osservare come questo dice che ogni campo finito ha cardinalità $p^n$ per un primo $p$ e $n>1$. In realtà questo è un se e solo se, cioè, per ogni $p^n$ esiste un campo di ordine $p^n$. Due strade: considerare lo splitting field $E$ di $x^{p^n}-x$ su $\mathbb{Z}/p\mathbb{Z}$ e mostrare che $|E|=p^n$, oppure costruire un polinomio irriducibile $f(x)$ di grado $n$ in $\mathbb{F}_p[x]$ e considerare il quoziente $\mathbb{F}_p[x]/\langle f \rangle$.

\	
	
\noindent Da questo segue anche che se $E,F$ sono campi finiti tali che $|E|=|F|=q$, allora $E^{\times}\simeq F^{\times}\simeq \mathbb{Z}/(q-1)\mathbb{Z}$ (questo perché se $G$ è un gruppo abeliano il cui esponente è $\exp(G)=|G|$, significa che $G$ è ciclico).

\begin{prop}[2.3.5]{}
Sia $A\subseteq K^{\times}$ sottoanello di $K$ campo, $|A|<\infty$. Allora, $A$ è un gruppo ciclico.
\end{prop}
\vspace{-4mm}
\begin{proof}
Sia $n=\exp(A)$, e sia $f=x^n-1$. Allora, $A\subseteq Z_f(K)$ (sta indicando con $Z_f(K)$ qualcosa che ha a che fare con gli zeri...) da cui $|A|\leq \deg^{\star}(f)=n=\exp(A)$. Poiché $\exp(A)\mid |A|$, deve essere $|A|=\exp(A)$, cioè $A$ è ciclico.
\end{proof}

\noindent Conclusione dimostrazione 2.3.4: Sia $\psi_q=(x^{q-1}-1)x$. Allora $Z_{\psi_q}(E)=E$ e $Z_{\psi_q}(F)=F$, quindi $\psi_q(x)=\prod\limits_{\lambda\in E}(x-\lambda)=\prod\limits_{\mu \in F}(x-\mu)$. Dunque, $E/E_0$ e $F/F_0$ sono campi di spezzamento di $\psi_q\in \mathbb{F}_p[x]$. Dunque, per il punto (b) del Teorema 2.3.1 sappiamo che $E\simeq F$.

\

\noindent Notare come questo dimostra che non è ambiguo denotare con $\mathbb{F}_p$ il campo di cardinalità $p$ primo, perché esso è effettivamente l'unico!

\

\noindent\textbf{Lezione del 12/11/2019} (appunti grezzi)

\begin{defn}[]{}
Sia $\mathbb{K}$ un campo con $\operatorname{char}(\mathbb{K})=p$. Allora, la mappa $F\colon \mathbb{K}\to \mathbb{K}$ definita come $F(x)=x^p$ è un omomorfismo di campi detto omomorfismo di Frobenius.
\end{defn}

\noindent Osserviamo che tale mappa è effettivamente un omomorfismo. Infatti, $F(0_K)=0_K$, $F(1_K)=1_K$ e $F(x+y)=\sum\limits_{k=0}^p \binom{p}{k}x^ky^{p-k}=x^p+y^p=F(x)+F(y)$ perché $\binom{p}{k}$ è divisibile per $p$ se $0<k<p$. Infine, è evidente che $F(xy)=(xy)^p=x^py^p=F(x)F(y)$. 

\

\noindent (Diventerà un Lemma) Osserviamo anche che $F$ è iniettiva, e se $|\mathbb{K}|<\infty$ è pure un automorfismo. Infatti, sappiamo che $\ker(F)\lhd \mathbb{K}$, ed essendo $1\not\in \ker(F)$, $\ker(F)=\{0_K\}$ perché gli unici ideali di un campo sono $\{0_K\}$ e $K$. Dunque $F$ è iniettiva. Se vale anche $|K|<\infty$, essendo $F$ iniettiva su un insieme finito, è chiaramente anche suriettiva, da cui è biettiva e quindi un automorfismo.

\begin{prop}[]{}
Sia $K$ un campo finito, $p=\operatorname{char}(K)$ e $K_0=\operatorname{Im}\chi_K$. Se $|K|=p^n$, $n=|K:K_0|$, allora $\operatorname{ord}(F)=n$, dove $\operatorname{ord}(F)=n$ è l'ordine di $F$ pensata come elemento di $\operatorname{Sym}(K)$.
\end{prop}
\vspace{-4mm}
\begin{proof}
Sappiamo che $K/K_0$ è campo di spezzamento di $x^{p^n}-x$, cioè $K$ è l'insieme degli zeri di $x^{p^n}-x$, il che è vero se e solo se $z^{p^n}=F^n(z)=z$ per ogni $z\in \mathbb{K}$. Dunque, $F^n=\operatorname{id}_K$. Resta da verificare che $n$ è effettivamente il minimo intero positivo per cui $F^k$ sia l'identità. Sia quindi $k\in \mathbb{N}^+$ con $k<n$ tale che $F^k=\operatorname{id}_K$. Allora, $z^{p^k}=F^k(z)=z$ per ogni $z\in K$, cioè $K$ è l'insieme degli zeri di $x^{p^k}-x$. Essendo $\deg^{\star}(x^{p^k}-x)=p^k$, tale polinomio ha al più $p^k$ radici, cioè $|K|=|Z_K(x^{p^k}-p)|\leq p^k<p^n$, assurdo. Dunque $F^k\neq \operatorname{id}_K$.
\end{proof}

\noindent Il grado di un'estensione si può chiamare ordine perché è l'ordine di un automorfismo (nel caso dei campi finiti).

\clearpage

\section{Teoria dei moduli}
\vspace{1.75mm}
\subsection{Moduli}

Introduciamo ora il concetto di modulo, una generalizzazione del concetto di spazio vettoriale in cui gli scalari costituiscono un anello e non necessariamente un campo.

\begin{defn}[]{}
Sia $R$ un anello. Un gruppo abeliano $(M,\oplus)$ dotato di un'operazione $\ast\colon R\times M\to M$ si dice \underline{$R$-modulo sinistro} se per ogni $r,r_1,r_2\in R$ e $m,m_1,m_2\in M$ si ha che:

\noindent (i) $(r_1+r_2)\ast m=r_1\ast m \oplus r_2\ast m$ (distributività sinistra);

\noindent (ii) $r\ast (m_1\oplus m_2)=r\ast m_1\oplus r\ast m_2$ (distributività destra);

\noindent (iii) $r_1\ast (r_2\ast m)=(r_1r_2)\ast m$ (associatività);

\noindent (iv) $1_R\ast m=m$.
\end{defn}

\noindent Analogamente, un $R$-modulo destro è un gruppo abeliano $(M,\oplus)$ dotato di un'operazione $\ast \colon M\times R\to M$ per cui valgono proprietà analoghe ma con gli elementi di $R$ scritti a destra. Se $R$ è un anello commutativo, i concetti di $R$-modulo destro e sinistro coincidono.\footnote{Ogni modulo destro è isomorfo al corrispondente modulo sinistro, e si parla infatti di modulo bilatero.}

\begin{exm}Ogni spazio vettoriale $V$ su un campo $\mathbb{K}$ può essere pensato come un $\mathbb{K}$-modulo, dove $\ast\colon \mathbb{K}\times V\to V$ è la moltiplicazione per scalari. Viceversa, essendo $\mathbb{K}$ commutativo, ogni $\mathbb{K}$-modulo è bilatero e può quindi essere pensato come uno spazio vettoriale su $\mathbb{K}. \ \square$\end{exm}

\begin{exm}Ogni gruppo abeliano $G$ può essere visto come un modulo sull'anello degli interi. Si consideri l'operazione $\ast\colon \mathbb{Z}\times G\to G$ definita come $0\ast g=0_G$, $n\ast g=g+g+...+g$ (somma di $n$ termini) e $(-n)\ast g=-(n\ast g)$ per ogni $n>0$ e $g\in G$. Si verifica facilmente che $G$ dotato di tale operazione soddisfa le proprietà (i)-(iv) ed è quindi uno $\mathbb{Z}$-modulo$. \ \square$\end{exm} 

\begin{exm}Sia $R$ un anello e sia $I\lhd R$ un ideale sinistro. Allora, $I$ è un $R$-modulo sinistro, dove $\ast\colon R \times I\to I$ è il prodotto dell'anello $R$, ed è ben definito in quanto per definizione di ideale sinistro $r\ast a=ra\in I$ per ogni $r\in R$ e $a\in I. \ \square$\end{exm}

\begin{exm}Sia $R$ un anello e sia $n$ un intero positivo. Si consideri il prodotto cartesiano $R^n=\{(r_1,...\,,r_n): r_1,...\,,r_n\in R\}$ dotato della moltiplicazione componente per componente $\ast\colon R\times R^n\to R^n$ definita come $r\ast (r_1,...\,,r_n)=(rr_1,...\,,rr_n)$. Si verifica facilmente che $R^n$ dotato di tale operazione soddisfa le proprietà (i)-(iv) ed è quindi un $R$-modulo sinistro$. \ \square$\end{exm}

\noindent L'esempio seguente è particolarmente importante nell'algebra lineare perché permette di dimostrare l'esistenza della forma canonica razionale e di Jordan di una matrice.\footnote{Riprenderemo questo argomento dopo il \emph{Teorema di struttura per i gruppi abeliani finitamente generati}.}

\begin{exm}Sia $V$ uno spazio vettoriale su un campo $\mathbb{K}$ e sia $\alpha\in \operatorname{End}_{\mathbb{K}}(V)$ un endomorfismo di $V$. Preso $f(x)=\sum\limits_{i=0}^n a_i x^i \in \mathbb{K}[x]$, si consideri l'operazione $\ast_{\alpha}\colon \mathbb{K}[x]\times V\to V$ definita come $f\ast_{\alpha} v =f_{\alpha}(v)$, dove $f_{\alpha}=\sum\limits_{i=0}^n a_i \alpha^i \in \operatorname{End}_{\mathbb{K}}(V)$.\footnote{Ricordiamo che l'insieme degli endomorfismi di un gruppo è un anello secondo le operazioni di somma puntuale e di composizione di funzioni. In questo caso, $a_i \alpha^i$ è l'endomorfismo che mappa $v\mapsto a_i\cdot \alpha^i(v)$, dove $\alpha^i$ indica la composizione $\underbrace{\alpha\circ \alpha\circ ... \circ \alpha}_{i \text{ volte}}$, inteso che $\alpha^0=\operatorname{id}_{V}$.} Allora, si verifica facilmente che $V$ dotato di tale operazione soddisfa le proprietà (i)-(iv) ed è quindi un $\mathbb{K}[x]$-modulo sinistro$. \ \square$\end{exm}

\noindent Dimostriamo ora due proprietà dei moduli.

\begin{prop}[3.1.1]{}
Sia $R$ un anello e sia $M$ un $R$-modulo sinistro. Allora, 

\noindent (a) $0_R\cdot m=0_M$ per ogni $m\in M$;

\noindent (b) $r\cdot 0_M=0_M$ per ogni $r\in R$.
\end{prop}
\vspace{-4mm}
\begin{proof}
(a) Per la distributività sinistra $0_R\cdot m=(0_R+0_R)\cdot m=0_R\cdot m+0_R\cdot m$. Dunque, sommando l'opposto $-0_R\cdot m$ ad entrambi i membri, otteniamo che $0_M=0_R\cdot m$.

\vspace{0.5mm}

\noindent (b) Per la distributività destra $r\cdot 0_M=r\cdot (0_M+0_M)=r\cdot 0_M+r\cdot 0_M$. Dunque, sommando l'opposto $-r\cdot 0_M$ ad entrambi i membri, otteniamo che $0_M=r\cdot 0_M$.
\end{proof}

\begin{defn}[]{}
Sia $R$ un anello e sia $M$ un $R$-modulo sinistro. Un sottogruppo abeliano $A\subseteq M$ si dice $R$-sottomodulo di $M$ se $r\cdot a\in A$ per ogni $r\in R$ e $a\in A$.
\end{defn}

\noindent Un sottomodulo è quindi un sottogruppo abeliano $A\subseteq M$ per cui $(A, \cdot_{R\times A} \colon R\times A\to A)$ è di nuovo un $R$-modulo (sto quindi effettuando una restrizione dell'operazione $\cdot$).

\begin{prop}[3.1.2]{}
Sia $R$ un anello, $M$ un $R$-modulo sinistro e sia $A\subseteq M$ un $R$-sottomodulo. Allora, $(M/A, \cdot \colon R\times M/A\to M/A)$ è un $R$-modulo sinistro, ove $r\cdot (m+A)=r\cdot m+A$ e $\overline{f}(r,m+A)=r\cdot m+A$.
\end{prop}
\vspace{-4mm}
\begin{proof}
Diagramma negli appunti cartacei. La dimostrazione è inesistente, ottimo.
\end{proof}

\begin{prop}[3.1.3]{}
Sia $R$ un anello, $M$ un $R$-modulo sinistro, $A,B\subseteq M$ sono $R$-sottomoduli. Allora, $A+B=\{a+b: a\in A,b\in B\}$ è un $R$-sottomodulo di $M$.
\end{prop}
\vspace{-4mm}
\begin{proof}
Sappiamo già che $A+B\subseteq M$ è un sottogruppo abeliano. Siano $a+b\in A+B$ e $r\in R$. Allora, $r\cdot (a+b)=r\cdot a+r\cdot b\in A+B$ perché $r\cdot a\in A$ e $r\cdot b\in B$ per definizione di sottomodulo.
\end{proof}


\begin{prop}[3.1.4]{}
Sia $R$ un anello e $M$ un $R$-modulo sinistro. Per $m\in M$ sappiamo che $R\cdot m=\{r\cdot m: r\in R\}$ è un $R$-sottomodulo di $M$. Siano $m_1,...\,,m_n\in M$. Allora, $\sum\limits_{i=1}^n R\cdot m_i=R\cdot m_1+...+R\cdot m_n=\{m\in M: \exists r_1,...\,,r_n\in R: m=\sum\limits_{i=1}^n r_i\cdot m_i\}$ è un $R$-sottomodulo di $M$.  
\end{prop}
\vspace{-4mm}
\begin{proof}
Usando distributività sx, $R\cdot m$ è un sottogruppo abeliano. Usando associatività, si conclude mostrando che $R\cdot m$ è un $R$-sottomodulo. Ora procediamo per induzione grazie alla \emph{Proposizione 3.1.3}.
\end{proof}

\begin{defn}[]{}
Sia $R$ un anello e sia $M$ un $R$-modulo sinistro. Definiamo \underline{numero minimo di generatori} $d_R(M)$ il più piccolo $n\in \mathbb{N}$ per cui esistano $m_1,...\,,m_n\in M$ tali che $M=\sum\limits_{i=1}^n R\cdot m_i$ Se tale $n\in \mathbb{N}$ non esiste, poniamo $d_R(M)=\infty.$ Diciamo che $M$ è \underline{finitamente generato} se $d_R(M)<\infty.$
\end{defn}

\noindent\textbf{Lezione del 13/11/2019} (appunti grezzi)

\noindent Manca tutto un primo pezzo, Trenord ti voglio bene anche io

\noindent Esistono i corrispondenti dei 3 teoremi di isomorfismo per gli $R$-moduli.

\begin{teo}[3.x.y: Primo teorema d'isomorfismo]{}
Sia $\phi\colon M\to N$ un omomorfismo di $R$-moduli, dove $R$ è un anello. Allora, l'omomorfismo indotto $\phi_{\star}\colon M/\ker(\phi)\to \operatorname{Im}(\phi)$ è un isomorfismo di $R$-moduli.
\end{teo}
\vspace{-4mm}
\begin{proof}
Dimostrazione mancante.
\end{proof}

\begin{teo}[3.x.y: Secondo teorema d'isomorfismo]{}
Sia $R$ un anello, $M$ un $R$-modulo, e siano $A,B\subseteq M$ degli $R$-sottomoduli. Allora, esiste un isomorfismo di $R$-moduli $\pi_{\star}\colon A/(A\cap B)\to (A+B)/B$.
\end{teo}
\vspace{-4mm}
\begin{proof}
Sia $\tau\colon M\to M/B$ la proiezione canonica, cioè $\tau(m)=m+B$, e sia la restrizione $\tau_{A}=\pi$. Allora, per il \emph{Primo teorema d'isomorfismo} la mappa $\pi_{star}\colon A/\ker(\pi)\to \operatorname{Im}(\pi)$ è un isomorfismo. Poiché $\ker(\pi)=\ker(\tau)\cap A=B\cap A$ e $\operatorname{Im}(\pi)=\{a+B: a\in A\}=(A+B)/B$, abbiamo concluso.
\end{proof}

\begin{teo}[3.x.y: Terzo teorema d'isomorfismo]{}
Sia $R$ un anello, $M$ un $R$-modulo, $A\subseteq B\subseteq M$ degli $R$-sottomoduli. Allora, esiste un isomorfismo di $R$-moduli $\psi_{\star}\colon (M/A)/(B/A)\to M/B$.
\end{teo}
\vspace{-4mm}
\begin{proof}
Sia $\psi\colon M/A\to B/A$ la mappa definita come $\psi(m+A)=m+B$. Poiché $\psi$ è un omomorfismo di $R$-moduli, per il \emph{Primo teorema d'isomorfismo} la mappa indotta $\psi_{\star}\colon (M/A)/\ker(\psi)\to \operatorname{Im}(\psi)$ è un isomorfismo. Essendo $\operatorname{Im}(\psi)=\{m+B: m\in M\}=M/B$ e $\ker(\psi)=\{m+A: m\in M, m+B=B\}=\{m+A: m\in B\}=B/A$, abbiamo concluso.
\end{proof}

\begin{prop}[]{}
Sia $R$ un anello, $M$ un $R$-modulo sinistro e $B\subseteq M$ un $R$-sottomodulo di $M$. Allora $d_R(M)\leq d_R(B)+d_R(M/B)$ e $d_R(M/B)\leq d_R(M)$.
\end{prop}
\vspace{-4mm}
\begin{proof}
Se $B$ o $M/B$ non sono finitamente generati, cioè $d_R(B)=\infty$ o $d_R(M/B)=\infty$, la prima equazione è banalmente vera. Siano quindi $d_R(B)=k<\infty$ e $d_R(M/B)=n<\infty$. Allora, esistono $m_1,...\,,m_k\in B$ tali che $B=\sum\limits_{i=1}^{k} R\cdot m_i$ ed esistono $t_1,...\,,t_n\in M$ tali che $M/B=\sum\limits_{i=1}^n R\cdot (t_i+B)$. Dunque, per ogni $m\in M$ esistono $r_1,...\,,r_n\in R$ tali che $m+B=\sum\limits_{i=1}^n r_i\cdot (t_i+B)$, cioè $m-\sum\limits_{i=1}^n r_i\cdot t_i\in B$. Allora, esistono $s_1,...\,,s_k\in R$ tali che $m-\sum\limits_{i=1}^n r_i\cdot t_i=\sum\limits_{j=1}^k s_j\cdot m_j$, da cui $m=\sum\limits_{i=1}^n r_i\cdot t_i+\sum\limits_{j=1}^k s_j\cdot m_j$, cioè $d_R(M)\leq n+k$. 

Per quanto riguarda la seconda disuguaglianza, possiamo assumere che $d_R(M)=n<\infty$, altrimenti è banalmente vera. Dunque, esistono $m_1,...\,,m_n\in M$ tali che $M=\sum\limits_{i=1}^n R\cdot m_i$, quindi per ogni $m\in M$ esistono $r_1,...\,,r_n\in R$ tali che $m=\sum\limits_{i=1^n}r_i\cdot m_i$, da cui $m+B=\sum\limits_{i=1}^n r_i\cdot (m_i+B)$, e per l'arbitrarietà di $M$ significa che $M/B=\sum\limits_{i=1}^n R\cdot (m_i+B)$. Dunque, $d_R(M/B)\leq n$ come desiderato.
\end{proof}

\begin{prop}[]{}
Sia $R$ un anello commutativo. Allora, $R$ è noetheriano se e solo se ogni sottomodulo di un $R$-modulo finitamente generato è finitamente generato.
\end{prop}
\vspace{-4mm}
\begin{proof}
Procediamo per induzione su $d=d_R(M)$. Se $d=1$, esiste $m\in M$ tale che $M=R\cdot m$. Sia $\tau_m\colon R\to M$ la mappa definita come $\tau_m(r)=r\cdot m$. Osserviamo che $\tau_m(0)=0$, $\tau_m(r_1+r_2)=\tau_m(r_1)+\tau_m(r_2)$ e $\tau_m(r\cdot r_1)=r\cdot r_1\cdot m =r\cdot \tau_m(r_1)$, quindi $\tau_m$ è un omomorfismo di $R$-moduli. Sia $B\subseteq M$ un $R$-sottomodulo e sia $I_B=\{r\in R: \tau_m(r)\in B \}\subseteq R$. Poiché $I_B$ è un sottogruppo abeliano e presi $a\in I_B$ e $r\in R$ sappiamo che $r\cdot a\in I_B$ essendo $B$ un sottomodulo, vale $I_B\lhd R$. Dunque, essendo $R$ noetheriano per ipotesi, esistono $a_1,...\,,a_n\in I_B$ tale che $I_B=\langle a_1,...\,,a_n \rangle$. Poiché $B=\tau_m(I_B)=\operatorname{Im}(\tau_m \raisebox{-.5em}{$\vert_{I_B}$})$, per la proposizione precedente concludiamo che $d_R(B)<\infty$. Supponiamo ora per induzione forte che tale affermazione valga per $k\leq d$, e mostriamo che vale per $d+1$. Sia $M$ un $R$-modulo sinistro con $d_R(M)=d+1$. Allora, esistono $m_0,...\,,m_{d}\in M$ tali che $M=\sum\limits_{k=0}^d R\cdot m_k$. Sia $B\subseteq M$ un sottomodulo e sia $M_{\star}=\sum\limits_{k=1}^d R\cdot m_k$. Poiché $d_R(M_{\star})\leq d$, $M/M_{\star}=R\cdot (m_0+M_{\star})$. Sia $\pi\colon M\to M/M_{\star}$ la proiezione canonica, dove $d_{R}(M/M_{\star})\leq 1$. Per ipotesi induttiva, $d_R(B\cap M_{\star})<\infty$, quindi $d_R(\pi(B))<\infty$. Poiché $\pi(B)=(B+M_{\star})/M_{\star}\subseteq M/M_{\star}$, per la proposizione precedente $d_R(B)\leq d_R(B\cap M_{\star})+d_R(B/(B\cap M_{\star}))$. Ma per ipotesi induttiva sappiamo che $d_R(B\cap M_{\star})<\infty$ e $B/(B\cap M_{\star})\simeq \pi(B)$ per il \emph{Secondo teorema d'isomorfismo}, quindi $d_R(B/(B\cap M_{\star}))<\infty$ e $d_R(B)<\infty$, da cui la tesi. 

Viceversa, sia $M=R$ con il prodotto di $R$ (tale $R$-modulo è detto $R$-modulo regolare).\footnote{Sto pensando $M=R$ come gruppo abeliano secondo il prodotto di $R$, essendo $R$ commutativo.} Poiché $B\subseteq R$ è un sottomodulo se e solo se $B\lhd R$ è un ideale, per ipotesi sappiamo che $d_R(B)<\infty$ pensando $B$ come sottomodulo, cioè $d_R(B)<\infty$ pensando ora $B$ come ideale, da cui $R$ è noetheriano.
\end{proof}

\clearpage

\subsection{Torsione}

\noindent Introduciamo ora un concetto fondamentale nello studio degli $R$-moduli.

\begin{defn}[]{}
Sia $R$ un anello e sia $M$ un $R$-modulo sinistro. Un elemento $m\in M$ si dice \underline{elemento di} \underline{torsione} se esiste almeno un $r\in R\setminus \{0_R\}$ tale che $r\cdot m=0_M$. Un $R$-modulo sinistro si dice \underline{modulo di torsione} se ogni suo elemento è di torsione.
\end{defn}

\noindent Denotiamo con $\tor_R(M)$ l'insieme degli elementi di torsione di $M$. Allora, è evidente che $m\in M$ è di torsione se e solo se $m\in \tor(M)$, e $M$ è di torsione se e solo se $M=\tor(M)$.

\begin{exm}Aggiungere esempio con $\mathbb{Z}$ e $\mathbb{Q}$ dall'esame di settembre. $\quad$\end{exm}

\begin{prop}[3.2.1]{}
Sia $R$ un dominio di integrità e sia $M$ un $R$-modulo sinistro. Allora, $\tor(M)$ è un $R$-sottomodulo di $M$.
\end{prop}
\vspace{-4mm}
\begin{proof}
Siano $m,n\in \tor(M)$ e sia $r\in R$. Allora, esistono $s_m,s_n\in R\setminus \{0_R\}$ tali che $s_m\cdot m=0_M$ e $s_n\cdot n=0_M$. Poiché $R$ è un dominio di integrità, $s_m\cdot s_n\neq 0_R$. Dunque, $s_m\cdot s_n\cdot (m+n)=s_n\cdot (s_m\cdot m)+s_m\cdot (s_n\cdot n)=0_M$ per la distributività destra, da cui $m+n \in \tor(M)$. Inoltre, $s_m\cdot (r\cdot m)= r\cdot (s_m\cdot m)=r\cdot 0_M=0_M$, dunque $r\cdot m\in \tor(M)$.
\end{proof}

\begin{defn}[]{}
Sia $R$ un dominio di integrità, $M$ un $R$-modulo sinistro e sia $A\subseteq M$ un $R$-sottomodulo di $M$. Definiamo saturazione di $A$ in $M$ l'insieme $\sat_M(A)$ degli elementi $m\in M$ tali che esiste $r\in R\setminus \{0_R\}$ con $r\cdot m\in A$.
\end{defn}

\begin{prop}[3.2.2]{}
Sia $R$ un dominio di integrità, $M$ un $R$-modulo sinistro e sia $A\subseteq M$ un $R$-sottomodulo di $M$. Allora,

\noindent (a) $\sat_M(A)\subseteq M$ è un $R$-sottomodulo di $M$;

\noindent (b) $\tor(M)=\sat_M(\{0_R\})$;

\noindent (c) $\tor(M/A)=\sat_M(A)/A$.
\end{prop}
\vspace{-4mm}
\begin{proof}
(a) Siano $m,n\in \sat_M(A)$ e sia $r\in R$. Allora, esistono $s_m,s_n\in R\setminus \{0_R\}$ tali che $s_m\cdot m\in A$ e $s_n\cdot n\in A$. Poiché $R$ è un dominio di integrità, $s_m\cdot s_n\neq 0_R$. Dunque, $s_m\cdot s_n\cdot (m+n)=s_n\cdot (s_m\cdot m)+s_m\cdot (s_n\cdot n)\in A$ poiché somma di elementi di $A$, da cui $n+m\in \sat_M(A)$. Inoltre, $s_m\cdot (r\cdot m)=r\cdot (s_m\cdot m)\in A$ essendo $s_m\cdot m\in A$.

\vspace{0.25mm}

\noindent (b) Ovvio per definizione

\vspace{0.25mm}

\noindent (c) Per definizione, $\tor(M/A)=\{m+A: \exists r\in R\setminus \{0_R\}: r\cdot (m+A)=0_{M/A}\}$. Poiché $r\cdot (m+A)=r\cdot m+ A=0_{M/A}=A$ se e solo se $r\cdot m\in A$, si ha $\tor(M/A)=\{m+A: \exists r\in R\setminus \{0_R\}: r\cdot m\in A\}=\{m+A: m\in \sat_M(A)\}=\sat_M(A)/A$.
\end{proof}

\begin{defn}[]{}
Sia $R$ un anello, $M$ un $R$-modulo sinistro e sia $m\in M$. Allora, si dice annullatore di $m$ in $R$ l'insieme $\Ann_R(m)=\{r\in R: r\cdot m=0_M\}$.
\end{defn}

\noindent Osserviamo che $m$ è di torsione se e solo se $\Ann_R(m)\neq \{0_R\}$. Chiaramente, si intende che \[\Ann_R(M)=\{r\in R: r\cdot m=0_M \ \forall m\in M\}=\bigcap_{m\in M} \Ann_R(m)\] e tale insieme si dice annullatore globale di $M$ in $R$. Si osservi che $\Ann_R(M)\subseteq \Ann_R(m)$ per ogni $m\in M$ e che $\Ann_R(M)\lhd R$ (andrebbe dimostrato).

\

\noindent Aggiungere esempi, aggiungere dimostrazione del $\sat(\sat(A))$ usata più avanti, commenti.

\clearpage

\begin{prop}[3.2.3]{}
Sia $A$ uno $\mathbb{Z}$-modulo finitamente generato. Allora, sono equivalenti:

\noindent (i) $A$ è di torsione;

\noindent (ii) $\Ann_{\mathbb{Z}}(A)\neq \{0\}$;

\noindent (iii) $|A|<\infty$.
\end{prop}
\vspace{-4mm}
\begin{proof} Poiché $A$ è finitamente generato, siano $a_1,...\,,a_n\in A$ tali che $A=\sum\limits_{i=1}^n \mathbb{Z}\cdot a_i$. 
\noindent (i) $\Rightarrow$ (ii) Essendo $A$ di torsione, in particolare anche $a_1,...\,,a_n\in \tor_{\mathbb{Z}}(A)$, quindi esistono $k_1,...\,,k_n\in \mathbb{Z}\setminus\{0\}$ tali che $k_i\cdot a_i=0_A$. Sia $m=\operatorname{mcm}(k_1,...\,,k_n)\neq 0$ e sia $a\in A$. Allora, esistono $z_1,...\,,z_n\in \mathbb{Z}$ tali che $a=\sum\limits_{i=1}^n z_i\cdot a_i$, e $$m\cdot a=\sum\limits_{i=1}^n mz_i\cdot a_i=\sum\limits_{i=1}^n z_i\cdot (m\cdot a_i)=0_A$$ perché $m\cdot a_i=0_A$ per ogni $i=1,...\,,n$. Dunque $m\in \Ann_{\mathbb{Z}}(a)$, e per l'arbitrarietà di $a$ si ha che $m\in \Ann_{\mathbb{Z}}(A)$, cioè $\Ann_{\mathbb{Z}}(A)\neq \{0\}$.

\vspace{0.5mm}

\noindent (ii) $\Rightarrow$ (iii) Sia $\phi\colon \mathbb{Z}^n\to A$ la mappa definita come $\phi(z_1,...\,,z_n)=\sum\limits_{i=1}^n z_i\cdot a_i$. Poiché $\phi$ è un omomorfismo suriettivo,\footnote{Che $\phi$ sia un omomorfismo è evidente; la suriettività segue dal fatto che $A$ è generato da $a_1,...\,,a_n$, quindi per ogni $a\in A$ esistono $z_1,...\,,z_n\in \mathbb{Z}$ tali che $a=\sum\limits_{i=1}^n z_i\cdot a_i$.} per il \emph{Primo teorema d'isomorfismo} si ha che $\mathbb{Z}^n/\ker(\phi)\simeq A$. D'altra parte, $\Ann_{\mathbb{Z}}(a_1)\times ... \times \Ann_{\mathbb{Z}}(a_n)\subseteq \ker(\phi)$, dunque $$|A|=|\mathbb{Z}^n/\ker(\phi)|\leq \left|\mathbb{Z}^n /\bigoplus\limits_{i=1}^n \Ann_{\mathbb{Z}}(a_i)\right| = \left|\bigoplus\limits_{i=1}^n \mathbb{Z}\,/\Ann_{\mathbb{Z}}(a_i)\right|.\footnotemark$$\noindent Poiché $\{0\}\neq \Ann_{\mathbb{Z}}(A)\subseteq \Ann_{\mathbb{Z}}(a_i)\lhd \mathbb{Z}$, \footnotetext{La disuguaglianza segue dal fatto che $|\Ann_{\mathbb{Z}}(a_1)\times ... \times \Ann_{\mathbb{Z}}(a_n)|\leq |\ker(\phi)|$, e l'uguaglianza perché tali anelli sono isomorfi. Andrebbe spiegato meglio, di fatto dice che ad esempio $\mathbb{Z}^2/\langle (2,3)\rangle\simeq \mathbb{Z}/2\mathbb{Z}\oplus \mathbb{Z}/3\mathbb{Z}$.}siano $k_1,...\,,k_n\in \mathbb{Z}$ con $\Ann_{\mathbb{Z}}(a_i)=k_i\mathbb{Z}$.\footnote{Infatti, essendo $\mathbb{Z}$ un PID, i suoi ideali sono tutti e soli quelli della forma $k\mathbb{Z}$ al variare di $k\in \mathbb{Z}$.} Allora, $$|A|\leq \left|\bigoplus\limits_{i=1}^n \mathbb{Z}\,/\Ann_{\mathbb{Z}}(a_i)\right| = k_1\cdot ...\cdot k_n<\infty.$$

\vspace{0.5mm}

\noindent (iii) $\Rightarrow$ (i) Sia $a\in A$ e sia $\phi_a\colon \mathbb{Z}\to A$ la mappa definita come $\phi_a(z)=z\cdot a$. Poiché $\phi_a$ è un omomorfismo di $\mathbb{Z}$-moduli e $\mathbb{Z}/\ker(\phi_a)\simeq A$ per il \emph{Primo teorema d'isomorfismo}, essendo $|\mathbb{Z}|=\infty$ e $|A|<\infty$, per il \emph{Principio dei cassetti} deve essere $\ker(\phi_a)\neq \{0\}$. Dunque $\ker(\phi_a)=\Ann_{\mathbb{Z}}(a)\neq \{0\}$, cioè $a$ è un elemento di torsione, e per l'arbitrarietà di $a$ concludiamo che $\tor_{\mathbb{Z}}(A)=A$, da cui $A$ è di torsione.
\end{proof}

\noindent Aggiungere da qualche parte la dimostrazione dell'isomorfismo citato nel punto 2. A questo punto però tanto vale usare la stessa strategia della proposizione seguente, cioè definire $\phi\colon \bigoplus\limits_{i=1}^n \mathbb{Z}/k_i\mathbb{Z}\to A$ la mappa $\phi(z_1+k_1\mathbb{Z}, ...\,, z_n+k_n\mathbb{Z})=\sum\limits_{i=1}^n z_i\cdot a_i$ e ragionando come sotto mostrare che è ben posto, suriettivo, e quindi $|A|\leq \left|\bigoplus\limits_{i=1}^n \mathbb{Z}/k_i\mathbb{Z}\right|=k_1\cdot ...\cdot k_n<\infty$. 

\clearpage

\noindent Vale una proposizione simile alla precedente anche nel caso dei $\mathbb{K}[x]$-moduli.

\begin{prop}[3.2.4]{}
Sia $\mathbb{K}$ un campo e sia $M$ un $\mathbb{K}[x]$-modulo sinistro finitamente generato. Allora, sono equivalenti:

\noindent (i) $M$ è di torsione;

\noindent (ii) $\Ann_{\mathbb{K}[x]}(M)\neq \{0_{\mathbb{K}}\}$;

\noindent (iii) $\operatorname{dim}_{\mathbb{K}}(M)<\infty$.
\end{prop}
\vspace{-4mm}
\begin{proof}Per ipotesi, esistono $m_1,...\,,m_n\in M$ tali che $M=\sum\limits_{i=1}^n \mathbb{K}[x]\cdot m_i$.

\noindent (i) $\Rightarrow$ (ii) Essendo $M$ di torsione, anche $m_1,...\,,m_n\in \tor_{\mathbb{K}[x]}(M)$, quindi esistono polinomi $f_1,...\,,f_n\in \mathbb{K}[x]\setminus\{0_{\mathbb{K}}\}$ tali che $f_i\cdot m_i=0_M$. Sia $g=\operatorname{mcm}(f_1,...\,,f_n)\neq 0_{\mathbb{K}}$ e sia $m\in M$. Allora, esistono $q_1,...\,,
q_n\in \mathbb{K}[x]$ tali che $m=\sum\limits_{i=1}^n q_i\cdot m_i$, e $$g\cdot m=\sum\limits_{i=1}^n (g \cdot q_i)\cdot m_i=\sum\limits_{i=1}^n q_i\cdot (g\cdot m_i)=0_M$$ perché $g\cdot m_i=0_M$ per ogni $i=1,...\,,n$. Dunque $g\in \Ann_{\mathbb{K}[x]}(m)$, e per l'arbitrarietà di $m$ si ha che $g\in \Ann_{\mathbb{K}[x]}(M)$, cioè $\Ann_{\mathbb{K}[x]}(M)\neq \{0_{\mathbb{K}}\}$.

\vspace{1.5mm}

\noindent (ii) $\Rightarrow$ (iii) Poiché $\{0_{\mathbb{K}}\}\neq \Ann_{\mathbb{K}[x]}(M)\subseteq \Ann_{\mathbb{K}[x]}(m_i)\lhd \mathbb{K}[x]$, sappiamo che esistono dei polinomi $f_1,...\,,f_n\in \mathbb{K}[x]\setminus \{0_{\mathbb{K}}\}$ tali che $\Ann_{\mathbb{K}[x]}(m_i)=\langle f_i\rangle$.\footnote{Infatti, essendo $\mathbb{K}[x]$ un PID, i suoi ideali sono tutti e soli quelli della forma $\langle f\rangle$ al variare di $f\in \mathbb{K}[x]$.} Sia $\phi\colon \bigoplus\limits_{i=1}^n \mathbb{K}[x]/\langle f_i\rangle\to M$ la mappa definita come $\phi(q_1+\langle f_1\rangle, ...\,, q_n+\langle f_n\rangle)=\sum\limits_{i=1}^n q_i\cdot m_i$. Poiché $\phi$ è un omomorfismo di $\mathbb{K}[x]$-moduli suriettivo,\footnote{Andrebbe dimostrato che $\phi$ è ben posto, il che segue dall'aver scelto come $f_i$ i generatori degli annullatori e ragionando componente per componente: se $q_i+\langle f_i\rangle = r_i +\langle f_i\rangle$, allora $q_i=r_i+h f_i$ per un certo $h\in \mathbb{K}[x]$, e la restrizione di $\phi$ alla $i$-esima componente è $\phi_i(q_i)=(r_i+hf_i)\cdot m_i=r_i\cdot m_i+hf_i\cdot m_i=r_i\cdot m_i=\phi_i(r_i)$. La suriettività invece risulta evidente dalla definizione.} essendo $\operatorname{dim}_{\mathbb{K}}(\mathbb{K}[x]/\langle f_i\rangle)=\deg^{\star}(f_i)$ concludiamo che $$\operatorname{dim}_{\mathbb{K}}(M)\leq \operatorname{dim}_{\mathbb{K}}\left(\bigoplus\limits_{i=1}^n \mathbb{K}[x]/\langle f_i\rangle \right)=\prod\limits_{i=1}^n \deg^{\star}(f_i)<\infty.$$


\vspace{0.5mm}

\noindent (iii) $\Rightarrow$ (i) Sia $m\in M$ e sia $\phi_m\colon \mathbb{K}[x]\to M$ la mappa definita come $\phi_m(f)=f\cdot m$. Poiché $\phi_m$ è un omomorfismo di $\mathbb{K}[x]$-moduli e per il \emph{Primo teorema d'isomorfismo} vale $\mathbb{K}[x]/\ker(\phi_m)\simeq M$, essendo $\operatorname{dim}_{\mathbb{K}}(\mathbb{K}[x])=\infty$ e $\operatorname{dim}_{\mathbb{K}}(M)<\infty$, deve essere $\ker(\phi_m)\neq \{0_{\mathbb{K}}\}$. Dunque $\ker(\phi_m)=\Ann_{\mathbb{K}[x]}(m)\neq \{0_{\mathbb{K}}\}$, cioè $m$ è un elemento di torsione, e per l'arbitrarietà di $m$ concludiamo che $\tor_{\mathbb{K}[x]}(M)=M$, da cui $M$ è di torsione.
\end{proof}

\noindent Aggiungere qualche commento e spostare l'osservazione finale (vedi foto) nel capitolo sugli endomorfismi. Qualche esempio pratico? Se mi viene in mente lo aggiungo.

\clearpage

\subsection{Endomorfismi}

\noindent \textbf{Lezione del 20/11/2019} (appunti grezzi)

\noindent Oggi parliamo del polinomio minimo di un endomorfismo di uno spazio vettoriale di dimensione finita.

\

\noindent Sia $K$ un campo, $V$ un $K$-spazio vettoriale con $\operatorname{dim}_K(V)<\infty$ e sia $\alpha\in \operatorname{End}_K(V)$. Sia $\phi_{\alpha}\colon K[x]\to \operatorname{End}_K(V)$ definita come $\phi_{\alpha}(f)=f(\alpha)$, cioè, preso $f(x)=\sum\limits_{k=0}^n a_kx^k$, $\phi_{\alpha}(f)=\sum\limits_{k=0}^n a_k \alpha^k$, dove $\alpha^k$ indica la composizione $k$ volte inteso che $\alpha^0=\operatorname{id}_V$. Allora, $\phi_{\alpha}$ è una mappa $K$-lineare, perché $\phi_{\alpha}(\lambda f+\mu h)=\lambda \phi_{\alpha}(f)+\mu\phi_{\alpha}(g)$ per ogni $\lambda,\mu\in K$ e per ogni $f,g\in K[x]$. Inoltre, tale mappa è un omomorfismo di anelli, essendo $\phi_{\alpha}(f\cdot g)=f(\alpha)\circ g(\alpha)$. Dunque, essendo $\operatorname{dim}(K[x])=\infty$ e $\dim(\operatorname{End}_K(V))=\dim_{K}(V)^2$, per il principio dei cassetti $\phi_{\alpha}$ non può essere iniettiva, cioè $\ker(\phi_{\alpha})\neq \{0_{K}\}$. Poiché $\ker(\phi_{\alpha})\lhd K[x]$ è non banale, esiste un unico generatore monico $\min_{\alpha}(x)\in \ker(\phi_{\alpha})$ cioè $\ker(\phi_{\alpha})=\langle \min_{\alpha}(x) \rangle$.

\begin{defn}[]{}
Tale polinomio $\min_{\alpha}(x)$ si dice polinomio minimo dell'endomorfismo $\alpha\in \operatorname{End}_K(V)$.
\end{defn}

\noindent Vogliamo ora fare due cose: innanzitutto capire come calcolare il polinomio minimo, e poi, analogamente a GAL, trovare un'opportuna base $\mathcal{B}$ di $V$ tale che $[\alpha]_{\mathcal{B}}$ abbia una forma piacevole (Teorema di Jordan). Adesso ci dedichiamo a fare la prima cosa. Per fare la seconda cosa, c'è un teorema molto generale detto Teorema fondamentale per moduli finitamente generati su un dominio a ideali principali. Applicando questo teorema a $(V,\ast_{\alpha})$ proveremo il Teorema di Jordan (per $K$ campo algebricamente chiuso), e applicandolo a $\mathbb{Z}$ troveremo il Teorema per gruppi abeliani finitamente generati. Inoltre, c'è un altro teorema detto di Decomposizione primaria che permette la caratterizzazione deglii endomorfismi diagonalizzabili. Tale seconda cosa è molto complessa, e ci staremo sopra fino a Natale. 

\begin{teo}[3.X.Y: Teorema di Cayley-Hamilton]{}
Sia $V$ un $K$-spazio vettoriale con $\dim_K(V)<\infty$ e sia $\alpha\in \operatorname{End}_K(V)$. Allora, $\min_{\alpha}(x)$ è un divisore del polinomio caratteristico $\operatorname{char}_{\alpha}(x)=\det(\alpha-x\cdot \operatorname{id}_V)$.
\end{teo}
\vspace{-4mm}
\begin{proof}
Basta provare che (non ha detto niente lol).
\end{proof}

\noindent Mettiamo a posto qualche pezzo di ieri, quando ha usato la somma diretta come se niente fosse. Sia $R$ un anello e siano $M$ e $N$ degli $R$-moduli sinistri. Allora, $M\oplus N=\{(m,n): m\in M,n\in N\}$ è un $R$-modulo sx, ove $(m_1,n_1)+(m_2,n_2)=(m_1+m_2,n_1+n_2)$ e $r\cdot (m,n)=(r\cdot m,r\cdot n)$. Analogamente, se $M_1,...\,,M_k$ sono $R$-moduli sinistri, poniamo $\oplus_{i=1}^n M_i=M_1\oplus ... \oplus M_k=\{(m_1,...\,,m_k): m_i\in M_i\}$ e questo è un $R$-modulo sx con le ovvie operazioni $(m_1,...\,,m_k)+(m'_1,...\,,m'_k)=(m_1+m'_1,...\,,m_k+m'_k)$ e $r\cdot (m_1,...\,,m_k)=(r\cdot m_1,...\,,r\cdot m_k)$.

\

\noindent Dimostriamo ora la proposizione che è l'analoga di quella di teoria dei gruppi, che serve per dimostrare che il prodotto diretto interno è isomorfo al prodotto diretto esterno sotto ipotesi ragionevoli (tra l'altro la seconda parte è più bella di come la sta facendo lui).

\begin{prop}[]{}
Sia $R$ un anello, $M$ un $R$-modulo sinistro e siano $A,B\subseteq M$ degli $R$-sottomoduli tali che $A\cap B=\{0_M\}$. Allora, $A+B\simeq A\oplus B$. In generale, se ho $A_1,...\,,A_k$ che sono $R$-sottomoduli di $M$ tali che $A_j \cap \sum\limits_{i\neq j}A_i=\{0_M\}$ per ogni $j=1,...\,,k$, ho che $\sum\limits_{i=1}^k A_i \simeq \oplus_{i=1}^k A_i$.
\end{prop}
\vspace{-4mm}
\begin{proof}
Sia $m\in A+B$; allora, esistono $a_m\in A$ e $b_m\in B$ con $m=a_m+b_m\in A+B$. Siano $a'\in A$, $b'\in B$ tali che $m=a'+b'$. Allora, $a_m+b_m=a+b$ se e solo se $a_m-a'=b'-b_m$. Poiché tale elemento appartiene a $A\cap B=\{0_M\}$, risulta $a'=a_m$ e $b'=b_m$, quindi ogni $m\in A+B$ si scrive in modo unico come somma $a_m+b_m$. Sia $\psi\colon A+B\to A\oplus B$ definita come $\psi(m)=(a_m,b_m)$ e sia $\eta\colon A\oplus B\to A+B$ definita come $\eta(a,b)=a+b$. Per l'unicità della scrittura di $m$, tali mappe sono ben definite. Inoltre, $\eta\circ\psi=\operatorname{id}_{A+B}$ e $\psi\circ\eta=\operatorname{id}_{A\oplus B}$, quindi è sufficiente mostrare che questi sono omomorfismi di $R$-moduli. Questo è facile: prendo $m=a_m+b_m$ e $n=a_n+b_n$, allora $m+n=(a_m+a_n)+(b_m+b_n)$, cioè $a_{m+n}=a_m+a_n$ e $b_{m+n}=b_m+b_n$, quindi $\psi$ è un omomorfismo di gruppi abeliani. Inoltre, preso $r\in R$, $r\cdot m=r\cdot a_m+r\cdot a_n=a_{r\cdot m}+b_{r\cdot m}$, da cui $\psi$ è un omomorfismo di $R$-moduli. Analogo per $\eta$, ho $\eta((a_1,b_1)+(a_2,b_2))=(a_1+a_2)+(b_1+b_2)=\eta(a_1,b_1)+\eta(a_2,b_2)$ e $\eta(r\cdot (a,b))=\eta(r\cdot a,r\cdot b)=r\cdot a+r\cdot b=r\cdot \eta(a,b)$.

Procediamo ora per induzione su $k$. Per $k=1$ non c'è nulla da dimostrare, per $k=2$ lo ho già fatto. Supponiamo quindi che $\sum\limits_{i=1}^{k-1}A_i\simeq \oplus_{i=1}^{k-1}A_i$ e dimostriamolo per $k$. Per ipotesi $A_k\cap \sum\limits_{i=1}^{k-1}=\{0\}$, quindi $\sum\limits_{i=1}^k A_i=\sum\limits_{i=1}^{k-1} A_i+A_k\simeq \oplus_{i=1}^{k-1} A_i\oplus A_k\simeq \oplus_{i=1}^k A_k$.
\end{proof}

\begin{prop}[]{}
Sia $V$ un $K$-spazio vettoriale di dimensione finita e sia $\alpha\in\operatorname{End}_{K}(V)$. Siano $U,W\leqslant V$ sottospazi vettoriali tali che $\alpha(U)=U$ e $\alpha(W)=W$, cioè $U$ e $W$ sono $\alpha$-invarianti. Siano $\alpha_U\in \operatorname{End}_{K}(U)$ e $\alpha_W\in \operatorname{End}_K(W)$ gli endomorfismi indotti. Se $U+W=V$ e $U\cap W=\{0_K\}$, allora $\min_{\alpha}(x)=\operatorname{mcm}(\min_{\alpha_U}(x),\min_{\alpha_W}(x))$.
\end{prop}
\vspace{-4mm}
\begin{proof}
Poiché $\ker(\phi_{\alpha})=\Ann_{K[x]}(V,\ast_{\alpha})=K[x]\min_{\alpha}(x)$,\footnote{Dimostrare l'uguaglianza tra $\ker$ e $\Ann$ usando le doppie inclusioni.} vale $(V,\ast_{\alpha})\simeq (U, \ast_{\alpha_U})\oplus (W, \ast_{\alpha_W})$. Dunque $\Ann_{K[x]}(V,\ast_{\alpha})=\Ann_{K[x]}(U,\ast_{\alpha_U})\cap \Ann_{K[x]}(W,\ast_{\alpha_W})$, da cui risulta $K[x]\operatorname{mcm}(\min_{\alpha_U}(x),\min_{\alpha_W}(x))=K[x]\min_{\alpha_U}(x)\cap K[x]\min_{\alpha_W}(x)$.
\end{proof}

\clearpage

\subsection{Moduli in domini a ideali principali}

\noindent \textbf{Lezione del 26/11/2019} (appunti grezzi, non so più cosa stia succedendo qui ad Algebra)

\begin{defn}[]{}
Sia $R$ un PID e sia $M$ un $R$-modulo finitamente generato di torsione. Sia $\mathfrak{p}\lhd R$ ideale primo di $R$. Definiamo $M_{\mathfrak{p}}=\{m\in M: x\cdot m=0\, \forall x\in \mathfrak{p}\}=\{m\in M: \mathfrak{p}\subseteq \Ann_R(M)\}$. Allora, tale $M_{\mathfrak{p}}$ si dice $\mathfrak{p}$-componente primaria di $M$ (o anche $\mathfrak{p}$-componente di Fitting).
\end{defn}

\begin{teo}[]{}
Sia $R$ un PID e sia $M$ un $R$-modulo sinistro finitamente generato di torsione con $\Ann_R(M)=\mathfrak{p}_1^{\alpha_1}...\mathfrak{p}_r^{\alpha_r}$, dove i $\mathfrak{p}_i\lhd R$ sono ideali primi non nulli. Allora, $M\simeq \oplus_{i=1}^r M\mathfrak{p}_i^{\alpha_i}$.
\end{teo}
\vspace{-4mm}
\begin{proof}
La facciamo la prossima volta, le ultime parole famose.
\end{proof}

\noindent Da qui comincia la lezione di oggi, ci sono cose sparse da spostare in torsione etc

\begin{prop}[]{}
Sia $R$ un dominio di integrità e sia $M$ un $R$-modulo sinistro finitamente generato. Allora, $M$ è di torsione se e solo se $\Ann_R(M)\neq 0$. 
\end{prop}
\vspace{-4mm}
\begin{proof}
Siano $m_1,...\,,m_n\in M$ tali che $M=\sum\limits_{i=1}^n R\cdot m_1$. Allora, $\Ann_R(M)=\bigcap\limits_{i=1}^n \Ann_R(m_i)$. Dunque, se $M$ è di torsione, sappiamo che ogni $\Ann_R(m_i)\neq \{0\}$ da cui $\bigcap\limits_{i=1}^n \Ann_R(m_i)\neq \{0\}$.\footnote{Infatti, presi $I,J$ ideali non banali di un dominio di integrità $R$, se per assurdo fosse $I\cap J=\{0\}$, essendo $IJ=\{ij: i\in I, j\in J\}\lhd R$ un ideale contenuto in $I\cap J=\{0\}$, avremmo che esistono $i\in I\setminus \{0\}$ e $j\in J\setminus\{0\}$ tali che $ij=0$, assurdo (perché siamo in un dominio di integrità). Il claim segue per induzione.} Il viceversa a quanto pare lo abbiamo già fatto.
\end{proof}

\noindent Osserviamo che se $R$ è un anello commutativo e $M$ è un $R$-modulo sinistro con $\Ann_R(M)\neq \{0\}$, essendo $\Ann_R(M)\lhd R$, $M$ è canonicamente un $\overline{R}=R/\Ann_R(\overline{M})$-modulo. Aggiungere qui il diagramma commutativo negli appunti cartacei. Verifichiamo che vale il Lemma della forbice. Presi $r_1,r_2\in R$, si ha che $\tau(r_1)=\tau(r_2)$ se e solo se $r_1-r_2\in \Ann_R(\overline{M})$, cioè $r_1=r_2+a$ con $a\in \Ann_R(\overline{M})$. Per ogni $m\in M$, si ha quindi che $r_1\cdot m=(r_2+a)\cdot m=r_2\cdot m+a\cdot m=r_2\cdot m$ essendo $a\cdot m=0$. Dunque, abbiamo dimostrato che $(r_1,m)\sim (r_2,m)$ implica $r_1\cdot m=r_2\cdot m$, quindi per il Lemma della forbice esista la mappa $\odot\colon \overline{R}\cdot M\to M$ tale che $(r+\Ann_R{M})\odot m=r\cdot m$.


\begin{teo}[3.X.Y: Teorema cinese del resto]{}
Sia $R$ un anello e siano $I_1,...\,,I_n\lhd R$ ideali a due a due coprimi (cioè tali che $I_j+I_k=R$ per ogni $j\neq k$). Sia $\pi\colon R\to \bigoplus\limits_{k=1}^n R/I_k$ la mappa definita come $\pi(r)=(r+I_1,...\,,r+I_n)$. Allora, $\phi$ è un omomorfismo di anelli suriettivo con $\ker(\pi)=\bigcap\limits_{k=1}^n I_k$.
\end{teo}
\vspace{-4mm}
\begin{proof}
Che $\pi$ sia un omomorfismo di anelli è evidente dalla definizione (a casa lo scrivo meglio). Inoltre, $\pi(r)=0$ se e solo se $r\in \bigcap\limits_{k=1}^n I_k$. Sia $J_k=\bigcup\limits_{j\neq k} I_j\lhd R$. Allora, $J_k$ e $I_k$ sono coprimi. Infatti, l'ipotesi che $I_k+I_j=R$ per $j\neq k$ implica che in particolare esistono $a_k\in I_k$ e $b_k\in I_j$ tali che $a_k+b_k=1_R$. Allora, \[1_R=(a_1+b_1)\cdot ...\cdot (a_k+b_k)=a_1a_2\cdot ...\cdot a_n+b_1a_2\cdot ...\cdot a_n+...+b_1b_2\cdot ...\cdot b_n\] dove detti $d_k=b_1b_2\cdot ...\cdot b_n\in I_1...I_{k-1}I_{k+1}...I_n\subseteq J_k$ e $e_k=$ tutti gli altri termini $\in I_k$, abbiamo che $d_k+e_k=1_R$, cioè $I_k$ e $J_k$ sono effettivamente coprimi. Sia $\pi_k\colon R\to R/I_k$ la proiezione canonica, cioè $\pi(r)=r+I_k$. Allora, $\pi_k(d_j)=0_{R/I_k}$ se $j\neq k$ e $\pi_k(d_j)=1_{R/I_k}=1_R+I_k$ per $j=k$. Dunque, $1_R+I_k=\pi_k(1_R)=\pi_k(d_k+e_k)=\pi_k(d_k)+\pi_k(e_k)=\pi_k(d_k)$ perché $\pi_k(e_k)=0$. Sia ora $y=(r_1+I_1,...\,,r_n+I_n)\in \bigoplus\limits_{k=1}^n R/I_k$ e sia $z=\sum\limits_{i=1}^n r_1\cdot d_i$. Allora, $\pi_k(z)=\sum\limits_{i=1}^n \pi_k(r_i)\cdot \pi_k(d_i)=\pi_k(r_k)\cdot \pi_k(d_k)=r_k+I_k$ essendo $\pi_k(r_k)=r_k+I_k$ e $\pi_k(d_k)=1_R+I_k$, da cui $\pi(z)=y$ e $\pi$ risulta quindi essere un omomorfismo suriettivo.
\end{proof}

\noindent Ora parliamo di ideali in domini a ideali principali (PID), dove $\mathfrak{p}\lhd R$ è primo se e solo se è massimale.

\begin{defn}[]{}
Sia $R$ un PID. Definiamo spettro di $R$ l'insieme $\spec(R)=\{\mathfrak{p}\lhd R: \mathfrak{p}\neq \{0\} \text{ è primo}\}$.
\end{defn}

\begin{prop}[]{}
Sia $R$ un PID e sia $I\lhd R$ un ideale non banale. Allora, esistono $n_{\mathfrak{p}}(I)$, $\mathfrak{p}\in \spec(R)$ e $n_{\mathfrak{p}}\in \mathbb{N}$ tali che $\operatorname{supp}(I)=\{\mathfrak{p}\in \spec(R): n_{\mathfrak{p}}(I)\neq 0\}$ è un insieme finito, e $I=\prod\limits_{\mathfrak{p}\in \spec(R)} \mathfrak{p}^{n_{\mathfrak{p}}(I)}$, dove si intende che $\mathfrak{p}^0=R$.
\end{prop}
\vspace{-4mm}
\begin{proof}
Sia $I=R\cdot a$. Se $a\in R^{\times}$, allora $n_{\mathfrak{p}}=0$ per ogni $\mathfrak{p}\in \spec(R)$. Poiché $I\neq \{0_R\}$, sappiamo che $a\neq 0_R$. Quindi, possiamo assumere che $a\in R^{\#}=R\setminus (R^{\times}\cup \{0_R\})$. Allora, esiste $u_a\in R^{\times}$ e $\varepsilon_p(a)\in \mathbb{N}$ tali che $a=u_a\cdot \prod\limits_{p\in \mathfrak{p}} p^{\varepsilon_p(a)}$ dove $\mathfrak{p}\subseteq \operatorname{prim}_0(R)$ è un sistema di rappresentanti rispetto a $\sim$ e $\{p\in \mathfrak{p}: \varepsilon_p(a)\neq 0\}$ è un insieme finito, cioè $|\operatorname{supp}(I)|<\infty$. Dunque $R\cdot a=\prod\limits_{p\in \mathfrak{p}} (R\cdot p)^{\varepsilon_p(a)}$. Dove finisce la dimostrazione? Boh...
\end{proof}

\noindent Sia $(m_{\mathfrak{p}})$ con $\mathfrak{p}\in \spec(R)$ una successione di interi non negativi tali che $\{\mathfrak{p}\in \spec(R): m_{\mathfrak{p}}\neq 0\}$ sia un insieme finito e $I=\prod\limits_{\mathfrak{p}\in \spec(R)} \mathfrak{p}^{m_{\mathfrak{p}}}$. Allora, $m_{\mathfrak{p}}=n_{\mathfrak{p}}(I)$ per ogni $\mathfrak{p}\in \spec(R)$ come conseguenza della univocitò della decomposizione in primi. Sia $I=\prod\limits_{\mathfrak{p}\in \spec(R)} \mathfrak{p}^{n_{\mathfrak{p}}(I)}=\prod\limits_{\mathfrak{p}\in \operatorname{supp}(I)} \mathfrak{p}^{n_{\mathfrak{p}}(I)} = \bigcap\limits_{\mathfrak{p}\in \operatorname{supp}(I)} \mathfrak{p}^{n_{\mathfrak{p}}(I)}$. (Ma sti cazzo di $n_{\mathfrak{p}}$ sono così o sono degli $\eta_{\mathfrak{p}}$?)

\

\noindent Sia $R$ un PID e sia $M$ un $R$-modulo sinistro di torsione. Allora, \[\operatorname{Ann}_R(M)=\prod\limits_{\mathfrak{p}\in \operatorname{supp}(\Ann_R(M))} \mathfrak{p}^{n_{\mathfrak{p}}(I)}=\bigcap\limits_{\mathfrak{p}\in \operatorname{supp}(\Ann_R(M))} \mathfrak{p}^{n_{\mathfrak{p}}(I)}\] da cui per il Teorema cinese e per il primo teorema d'isomorfismo si ha che $\overline{R}=R/\Ann_R(M)\simeq \bigoplus\limits_{\mathfrak{p}\in \operatorname{supp}(\Ann_R(M))} R/\mathfrak{p}^{n_{\mathfrak{p}}}$. Sia $d_{\mathfrak{p}}\in \overline{R}$, $d_{\mathfrak{p}}\in \bigcap\limits_{\mathfrak{q}\neq \mathfrak{p}} \mathfrak{q}^{n_{\mathfrak{q}}}$ dove $\mathfrak{q}\in \operatorname{supp}(\Ann_R(M))$. Allora, $d_{\mathfrak{p}}+\mathfrak{p}^{n_{\mathfrak{p}}} = 1+\mathfrak{p}^{n_{\mathfrak{p}}}$. Detto $\Omega=\{ \mathfrak{p}^{n_{\mathfrak{p}}}: \mathfrak{p}\in \operatorname{supp}(\Ann_R(M))\}$, se $\mathfrak{p},\mathfrak{q}\in \spec(R)$ e $\mathfrak{p}\neq \mathfrak{q}$, significa che $\mathfrak{p}^m+\mathfrak{q}^n=R$ per ogni $m,n\in \mathbb{N}$, cioè $\Omega$ sono a due a due coprimi. Infine, si ha quindi che $1_{\overline{R}}=\sum\limits_{\mathfrak{p}\in \operatorname{supp}(\Ann_R(M))} d_{\mathfrak{p}}$.

\
 
\noindent \textbf{Lezione del 27/11/2019} (vedi appunti cartacei)

\

\noindent \textbf{Lezione del 03/12/2019} (appunti grezzi)

\

\noindent Facciamo un recap. Se $R$ è un PID e $M$ è un $R$-modulo sinistro finitamente generato di torsione, allora $\Ann_R(M)\neq \{0\}$ ed esistono $\mathfrak{p}_1,...\,,\mathfrak{p}_r\in \spec(R)$ e $\alpha_i\in \mathbb{N}$ tali che $\Ann_R(M)=\prod\limits_{i=1}^r \mathfrak{p}_i^{\alpha_i}$. Sappiamo anche che i $\mathfrak{p}_i^{\alpha_i}$, $\mathfrak{p}_j^{\alpha_j}$ sono a due a due coprimi. Abbiamo visto poi che vale il Teorema cinese del resto, cioè $\overline{R}=R/\Ann_R(M)\simeq \bigoplus\limits_{i=1}^r R/\mathfrak{p}_i^{\alpha_i}$ mediante la mappa $\pi$. Inoltre, se prendo $d_1,...\,,d_r\in R$ tali che $\pi(d_i+\Ann_R(M))=(0,...\,,1,0,...\,,0)$ dove $1$ è in posizione $i$-esima, sappiamo che gli $M_i=d_i\cdot M$ sono $R$-sottomoduli di $M$ e $M=\bigoplus\limits_{i=1}^r M_i$.

\

\noindent Abbiamo applicato la teoria generale al caso particolare in cui $R=\mathbb{K}[x]$ con $\mathbb{K}$ campo e $(M,\cdot)=(M,\ast_{\alpha})$. In questo caso, $\Ann_{\mathbb{K}[x]}(M)=\mathbb{K}[x]\cdot \min_{\alpha}(x)\cdot \mathbb{K}[x]$ (forse c'è un $\mathbb{K}[x]$ di troppo), e abbiamo dimostrato che $\alpha$ è un endomorfismo diagonalizzabile se e solo se $\min_{\alpha}(x)=\prod\limits_{i=1}^k (x-\lambda_i)$ con $\lambda_i\neq \lambda_j$ se $i\neq j$, cioè se e solo se il polinomio minimo splitta completamente in fattori lineari distinti su $\mathbb{K}[x]$.

\begin{prop}[]{}
Si ha che $M_i=M_{\mathfrak{p}_i^{\alpha_i}}=\{m\in M: \mathfrak{p}_i^{\alpha_i}\cdot m=0\}$.
\end{prop}
\vspace{-4mm}
\begin{proof}
Osserviamo che $d_i\in \mathfrak{p}_j^{\alpha_j}$ per $j\neq i$, quindi $d_i\in \bigcap\limits_{j\neq i} \mathfrak{p}_j^{\alpha_j}$. Sia $m\in d_i\cdot M$. Allora, $m=d_i\cdot m$ perché $(d_i+\Ann_R(M))^2=d_i+\Ann_R(M)$, cioè esiste $y\in M$ tale che $m=d_i\cdot y=d_i^2\cdot y=d_i(d_i\cdot y)=d_i\cdot m$. Per ogni $z\in \mathfrak{p}^{\alpha_i}$ tale che $z\cdot d_i\cdot m=0$ osserviamo che $z\cdot d_i$ (qualcosa, forse è appartiene?) $\mathfrak{p}_i^{\alpha_i}\cap \prod\limits_{j\neq i} \mathfrak{p}_j^{\alpha_j}=\prod\limits_{k=1}^r \mathfrak{p}_k^{\alpha_k}=\mathfrak{p}_1^{\alpha_1}\cap ... \cap \mathfrak{p}_k^{\alpha_k}=\Ann_R(M)$, e questo prova che $M_i\in M_{\mathfrak{p}_i^{\alpha_i}}$. Sia ora $m\in M_i\in M_{\mathfrak{p}_i^{\alpha_i}}$. Poiché $m=\cdot m$ e $1_{\overline{R}}=\sum\limits_{i=1}^r d_i + \Ann_R(M)$, sappiamo che $m=\sum\limits_{k=1}^r d_k\cdot m=d_i\cdot m$. Per $k\neq i$, l'elemento $d_k\in \bigcap\limits_{j\neq k} \mathfrak{p}_j^{\alpha_j}\subseteq \mathfrak{p}_i^{\alpha_i}$. Dunque $d_k\cdot m=0$ perché $m\in M_{\mathfrak{p}_i^{\alpha_i}}$, da cui $M_{\mathfrak{p}_i^{\alpha_i}}\subseteq d_i\cdot M=M_i$ come desiderato.
\end{proof}

\noindent Come si applica questa cosa? Sia $R=\mathbb{Z}$ e sia $A$ uno $\mathbb{Z}$-modulo finitamente generato di torsione. Allora, avevamo visto che $|A|<\infty$, cioè $A$ è un gruppo abeliano finito.\footnote{Ricordiamo che per ogni $g\in G$ gruppo, la mappa $\chi_g\colon \mathbb{Z}\to G$ definita come $\chi_g(k)=g^k$ è un omomorfismo di gruppi. Definiamo esponente di $G$ l'intero positivo $\exp(G)$ tale che $\exp(G) \mathbb{Z}=\bigcap\limits_{g\in G} \ker(\chi_g)$. In realtà c'è una definizione molto più facile ma a lui piace complicarsi la vita.} Per quanto appena provato, possiamo scrivere $A=\bigoplus\limits_{i=1}^r A_i$, dove $A_i=A_{p_i^{\alpha_i}\mathbb{Z}}={a\in A: p_i^{\alpha_i}\cdot a=0}\in \operatorname{Syl}_p(A)$. Sia $|A|=p_1^{n_1}\cdot ...\cdot p_r^{n_r}\cdot p_{r+1}^{n_{r+1}}\cdot ...\cdot p_{r+k}^{n_{r+k}}$. Allora, $A_i\subseteq A$ è un sottogruppo, anzi è un $p_i$-sottogruppo, e $|A_i|=p_i^{\beta_i}$. Infatti, se per assurdo fosse $|A_i|=p_i^{\beta_i}\cdot q^{\beta}\cdot r$ con $q\neq p_i$ primo e $r$ intero coprimo a $p_i$ e $q$, dove ovviamente $\beta\geq 1$, per il Teorema di Sylow esiste $Q\subseteq \operatorname{Syl_q}(A_i)\subseteq A_i$ tale che $|Q|=q^{\beta}\neq 1$, cioè esiste $g\in Q\setminus\{1\}$. Dunque, $g\in \operatorname{Syl}_q(A_i)\subseteq A_i$ da cui, essendo $g^{p_i^{\alpha_i}}=1$ e $\langle g \rangle\subseteq Q$, per Lagrange $g^{|Q|}=g^{q^{\beta}}=1$. Dunque, essendo $\gcd(p,q)=1$, deve essere $g=1$, il che è assurdo perché questo forza $Q=\{1\}$. Dunque, essendo $A=\bigoplus\limits_{i=1}^r A_i$, abbiamo che $|A|=\prod p_i^{\beta_i}$, dove $\beta_i$ è la massima potenza di $p_i$ che divide $|A|$, da cui $A_i\in \operatorname{Syl}_{p_i}(A)$. (In entrambi gli esempi, ho mostrato che un modulo è somma diretta di sottomoduli che si annullano su ideali particolari che contengolo l'annullatore globale, credo abbia detto così). 

\begin{exm}Se $|G|=35$, allora $G\simeq \mathbb{Z}/35\mathbb{Z}$. Infatti, per quanto appena detto si ha che $G\simeq \mathbb{Z}/5\mathbb{Z}\times \mathbb{Z}/7\mathbb{Z}=\mathbb{Z}/35\mathbb{Z}$, cioè $G$ è ciclico. L'ide è che ho un solo $5$-sottogruppo di Sylow e un solo $7$-sottogruppo di Sylow, da cui essi sono normali, e si conclude facilmente$. \ \square$\end{exm}

\noindent Vogliamo arrivare al teorema seguente. Per farlo dovremo prima introdurre i moduli liberi.

\begin{teo}[3.X.Y: Teorema fondamentale sui moduli f.g. per PID]{}
Sia $M$ un $R$-modulo sinistro finitamente generato di torsione. Allora, esistono degli ideali $\mathfrak{a}_1,...\,,\mathfrak{a}_k \lhd R$ tali che $M\simeq \bigoplus\limits_{i=1}^k R/\mathfrak{a}_i$.
\end{teo}

\begin{exm}Se $R=\mathbb{Z}$ e $A$ è uno $\mathbb{Z}$-modulo di torsione con $|A|=27$, allora $\mathfrak{a}_1,\mathfrak{a_2},\mathfrak{a}_3\in \{3\mathbb{Z}, 9\mathbb{Z},27\mathbb{Z}\}$ e $A$ è isomorfo a uno tra $\mathbb{Z}/27\mathbb{Z}$, $\mathbb{Z}/9\mathbb{Z}\times \mathbb{Z}/3\mathbb{Z}$ e $\mathbb{Z}/3\mathbb{Z}\times \mathbb{Z}/3\mathbb{Z}\times \mathbb{Z}/3\mathbb{Z}. \ \square$\end{exm}

\clearpage

\subsection{Moduli liberi}

\begin{defn}[]{}
Sia $R$ un anello e sia $X$ un insieme. Un $R$-modulo sinistro $L$ dotato di una mappa $i_X\colon X\to L$ si dice libero su $X$ se per ogni $\phi\colon X\to M$ con $M$ che è $R$-modulo sinistro, esiste un unico $\phi_{\star}\colon L\to M$ omomorfismo di $R$-moduli tale che $\phi=\phi_{\star}\circ i_X$.
\end{defn}

\noindent Aggiungere diagrammino dagli appunti. Esistono definizioni analoghe per i gruppi, per le algebre, etc. Il concetto di libero è una generalizzazione del concetto di funtore aggiunto. Ma proseguiamo la prossima volta. Se prendo $R=\mathbb{K}$ campo, $M=V$ spazio vettoriale, $X=\mathcal{B}$ base di $V$ e $i_X$ l'inclusione canonica, allora lo spazio vettoriale $V$ lo possiamo vedere come modulo libero sulla base $\mathcal{B}$. L'idea è che basta definire i valori di una mappa $\mathbb{K}$-lineare sulla base, e so già come si comporta in tutto lo spazio $V$.

\

\noindent \textbf{Lezione del 10/12/2019} (vedi appunti cartacei)

\

\noindent La lezione del 10/12/2019 la ho negli appunti cartacei per ora. Le cose su teoria dei moduli sono davvero troppo a caso come ordine, dovrei davvero risistemarle.

\

\

\noindent \textbf{Lezione del 18/12/2019} (appunti grezzi)

\

\noindent Scopo di questa lezione è arrivare al teorema che mostri che se $R$ è un PID, allora ogni $R$-modulo finitamente generato senza torsione possiamo in realtà vederlo come $R$-modulo libero su un opportuno insieme finito. Per fare ciò, procediamo step by step. 

\vspace{1.5mm}

\noindent La somma diretta: sia $R$ un anello e $M$ un $R$-modulo sinistro. Allora, $M\simeq A\oplus B$, dove $A$ e $B$ sono $R$-sottomoduli di $M$, se e solo se dette $\iota_A\colon A\to M$, $\iota_B\colon B\to M$ le inclusioni e $\pi_A\colon M\to A$ e $\pi_B\colon M\to B$ le rispettive proiezioni sul quoziente, accade che $\pi_A\circ \iota_A=\operatorname{id}_A$, $\pi_B\circ \iota_B=\operatorname{id}_B$ e $\iota_A\circ \pi_A+\iota_B\circ \pi_B=\operatorname{id}_M$.

\begin{prop}[3.5.4]{}
Sia $R$ un anello, $M$ un $R$-modulo sinistro, $A$ un $R$-sottomodulo di $M$ e $\iota_A\colon A\to M$ e $\pi_A\colon M\to A$ omomorfismi di $R$-moduli tali che $\pi_A\circ \iota_A=\operatorname{id}_A$. Allora, $M\simeq A\oplus \ker(\pi_A)$.
\end{prop}
\vspace{-4mm}
\begin{proof}
Sia $\phi\colon A\oplus \ker(\pi_A)\to M$ la mappa definita come $\phi(a,x)=\iota_A(a)+x$, dove $a\in A$ e $x\in \ker(\pi_A)$. Chiaramente tale mappa è un omomorfismo di $R$-moduli. Inoltre, se $\phi(a,x)=0$, allora $\iota_A(a)=-x$, cioè $a=\pi_A(\iota_A(a))=\pi_A(-x)=0$, da cui $a=0$, cioè $-x=0$ e quindi $x=0$, dunque $(a,x)=(0,0)$ il che mostra che $\phi$ è iniettiva. Infine, $\phi$ è anche suriettiva. Infatti, sia $z\in M$ e sia $y=z-\iota_A(\pi_A(z))\in M$. Allora, $\pi_A(y)=\pi_A(z)-\pi_A(\iota_A(\pi_A(z)))=\pi_A(z)-\pi_A(z)=0$, dove nell'ultimo passaggio abbiamo usato che $\pi_A\circ \iota_A=\operatorname{id}_A$, da cui $y\in \ker(\pi_A)$. Dunque, $z=\phi(\pi_A(x),y)=\iota_A(\pi_A(z))+y$, e questo prova la suriettività di $\phi$, da cui esso è quindi un isomorfismo e vale quindi $M\simeq A\oplus \ker(\pi_A)$.
\end{proof}

\noindent Vale una proposizione simile nel caso dei moduli liberi.

\

\begin{prop}[3.5.5]{}
Sia $M$ un $R$-modulo sinistro, $\pi\colon M\to F$ un omomorfismo suriettivo e $F$ un $R$-modulo sinistro libero su un insieme $Y$. Allora, $M\simeq F\oplus \ker(\pi)$.
\end{prop}
\vspace{-4mm}
\begin{proof}
Sia $\iota_X\colon X\to F$ una mappa tale che $(F,\iota_X)$ sia libero su $X$, e per ogni $x\in X$ sia $m_x\in M$ tale che $\pi(m_x)=\iota_X(x)$. Sia $\psi\colon X\to M$ la mappa definita come $\psi(x)=m_x$.
\[
\begin{tikzcd}[column sep=small]
X \arrow[rd, "\psi"'] \arrow[rr, "\iota_X"] &                      & F \arrow[ld, "\psi_{\star}", dashed, bend left=40] \\
                                            & M \arrow[ru, swap, "\pi"'] &                                                   
\end{tikzcd}
\]
\noindent Essendo $F$ libero su $X$, sappiamo che esiste un'unica mappa $\psi_{\star}\colon F\to M$ tale che $\psi_{\star}\circ \iota_X=\psi$. Resta da verificare che $\pi\circ \psi_{\star}=\operatorname{id}_F$. Poiché $\pi(\psi_{\star}(\iota_X(x)))=\pi(\psi(x))=\pi(m_x)=\iota_X(x)$, abbiamo che $(\pi\circ \psi_{\star}) (\iota_X(x))=\iota_X(x)$ per ogni $x\in X$. Abbiamo quindi trovato due mappe che fanno commutare il diagramma seguente: 
\[
\begin{tikzcd}[column sep=small]
X \arrow[rd, "\iota_X"'] \arrow[rr, "\iota_X"] &   & F \arrow[ld, "\operatorname{id}_F"'] \arrow[ld, "\pi\circ \psi_{\star}", bend left=40] \\
                                               & F &                                                                                       
\end{tikzcd}
\]
\noindent Tuttavia, essendo $F$ libero, la mappa che fa commutare tale diagramma è unica, da cui $\pi\circ \psi_{\star}=\operatorname{id}_F$. Dunque, presa $\iota_F=\psi_{\star}$, per la \emph{Proposizione 3.5.4} vale $M\simeq F\oplus \ker(\pi)$.
\end{proof}

\noindent Per dimostrare il Teorema, vogliamo procedere per induzione sul numero di generatori di $M$. Tuttavia, per fare ciò dobbiamo prima essere in grado di dimostrare il passo base e lo step induttivo. Ci servono quindi altre due proposizioni.

\begin{prop}[3.5.6]{}
Sia $R$ un PID, $\mathbb{K}=\quot(R)$ e sia $M\subseteq \mathbb{K}$ un $R$-sottomodulo finitamente generato. Allora, $M\simeq R$ oppure $M=\{0\}$.
\end{prop}
\vspace{-4mm}
\begin{proof}
Poiché $M$ è finitamente generato, esistono $m_1,...\,,m_n\in M$ tali che $M=\sum\limits_{i=1}^n R\cdot m_i$. Essendo $M\subseteq \mathbb{K}$, sappiamo che ogni $m_i$ è della forma $m_i=\frac{a_i}{s_i}$ per degli opportuni $a_i\in R$ e $s_i\in R\setminus \{0\}$. Sia $s=s_1\cdot ...\cdot s_n$, così che $s\cdot M\subseteq R$ sia un $R$-sottomodulo (perché?). Siano $r_1,...\,,r_n\in R$; allora, $s\cdot \sum\limits_{i=1}^n r_i\cdot \frac{a_i}{s_i}=\sum\limits_{i=1}^n r_i s_i^{\times} a_i$ dove $s_i^{\times}=\prod\limits_{j\neq i} s_j$ (non so cosa stia facendo qui). Dunque, essendo $s\cdot M$ un ideale di $R$, poiché $R$ è un PID ogni suo ideale è principale, quindi esiste $b\in R$ tale che $s\cdot M=\langle b\rangle$, da cui $M=R\cdot \frac{b}{s}$. Allora, la mappa $\phi_{b/s}\colon R\to M$ definita come $\phi_{b/s}(r)=r\cdot \frac{b}{s}$ è un omomorfismo suriettivo. Se $b=0$, allora banalmente $M=\{0\}$. Se $b\neq 0$, allora $\ker(\phi_{b/s})=\{0\}$ e $\phi_{b/s}$ è quindi un isomorfismo.
\end{proof}

\noindent Manca ancora un'ultima (spero meno dubbia della precedente) proposizione prima di poter dimostrare il Teorema. Altro che sagra della primavera, qui è la sagra delle proposizioni.

\begin{prop}[3.5.7]{}
Sia $R$ un anello, $F_1$ un $R$-modulo sinistro libero su $X$ e $F_2$ un $R$-modulo sinistro libero su $Y$. Allora, $F_1\oplus F_2$ è un $R$-modulo libero su $X\sqcup Y$.
\end{prop}
\vspace{-4mm}
\begin{proof}
Siano $\iota_X\colon X\to F_1$ e $\iota_Y\colon Y\to F_2$ le mappe dei moduli liberi $F_1$ e $F_2$, rispettivamente, e sia $\iota_{X\sqcup Y}\colon X\sqcup Y\to F_1\oplus F_2$ la mappa definita come $\iota_{X\sqcup Y}(x)=\iota_X(x)$ e $\iota_{X\sqcup Y}(y)=\iota_Y(y)$ per ogni $x\in X$ e $y\in Y$ (sappiamo che tale mappa è ben definita per le proprietà dell'unione disgiunta). Sia $M$ un $R$-modulo sinistro e sia $\phi \colon X\sqcup Y\to M$ una mappa qualunque. Allora, detta $\phi_{\star}\colon F_1\oplus F_2\to M$ la mappa $\phi_{\star}(f_1,f_2)=\phi_1(f_1)+\phi_2(f_2)$, dove $\phi_1\colon F_1\to M$ e $\phi_2\colon F_2\to M$ sono gli omomorfismi di $R$-moduli tali che $\phi \raisebox{-.5em}{$\vert_{X}$}=\phi_1\circ \iota_X$ e $\phi \raisebox{-.5em}{$\vert_{Y}$}=\phi_2\circ \iota_Y$ (che credo esistano essendo $F_1$ e $F_2$ moduli liberi), si ha che $\phi_{\star}\circ \iota_{X\sqcup Y}=\phi$, il che prova l'esistenza. Resta da mostrare la unicità di tale mappa $\phi_{\star}$ per concludere che $F_1\oplus F_2$ è libero. D'altra parte, se $\psi\colon F_1\oplus F_2\to M$ è una mappa tale che $\psi\circ \iota_{X\sqcup Y}=\phi$, in particolare deve essere $\psi \raisebox{-.5em}{$\vert_{X}$}=\phi_1$ e $\psi \raisebox{-.5em}{$\vert_{Y}$}=\phi_2$, da cui $\psi(f_1,f_2)=\psi(f_1,0)+\psi(0,f_2)=\phi_1(f_1)+\phi_2(f_2)=\phi_{\star}(f_1,f_2)$, da cui $\psi=\phi_{\star}$ provando l'unicità di $\phi_{\star}$.
\end{proof}

\noindent It's time for the big theorem, boi :)

\begin{teo}[3.5.8]{}
Sia $R$ un PID e sia $M$ un $R$-modulo sinistro finitamente generato con $\tor_R(M)=\{0\}$. Allora, esiste un insieme finito $X$ con $|X|=d_R(M)$ tale che $M$ è libero su $X$.
\end{teo}
\vspace{-4mm}
\begin{proof}
Procediamo per induzione sul numero di generatori $d_R(M)$. Se $d_R(M)=1$, esiste $m\in M$ tale che $M=R\cdot m$. Allora, $\phi_m\colon R\to M$ definita come $\phi_r(m)=r\cdot m$ è un omomorfismo di moduli suriettivo, e $\ker(\phi_m)=\Ann_R(m)=\{0\}$ perché per ipotesi $\tor_R(M)=\{0\}$. Dunque $\phi_m$ è iniettivo, da cui $M\simeq R$, quindi il teorema vale (perchè ogni anello è un modulo libero su se stesso con $1$ generatore, in quanto $R=\langle 1_R\rangle$, cioè $\{1_R\}$ è una base). Supponiamo ora che la tesi valga per $d_R(M)\leq n$. Sia $M$ con $d_R(M)=n+1$ e $\tor_R(M)=\{0\}$. Allora, esistono $m_0,...\,,m_n\in M$ tali che $M=\sum\limits_{i=0}^n R\cdot m_i$. Sia $M_0=\sat_M(R\cdot m_0)$. Allora, $\sat_M(M_0)=\sat_M(\sat_M(M_0))=M_0$ (il passaggio in mezzo è inutile, il punto è che il sat del sat è ancora il sat), dunque per la \emph{Proposizione 3.2.2} si ha che $\tor_R(M/M_0)=\sat_M(M_0)/M_0=\{0\}$ (perché il quoziente è $M_0/M_0$). Poiché $d_R(M/M_0)\leq n$, per ipotesi induttiva $M/M_0$ è libero e per la \emph{Proposizione 3.5.5} vale $M\simeq M_0\oplus M/M_0$. Dunque, basta far vedere che anche $M_0$ è libero. Preso $x\in M_0$, (da qui in poi è delirio) sappiamo che esistono $r_x\in R$ e $s_x\in R\setminus \{0\}$ tali che $s_x\cdot x=r_x\cdot m_0$. Sia $\alpha\colon M_0\to \quot(R)$ la mappa $\alpha(x)=\frac{r_x}{s_x}$ se $x\neq 0$ e $\alpha(0)=0$. Siano $r,r'\in R$ e $s,s'\in R\setminus \{0\}$ con $s\cdot x=r\cdot m_0$ e $s'\cdot x=r'\cdot m_0$. Allora, $ss'\cdot x=s'r\cdot m_0=sr'\cdot m_0$, cioè $(s'r-sr')\cdot m_0=0$, da cui $s'r-sr'\in Ann_R(m_0)=\{0\}$ e quindi $s'r-sr'=0$, cioè $\frac{r}{s}=\frac{r'}{s'}$ (a che serve sta cosa?). Mostriamo che $\alpha$ è un omomorfismo iniettivo di $R$-moduli. Infatti, presi $x,y\in M_0$, siano $s_x\cdot x=r_x\cdot m_0$ e $s_y\cdot y=r_y\cdot m_0$, così che moltiplicando la prima equazione per $s_y$ e la seconda per $s_x$ e sommandole, valga $s_xs_y(x+y)=(s_yr_x+s_xr_y)\cdot m_0$, da cui $\alpha(x+y)=\frac{s_yr_x+s_xr_y}{s_xs_y}=\frac{r_x}{s_x}+\frac{r_y}{s_y}=\alpha(x)+\alpha(y)$. Inoltre, preso $r\neq 0$, $rs_x\cdot x=rr_x\cdot m_0$, quindi $\alpha(r\cdot x)=\frac{r\cdot r_x}{s_x}=r\cdot \alpha(x)$. Per l'iniettività, se $\alpha(x)=0$ esiste $s_x\in R\setminus \{0\}$ tale che $s_x\cdot x=0$, cioè $x\in \tor_R(M_0)\subseteq M$, da cui $x=0$ essendo $\tor_R(M)=\{0\}$. Dunque, per il \emph{Primo teorema d'isomorfismo} si ha $M_0\simeq \operatorname{Im}(\alpha)\subseteq M$. Tuttavia, per la \emph{Proposizione 3.5.6}, essendo $\operatorname{Im}(\alpha)$ un $R$-sottomodulo di $\quot(R)$, vale $\operatorname{Im}(\alpha)\simeq R$, quindi $M_0\simeq R$. Poiché $R$ è libero su $\{\cdot \}$ (come detto prima la base è un insieme di cardinalità 1) e per ipotesi induttiva $M/M_0$ è libero su $X'$ di cardinalità $|X'|=d_R(M)-1$, concludiamo che $M$ è libero su $X=X'\sqcup \{\cdot \}$ e $|X|=d_R(M)$ come desiderato. 
\end{proof}

\noindent Ci sono un sacco di punti che non mi sono chiari: perchè il sat del sat è il sat? che succede quando compare un $m_0$ selvaggio con tutto il delirio degli $r_x$ e $s_x$? Alla fine che succede?

\clearpage

\subsection{Divisori elementari}

\noindent \textbf{Lezione del 07/01/2020} (appunti grezzi)

\noindent Sia $R$ un PID e sia $F$ un $R$-modulo sinistro libero su $X=\{x_1,...\,,x_n\}\subseteq F$, dove $X$ è una base di $F$. Preso un elemento $z\in F$, siano $r_1,...\,,r_n\in R$ tali che $z=\sum\limits_{i=1}^n r_i\cdot x_i$.

\begin{defn}[]{}
L'ideale $\con(z)=\langle r_1,...\,,r_n \rangle\lhd R$ si dice \underline{contenuto} di $z$.
\end{defn}

\noindent A priori, tale definizione è strana: per come lo abbiamo posto, sembra che $\con(z)$ dipenda dalla particolare base $X$ scelta. Tuttavia questo non è vero, come mostra la proposizione seguente. Per comodità di notazione, sia $F^{\star}=\operatorname{Hom}(F,R)$. 

\begin{prop}[3.6.1]{}
Sia $z\in F$ e sia $I_z=\{\phi(z): \phi\in F^{\star}\}$. Allora, $I_z$ è un ideale di $R$ e $\con(z)=I_z$.
\end{prop}
\vspace{-4mm}
\begin{proof}
Sia $x_i^{\star}\in F^{\star}$ definito come $x_i^{\star}(x_j)=\delta_{i,j}$, così che $r_i=x_i^{\star}(z)$, cioè $r_i\in I_z$.\footnote{Osserviamo che tali $x_i^{\star}$ sono una base del duale.} Mostriamo ora che $I_z\lhd R$. Innanzitutto, sappiamo che $F^{\star}$ è un $R$-modulo, perché presi $\phi,\psi\in F^{\star}$ anche $\phi+\psi\in F^{\star}$ e $r\cdot \phi\in F^{\star}$ per ogni $r\in R$. Dunque, la mappa $\underline{\ }(z)\colon F^{\star}\to R$ è un omomorfismo di $R$-moduli (ma perché chiama le mappe con il trattino, e che cacchio) da cui $\operatorname{Im}(\underline{\ }(z))=I_z$, cioè $I_z$ è un $R$-modulo (e quindi anche un ideale di $R$). Chiaramente $\con(z)\subseteq I_z$. D'altra parte, preso $\phi\in F^{\star}$, osserviamo che $\phi(z)=\phi\left( \sum\limits_{i=1}^n r_i\cdot x_i \right)=\sum\limits_{i=1}^n r_i\cdot \phi(x_i)\in \con(z)$, da cui $I_z\subseteq \con(I_z)$ e quindi seque che $\con(z)=I_z$ come richiesto.
\end{proof}

\begin{lem}[3.6.2]{}
Sia $R$ un PID, $F$ un $R$-modulo sinistro libero su $X=\{x_1,...\,,x_n\}$ e sia $M\subseteq F$ un $R$-sottomodulo di $F$. Sia $z\in F$. Allora,

\noindent (a) esiste $\phi\in F^{\star}$ tale che $\con(z)=\langle \phi(z) \rangle$; (sono ideali o moduli? lui scrive $R\cdot \phi(z)$)

\noindent (b) per ogni $\psi\in F^{\star}$ si ha che $\psi(z)\in \con(z)$;

\noindent (c) esiste $x_0\in M$ tale che per ogni $y\in M$ si abbia $\con(y)\subseteq \con(x_0)$.
\end{lem}
\vspace{-4mm}
\begin{proof}
(a) Poiché $R$ è un PID, ogni suo ideale è principale, da cui essendo $\con(z)\lhd R$ sappiamo che esiste $c\in \con(z)$ tale che $\con(z)=\langle c\rangle$. Dunque, per la \emph{Proposizione 3.6.1} esiste $\phi\in F^{\star}$ tale che $c=\phi(z)$.

\vspace{1mm}

\noindent (b) Segue banalmente dalla \emph{Proposizione 3.6.1} essendo $\con(z)=I_z$.

\vspace{1mm}

\noindent (c) Poiché $R$ è un PID, esso è noetheriano, dunque esiste $x_0\in M$ tale che $\con(x_0)$ è massimale in $\{\con(y): y\in M\}$, cioè se $\con(x_0)\subseteq \con(z)$ per un certo $z\in M$, allora $\con(z)=\con(x_0)$. Resta da mostrare che $x_0$ soddisfa (c). Per (a), sappiamo che esiste $\phi\in F^{\star}$ tale che $\con(x_0)=\langle \phi(x_0)\rangle$. Ora, per la \emph{Proposizione 3.6.1} basta verificare che $\phi(z)\subseteq \langle \phi(x_0)\rangle$ per ogni $z\in M$ e $\psi\in F^{\star}$. Sia $R\cdot d=R\cdot \phi(x_0) + R\cdot z_0$. Allora, esistono $a,b\in R$ tali che $d=a\cdot \phi(x_0)+b\cdot \phi(z)$, cioè $d=\phi(ax_0+bz)\in \con(ax_0+bz)$ per la \emph{Proposizione 3.6.1}. Allora, $\con(x_0)=R\cdot \phi(x_0)\subseteq R\cdot d\in \con(ax_0+bz)$, da cui $\con(x_0)=\con(ax_0+bz)$. Dunque, $d\in R\cdot \phi(x_0)$, cioè $\phi(z)\in R\cdot d\subseteq R\cdot \phi(x_0)=\con(x_0)$. Manca da mostrare che $\psi(z)\in \con(x_0)$ per ogni $\psi\in F^{\star}$. Sappiamo che $\psi(x_0)\in \con(x_0)$ per ogni $\psi\in F^{\star}$. Sia $z_0=z-\frac{\phi(z)}{\phi(x_0)}\cdot x_0$, dove quindi $\frac{\phi(z)}{\phi(x_0)}\in R$. Allora $\phi(z_0)=0$. Basta dimostrare che $\psi(z_0)\in \con(x_0)$ (nota: mi sono perso). Sia $\psi_0\in F^{\star}$ tale che $\psi_0=\psi-\frac{\psi(x_0)}{\phi(x_0)}\cdot \phi$. Osserviamo che $\psi_0(z_0)=\psi(z)$ e $\psi_0(x_0)=0$. Ora basta mostrare che $\psi_0(z_0)\in \con(x_0)$. Usiamo lo stesso trucco di prima. Sia $R\cdot c=R\cdot \psi_0(z_0)+R\cdot \psi_0(x_0)$. Allora, esistono $p,q\in R$ tali che $x=p\cdot \psi_0(z_0)+q\cdot \psi_0(x_0)$. Dunque, \[ (\phi+\psi_0)(pz_0+qx_0)=\phi(pz_0)+\phi(qx_0)+\psi_0(pz_0)+\psi_0(qx_0)=q\cdot \phi(x_0)+p\cdot \psi_0(z_0)=c\] in quanto gli altri due termini sono nulli. Quindi per la \emph{Proosizione 3.6.1} vale $c=(\phi+\psi_0)(pz_0+qx_0)\in \con(pz_0+qx_0)$, da cui $R\cdot c\subseteq \con(pz_0+qx_0)$, dove $\con(x_0)=R\cdot \phi(x_0)\subseteq R\cdot c$. Dunque, $R\cdot \phi(x_0)=\con(pz_0+qx_0)\ni c$, cioè $R\cdot c=R\cdot \phi(x_0)$, quindi $\psi_0(z_0)\in R\cdot \psi(x_0)=\con(x_0)$ come desiderato.
\end{proof}

\begin{teo}[3.6.3]{}
Sia $R$ un PID, $F$ un $R$-modulo libero su $\{y_1,...\,,y_n\}$ e sia $M\subseteq F$ un $R$-sottomodulo di $F$. Allora, esistono una base $\{x_1,...\,,x_n\}$ di $F$ e degli elementi $\alpha_1,...\,,\alpha_m\in R\setminus\{0\}$ tali che $\{\alpha_1 x_1,...\,,\alpha_m x_m\}$ sia una base di $M$. Inoltre, la successione $(R\cdot \alpha_1, ...\,,R\cdot \alpha_m)$ è univocamente determinata da $M$.
\end{teo}
\vspace{-4mm}
\begin{proof}
Dannazione, me la sono persa per lo sciopero, ma c'è sulle sue note.
\end{proof}

\begin{defn}[]{}
Tali $R\cdot \alpha_i$ si dicono divisori elementari di $M$.
\end{defn}

\begin{cor}[3.6.4]{}
Sia $R$ un PID e sia $A$ un $R$-modulo di torsione finitamente generato. Allora, esistono degli ideali $I_1,...\,,I_n\lhd R$ con $\Ann_R(A)\subseteq I_n\subseteq ... \subseteq I_1\varsubsetneq R$ tali che $A\simeq \bigoplus\limits_{k=1}^n R/I_k$.
\end{cor}

\noindent Ha detto qualcosa su come applicarlo ai gruppi abeliani. Notare come tale teorema+corollario implica il Teorema di Jordan.

\

\noindent \textbf{Lezione del 08/01/2020} (manca, ha dimostrato le cose scritte nelle sue note sui divisori elementari)

\clearpage

\subsection{Forma canonica di Jordan}

\noindent \textbf{Lezione del 10/01/2020} (manca, ha parlato di forma canonica e Teorema di Jordan)

\

\noindent \textbf{Lezione del 14/01/2020} (manca, corollari di Jordan e Cayley-Hamilton)

\

\noindent \textbf{Lezione del 15/01/2020} (manca, recap del corso, aka questo fottutissimo delirio)





\end{document}