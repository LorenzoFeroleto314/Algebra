\documentclass{article}

\usepackage[utf8]{inputenc}
\usepackage[english]{babel}
\usepackage{amsmath}
\usepackage{amssymb}
\usepackage{amsthm}
\usepackage{yhmath}
\usepackage{gensymb}
\usepackage{graphicx}
\usepackage{siunitx}
\usepackage{amscd}
\usepackage{sectsty}
\usepackage{stmaryrd}
\usepackage{tikz-cd}
\usepackage{wrapfig}
\usepackage{xcolor}
\usepackage[margin=1.5in]{geometry}
\usepackage[framemethod=TikZ]{mdframed}

\theoremstyle{definition}
\newtheorem{thm}{Teorema}[subsection]
\newtheorem*{exm}{Esempio}
\renewcommand\qedsymbol{$\blacksquare$}
\addto\captionsenglish{\renewcommand*{\proofname}{Dimostrazione}}
\addto\captionsenglish{\renewcommand{\contentsname}{Indice}}

\sectionfont{\fontsize{20}{15}\selectfont}
\subsectionfont{\fontsize{14}{15}\selectfont}

\renewcommand\thefootnote{\textcolor{red}{\arabic{footnote}}}

\newcommand{\quot}{\operatorname{quot}}
\newcommand{\tor}{\operatorname{tor}}
\newcommand{\sat}{\operatorname{sat}}
\newcommand{\Ann}{\operatorname{Ann}}
\newcommand{\spec}{\operatorname{spec}}
\newcommand{\id}{\operatorname{id}}

\newenvironment{teo}[2][]{%
\ifstrempty{#1}%
{\mdfsetup{%
frametitle={%
\tikz[baseline=(current bounding box.east),outer sep=0pt]
\node[anchor=east,rectangle,fill=blue!25]
{\strut Teorema};}}
}%
{\mdfsetup{%
frametitle={%
\tikz[baseline=(current bounding box.east),outer sep=0pt]
\node[anchor=east,rectangle,fill=blue!25]
{\strut Teorema~#1};}}%
}%
\mdfsetup{innertopmargin=1.5pt,linecolor=blue!25,%
linewidth=1.75pt,topline=true,%
frametitleaboveskip=\dimexpr-\ht\strutbox\relax
}
\begin{mdframed}[]\relax%
}{\end{mdframed}}

\newenvironment{prop}[2][]{%
\ifstrempty{#1}%
{\mdfsetup{%
frametitle={%
\tikz[baseline=(current bounding box.east),outer sep=0pt]
\node[anchor=east,rectangle,fill=blue!25]
{\strut Proposizione};}}
}%
{\mdfsetup{%
frametitle={%
\tikz[baseline=(current bounding box.east),outer sep=0pt]
\node[anchor=east,rectangle,fill=blue!25]
{\strut Proposizione~#1};}}%
}%
\mdfsetup{innertopmargin=1.5pt,linecolor=blue!25,%
linewidth=1.75pt,topline=true,%
frametitleaboveskip=\dimexpr-\ht\strutbox\relax
}
\begin{mdframed}[]\relax%
}{\end{mdframed}}

\newenvironment{cor}[2][]{%
\ifstrempty{#1}%
{\mdfsetup{%
frametitle={%
\tikz[baseline=(current bounding box.east),outer sep=0pt]
\node[anchor=east,rectangle,fill=blue!25]
{\strut Corollario};}}
}%
{\mdfsetup{%
frametitle={%
\tikz[baseline=(current bounding box.east),outer sep=0pt]
\node[anchor=east,rectangle,fill=blue!25]
{\strut Corollario~#1};}}%
}%
\mdfsetup{innertopmargin=1.5pt,linecolor=blue!25,%
linewidth=1.75pt,topline=true,%
frametitleaboveskip=\dimexpr-\ht\strutbox\relax
}
\begin{mdframed}[]\relax%
}{\end{mdframed}}

\newenvironment{lem}[2][]{%
\ifstrempty{#1}%
{\mdfsetup{%
frametitle={%
\tikz[baseline=(current bounding box.east),outer sep=0pt]
\node[anchor=east,rectangle,fill=blue!25]
{\strut Lemma};}}
}%
{\mdfsetup{%
frametitle={%
\tikz[baseline=(current bounding box.east),outer sep=0pt]
\node[anchor=east,rectangle,fill=blue!25]
{\strut Lemma~#1};}}%
}%
\mdfsetup{innertopmargin=1.5pt,linecolor=blue!25,%
linewidth=1.75pt,topline=true,%
frametitleaboveskip=\dimexpr-\ht\strutbox\relax
}
\begin{mdframed}[]\relax%
}{\end{mdframed}}

\newenvironment{defn}[2][]{%
\ifstrempty{#1}%
{\mdfsetup{%
frametitle={%
\tikz[baseline=(current bounding box.east),outer sep=0pt]
\node[anchor=east,rectangle,fill=green!35]
{\strut Definizione};}}
}%
{\mdfsetup{%
frametitle={%
\tikz[baseline=(current bounding box.east),outer sep=0pt]
\node[anchor=east,rectangle,fill=green!35]
{\strut Definizione:~#1};}}%
}%
\mdfsetup{innertopmargin=1.5pt,linecolor=green!35,%
linewidth=1.75pt,topline=true,%
frametitleaboveskip=\dimexpr-\ht\strutbox\relax
}
\begin{mdframed}[]\relax%
}{\end{mdframed}}

%%%%%%%%%%%%%%%%%%%%%%%%%%%%%%%%%%%%%%%%%%%%%%%%%%%%%%%%%%%%%%%%%%%%%

\begin{document}

\subsection{Riducibilità di polinomi}

Concludiamo lo studio degli anelli di polinomi affrontandone il problema della riducibilità.

\begin{defn}[]{}
Sia $R$ un dominio di integrità e sia $f(x)\in R[x]$ un polinomio non invertibile\footnotemark\, e non nullo. Allora, $f(x)$ si dice \underline{irriducibile in $R[x]$} se ogni volta che esprimiamo $f(x)$ come un prodotto $f(x)=g(x)h(x)$ di polinomi $g(x),h(x)\in R[x]$, almeno uno fra $g(x)$ e $h(x)$ è invertibile. Se $f(x)$ non è irriducibile in $R[x]$, diciamo che $f(x)$ è \underline{riducibile in $R[x]$}.
\end{defn}\footnotetext{Si intende rispetto al prodotto, cioè per la \emph{Proposizione 1.1.2} prendiamo $f(x)\not\in R^{\times}$.}

\noindent La riducibilità di un polinomio non è un fatto generale, ma dipende dal particolare dominio di integrità preso in esame: non ha alcun senso parlare di ``polinomio irriducibile'' senza specificare quale sia il dominio d'integrità considerato.

\begin{exm}Il polinomio $f(x)=2x+4$ è irriducibile in $\mathbb{Q}[x]$ ma riducibile in $\mathbb{Z}[x]$. Infatti, se fosse $f(x)=g(x)h(x)$, per la \emph{Proposizione 1.1.1} si avrebbe $\deg^{\star}(f)=1=\deg^{\star}(g)+\deg^{\star}(h)$. Dunque, almeno uno fra $g(x)$ e $h(x)$ ha grado $0$ e risulta quindi invertibile essendo $\mathbb{Q}$ un campo, da cui $f(x)$ è irriducibile in $\mathbb{Q}[x]$. D'altra parte, $2x+4=2(x+2)$ e né $2$ né $x+2$ sono elementi invertibili in $\mathbb{Z}[x]$, quindi $f(x)$ è riducibile in $\mathbb{Z}[x]. \ \square$\end{exm}

\noindent Nel caso in cui il dominio di integrità sia un campo $\mathbb{K}$, poiché ogni elemento non nullo di $\mathbb{K}$ è invertibile, un polinomio non costante $f(x)\in \mathbb{K}[x]$ è riducibile in $\mathbb{K}[x]$ se e solo se può essere espresso come prodotto di due polinomi non costanti di grado minore di $\deg^{\star}(f)$.

\begin{exm}Il polinomio $f(x)=x^2+1$ è irriducibile in $\mathbb{R}[x]$ ma riducibile in $\mathbb{C}[x]$. Infatti, se $f(x)$ fosse riducibile in $\mathbb{R}[x]$, per quanto appena detto esso sarebbe il prodotto di due termini di grado $1$, il che è impossibile poiché $f(x)$ non ha radici reali. D'altra parte, sappiamo che $x^2+1=(x+i)(x-i)$, dunque $f(x)$ è riducibile in $\mathbb{C}[x]. \ \square$\end{exm}

\noindent In generale, stabilire se un polinomio sia o meno irriducibile in un certo dominio di integrità è un problema complesso. Tuttavia, esistono alcuni casi particolari in cui ciò è molto semplice.

\begin{teo}[1.5.1: Criterio del grado]{}
Sia $\mathbb{K}$ un campo e sia $f(x)\in \mathbb{K}[x]$ un polinomio di grado $2$ o $3$. Allora, $f(x)$ è riducibile in $\mathbb{K}[x]$ se e solo se $f(x)$ ha una radice in $\mathbb{K}$.
\end{teo}
\vspace{-4mm}
\begin{proof}
Supponiamo che $f(x)$ sia riducibile in $\mathbb{K}[x]$. Allora, per definizione esistono $g(x),h(x)\in \mathbb{K}[x]$ non costanti di grado minore di $\deg^{\star}(f)$ tali che $f(x)=g(x)h(x)$. Poiché per ipotesi $\deg^{\star}(g)+\deg^{\star}(h)=\deg^{\star}(f)\leq 3$, almeno uno fra $g(x)$ e $h(x)$ ha grado $1$, e senza perdita di generalità sia esso $g(x)=ax+b$. Essendo $\mathbb{K}$ un campo, $\alpha=-a^{-1}b\in \mathbb{K}$, da cui $g(\alpha)=a(-a^{-1}b)+b=0_{\mathbb{K}}$. Dunque, $f(\alpha)=g(\alpha)h(\alpha)=0_{\mathbb{K}}$, cioè $\alpha$ è una radice di $f(x)$.

Viceversa, supponiamo che esista $\alpha\in \mathbb{K}$ tale che $f(\alpha)=0_{\mathbb{K}}$. Per il \emph{Teorema di Ruffini} sappiamo che $(x-\alpha)$ divide $f(x)$, cioè $f(x)=(x-\alpha)q(x)$ per un opportuno $q(x)\in \mathbb{K}[x]$. Poiché $\deg^{\star}(q)=\deg^{\star}(f)-\deg^{\star}(x-\alpha)\geq 2-1=1$, si ha che $f(x)$ è riducibile in $\mathbb{K}[x]$.
\end{proof}

\noindent Tale teorema è particolarmente comodo nel caso dei campi finiti, poiché per stabilire la riducibilità di $f(x)\in \mathbb{F}_p[x]$ è sufficiente verificare se $f(n)\equiv 0\pmod{p}$ per $n=0,1,...\,,p-1$.

\begin{exm}Il polinomio $f(x)=x^3+x+1$ è irriducibile in $\mathbb{F}_2[x]$ ma riducibile in $\mathbb{F}_3[x]$. Infatti, $f(0)\equiv f(1)\equiv 1\not\equiv 0\pmod{2}$ in $\mathbb{F}_2$, ma $f(1)=3\equiv 0\pmod{3}$ in $\mathbb{F}_3. \ \square$\end{exm} \clearpage

\noindent Osserviamo che il \emph{Teorema 1.5.1} vale solo nei campi, dunque non è applicabile in $\mathbb{Z}$. Inoltre, esistono polinomi riducibili di grado maggiore o uguale a $4$ che non hanno radici.

\begin{exm}Entrambi i polinomi $f(x)=x^4+1$ e $g(x)=x^6+1$ non ammettono chiaramente radici reali. Tuttavia, osserviamo che $x^4+1=(x^2+\sqrt{2}x+1)(x^2-\sqrt{2}x+1)$ e possiamo scomporre $x^6+1=(x^2+1)(x^4-x^2+1)$, dunque $f(x)$ e $g(x)$ sono riducibili in $\mathbb{R}[x]. \ \square$\end{exm}

\noindent Di qui in seguito ci concentreremo principalmente sul problema della riducibilità in $\mathbb{Z}[x]$.

\begin{defn}[]{}
Sia $f(x)=\sum\limits_{i=0}^n a_i x^i \in \mathbb{Z}[x]$ un polinomio non nullo. Si definisce \underline{contenuto} di $f$ il valore di $\operatorname{MCD}(a_0,...\,,a_n)$. Un polinomio si dice \underline{primitivo} se il suo contenuto è $1$.
\end{defn}

\begin{exm}Il polinomio $f(x)=2x^2+3x+4$ è primitivo perché $\operatorname{MCD}(2,3,4)=1$. D'altra parte, il polinomio $g(x)=2x^2+4$ non è primitivo poiché $\operatorname{MCD}(2,0,4)=2\neq 1. \ \square$\end{exm}

\noindent Osserviamo che presi i due polinomi primitivi $f(x)=x+1$ e $g(x)=2x+3$, anche il loro prodotto $f(x)g(x)=2x^2+5x+3$ è primitivo, poiché il suo contenuto è $\operatorname{MCD}(2,5,3)=1$. Questo è un fatto generale, come dimostrato dal lemma seguente.

\begin{lem}[1.5.2: Lemma di Gauss]{}
Il prodotto di due polinomi primitivi è un polinomio primitivo.
\end{lem}
\vspace{-4mm}
\begin{proof}
Siano $f(x),\,g(x)\in \mathbb{Z}[x]$ polinomi primitivi, e supponiamo per assurdo che $f(x)g(x)$ non sia primitivo. Allora, esiste $p$ primo che divide tutti i coefficienti di $f(x)g(x)$, cioè $f(x)g(x)\equiv 0$ in $\mathbb{F}_p[x]$. Poiché $\mathbb{F}_p[x]$ è un dominio di integrità, deve essere $f(x)\equiv 0$ oppure $g(x)\equiv 0$, da cui $p$ divide tutti i coefficienti di almeno uno fra $f(x)$ e $g(x)$, e tale polinomio risulta quindi non primitivo, assurdo. Dunque, $f(x)g(x)$ è primitivo.
\end{proof}

\noindent Esiste una stretta relazione tra la riducibilità in $\mathbb{Z}[x]$ e quella in $\mathbb{Q}[x]$.

\begin{teo}[1.5.3]{}
Sia $f(x)\in \mathbb{Z}[x]$ un polinomio irriducibile in $\mathbb{Z}[x]$. Allora, $f(x)$ è irriducibile in $\mathbb{Q}[x]$.
\end{teo}
\vspace{-4mm}
\begin{proof}
Supponiamo per assurdo che $f(x)$ sia riducibile in $\mathbb{Q}[x]$. Allora, esistono $g(x),\,h(x)\in \mathbb{Q}[x]$ non costanti tali che $f(x)=g(x)h(x)$, dove, a meno di dividere $g(x)$ per il contenuto di $f$, possiamo assumere senza perdita di generalità che $f(x)$ sia primitivo. Siano $a$ e $b$ il minimo comune multiplo dei denominatori dei coefficienti di $g(x)$ e $h(x)$, rispettivamente, così che $ag(x)$ e $bh(x)$ siano polinomi a coefficienti interi. Detti $c_1$ e $c_2$ il contenuto di $ag(x)$ e $bh(x)$, rispettivamente, si ha che $ag(x)=c_1 g'(x)$ e $bh(x)=c_2h'(x)$, dove $g'(x)$ e $h'(x)$ sono polinomi primitivi. Poiché $abf(x)=ag(x)\,bh(x)=c_1c_2\,g'(x)h'(x)$ e per il \emph{Lemma 1.5.2} anche $g'(x)h'(x)$ è primitivo, deve essere $ab=c_1c_2$. Dunque, si ha che $f(x)=g'(x)h'(x)$ dove $g'(x),\,h'(x)\in \mathbb{Z}[x]$, cioè $f(x)$ è riducibile in $\mathbb{Z}[x]$, assurdo.
\end{proof}

\noindent Sebbene $\mathbb{Q}$ sia un campo più grande di $\mathbb{Z}$, tale teorema mostra che esso non è abbastanza grande per permettere di scomporre in $\mathbb{Q}[x]$ un polinomio irriducibile in $\mathbb{Z}[x]$, ed è quindi necessario passare a campi ancora più grandi quali $\mathbb{R}$ e $\mathbb{C}$. Inoltre, la dimostrazione mostra che se un polinomio $f(x)\in \mathbb{Z}[x]$ è riducibile in $\mathbb{Q}[x]$, allora esso è riducibile anche in $\mathbb{Z}[x]$.

\begin{exm}Sia $f(x)=6x^2-5x+1=(3x-\frac{3}{2})(2x-\frac{2}{3})$ un polinomio riducibile in $\mathbb{Q}[x]$. Utilizzando la notazione del \emph{Teorema 1.5.3}, definiamo $g(x)=(3x-\frac{3}{2})$ e $h(x)=(2x-\frac{2}{3})$. Allora, $a=2$ e $b=3$, da cui $ag(x)=6x-3$ e $bh(x)=6x-2$. Dunque, $c_1=\operatorname{MCD}(6,3)=3$ e $c_2=\operatorname{MCD}(6,2)=2$, da cui $g'(x)=2x-1$ e $h'(x)=3x-1$ sono polinomi primitivi e $f(x)=g'(x)h'(x)=(2x-1)(3x-1)$ risulta quindi riducibile in $\mathbb{Z}[x]. \ \square$\end{exm}

\

\noindent Sketch del capitolo: riduzione mod p, Eisenstein, polinomi ciclotomici, tanti esempi, e tutto quello che Weigel dà per scontato sia stato fatto ad Algebra 1.

\end{document}