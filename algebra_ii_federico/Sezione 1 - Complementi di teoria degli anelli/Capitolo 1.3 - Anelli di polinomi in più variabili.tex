\documentclass{article}

\usepackage[utf8]{inputenc}
\usepackage[english]{babel}
\usepackage{amsmath}
\usepackage{amssymb}
\usepackage{amsthm}
\usepackage{yhmath}
\usepackage{gensymb}
\usepackage{graphicx}
\usepackage{siunitx}
\usepackage{amscd}
\usepackage{sectsty}
\usepackage{stmaryrd}
\usepackage{tikz-cd}
\usepackage{wrapfig}
\usepackage{xcolor}
\usepackage[margin=1.5in]{geometry}
\usepackage[framemethod=TikZ]{mdframed}

\theoremstyle{definition}
\newtheorem{thm}{Teorema}[subsection]
\newtheorem*{exm}{Esempio}
\renewcommand\qedsymbol{$\blacksquare$}
\addto\captionsenglish{\renewcommand*{\proofname}{Dimostrazione}}
\addto\captionsenglish{\renewcommand{\contentsname}{Indice}}

\sectionfont{\fontsize{20}{15}\selectfont}
\subsectionfont{\fontsize{14}{15}\selectfont}

\renewcommand\thefootnote{\textcolor{red}{\arabic{footnote}}}

\newcommand{\quot}{\operatorname{quot}}
\newcommand{\tor}{\operatorname{tor}}
\newcommand{\sat}{\operatorname{sat}}
\newcommand{\Ann}{\operatorname{Ann}}
\newcommand{\spec}{\operatorname{spec}}
\newcommand{\id}{\operatorname{id}}

\newenvironment{teo}[2][]{%
\ifstrempty{#1}%
{\mdfsetup{%
frametitle={%
\tikz[baseline=(current bounding box.east),outer sep=0pt]
\node[anchor=east,rectangle,fill=blue!25]
{\strut Teorema};}}
}%
{\mdfsetup{%
frametitle={%
\tikz[baseline=(current bounding box.east),outer sep=0pt]
\node[anchor=east,rectangle,fill=blue!25]
{\strut Teorema~#1};}}%
}%
\mdfsetup{innertopmargin=1.5pt,linecolor=blue!25,%
linewidth=1.75pt,topline=true,%
frametitleaboveskip=\dimexpr-\ht\strutbox\relax
}
\begin{mdframed}[]\relax%
}{\end{mdframed}}

\newenvironment{prop}[2][]{%
\ifstrempty{#1}%
{\mdfsetup{%
frametitle={%
\tikz[baseline=(current bounding box.east),outer sep=0pt]
\node[anchor=east,rectangle,fill=blue!25]
{\strut Proposizione};}}
}%
{\mdfsetup{%
frametitle={%
\tikz[baseline=(current bounding box.east),outer sep=0pt]
\node[anchor=east,rectangle,fill=blue!25]
{\strut Proposizione~#1};}}%
}%
\mdfsetup{innertopmargin=1.5pt,linecolor=blue!25,%
linewidth=1.75pt,topline=true,%
frametitleaboveskip=\dimexpr-\ht\strutbox\relax
}
\begin{mdframed}[]\relax%
}{\end{mdframed}}

\newenvironment{cor}[2][]{%
\ifstrempty{#1}%
{\mdfsetup{%
frametitle={%
\tikz[baseline=(current bounding box.east),outer sep=0pt]
\node[anchor=east,rectangle,fill=blue!25]
{\strut Corollario};}}
}%
{\mdfsetup{%
frametitle={%
\tikz[baseline=(current bounding box.east),outer sep=0pt]
\node[anchor=east,rectangle,fill=blue!25]
{\strut Corollario~#1};}}%
}%
\mdfsetup{innertopmargin=1.5pt,linecolor=blue!25,%
linewidth=1.75pt,topline=true,%
frametitleaboveskip=\dimexpr-\ht\strutbox\relax
}
\begin{mdframed}[]\relax%
}{\end{mdframed}}

\newenvironment{lem}[2][]{%
\ifstrempty{#1}%
{\mdfsetup{%
frametitle={%
\tikz[baseline=(current bounding box.east),outer sep=0pt]
\node[anchor=east,rectangle,fill=blue!25]
{\strut Lemma};}}
}%
{\mdfsetup{%
frametitle={%
\tikz[baseline=(current bounding box.east),outer sep=0pt]
\node[anchor=east,rectangle,fill=blue!25]
{\strut Lemma~#1};}}%
}%
\mdfsetup{innertopmargin=1.5pt,linecolor=blue!25,%
linewidth=1.75pt,topline=true,%
frametitleaboveskip=\dimexpr-\ht\strutbox\relax
}
\begin{mdframed}[]\relax%
}{\end{mdframed}}

\newenvironment{defn}[2][]{%
\ifstrempty{#1}%
{\mdfsetup{%
frametitle={%
\tikz[baseline=(current bounding box.east),outer sep=0pt]
\node[anchor=east,rectangle,fill=green!35]
{\strut Definizione};}}
}%
{\mdfsetup{%
frametitle={%
\tikz[baseline=(current bounding box.east),outer sep=0pt]
\node[anchor=east,rectangle,fill=green!35]
{\strut Definizione:~#1};}}%
}%
\mdfsetup{innertopmargin=1.5pt,linecolor=green!35,%
linewidth=1.75pt,topline=true,%
frametitleaboveskip=\dimexpr-\ht\strutbox\relax
}
\begin{mdframed}[]\relax%
}{\end{mdframed}}

%%%%%%%%%%%%%%%%%%%%%%%%%%%%%%%%%%%%%%%%%%%%%%%%%%%%%%%%%%%%%%%%%%%%%

\begin{document}

\subsection{Anelli di polinomi in più variabili}

\noindent Vogliamo ora generalizzare il concetto di anello di polinomi ad un numero qualsiasi di variabili, anche infinito. Sia $X$ un insieme non vuoto e sia $\mathcal{F}^{\times}=\mathcal{F}^{\times}(X,\mathbb{N})$ l'insieme delle funzioni $\underline{\alpha}\colon X\to \mathbb{N}$ che hanno supporto finito.\footnote{Notare come a differenza dei polinomi in $n$ variabili, ora richiediamo esplicitamente che tali funzioni $\underline{\alpha}$ abbiano supporto finito. Infatti, nel caso dei polinomi in $n$ variabili, $X$ è un insieme finito con $n$ elementi, quindi ogni funzione $\underline{\alpha}\colon X\to \mathbb{N}$ ha in realtà supporto finito perché $\operatorname{supp}(\underline{\alpha})\subseteq X,$ che è finito. Dunque, se $|X|<\infty$, non vi è differenza tra $\mathcal{F}^{\times}(X, \mathbb{N})=\mathcal{F}(X,\mathbb{N})$.}
\vspace{-0.75mm}
\begin{defn}[]{}
Sia $X$ un insieme. Denotiamo con $M=\operatorname{mon}\{X\}$ l'\underline{insieme dei monomi di $X$}, cioè $M=\left\{X^{\underline{\alpha}}: \underline{\alpha}\in \mathcal{F}^{\times}\right\}$ dove $X^{\underline{\alpha}}=\prod\limits_{x\in X} x^{\underline{\alpha}(x)}$.
\end{defn}
\vspace{-1.5mm}
\noindent Poiché abbiamo scelto $\underline{\alpha}$ con supporto finito, osserviamo che ogni monomio di $X$ è il prodotto di un numero finito di elementi di $X$, anche nel caso in cui $X$ sia un insieme infinito. Inoltre, analogamente al caso dei polinomi in $n$ variabili, $M$ è un monoide commutativo ed esiste una corrispondenza biunivoca tra i monomi di $M$ e le funzioni di $\mathcal{F}^{\times}$.

\vspace{1.5mm}

\noindent Sia $R$ un anello commutativo e sia $\mathcal{F}^{\times}(\mathcal{F}^{\times}, R)=\left\{ f\colon \mathcal{F}^{\times}\to R : |\operatorname{supp}(f)| <\infty \right\}$, cioè l'insieme delle funzioni che associano ad ogni funzione di $\mathcal{F}^{\times}$ un elemento dell'anello $R$, e che sono diverse da $0_R$ solo per un numero finito elementi di $\mathcal{F}^{\times}$. Al variare di $\underline{\alpha}\in \mathcal{F}^{\times}$, sia $r\underline{\, \, \,}\in \mathcal{F}^{\times}(\mathcal{F}^{\times},R)$ la funzione che associa ad ogni $\underline{\alpha}\in \mathcal{F}^{\times}$ l'elemento $r_{\underline{\alpha}}\in R.$ Osserviamo che possiamo definire un polinomio a variabili in $X$ ponendo {\setlength{\belowdisplayskip}{2.75pt} \setlength{\abovedisplayskip}{4.75pt} \[ f(X)=\sum\limits_{\underline{\alpha}\in \mathcal{F}^{\times}}r_{\underline{\alpha}} X^{\underline{\alpha}}.\]}\noindent Infatti, $f(X)$ è la somma di un numero finito di monomi non nulli, ognuno con un numero finito di variabili e preceduto dal relativo coefficiente $r_{\underline{\alpha}}.$ 

\vspace{0.6mm}

\noindent Sia $R[X]=\left\{ \sum\limits_{\underline{\alpha}\in \mathcal{F}^{\times}} r_{\underline{\alpha}} X^{\underline{\alpha}} : r\underline{\, \, \,}\in \mathcal{F}^{\times}(\mathcal{F}^{\times},R) \right\}$. Presi due elementi $f(X)=\sum\limits_{\underline{\alpha}\in \mathcal{F}^{\times}}r_{\underline{\alpha}} X^{\underline{\alpha}}\,$ e $g(X)=\sum\limits_{\underline{\beta}\in \mathcal{F}^{\times}}s_{\underline{\beta}} X^{\underline{\beta}}\,$ di $R[X],$ definiamo su $R[X]$ le operazioni binarie di somma e prodotto {\setlength{\belowdisplayskip}{2pt} \setlength{\abovedisplayskip}{-4.5pt}\[f(X)+g(X)=\sum\limits_{\underline{\alpha}\in \mathcal{F}^{\times}}(r_{\underline{\alpha}}+s_{\underline{\alpha}}) X^{\underline{\alpha}}\]} \[f(X)\cdot g(X)=\sum\limits_{\underline{\gamma}\in \mathcal{F}^{\times}}t_{\underline{\gamma}}X^{\underline{\gamma}}\] dove abbiamo posto $\underline{\gamma}=\underline{\alpha}+\underline{\beta}$ e $t_{\underline{\gamma}}=\sum\limits_{\underline{\alpha}+\underline{\beta}=\underline{\gamma}}r_{\underline{\alpha}}s_{\underline{\beta}}$. In modo del tutto analogo a quanto visto nel caso di $R[x_1,...\,,x_n]$, si dimostra che tali operazioni sono ben poste e che $R[X]$ dotato di tali operazioni di somma e prodotto è un anello commutativo con elemento neutro il polinomio nullo $\sum\limits_{\underline{\alpha}\in \mathcal{F}^{\times}} 0_{\underline{\alpha}}X^{\underline{\alpha}}=0_R$ e unità il monomio banale $X^{\underline{0}}=1_R$.
\vspace{-3mm}
\begin{defn}[]{}
Sia $R$ un anello commutativo e sia $X$ un insieme non vuoto. Allora, l'insieme $R[X]$ è detto \underline{anello dei polinomi a coefficienti in $R$ e a variabili in $X$}.
\end{defn}
\clearpage
\noindent Anche per gli anelli di polinomi in più variabili vale la \emph{Proprietà universale}.

\begin{teo}[1.3.1: Proprietà universale]{}
Sia $X$ un insieme e sia $R$ un anello commutativo. Allora, per ogni anello commutativo $S\supseteq R$ e per ogni mappa $\varphi\colon X\to S$ esiste un unico omomorfismo di anelli $\phi\colon R[X]\to S$ tale che $\phi(X^{\underline{\delta}_x})=\varphi(x) \ \forall x\in X \text{ e } \phi \raisebox{-.5em}{$\vert_{R}$}= \operatorname{id}_R$, dove $\underline{\delta}_x\colon X\to \mathbb{N}$, $\underline{\delta}_x(y)=\begin{cases}1 \text{ \ se }y=x\\0 \text{ \ se } y\neq x\end{cases}$
\end{teo}
\vspace{-3mm}
\begin{proof}Siano $f=\sum\limits_{\underline{\alpha}\in \mathcal{F}^{\times}} r_{\underline{\alpha}}X^{\underline{\alpha}}\,$ e $g=\sum\limits_{\underline{\beta}\in \mathcal{F}^{\times}} s_{\underline{\beta}}X^{\underline{\beta}}\,$ due elementi di $R[X].$ Per ogni monomio $X^{\underline{\alpha}}\in M$, sia $\phi(X^{\underline{\alpha}})=\prod\limits_{x\in X} \varphi(x)^{\underline{\alpha}(x)}$, e sia quindi $\phi(f)=\sum\limits_{\underline{\alpha}\in \mathcal{F}^{\times}} r_{\underline{\alpha}}\phi(X^{\underline{\alpha}}).$ Poiché $r_{\underline{\alpha}}\in R\subseteq S$ per ipotesi e $\phi(f)\in S$ perché somma di prodotti di elementi di $S,$ che in quanto anello è chiuso rispetto a somma e prodotto, $\phi$ è ben definita. Inoltre, $\phi(X^{\underline{\delta}_x})=\varphi(x)$ e $\phi(\rho)=\rho$ per ogni $\rho\in R,$ quindi $\phi$ soddisfa le condizioni richieste.\footnote{$\underline{\delta}_x$ è la funzione tale che per ogni $x\in X$ si abbia $X^{\underline{\delta}_x}=x$. Infatti, $X^{\underline{\delta}_x}=\prod\limits_{y\in X}y^{\underline{\delta}_x(y)}=x^{\underline{\delta}_x(x)}=x^1=x$ perché tutti gli altri termini del prodotto hanno esponente $0$, essendo per definizione $\underline{\delta}_x(y)=0$ se $y\neq x$.} Mostriamo ora che è un omomorfismo di anelli. Infatti, \[\phi\left(f+g\right)=\sum\limits_{\underline{\alpha}\in \mathcal{F}^{\times}}(r_{\underline{\alpha}}+s_{\underline{\alpha}}) \phi(X^{\underline{\alpha}})=\sum\limits_{\underline{\alpha}\in \mathcal{F}^{\times}}r_{\underline{\alpha}}\phi(X^{\underline{\alpha}})+\sum\limits_{\underline{\alpha}\in \mathcal{F}^{\times}}s_{\underline{\alpha}} \phi(X^{\underline{\alpha}})=\phi(f)+\phi(g)\] per la proprietà distributiva del prodotto rispetto alla somma, essendo $S$ un anello, e \[\phi(f\cdot g)=\sum\limits_{\underline{\gamma}\in \mathcal{F}^{\times}} \sum\limits_{\underline{\alpha}+\underline{\beta}=\underline{\gamma}} r_{\underline{\alpha}}s_{\underline{\beta}} \ \phi(X^{\underline{\gamma}})=\left(\sum\limits_{\underline{\alpha}\in \mathcal{F}^{\times}} r_{\underline{\alpha}}\phi(X^{\underline{\alpha}})\right)\cdot \left(\sum\limits_{\underline{\beta}\in \mathcal{F}^{\times}} s_{\underline{\beta}}\phi(X^{\underline{\beta}})\right)=\phi(f)\cdot \phi(g)\] perché $\phi(X^{\underline{\gamma}})=\prod\limits_{x\in X} \varphi(x)^{\underline{\gamma}(x)}=\prod\limits_{x\in X} \varphi(x)^{\underline{\alpha}(x)}\cdot \prod\limits_{x\in X} \varphi(x)^{\underline{\beta}(x)}=\phi(X^{\underline{\alpha}})\cdot \phi(X^{\underline{\beta}})$. Poiché $\phi(0_R)=0_S$ e $ \phi(1_R)=1_S$, concludiamo che $\phi$ è un omomorfismo di anelli.

\vspace{1mm}

Mostriamo ora che $\phi$ è unico. Sia $\psi\colon R[X]\to S$ un omomorfismo di anelli tale che $\psi(X^{\underline{\delta}_x})=\varphi(x)$ e $\psi \raisebox{-.5em}{$\vert_{R}$}= \operatorname{id}_R.$ Allora, per ogni monomio $X^{\underline{\alpha}}\in M$ vale \[ \psi(X^{\underline{\alpha}})=\psi\left(\prod\limits_{x\in X} x^{\underline{\alpha}(x)}\right)=\prod\limits_{x\in X} \psi\left(x^{\underline{\alpha}(x)}\right)=\prod\limits_{x\in X} \psi(X^{\underline{\delta}_x})^{\underline{\alpha}(x)}=\prod\limits_{x\in X} \varphi(x)^{\underline{\alpha}(x)}=\phi(X^{\underline{\alpha}}). \] Poiché $\psi$ è un omomorfismo, per ogni $f=\sum\limits_{\underline{\alpha}\in \mathcal{F}^{\times}} r_{\underline{\alpha}}X^{\underline{\alpha}}\in R[X]$ si ha che \[ \psi(f)=\psi\left(\sum\limits_{\underline{\alpha}\in \mathcal{F}^{\times}} r_{\underline{\alpha}}X^{\underline{\alpha}}\right)=\sum\limits_{\underline{\alpha}\in \mathcal{F}^{\times}} \psi(r_{\underline{\alpha}}X^{\underline{\alpha}})=\sum\limits_{\underline{\alpha}\in \mathcal{F}^{\times}} \psi(r_{\underline{\alpha}})\psi(X^{\underline{\alpha}})=\sum\limits_{\underline{\alpha}\in \mathcal{F}^{\times}} r_{\underline{\alpha}}\phi(X^{\underline{\alpha}})=\phi(f)\] essendo $\psi(r_{\underline{\alpha}})=r_{\underline{\alpha}}$ perché $r_{\underline{\alpha}}\in R$ e $\psi(X^{\underline{\alpha}})=\phi(X^{\underline{\alpha}})$ per quanto appena mostrato. Dunque, $\psi$ coincide con $\phi,$ che risulta quindi essere unico.\end{proof}

\noindent In modo del tutto analogo al \emph{Teorema 1.1.4} è possibile mostrare che, a meno di isomorfismi, $R[X]$ è l'unico anello contenente $R$ avente questa proprietà. \clearpage

\noindent Sia $R[x]$ l'anello dei polinomi a coefficienti in $R$ nella variabile $x.$ Possiamo considerare $R[x]$ stesso come anello dei coefficienti per l'anello dei polinomi nella variabile $y$, cioè \[(R[x])[y]=\left\{\sum\limits_{i=0}^n f_i\,y^i : f_i\in R[x],\, n\in \mathbb{N}\right\}.\] Poiché ogni polinomio di $(R[x])[y]$ può essere visto come un polinomio in due variabili di $R[x,y]$ e ogni polinomio di $R[x,y]$ può essere pensato come un polinomio di $(R[x])[y]$ raccogliendo i termini dello stesso grado in $y$, questo suggerisce che $(R[x])[y]\simeq R[x,y]$.

\begin{exm}Sia $f(y)=(x^2+1)y^2+(2x)y+3 \in (\mathbb{Z}[x])[y]$. Allora, possiamo vedere $f(y)$ come un polinomio in due variabili $g(x,y)=x^2y^2+y^2+2xy+3 \in \mathbb{Z}[x,y]$. Viceversa, preso $p(x,y)=xy^2+2xy+3y+4\in \mathbb{Z}[x,y]$, raccogliendo i termini dello stesso grado in $y$ possiamo pensare $p(x,y)$ come un polinomio $q(y)=(x)y^2+(2x+3)y+4 \in (\mathbb{Z}[x])[y]. \ \square$\end{exm}

\noindent In generale, se $X$ e $Y$ sono insiemi non vuoti e $(R[X])[Y]$ è l'anello dei polinomi a coefficienti in $R[X]$ e a variabili in $Y$, detta $X \sqcup Y$ l'unione disgiunta,\footnote{Ricordiamo che l'unione disgiunta di una famiglia di insiemi $\{A_i\}_{i\in I}$ è l'insieme $\bigsqcup\limits_{i\in I} A_i = \bigcup\limits_{i\in I} (A \times \{i\})$. Ad esempio, presi $A_0=\{3,4,5\}$ e $A_1=\{5,6\}$, si ha che $A_0 \sqcup A_1=\{(3,0), (4,0), (5,0), (5,1), (6,1)\}$.} vale il teorema seguente.

\begin{teo}[1.3.2]{}
Sia $R$ un anello commutativo e siano $X$ e $Y$ non vuoti. Allora, $R[X\sqcup Y]\simeq (R[X])[Y]$.
\end{teo}
\vspace{-4mm}
\begin{proof}
Sia $S$ un anello commutativo tale che $R\subseteq R[X]\subseteq S$ e sia $\varphi_X\colon X\to S$ definita come $\varphi_X(x)=X^{\underline{\delta}_x}$. Presa una qualunque funzione $\varphi_Y\colon Y\to S$, sia $\widetilde{\varphi}\colon X\sqcup Y\to S$ l'unica mappa tale che $\widetilde{\varphi} \raisebox{-.5em}{$\vert_{X}$}=\varphi_X$ e $\widetilde{\varphi} \raisebox{-.5em}{$\vert_{Y}$}=\varphi_Y$. Allora, per il \emph{Teorema 1.3.1} esiste un unico omomorfismo $\widetilde{\phi}\colon R[X\sqcup Y]\to S$ tale che $\widetilde{\phi}(Z^{\underline{\delta}_z})=\widetilde{\varphi}(z)$ per ogni $z\in X\sqcup Y$ e $\widetilde{\phi} \raisebox{-.5em}{$\vert_{R}$}= \operatorname{id}_R$. Per ogni $\underline{\alpha}\in \mathcal{F}^{\times}(X,\mathbb{N})$, sia $\underline{\widetilde{\alpha}}\in \mathcal{F}^{\times}(X \sqcup Y,\mathbb{N})$ l'unica funzione tale che $\underline{\widetilde{\alpha}} \raisebox{-.5em}{$\vert_{X}$} = \underline{\alpha}$ e $\underline{\widetilde{\alpha}} \raisebox{-.5em}{$\vert_{Y}$} = \underline{0}$. Allora, possiamo pensare ogni monomio $X^{\underline{\alpha}}$ di $R[X]$ come monomio $Z^{\widetilde{\underline{\alpha}}}$ di $R[X\sqcup Y]$, da~cui 
{\setlength{\belowdisplayskip}{2.5pt} \setlength{\abovedisplayskip}{5pt}\begin{align*} \widetilde{\phi}(Z^{\widetilde{\underline{\alpha}}}) &= \widetilde{\phi}\left( \prod\limits_{z\in X\sqcup Y} z^{\widetilde{\underline{\alpha}}(z)} \right) = \prod\limits_{z\in X\sqcup Y} \widetilde{\phi}\left(z^{\widetilde{\underline{\alpha}}(z)}\right) = \prod\limits_{z\in X\sqcup Y} \widetilde{\phi}\left(Z^{\underline{\delta}_z}\right)^{\widetilde{\underline{\alpha}}(z)} = \prod\limits_{z\in X\sqcup Y} \widetilde{\varphi}(z)^{\widetilde{\underline{\alpha}}(z)} \\ &= \prod\limits_{x\in X} \varphi_X(x)^{\underline{\alpha}(x)}\cdot \prod\limits_{y\in Y} \varphi_Y(y)^{\underline{0}} = \prod\limits_{x\in X} (X^{\underline{\delta}_x})^{\underline{\alpha}(x)}\cdot 1_R = X^{\underline{\alpha}} \end{align*}}\noindent per come abbiamo definito $\widetilde{\varphi}$ e $\underline{\widetilde{\alpha}}$ ed usando il fatto che $\widetilde{\phi}$ è un omomorfismo. Quindi, preso $f=\sum\limits_{\underline{\alpha}\in \mathcal{F}^{\times}} r_{\underline{\alpha}} X^{\underline{\alpha}}\in R[X]$, pensando $f$ come elemento $\widetilde{f}=\sum\limits_{\widetilde{\underline{\alpha}}\in \mathcal{F}^{\times}} r_{\widetilde{\underline{\alpha}}} Z^{\widetilde{\underline{\alpha}}}\in R[X\sqcup Y]$ si ha che \[ \widetilde{\phi}(\widetilde{f}\,) = \sum\limits_{\widetilde{\underline{\alpha}}\in \mathcal{F}^{\times}} \widetilde{\phi}(r_{\widetilde{\underline{\alpha}}}) \widetilde{\phi}(Z^{\widetilde{\underline{\alpha}}}) = \sum\limits_{\widetilde{\underline{\alpha}}\in \mathcal{F}^{\times}} r_{\widetilde{\underline{\alpha}}}\,\widetilde{\phi}(Z^{\widetilde{\underline{\alpha}}}) = \sum\limits_{\underline{\alpha}\in \mathcal{F}^{\times}} r_{\underline{\alpha}} X^{\underline{\alpha}} = f \] perché $\widetilde{\phi}(r_{\widetilde{\underline{\alpha}}})=r_{\widetilde{\underline{\alpha}}}$ essendo $\widetilde{\phi} \raisebox{-.5em}{$\vert_{R}$}= \operatorname{id}_R$, da cui $\widetilde{\phi} \raisebox{-.5em}{$\vert_{R[X]}$}=\operatorname{id}_{R[X]}$. Inoltre, per ogni $y\in Y$ si ha che $\widetilde{\phi}(Z^{\underline{\delta}_y})=\widetilde{\varphi}(y)=\varphi_Y(y)$. Poiché $R[X\sqcup Y]$ è un anello commutativo contenente $R[X]$ che soddisfa la proprietà universale di $(R[X])[Y]$,\footnote{Infatti, abbiamo appena mostrato che per ogni anello $S\supseteq R[X]$ e per ogni mappa $\varphi_Y\colon Y\to S$, esiste un unico omomorfismo $\widetilde{\phi}\colon R[X\sqcup Y]\to S$ tale che $\widetilde{\phi}(Z^{\underline{\delta}_y})=\varphi_Y(y)$ per ogni $y\in Y$ e $\widetilde{\phi} \raisebox{-.5em}{$\vert_{R[X]}$}=\operatorname{id}_{R[X]}$.} per la generalizzazione del \emph{Teorema 1.1.4} possiamo effettivamente concludere che $R[X \sqcup Y]\simeq (R[X])[Y]$.
\end{proof}
\clearpage

\noindent Nel caso in cui l'insieme delle variabili sia finito, vale il corollario seguente.

\begin{cor}[1.3.3]{}
Sia $n$ un intero positivo. Allora, $R[x_1,...\,,x_n]\simeq (\cdots((R[x_1])[x_2])\cdots )[x_n]$.
\end{cor}
\vspace{-4mm}
\begin{proof}
Procediamo per induzione sul numero $n$ di variabili. Chiaramente, se $n=1$ allora $R[x_1]\simeq R[x_1]$. Supponiamo quindi che la tesi sia vera per un certo intero $n\geq 1.$ Detti $X=\{x_1,...\,,x_n\}$ e $Y=\{x_{n+1}\}$, per il \emph{Teorema 1.3.2} si ha che $R[X\sqcup Y]\simeq (R[X])[Y]$ da cui $R[x_1,...\,,x_{n+1}]\simeq (R[x_1,...\,,x_n])[x_{n+1}]\simeq ((\cdots((R[x_1])[x_2])\cdots )[x_n])[x_{n+1}]$.\end{proof}

\noindent Possiamo quindi estendere agli anelli di polinomi in più variabili anche la \emph{Proposizione 1.1.1}. Per fare ciò, osserviamo innanzitutto che ogni polinomio di $R[X]$ è la somma di un numero finito di monomi non nulli, ognuno con un numero finito di variabili. Dunque, ogni polinomio di $R[X]$ può essere pensato come un polinomio in un numero finito di variabili, o meglio, per ogni $f\in R[X]$ esiste un sottoinsieme delle variabili $X_f\subseteq X$ finito tale che $f\in R[X_f]$.\footnote{Più formalmente, preso $f=\sum\limits_{\underline{\alpha}\in \mathcal{F}^{\times}}r_{\underline{\alpha}} X^{\underline{\alpha}}\in R[X]$ sappiamo che $\Omega_f=\operatorname{supp}(r\underline{\, \, \,}\,)\subseteq \mathcal{F}^{\times}$ è finito, quindi esiste solo un numero finito di funzioni $\underline{\alpha}\in \mathcal{F}^{\times}$ per cui il monomio $X^{\underline{\alpha}}$ ha un coefficiente $r_{\underline{\alpha}}$ non nullo. Poiché ogni $\underline{\alpha}\in \mathcal{F}^{\times}$ ha supporto finito, $X_f=\bigcup\limits_{\underline{\alpha}\in \Omega_f}\operatorname{supp}(\underline{\alpha})$ è finito in quanto unione finita di insiemi finiti.}

\begin{prop}[1.3.4]{}
Sia $X$ un insieme non vuoto e sia $R$ un dominio di integrità. Allora, anche l'anello dei polinomi $R[X]$ è un dominio di integrità.
\end{prop}
\vspace{-4mm}
\begin{proof}
Siano $f,g\in R[X]$ e siano $X_f, X_g\subseteq X$ finiti tali che $f\in R[X_f]$ e $g\in R[X_g]$. Osserviamo innanzitutto che $X_f\cup X_g$ è un sottoinsieme finito di $X$ e $f\cdot g\in R[X_f\cup X_g]$. Dunque, detto $X_f\cup X_g=\{x_1,...\,,x_n\}$, per dimostrare che $R[X]$ è un dominio di integrità~è sufficiente provare che $R[x_1,...\,,x_n]$ è un dominio di integrità.\footnote{Se il polinomio $f\cdot g$ si annulla in $R[X]$, allora si annulla anche pensato come polinomio di $R[X_f\cup X_g]$. Dunque, se $R[X_f\cup X_g]$ è un dominio di integrità per ogni $f,g\in R[X]$, allora anche $R[X]$ deve essere un dominio di integrità. Infatti, se esistessero $f,g\in R[X]$ divisori dello zero, per quanto appena detto essi sarebbero divisori dello zero anche in $R[X_f\cup X_g]$, il che contraddice la definizione di dominio di integrità.}\,Per fare ciò, procediamo~per induzione sul numero di variabili. Se $n=1$, per la \emph{Proposizione 1.1.1} sappiamo che $R[y_1]$ è un dominio di integrità. Supponiamo quindi che la tesi valga per un certo intero $n\geq 1$. Allora, per il \emph{Corollario 1.3.3} si ha che $R[y_1,...\,,y_{n+1}]\simeq (R[y_1,...\,,y_n])[y_{n+1}]$, ed essendo $R[y_1,...\,,y_n]$ un dominio di integrità per ipotesi induttiva, per la \emph{Proposizione 1.1.1} anche $(R[y_1,...\,,y_n])[y_{n+1}]$ è un dominio di integrità, da cui lo è pure $R[y_1,...\,,y_{n+1}]$. Dunque,~$R[Y]$ è un dominio di integrità per ogni insieme finito $Y$, ed in particolare lo è per $Y=X_f\cup X_g$. Per l'arbitrarietà di $f,g\in R[X]$, possiamo concludere che $R[X]$ è un dominio di integrità.
\end{proof}

\noindent Concludiamo con un'osservazione che acquisirà importanza quando passeremo allo studio dell'estensione di campi. Preso un anello commutativo $R$ e un qualunque oggetto $x\not\in R$, l'anello dei polinomi $R[x]$ è il più piccolo anello contenente $R$ e $x$. Infatti, se $S$ è un anello contenente $R$ e $x$, per la chiusura di $S$ rispetto a somma e prodotto esso conterrà tutte le potenze non negative $\{x^0, x^1, x^2, ...\}$ di $x$ e tutte le combinazioni lineari tra potenze di $x$ ed elementi di $R$, cioè tutti gli elementi della forma $a_nx^n + ... + a_1x+a_0$ con $a_0,...\,,a_n\in R$. In generale, se $X$ è un insieme non vuoto, possiamo quindi vedere $R[X]$ come la più piccola ``estensione'' di $R$ contenente $X$, cioè come il più piccolo anello contenente sia $R$ che $X$.

\end{document}