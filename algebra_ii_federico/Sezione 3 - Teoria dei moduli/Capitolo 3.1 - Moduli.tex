\documentclass{article}

\usepackage[utf8]{inputenc}
\usepackage[english]{babel}
\usepackage{amsmath}
\usepackage{amssymb}
\usepackage{amsthm}
\usepackage{yhmath}
\usepackage{gensymb}
\usepackage{graphicx}
\usepackage{siunitx}
\usepackage{amscd}
\usepackage{sectsty}
\usepackage{stmaryrd}
\usepackage{tikz-cd}
\usepackage{wrapfig}
\usepackage{xcolor}
\usepackage[margin=1.5in]{geometry}
\usepackage[framemethod=TikZ]{mdframed}

\theoremstyle{definition}
\newtheorem{thm}{Teorema}[subsection]
\newtheorem*{exm}{Esempio}
\renewcommand\qedsymbol{$\blacksquare$}
\addto\captionsenglish{\renewcommand*{\proofname}{Dimostrazione}}
\addto\captionsenglish{\renewcommand{\contentsname}{Indice}}

\sectionfont{\fontsize{20}{15}\selectfont}
\subsectionfont{\fontsize{14}{15}\selectfont}

\renewcommand\thefootnote{\textcolor{red}{\arabic{footnote}}}

\newcommand{\quot}{\operatorname{quot}}
\newcommand{\tor}{\operatorname{tor}}
\newcommand{\sat}{\operatorname{sat}}
\newcommand{\Ann}{\operatorname{Ann}}
\newcommand{\spec}{\operatorname{spec}}
\newcommand{\id}{\operatorname{id}}

\newenvironment{teo}[2][]{%
\ifstrempty{#1}%
{\mdfsetup{%
frametitle={%
\tikz[baseline=(current bounding box.east),outer sep=0pt]
\node[anchor=east,rectangle,fill=blue!25]
{\strut Teorema};}}
}%
{\mdfsetup{%
frametitle={%
\tikz[baseline=(current bounding box.east),outer sep=0pt]
\node[anchor=east,rectangle,fill=blue!25]
{\strut Teorema~#1};}}%
}%
\mdfsetup{innertopmargin=1.5pt,linecolor=blue!25,%
linewidth=1.75pt,topline=true,%
frametitleaboveskip=\dimexpr-\ht\strutbox\relax
}
\begin{mdframed}[]\relax%
}{\end{mdframed}}

\newenvironment{prop}[2][]{%
\ifstrempty{#1}%
{\mdfsetup{%
frametitle={%
\tikz[baseline=(current bounding box.east),outer sep=0pt]
\node[anchor=east,rectangle,fill=blue!25]
{\strut Proposizione};}}
}%
{\mdfsetup{%
frametitle={%
\tikz[baseline=(current bounding box.east),outer sep=0pt]
\node[anchor=east,rectangle,fill=blue!25]
{\strut Proposizione~#1};}}%
}%
\mdfsetup{innertopmargin=1.5pt,linecolor=blue!25,%
linewidth=1.75pt,topline=true,%
frametitleaboveskip=\dimexpr-\ht\strutbox\relax
}
\begin{mdframed}[]\relax%
}{\end{mdframed}}

\newenvironment{cor}[2][]{%
\ifstrempty{#1}%
{\mdfsetup{%
frametitle={%
\tikz[baseline=(current bounding box.east),outer sep=0pt]
\node[anchor=east,rectangle,fill=blue!25]
{\strut Corollario};}}
}%
{\mdfsetup{%
frametitle={%
\tikz[baseline=(current bounding box.east),outer sep=0pt]
\node[anchor=east,rectangle,fill=blue!25]
{\strut Corollario~#1};}}%
}%
\mdfsetup{innertopmargin=1.5pt,linecolor=blue!25,%
linewidth=1.75pt,topline=true,%
frametitleaboveskip=\dimexpr-\ht\strutbox\relax
}
\begin{mdframed}[]\relax%
}{\end{mdframed}}

\newenvironment{lem}[2][]{%
\ifstrempty{#1}%
{\mdfsetup{%
frametitle={%
\tikz[baseline=(current bounding box.east),outer sep=0pt]
\node[anchor=east,rectangle,fill=blue!25]
{\strut Lemma};}}
}%
{\mdfsetup{%
frametitle={%
\tikz[baseline=(current bounding box.east),outer sep=0pt]
\node[anchor=east,rectangle,fill=blue!25]
{\strut Lemma~#1};}}%
}%
\mdfsetup{innertopmargin=1.5pt,linecolor=blue!25,%
linewidth=1.75pt,topline=true,%
frametitleaboveskip=\dimexpr-\ht\strutbox\relax
}
\begin{mdframed}[]\relax%
}{\end{mdframed}}

\newenvironment{defn}[2][]{%
\ifstrempty{#1}%
{\mdfsetup{%
frametitle={%
\tikz[baseline=(current bounding box.east),outer sep=0pt]
\node[anchor=east,rectangle,fill=green!35]
{\strut Definizione};}}
}%
{\mdfsetup{%
frametitle={%
\tikz[baseline=(current bounding box.east),outer sep=0pt]
\node[anchor=east,rectangle,fill=green!35]
{\strut Definizione:~#1};}}%
}%
\mdfsetup{innertopmargin=1.5pt,linecolor=green!35,%
linewidth=1.75pt,topline=true,%
frametitleaboveskip=\dimexpr-\ht\strutbox\relax
}
\begin{mdframed}[]\relax%
}{\end{mdframed}}

%%%%%%%%%%%%%%%%%%%%%%%%%%%%%%%%%%%%%%%%%%%%%%%%%%%%%%%%%%%%%%%%%%%%%

\begin{document}

\section{Teoria dei moduli}
\vspace{1.75mm}
\subsection{Moduli}

Introduciamo ora il concetto di modulo, una generalizzazione del concetto di spazio vettoriale in cui gli scalari costituiscono un anello e non necessariamente un campo.

\begin{defn}[]{}
Sia $R$ un anello. Un gruppo abeliano $(M,\oplus)$ dotato di un'operazione $\ast\colon R\times M\to M$ si dice \underline{$R$-modulo sinistro} se per ogni $r,r_1,r_2\in R$ e $m,m_1,m_2\in M$ si ha che:

\noindent (i) $(r_1+r_2)\ast m=r_1\ast m \oplus r_2\ast m$ (distributività sinistra);

\noindent (ii) $r\ast (m_1\oplus m_2)=r\ast m_1\oplus r\ast m_2$ (distributività destra);

\noindent (iii) $r_1\ast (r_2\ast m)=(r_1r_2)\ast m$ (associatività);

\noindent (iv) $1_R\ast m=m$.
\end{defn}

\noindent Analogamente, un $R$-modulo destro è un gruppo abeliano $(M,\oplus)$ dotato di un'operazione $\ast \colon M\times R\to M$ per cui valgono proprietà analoghe ma con gli elementi di $R$ scritti a destra. Se $R$ è un anello commutativo, i concetti di $R$-modulo destro e sinistro coincidono.\footnote{Ogni modulo destro è isomorfo al corrispondente modulo sinistro, e si parla infatti di modulo bilatero.}

\begin{exm}Ogni spazio vettoriale $V$ su un campo $\mathbb{K}$ può essere pensato come un $\mathbb{K}$-modulo, dove $\ast\colon \mathbb{K}\times V\to V$ è la moltiplicazione per scalari. Viceversa, essendo $\mathbb{K}$ commutativo, ogni $\mathbb{K}$-modulo è bilatero e può quindi essere pensato come uno spazio vettoriale su $\mathbb{K}. \ \square$\end{exm}

\begin{exm}Ogni gruppo abeliano $G$ può essere visto come un modulo sull'anello degli interi. Si consideri l'operazione $\ast\colon \mathbb{Z}\times G\to G$ definita come $0\ast g=0_G$, $n\ast g=g+g+...+g$ (somma di $n$ termini) e $(-n)\ast g=-(n\ast g)$ per ogni $n>0$ e $g\in G$. Si verifica facilmente che $G$ dotato di tale operazione soddisfa le proprietà (i)-(iv) ed è quindi uno $\mathbb{Z}$-modulo$. \ \square$\end{exm} 

\begin{exm}Sia $R$ un anello e sia $I\lhd R$ un ideale sinistro. Allora, $I$ è un $R$-modulo sinistro, dove $\ast\colon R \times I\to I$ è il prodotto dell'anello $R$, ed è ben definito in quanto per definizione di ideale sinistro $r\ast a=ra\in I$ per ogni $r\in R$ e $a\in I. \ \square$\end{exm}

\begin{exm}Sia $R$ un anello e sia $n$ un intero positivo. Si consideri il prodotto cartesiano $R^n=\{(r_1,...\,,r_n): r_1,...\,,r_n\in R\}$ dotato della moltiplicazione componente per componente $\ast\colon R\times R^n\to R^n$ definita come $r\ast (r_1,...\,,r_n)=(rr_1,...\,,rr_n)$. Si verifica facilmente che $R^n$ dotato di tale operazione soddisfa le proprietà (i)-(iv) ed è quindi un $R$-modulo sinistro$. \ \square$\end{exm}

\noindent L'esempio seguente è particolarmente importante nell'algebra lineare perché permette di dimostrare l'esistenza della forma canonica razionale e di Jordan di una matrice.\footnote{Riprenderemo questo argomento dopo il \emph{Teorema di struttura per i gruppi abeliani finitamente generati}.}

\begin{exm}Sia $V$ uno spazio vettoriale su un campo $\mathbb{K}$ e sia $\alpha\in \operatorname{End}_{\mathbb{K}}(V)$ un endomorfismo di $V$. Preso $f(x)=\sum\limits_{i=0}^n a_i x^i \in \mathbb{K}[x]$, si consideri l'operazione $\ast_{\alpha}\colon \mathbb{K}[x]\times V\to V$ definita come $f\ast_{\alpha} v =f_{\alpha}(v)$, dove $f_{\alpha}=\sum\limits_{i=0}^n a_i \alpha^i \in \operatorname{End}_{\mathbb{K}}(V)$.\footnote{Ricordiamo che l'insieme degli endomorfismi di un gruppo è un anello secondo le operazioni di somma puntuale e di composizione di funzioni. In questo caso, $a_i \alpha^i$ è l'endomorfismo che mappa $v\mapsto a_i\cdot \alpha^i(v)$, dove $\alpha^i$ indica la composizione $\underbrace{\alpha\circ \alpha\circ ... \circ \alpha}_{i \text{ volte}}$, inteso che $\alpha^0=\operatorname{id}_{V}$.} Allora, si verifica facilmente che $V$ dotato di tale operazione soddisfa le proprietà (i)-(iv) ed è quindi un $\mathbb{K}[x]$-modulo sinistro$. \ \square$\end{exm}

\noindent Dimostriamo ora due proprietà dei moduli.

\begin{prop}[3.1.1]{}
Sia $R$ un anello e sia $M$ un $R$-modulo sinistro. Allora, 

\noindent (a) $0_R\cdot m=0_M$ per ogni $m\in M$;

\noindent (b) $r\cdot 0_M=0_M$ per ogni $r\in R$.
\end{prop}
\vspace{-4mm}
\begin{proof}
(a) Per la distributività sinistra $0_R\cdot m=(0_R+0_R)\cdot m=0_R\cdot m+0_R\cdot m$. Dunque, sommando l'opposto $-0_R\cdot m$ ad entrambi i membri, otteniamo che $0_M=0_R\cdot m$.

\vspace{0.5mm}

\noindent (b) Per la distributività destra $r\cdot 0_M=r\cdot (0_M+0_M)=r\cdot 0_M+r\cdot 0_M$. Dunque, sommando l'opposto $-r\cdot 0_M$ ad entrambi i membri, otteniamo che $0_M=r\cdot 0_M$.
\end{proof}

\begin{defn}[]{}
Sia $R$ un anello e sia $M$ un $R$-modulo sinistro. Un sottogruppo abeliano $A\subseteq M$ si dice $R$-sottomodulo di $M$ se $r\cdot a\in A$ per ogni $r\in R$ e $a\in A$.
\end{defn}

\noindent Un sottomodulo è quindi un sottogruppo abeliano $A\subseteq M$ per cui $(A, \cdot_{R\times A} \colon R\times A\to A)$ è di nuovo un $R$-modulo (sto quindi effettuando una restrizione dell'operazione $\cdot$).

\begin{prop}[3.1.2]{}
Sia $R$ un anello, $M$ un $R$-modulo sinistro e sia $A\subseteq M$ un $R$-sottomodulo. Allora, $(M/A, \cdot \colon R\times M/A\to M/A)$ è un $R$-modulo sinistro, ove $r\cdot (m+A)=r\cdot m+A$ e $\overline{f}(r,m+A)=r\cdot m+A$.
\end{prop}
\vspace{-4mm}
\begin{proof}
Diagramma negli appunti cartacei. La dimostrazione è inesistente, ottimo.
\end{proof}

\begin{prop}[3.1.3]{}
Sia $R$ un anello, $M$ un $R$-modulo sinistro, $A,B\subseteq M$ sono $R$-sottomoduli. Allora, $A+B=\{a+b: a\in A,b\in B\}$ è un $R$-sottomodulo di $M$.
\end{prop}
\vspace{-4mm}
\begin{proof}
Sappiamo già che $A+B\subseteq M$ è un sottogruppo abeliano. Siano $a+b\in A+B$ e $r\in R$. Allora, $r\cdot (a+b)=r\cdot a+r\cdot b\in A+B$ perché $r\cdot a\in A$ e $r\cdot b\in B$ per definizione di sottomodulo.
\end{proof}


\begin{prop}[3.1.4]{}
Sia $R$ un anello e $M$ un $R$-modulo sinistro. Per $m\in M$ sappiamo che $R\cdot m=\{r\cdot m: r\in R\}$ è un $R$-sottomodulo di $M$. Siano $m_1,...\,,m_n\in M$. Allora, $\sum\limits_{i=1}^n R\cdot m_i=R\cdot m_1+...+R\cdot m_n=\{m\in M: \exists r_1,...\,,r_n\in R: m=\sum\limits_{i=1}^n r_i\cdot m_i\}$ è un $R$-sottomodulo di $M$.  
\end{prop}
\vspace{-4mm}
\begin{proof}
Usando distributività sx, $R\cdot m$ è un sottogruppo abeliano. Usando associatività, si conclude mostrando che $R\cdot m$ è un $R$-sottomodulo. Ora procediamo per induzione grazie alla \emph{Proposizione 3.1.3}.
\end{proof}

\begin{defn}[]{}
Sia $R$ un anello e sia $M$ un $R$-modulo sinistro. Definiamo \underline{numero minimo di generatori} $d_R(M)$ il più piccolo $n\in \mathbb{N}$ per cui esistano $m_1,...\,,m_n\in M$ tali che $M=\sum\limits_{i=1}^n R\cdot m_i$ Se tale $n\in \mathbb{N}$ non esiste, poniamo $d_R(M)=\infty.$ Diciamo che $M$ è \underline{finitamente generato} se $d_R(M)<\infty.$
\end{defn}

\noindent\textbf{Lezione del 13/11/2019} (appunti grezzi)

\noindent Manca tutto un primo pezzo, Trenord ti voglio bene anche io

\noindent Esistono i corrispondenti dei 3 teoremi di isomorfismo per gli $R$-moduli.

\begin{teo}[3.x.y: Primo teorema d'isomorfismo]{}
Sia $\phi\colon M\to N$ un omomorfismo di $R$-moduli, dove $R$ è un anello. Allora, l'omomorfismo indotto $\phi_{\star}\colon M/\ker(\phi)\to \operatorname{Im}(\phi)$ è un isomorfismo di $R$-moduli.
\end{teo}
\vspace{-4mm}
\begin{proof}
Dimostrazione mancante.
\end{proof}

\begin{teo}[3.x.y: Secondo teorema d'isomorfismo]{}
Sia $R$ un anello, $M$ un $R$-modulo, e siano $A,B\subseteq M$ degli $R$-sottomoduli. Allora, esiste un isomorfismo di $R$-moduli $\pi_{\star}\colon A/(A\cap B)\to (A+B)/B$.
\end{teo}
\vspace{-4mm}
\begin{proof}
Sia $\tau\colon M\to M/B$ la proiezione canonica, cioè $\tau(m)=m+B$, e sia la restrizione $\tau_{A}=\pi$. Allora, per il \emph{Primo teorema d'isomorfismo} la mappa $\pi_{star}\colon A/\ker(\pi)\to \operatorname{Im}(\pi)$ è un isomorfismo. Poiché $\ker(\pi)=\ker(\tau)\cap A=B\cap A$ e $\operatorname{Im}(\pi)=\{a+B: a\in A\}=(A+B)/B$, abbiamo concluso.
\end{proof}

\begin{teo}[3.x.y: Terzo teorema d'isomorfismo]{}
Sia $R$ un anello, $M$ un $R$-modulo, $A\subseteq B\subseteq M$ degli $R$-sottomoduli. Allora, esiste un isomorfismo di $R$-moduli $\psi_{\star}\colon (M/A)/(B/A)\to M/B$.
\end{teo}
\vspace{-4mm}
\begin{proof}
Sia $\psi\colon M/A\to B/A$ la mappa definita come $\psi(m+A)=m+B$. Poiché $\psi$ è un omomorfismo di $R$-moduli, per il \emph{Primo teorema d'isomorfismo} la mappa indotta $\psi_{\star}\colon (M/A)/\ker(\psi)\to \operatorname{Im}(\psi)$ è un isomorfismo. Essendo $\operatorname{Im}(\psi)=\{m+B: m\in M\}=M/B$ e $\ker(\psi)=\{m+A: m\in M, m+B=B\}=\{m+A: m\in B\}=B/A$, abbiamo concluso.
\end{proof}

\begin{prop}[]{}
Sia $R$ un anello, $M$ un $R$-modulo sinistro e $B\subseteq M$ un $R$-sottomodulo di $M$. Allora $d_R(M)\leq d_R(B)+d_R(M/B)$ e $d_R(M/B)\leq d_R(M)$.
\end{prop}
\vspace{-4mm}
\begin{proof}
Se $B$ o $M/B$ non sono finitamente generati, cioè $d_R(B)=\infty$ o $d_R(M/B)=\infty$, la prima equazione è banalmente vera. Siano quindi $d_R(B)=k<\infty$ e $d_R(M/B)=n<\infty$. Allora, esistono $m_1,...\,,m_k\in B$ tali che $B=\sum\limits_{i=1}^{k} R\cdot m_i$ ed esistono $t_1,...\,,t_n\in M$ tali che $M/B=\sum\limits_{i=1}^n R\cdot (t_i+B)$. Dunque, per ogni $m\in M$ esistono $r_1,...\,,r_n\in R$ tali che $m+B=\sum\limits_{i=1}^n r_i\cdot (t_i+B)$, cioè $m-\sum\limits_{i=1}^n r_i\cdot t_i\in B$. Allora, esistono $s_1,...\,,s_k\in R$ tali che $m-\sum\limits_{i=1}^n r_i\cdot t_i=\sum\limits_{j=1}^k s_j\cdot m_j$, da cui $m=\sum\limits_{i=1}^n r_i\cdot t_i+\sum\limits_{j=1}^k s_j\cdot m_j$, cioè $d_R(M)\leq n+k$. 

Per quanto riguarda la seconda disuguaglianza, possiamo assumere che $d_R(M)=n<\infty$, altrimenti è banalmente vera. Dunque, esistono $m_1,...\,,m_n\in M$ tali che $M=\sum\limits_{i=1}^n R\cdot m_i$, quindi per ogni $m\in M$ esistono $r_1,...\,,r_n\in R$ tali che $m=\sum\limits_{i=1^n}r_i\cdot m_i$, da cui $m+B=\sum\limits_{i=1}^n r_i\cdot (m_i+B)$, e per l'arbitrarietà di $M$ significa che $M/B=\sum\limits_{i=1}^n R\cdot (m_i+B)$. Dunque, $d_R(M/B)\leq n$ come desiderato.
\end{proof}

\begin{prop}[]{}
Sia $R$ un anello commutativo. Allora, $R$ è noetheriano se e solo se ogni sottomodulo di un $R$-modulo finitamente generato è finitamente generato.
\end{prop}
\vspace{-4mm}
\begin{proof}
Procediamo per induzione su $d=d_R(M)$. Se $d=1$, esiste $m\in M$ tale che $M=R\cdot m$. Sia $\tau_m\colon R\to M$ la mappa definita come $\tau_m(r)=r\cdot m$. Osserviamo che $\tau_m(0)=0$, $\tau_m(r_1+r_2)=\tau_m(r_1)+\tau_m(r_2)$ e $\tau_m(r\cdot r_1)=r\cdot r_1\cdot m =r\cdot \tau_m(r_1)$, quindi $\tau_m$ è un omomorfismo di $R$-moduli. Sia $B\subseteq M$ un $R$-sottomodulo e sia $I_B=\{r\in R: \tau_m(r)\in B \}\subseteq R$. Poiché $I_B$ è un sottogruppo abeliano e presi $a\in I_B$ e $r\in R$ sappiamo che $r\cdot a\in I_B$ essendo $B$ un sottomodulo, vale $I_B\lhd R$. Dunque, essendo $R$ noetheriano per ipotesi, esistono $a_1,...\,,a_n\in I_B$ tale che $I_B=\langle a_1,...\,,a_n \rangle$. Poiché $B=\tau_m(I_B)=\operatorname{Im}(\tau_m \raisebox{-.5em}{$\vert_{I_B}$})$, per la proposizione precedente concludiamo che $d_R(B)<\infty$. Supponiamo ora per induzione forte che tale affermazione valga per $k\leq d$, e mostriamo che vale per $d+1$. Sia $M$ un $R$-modulo sinistro con $d_R(M)=d+1$. Allora, esistono $m_0,...\,,m_{d}\in M$ tali che $M=\sum\limits_{k=0}^d R\cdot m_k$. Sia $B\subseteq M$ un sottomodulo e sia $M_{\star}=\sum\limits_{k=1}^d R\cdot m_k$. Poiché $d_R(M_{\star})\leq d$, $M/M_{\star}=R\cdot (m_0+M_{\star})$. Sia $\pi\colon M\to M/M_{\star}$ la proiezione canonica, dove $d_{R}(M/M_{\star})\leq 1$. Per ipotesi induttiva, $d_R(B\cap M_{\star})<\infty$, quindi $d_R(\pi(B))<\infty$. Poiché $\pi(B)=(B+M_{\star})/M_{\star}\subseteq M/M_{\star}$, per la proposizione precedente $d_R(B)\leq d_R(B\cap M_{\star})+d_R(B/(B\cap M_{\star}))$. Ma per ipotesi induttiva sappiamo che $d_R(B\cap M_{\star})<\infty$ e $B/(B\cap M_{\star})\simeq \pi(B)$ per il \emph{Secondo teorema d'isomorfismo}, quindi $d_R(B/(B\cap M_{\star}))<\infty$ e $d_R(B)<\infty$, da cui la tesi. 

Viceversa, sia $M=R$ con il prodotto di $R$ (tale $R$-modulo è detto $R$-modulo regolare).\footnote{Sto pensando $M=R$ come gruppo abeliano secondo il prodotto di $R$, essendo $R$ commutativo.} Poiché $B\subseteq R$ è un sottomodulo se e solo se $B\lhd R$ è un ideale, per ipotesi sappiamo che $d_R(B)<\infty$ pensando $B$ come sottomodulo, cioè $d_R(B)<\infty$ pensando ora $B$ come ideale, da cui $R$ è noetheriano.
\end{proof}

\end{document}